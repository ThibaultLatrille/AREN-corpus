\documentclass[8pt]{article}

\usepackage[T1]{fontenc}
\usepackage[utf8]{inputenc}
\usepackage{graphicx}
\usepackage{lmodern}
\usepackage{amsmath}
\usepackage{xfrac}
\usepackage{amsthm}
\usepackage{listings}
\usepackage{enumerate}
\usepackage{amssymb}
\usepackage{cancel}
\usepackage{amsfonts}
\usepackage{float}
\usepackage{fullpage}

\DeclareUnicodeCharacter{200A}{ } 
\renewcommand*\contentsname{Table des matières}

\PassOptionsToPackage{hyphens}{url}\usepackage{hyperref}

\usepackage{listings}
\author{ControverSciences\textit{ et al} }
\title{Projet AREN - Corpus de ressources \\  Énergies fossiles et renouvelables.}
\date{7 Mars 2018}

\begin{document}
\maketitle

\tableofcontents

\newpage
\section{Textes à débattre}
\subsection{}
\begin{itemize}
	\item \textbf{Lien : }  \url{} 
	\item \textbf{Auteur : } 
	\item \textbf{Date : }  
	\item \textbf{Source : }  
\end{itemize}

\newpage

\newpage
\section{Corpus de ressources}

\subsection{Charbon : le fossile qui a de beaux restes}
\begin{itemize}
	\item \textbf{Lien : }  \url{https://www.youtube.com/watch?v=kk0TIhy2D3g} 
	\item \textbf{Auteur : } Julien Goetz
	\item \textbf{Date : } 22 nov. 2014
	\item \textbf{Durée : } 3 minutes 37 secondes
	\item \textbf{Description : } Vous pensez que le XIXeme siècle, ce temps lointain de l'invention de la machine à vapeur, était le siècle du charbon ? Perdu. Nous n'avons jamais consommé autant de charbon qu'aujourd'hui et si l'on suit la tendance actuelle, ce n'est pas près de s'arrêter. Certes c'est le combustible fossile le plus polluant mais c'est aussi l'un des moins chers. Bienvenue dans un monde de suie.
	\item \textbf{Source : } DataGueule, chaque épisode de cette émission hebdomadaire produite par france4 tente de révéler et décrypter les mécanismes de la société et leurs aspects méconnus.
\end{itemize}

\subsection{Énergies fossiles : mortelles subventions}
\begin{itemize}
	\item \textbf{Lien : }  \url{https://www.youtube.com/watch?v=aUmJ35kMq1Q} 
	\item \textbf{Auteur : } Julien Goetz
	\item \textbf{Date : } 8 juillet 2015
	\item \textbf{Durée : } 4 minutes 17 secondes
	\item \textbf{Description : } Alors que nos émissions de CO2 dans l'atmosphère continuent de battre des records et que le changement climatique devient notre compagnon de route, pourquoi est-il si difficile de quitter les énergies fossiles - charbon, pétrole et gaz - pourtant responsables d'une grande partie des dégâts ? Peut-être parce que tout un système de subventions les rends particulièrement attrayantes et peu chères, tout autant pour l'industrie que pour nous, les consommateurs finaux. 
	\item \textbf{Source : } DataGueule, chaque épisode de cette émission hebdomadaire produite par france4 tente de révéler et décrypter les mécanismes de la société et leurs aspects méconnus.
\end{itemize}

\subsection{Ensemble, tout devient fossile - " 2° avant la fin du monde}
\begin{itemize}
	\item \textbf{Lien : }  \url{https://www.youtube.com/watch?v=aUmJ35kMq1Q} 
	\item \textbf{Auteur : } Julien Goetz
	\item \textbf{Date : } 9 novembre 2015
	\item \textbf{Durée : } 4 minutes 32 secondes
	\item \textbf{Description : } Que s’est-il passé ces deux cent dernières années ? Comment en sommes-nous arrivés à ce paradoxe moderne : une société dont nous chérissons le confort mais dont l’obtention de ce même confort s’est fait au prix de dégâts toujours plus croissants sur les écosystèmes de la planète ? Creusons les rouages de nos sociétés comme nous avons creusé le globe à la recherche d’énergies pour faire tourner notre monde.
	\item \textbf{Source : } DataGueule, chaque épisode de cette émission hebdomadaire produite par france4 tente de révéler et décrypter les mécanismes de la société et leurs aspects méconnus.
\end{itemize}

\end{document}


