\documentclass[8pt]{article}

\usepackage[T1]{fontenc}
\usepackage[utf8]{inputenc}
\usepackage{graphicx}
\usepackage{lmodern}
\usepackage{amsmath}
\usepackage{xfrac}
\usepackage{amsthm}
\usepackage{listings}
\usepackage{enumerate}
\usepackage{amssymb}
\usepackage{cancel}
\usepackage{amsfonts}
\usepackage{float}
\usepackage{fullpage}
\usepackage{pdfpages}

\DeclareUnicodeCharacter{200A}{ } 
\renewcommand*\contentsname{Table des matières}

\PassOptionsToPackage{hyphens}{url}\usepackage{hyperref}

\usepackage{listings}
\author{ControverSciences\textit{ et al} }
\title{Projet AREN - Corpus de ressources \\  Énergies fossiles et renouvelables}
\date{7 Mars 2018}

\begin{document}
\maketitle

\tableofcontents

\newpage
\section{Textes à débattre}

\subsection{Jurassic Fuels}
\begin{itemize}
	\item \textbf{Lien : }  \url{http://www.sciencepresse.qc.ca/blogue/vincent-jase/2017/09/20/jurassic-fuels-carburants-fossiles} 
	\item \textbf{Auteur : } Vincent Jase est un Youtubeur québécois se dédiant à la communication et à la vulgarisation scientifique.
	\item \textbf{Date : }  20 septembre 2017
	\item \textbf{Source : }  L'Agence Science Presse est un média indépendant, à but non lucratif, fondé à Montréal en 1978. Sa mission est d’alimenter les médias en nouvelles scientifiques. 
\end{itemize}

Les carburants fossiles qui comprennent le charbon, le pétrole et le gaz naturel sont la source d’énergie la plus utilisée dans la société moderne. Votre voiture brûle probablement de l’essence et si vous vivez aux États-Unis, c’est le deux tiers de l’électricité qui est produite dans des centrales thermiques fonctionnant aux carburants fossiles (presque moitié-moitié charbon et gaz naturel). Pourtant les impacts environnementaux de ces sources d’énergie sont multiples et sérieux.\\

La combustion des carburants fossiles est l’un des principaux contributeurs aux changements climatiques, mais l’empreinte écologique de leur utilisation ne se limite pas au CO2. En effet, leur combustion entraîne la libération dans l’atmosphère de microparticules qui sont responsables de l’augmentation des troubles respiratoires, spécialement chez les populations à risques comme les enfants. Bien que les carburants fossiles soient principalement constitués de chaînes de carbone, leur combustion entraine également la dispersion de divers autres contaminants comme le souffre, qui contribue au phénomène des pluies acides. L’extraction de la ressource est également à considérer, comme dans le cas des sables bitumineux (pétrole) et de la fracturation hydraulique (gaz naturel) qui sont mis en cause dans la pollution de cours d’eau.\\

Mais bien entendu, les carburants fossiles n’ont pas que des défauts, sinon leur remplacement aurait été complété il y a longtemps. Leur densité et leur efficacité énergétique sont deux facteurs qui contribuent à les rendre aussi tenaces. Pourtant avec l’émergence de sources d’énergies renouvelables, leurs heures sont comptées.\\

Afin de mieux comprendre les enjeux entourant le plus grand défi de la prochaine décennie, soit celui de la transition d’une économie basée sur la combustion du carbone vers une économie plus verte basée sur des sources d’énergies renouvelables, il faut poser un regard nuancé et critique sur chaque forme d’énergie qui est à notre disposition. Quels sont exactement les avantages et les inconvénients de l’utilisation des carburants fossiles ? Est-ce que le charbon, le pétrole et le gaz naturel sont aussi polluants les uns que les autres ? Quel avenir pour les carburants fossiles dans un monde plongé dans les changements climatiques ?

\newpage

\subsection{"N'investissez plus dans les énergies fossiles" : le cri d'alarme de 80 économistes}
\begin{itemize}
	\item \textbf{Lien : }  \url{https://www.nouvelobs.com/planete/20171207.OBS8908/n-investissez-plus-dans-les-energies-fossiles-le-cri-d-alarme-de-80-economistes.html} 
	\item \textbf{Auteur : } Benjamin Aleberteau est journaliste à l'OBS et rédacteur en chef et vice-président de Worldzine.
	\item \textbf{Date : }  7 décembre 2017 
	\item \textbf{Source : }  Le Nouvel Observateur est un magazine d'actualité hebdomadaire français est classé à gauche, avec une ligne « sociale-démocrate ».
\end{itemize}


Quatre-vingt économistes réunis contre les énergies fossiles. Dans une déclaration publiée jeudi 7 décembre 2017 par l’ONG 350.org, les économistes d’une vingtaine de nationalités différentes appellent à ne plus investir dans les énergies fossiles. \\

\textbf{La fin des énergies fossiles}\\

Les 80 économistes déplorent l'absence dans les débats d'un problème qu'ils jugent primordial : le financement des énergies fossiles. Selon eux, des acteurs continuent à investir "dans de nouveaux projets de production et d’infrastructures charbonnières, gazières et pétrolières", ce qui représente un danger pour la planète. La déclaration, signée notamment par plusieurs Français, à l’image de Cécile Renouard ou Patrick Criqui, et par l'ancien ministre grec Yanis Varoufakis, expose son objectif :
"Nous appelons à la fin immédiate de tout investissement dans de nouveaux projets de production et d'infrastructure de combustibles fossiles, et encourageons une hausse significative du financement des énergies renouvelables."\\

Les signataires constatent que, partout dans le monde, "le changement climatique et les destructions environnementales prennent une ampleur sans précédent". Des actions inédites doivent faire leur apparition, selon leur propos, pour limiter les conséquences néfastes de notre dépendance au pétrole, au charbon et au gaz. Ils rappellent que ces actions doivent être menées "sans délai", afin d’en finir avec "l’exploration et l’expansion des projets fossiles".\\

\textbf{Développer les énergies renouvelables}\\

Les études citées par la déclaration montrent que le CO2 des gisements de combustibles fossiles exploités sur la planète suffit à placer la Terre au-delà du seuil de réchauffement climatique critique. "Il n’y a plus de place pour de nouvelles infrastructures fossiles et il n’y a aucune raison de continuer d’investir dans le secteur", signale le groupe d’experts en économie.\\

Pour faire entendre leur appel, les économistes se tournent vers les institutions de développement et les investisseurs privés et publics qui doivent, selon eux, se diriger "vers les énergies renouvelables sûres" et "montrer la voie en mettant fin à l’exploitation des combustibles fossiles". La déclaration conclut sur l’idée que le développement des énergies renouvelables est actuellement bloqué par l’hypothèse erronée que portent certains acteurs sur la prétendue "nécessité" des énergies fossiles.\\

\newpage


\section{Corpus de ressources}

\subsection{Charbon : le fossile qui a de beaux restes}
\begin{itemize}
	\item \textbf{Lien : }  \url{https://www.youtube.com/watch?v=kk0TIhy2D3g} 
	\item \textbf{Auteur : } Julien Goetz
	\item \textbf{Date : } 22 nov. 2014
	\item \textbf{Durée : } 3 minutes 37 secondes
	\item \textbf{Description : } Vous pensez que le XIXeme siècle, ce temps lointain de l'invention de la machine à vapeur, était le siècle du charbon ? Perdu. Nous n'avons jamais consommé autant de charbon qu'aujourd'hui et si l'on suit la tendance actuelle, ce n'est pas près de s'arrêter. Certes c'est le combustible fossile le plus polluant mais c'est aussi l'un des moins chers. Bienvenue dans un monde de suie.
	\item \textbf{Source : } DataGueule, chaque épisode de cette émission hebdomadaire produite par france4 tente de révéler et décrypter les mécanismes de la société et leurs aspects méconnus.
\end{itemize}

\subsection{Énergies fossiles : mortelles subventions}
\begin{itemize}
	\item \textbf{Lien : }  \url{https://www.youtube.com/watch?v=aUmJ35kMq1Q} 
	\item \textbf{Auteur : } Julien Goetz
	\item \textbf{Date : } 8 juillet 2015
	\item \textbf{Durée : } 4 minutes 17 secondes
	\item \textbf{Description : } Alors que nos émissions de CO2 dans l'atmosphère continuent de battre des records et que le changement climatique devient notre compagnon de route, pourquoi est-il si difficile de quitter les énergies fossiles - charbon, pétrole et gaz - pourtant responsables d'une grande partie des dégâts ? Peut-être parce que tout un système de subventions les rends particulièrement attrayantes et peu chères, tout autant pour l'industrie que pour nous, les consommateurs finaux. 
	\item \textbf{Source : } DataGueule
\end{itemize}

\subsection{Ensemble, tout devient fossile - 2° avant la fin du monde}
\begin{itemize}
	\item \textbf{Lien : }  \url{https://www.youtube.com/watch?v=KAPzxR9VYRI} 
	\item \textbf{Auteur : } Julien Goetz
	\item \textbf{Date : } 9 novembre 2015
	\item \textbf{Durée : } 4 minutes 32 secondes
	\item \textbf{Description : } Que s’est-il passé ces deux cent dernières années ? Comment en sommes-nous arrivés à ce paradoxe moderne : une société dont nous chérissons le confort mais dont l’obtention de ce même confort s’est fait au prix de dégâts toujours plus croissants sur les écosystèmes de la planète ? Creusons les rouages de nos sociétés comme nous avons creusé le globe à la recherche d’énergies pour faire tourner notre monde.
	\item \textbf{Source : } DataGueule
\end{itemize}

\subsection{L'essentiel sur l'énergie}
\begin{itemize}
	\item \textbf{Lien : }  \url{http://www.cea.fr/comprendre/Pages/energies/essentiel-sur-energies.aspx}  
	\item \textbf{Date : }   20 décembre 2017 
	\item \textbf{Source : }  Le Commissariat à l’énergie atomique et aux énergies alternatives (CEA) est un organisme public de recherche à caractère scientifique, technique et industriel. Le CEA intervient dans quatre domaines : la défense et la sécurité, les énergies nucléaire et renouvelables, la recherche technologique pour l'industrie et la recherche fondamentale.
\end{itemize}

Nous sommes tous entourés d'énergie : dans notre corps, notre maison, notre environnement... Elle est là, dans notre quotidien, mais qu'est-ce que l'énergie ? Quelles sont les formes de l'énergie ? Ses sources ? Que signifient les expressions "énergies primaires", "énergies secondaires", "énergies renouvelables", "énergies non-renouvelables", "énergies fossiles" ? \\

\textbf{I. Qu'est ce que l'énergie ?}\\

Le mot « énergie » vient du Grec Ancien « énergéia », qui signifie « La force en action ». Ce concept scientifique est apparu avec Aristote et a fortement évolué au cours du temps. Aujourd’hui, l’énergie désigne  « la capacité à effectuer des transformations ». Par exemple, l’énergie c’est ce qui permet de fournir du travail, de produire un mouvement, de modifier la température ou de changer l’état de la matière. Toute action humaine requiert de l’énergie : le fait de se déplacer, de se chauffer, de fabriquer des objets et même de vivre. 
 L’énergie est partout présente autour de nous : dans la rivière qui fait tourner la roue du moulin, dans le moteur d’une voiture, dans l’eau de la casserole que l’on chauffe, dans la force du vent qui fait tourner les éoliennes… et même dans notre corps humain.\\

\textbf{II. Les différentes formes d'énergie}\\

L’énergie peut exister sous plusieurs formes. Parmi les principales :

\begin{itemize}
	\setlength\itemsep{-0.2em}
	\item L’énergie thermique, qui génère de la chaleur ;
	\item L’énergie électrique ou électricité, qui fait circuler les particules – électrons - dans les fils électriques ;
	\item L’énergie mécanique, qui permet de déplacer des objets ;
	\item L’énergie chimique, qui lie les atomes dans les molécules ;
	\item L’énergie de rayonnement ou énergie lumineuse, qui génère de la lumière ;
	\item L’énergie musculaire qui fait bouger les muscles.
\end{itemize}

\textbf{II.1 Conservation de l'énergie}\\

L’énergie se conserve. La quantité totale d'énergie dans un système donné ne change pas, on ne peut donc ni la créer, ni la détruire. L'énergie est transmise d'un élément vers un autre, souvent sous une forme différente. 
Un exemple : quand on chauffe de l'eau, différentes transformations d’énergie ont lieu. En brûlant dans l’air, le bois libère son énergie chimique. Cette énergie se transforme en chaleur, l’énergie thermique, et en lumière, l’énergie de rayonnement. Lors de cette réaction, la quantité d'énergie totale ne change pas, elle change simplement de forme. \\


Un autre exemple : lorsqu’une voiture fonctionne, l’essence libère son énergie chimique en brûlant dans l’air. Elle chauffe le moteur et pousse les pistons (énergie thermique et énergie mécanique). Les pistons font tourner le moteur et les roues, transfert d’énergie mécanique, et la voiture se déplace (énergie cinétique). Au passage, la courroie fait tourner l’alternateur qui transforme une petite partie de l’énergie mécanique en électricité qui sera stockée dans la batterie.\\

\textbf{III. Les sources d'énergie}\\

L’énergie est issue de différentes sources d’énergie qui peuvent être classifiées en deux groupes : les énergies non renouvelables, dont les sources ont des stocks sur Terre limités et les énergies renouvelables qui dépendent d’éléments que la nature renouvelle en permanence. \\

\textbf{III.1 Les sources d'énergie non renouvelables}\\

\textbf{III.1.1 Énergies fossiles }\\

Dans les énergies non renouvelables, on trouve les énergies dites fossiles : ce sont les résidus des matières végétales et organiques accumulés sous terre pendant des centaines de millions d’années. Ces résidus se transforment en hydrocarbure (pétrole, gaz naturel et de schiste, charbon…). Pour pouvoir les exploiter, il faut puiser dans ces ressources qui ne sont pas illimitées, c’est pourquoi les énergies fossiles ne sont pas renouvelables. \\

\textbf{III.1.2 Énergie nucléaire}\\

L’énergie nucléaire est « localisée » dans le noyau des atomes. Dans les centrales nucléaires actuelles, on utilise la fission (cassure) des noyaux d’uranium, élément que l’on retrouve sur Terre dans les mines. Les mines d’uranium s’épuiseront un jour tout comme le charbon, le gaz et le pétrole.
Au rythme de l’utilisation des ressources actuellement exploitées, on estime les réserves de pétrole à 40 ans, de gaz naturel conventionnel à 60 ans et de charbon à 120 ans. Les réserves d’uranium, combustible de l’énergie nucléaire, à 100 ans avec les réacteurs actuels. \\

\textbf{III.2 Les sources d'énergies renouvelables}\\

Le soleil, le vent, l’eau, la biomasse et la géothermie sont des sources qui ne s’épuisent pas et sont renouvelées en permanence. La biomasse et la géothermie sont deux sources d’énergies bien distinctes.
La géothermie est l’énergie générée par la chaleur des profondeurs de la Terre et sa radioactivité. Le mot « géothermie » vient du grec « geo » (la terre) et « thermos » (la chaleur). On l’exploite pour chauffer des habitations grâce à des forages légers. 
La biomasse a, quant à elle, pour source le Soleil dont l’énergie de rayonnement est transformée en énergie chimique par les matières organiques d’origine végétale (bois), animale, bactérienne ou fongique (champignons). Il existe des centrales « biomasse » qui produisent de l’électricité avec la combustion de matières organiques.  \\

Parmi toutes ces sources d’énergie, on distingue les énergies primaires des énergies secondaires.\\

\textbf{III.3 Énergie primaire}\\

Une énergie primaire est une énergie brute n’ayant pas subi de transformation, dont la source se trouve à l’état pur dans l’environnement. Le vent, le Soleil, l’eau, la biomasse, la géothermie, le pétrole, le charbon, le gaz ou l’uranium sont des sources d’énergies primaires.\\

\textbf{III.4 Énergie secondaire}\\

On appelle « énergie secondaire » une énergie qui est obtenue par la transformation d’une énergie primaire. 
Par exemple, l’électricité est une énergie secondaire qu’on obtient à partir de plusieurs énergies primaires : l’énergie solaire avec des panneaux, l’énergie nucléaire avec des réacteurs, l’énergie hydraulique avec des barrages ou encore l’énergie du vent avec des éoliennes. Il n’existe pas d’électricité à l’état naturel.\\

L’essence, le gasoil et les biocarburants sont également des énergies secondaires ; on les obtient par la transformation du pétrole, qui lui, est brut ou de la biomasse. L’hydrogène, qui n'existe pas à l'état pur, est également une énergie chimique secondaire car il faut le produire.\\

\textbf{III.4 Énergie et puissance}\\

On mesure l’énergie à l’aide d’une unité particulière nommée le joule. Son nom vient du physicien anglais James Prescott Joule. Un joule représente par exemple l'énergie requise pour élever une pomme de 100 grammes d'un mètre ou encore l'énergie nécessaire pour élever la température d'un gramme (un litre) d'air sec de un degré Celsius.
Dans le domaine de la nutrition, c’est la kilocalorie qui est utilisée. 1 kilocalorie équivaut à 4,2 kilojoules. Pour évaluer l’énergie utilisée sur une année, on utilise généralement la tonne équivalent pétrole, tep. 1 tep est égale à 41 868 000 000 joules.\\ 

La puissance correspond, quant à elle, à la vitesse à laquelle l'énergie est délivrée. Elle se mesure en watt, ce qui correspond à un joule par seconde. 
Par exemple, si pour faire bouillir un litre d’eau, on utilise d’un côté une flamme d’un gros feu de bois et de l’autre, la flamme d’une bougie : dans les deux cas, la même quantité d’énergie sera utilisée pour faire bouillir l’eau. Seulement, ce sera fait plus rapidement avec un feu qu’avec une bougie. L'énergie est dégagée plus rapidement avec le feu de bois qu'avec la flamme de la bougie. Le feu de bois est donc plus puissant que la flamme de la bougie. \\

\textbf{IV. Utilisation des énergies en France et environnement}\\

L’énergie, en France, est surtout utilisée pour le transport, l’habitat (chauffage), l’industrie, le tertiaire et l’agriculture.
Bien que la dépendance énergétique de la France se soit réduite depuis 1973 grâce à la construction du parc nucléaire, son mix énergétique dépend encore fortement des énergies fossiles qui couvrent près de 50\% de la consommation d’énergie primaire. A eux seuls, le transport et l’habitat représentent en France près de 80\% de la consommation finale. Le bâtiment dépend à plus de 50\% des combustibles fossiles et le transport à 95\% du pétrole. Ces deux secteurs sont à l’origine de plus de 50\% des émissions de CO2, l’un des principaux gaz à effet de serre. 
Ces émissions impactent directement le climat en contribuant au réchauffement climatique. Face à ce défi climatique majeur, il devient indispensable de disposer de sources d’énergie à la fois compétitives et bas carbone (faiblement émettrices de gaz à effet de serre) et de faire évoluer le mix énergétique de la France. \\

\textbf{IV.1 Les défis énergétiques}\\

Toute action humaine requiert de l’énergie. Depuis toujours, l’Homme a cherché à accéder à des sources d’énergie abondantes et peu chères pour satisfaire ses besoins. Mais depuis le début de la révolution industrielle, la société moderne utilise sans compter de l’énergie provenant de sources, qui sont, pour la plupart, non renouvelables. Conséquence, les ressources s’épuisent et la quantité d’émission de gaz à effet de serre dans l’atmosphère, issue de l’exploitation des ressources fossiles, menace le climat. Face à ces réalités, il devient nécessaire de :

\begin{itemize}
	\setlength\itemsep{-0.2em}
	\item Mieux gérer l’utilisation des énergies en faisant notamment moins de gaspillage.
	\item Repenser notre mix énergétique en utilisant des sources d’énergie bas carbone tels que le nucléaire et les énergies renouvelables.
	\item Améliorer les technologies de stockage de l’énergie (batteries, hydrogène).
	\item Continuer à travailler sur les énergies du futur : nucléaire du futur (fission et fusion nucléaire), solaire, éolien, bioénergies.
\end{itemize}

\newpage

\subsection{Sortir de l’âge des fossiles, la bataille du siècle}
\begin{itemize}
	\item \textbf{Lien : }  \url{https://theconversation.com/sortir-de-lage-des-fossiles-la-bataille-du-siecle-87534} 
	\item \textbf{Auteur : } Patrick Criqui est Directeur de recherche émérite au CNRS à l'Université Grenoble Alpes, et Michel Damian est Professeur émérite à l'Université Grenoble Alpes.
	\item \textbf{Date : } 11 décembre 2017
	\item \textbf{Source : } The Conversation France est un média en ligne d'information et d'analyse de l'actualité indépendant, qui publie des articles grand public écrits par les chercheurs et les universitaires. 
\end{itemize}


En 1896 et pour la première fois, un scientifique – le chimiste suédois Svante Arrhenius – estimait qu’un doublement de la teneur de l’atmosphère en CO2 accroîtrait les températures de l’ordre de 5 °C. Un doublement qui ne devait intervenir selon lui qu’après 3 000 ans.
Depuis Arrhenius, les incertitudes sur l’évolution du climat de la planète persistent. Jugez plutôt : entre le premier rapport des experts du GIEC publié en 1990 et le plus récent, paru en 2013, les fourchettes de réchauffement à l’horizon 2100 sont passées de 2-5 °C à 1,5–4,8 °C, en fonction des scénarios d’émission et des incertitudes des modèles !
La réduction de ces incertitudes ne viendra certainement que des manifestations explicites du changement climatique ; ainsi les preuves arriveront toujours trop tard.\\

\textbf{La nouvelle donne climatique}\\

Nous sommes face à un cas d’école pour l’application du principe de précaution : pas de preuves, mais des indications scientifiques convergentes et fiables. D’où l’objectif des 2 °C de réchauffement à ne pas dépasser sur le siècle et, depuis la COP21 de Paris fin 2015, celui de se rapprocher de 1,5 °C.
Malgré ce contexte incertain, on peut toutefois affirmer qu’un mouvement global se dessine en vue d’affronter la nouvelle donne climatique. Cette « transition » est avant tout énergétique : elle vise à affranchir l’économie mondiale de sa dépendance aux énergies fossiles. Car en brûlant charbon, pétrole et gaz naturel pour leurs activités, les hommes libèrent chaque année, par milliards de tonnes, des gaz à effet de serre qui s’accumulent dans l’atmosphère et perturbent le climat.\\

Il est bien difficile de prédire la portée de cette transition. Les plus pessimistes diront qu’il est déjà trop tard, pointant le fossé abyssal qui existe entre l’urgence climatique et le temps long des transformations énergies-climat-sociétés.
Pourtant, quelle qu’en soit l’issue, cette transition est déjà à l’œuvre. Et elle constitue une bataille, au moins pour tout le siècle, qui vaut la peine d’être menée.
Ces politiques « bas carbone » engagées reposent pour l’heure sur la réduction des émissions de gaz à effet de serre en s’appuyant sur deux axes principaux : la maîtrise de la demande d’énergie et le développement d’une offre énergétique décarbonée, grâce en particulier aux énergies renouvelables. Remarquons également que les progrès de la digitalisation pourraient singulièrement accélérer la mise en œuvre de nouveaux systèmes énergétiques plus décentralisés.\\

Un dernier axe, encore hypothétique et qui doit faire face à de nombreux défis, concerne les « émissions négatives ». Elles sont considérées comme indispensables dans les scénarios les plus ambitieux, ceux qui visent à contenir l’augmentation de la température globale à moins de 2 °C. Par émissions négatives, il faut entendre ici l’ensemble des pratiques (comme la reforestation ou la séquestration du carbone dans les sols) et des techniques (stockage du carbone récupéré ou « géo-ingénierie ») qui permettront de réduire le stock de gaz à effet de serre accumulé dans l’atmosphère depuis la révolution industrielle amorcée au XVIIIe siècle.

 \newpage


\textbf{Maîtriser la demande d’énergie}\\

Premier axe des politiques bas carbone, donc : la réduction des émissions de gaz à effet de serre par la maîtrise ou la réduction de la demande énergétique. Dans ce domaine, il faut d’abord faire la part des évolutions structurelles de l’économie mondiale sur une longue période : comme l’illustre bien la dynamique chinoise, le développement des industries lourdes, fortement consommatrices en énergies fossiles, n’est qu’une phase dans le processus de développement économique ; la croissance de l’Empire du Milieu reposera dans les prochaines décennies sur la consommation intérieure, les nouvelles technologies et les services.
Ce sont ces évolutions structurelles qui expliquent la stabilisation des émissions de gaz à effet de serre dans les pays les plus riches depuis maintenant plus de dix ans et le timide ralentissement, plus récent, de la croissance des émissions dans les pays en développement. Dans cette perspective, l’augmentation des émissions de la Chine en 2017, prévue par le Global Carbon Project, ne doit pas être extrapolée sur le long terme.\\


Au-delà des évolutions structurelles, il y a aussi les actions volontaires entreprises, d’abord après les chocs pétroliers puis à partir des années 1990. Elles visent explicitement la réduction de la consommation des énergies fossiles. Ces politiques bas carbone au sens strict du terme figurent au cœur de l’Accord de Paris, conclu en décembre 2015.
Il y a, enfin, la prise de conscience très récente des effets dévastateurs de la pollution atmosphérique et, en particulier, des émissions de particules très fines. Cette pollution devient un problème majeur dans tous les pays, qu’ils soient émergents ou d’industrialisation ancienne. Selon des estimations récentes parues dans The Lancet, la pollution de l’air serait responsable chaque année de quelque 6,5 millions de décès prématurés.\\


Aujourd’hui, la réduction de ces polluants atmosphériques va de pair avec celle des gaz à effet de serre. Et la baisse de ces émissions – en remplaçant les combustibles traditionnels par des énergies modernes pour la cuisson, en fermant des centrales thermiques au charbon, en réduisant la motorisation diesel ou essence pour l’abandonner un jour – présente des co-bénéfices pour la santé et le climat.
Cet aspect explique en bonne partie l’engagement de la Chine et des autres grands pays émergents dans l’Accord de Paris et contribue aussi à l’action de plus en plus déterminée des villes dans la lutte pour réduire ces émissions nocives.\\

\textbf{Développer les énergies renouvelables}\\


Second axe de la transition énergétique : le développement des énergies renouvelables (hydraulique, éolien, solaire, biomasse…).
Uniques sources d’énergie des sociétés préindustrielles, elles ont été délaissées à l’âge des fossiles dont l’exploitation massive remonte à la première révolution industrielle en Angleterre. Pendant longtemps, l’énergie hydraulique – « inventée » dans les Alpes par Aristide Bergès – constituera une exception, en raison de l’importance et de la facilité d’exploitation de son potentiel.
Les autres renouvelables ne réapparaissent dans le paysage énergétique des pays les plus riches qu’au début des années 1970. Aux États-Unis, en juin 1973 – c’est-à-dire quatre mois avant le premier choc pétrolier –, le président Richard Nixon dissout la Commission à l’énergie atomique et la remplace par une agence chargée des énergies non-conventionnelles et renouvelables. À l’époque, l’énergie nucléaire marque déjà outre Atlantique les limites de ses promesses, même si pour un quart de siècle encore, les recherches et financements massifs qui lui seront octroyés limiteront ceux consacrés aux renouvelables.\\


En France, on se souviendra que le CNRS a lancé, en 1975, un premier programme interdisciplinaire sur l’énergie solaire. En 1977, le congrès de la Société internationale de l’énergie solaire se tient à New Delhi et permet à des scientifiques du monde entier de partager leurs avancées en ce domaine.
Après ce premier enthousiasme vient le temps du recul dans les années 1980, et ce en dépit d’un second choc pétrolier. La relance arrivera progressivement, d’abord du côté de l’énergie éolienne avec les succès danois, puis de l’énergie solaire sous l’impulsion de politiques publiques d’incitation actives aux États-Unis, en Europe, puis dans les pays émergents et tout particulièrement en Chine et en Inde.\\


Partout dans le monde aujourd’hui, le secteur de l’énergie est bousculé par le développement des renouvelables… même si, hors hydro-électricité, elles ne fournissent encore que 3\% de l’énergie mondiale et que de nombreux obstacles et inerties doivent encore être vaincus. En 2016, plus d’un milliard de personnes n’ont toujours pas accès à l’électricité et près de trois milliards utilisent pour la cuisson encore exclusivement du bois ou des combustibles très polluants. \\


Arrêtons-nous un instant sur la place du nucléaire dans la transition énergétique. Si cette énergie n’entraîne en effet pas d’émissions directes de gaz à effet de serre, elle n’est pas à proprement parler une énergie renouvelable, puisqu’elle nécessite des ressources fissiles pour fonctionner. Ces ressources étant cependant importantes, le nucléaire pourrait contribuer au niveau mondial à la sortie des fossiles. Mais on ne peut s’attendre qu’à une augmentation assez modeste de sa part dans la production mondiale d’électricité – de 11\% aujourd’hui à 15\% en 2040 dans les scénarios les plus optimistes – compte tenu de nombreux obstacles à surmonter.\\

\textbf{La révolution digitale, un coup de pouce ?}\\

On l’a vu, la maîtrise de la demande d’énergie et le développement des énergies renouvelables constituent les deux piliers des politiques de transition dans tous les grands pays, comme le souligne une vaste étude sur la « décarbonation profonde » conduite en 2015.
La convergence de ces deux axes pourrait être singulièrement amplifiée par la diffusion massive des technologies digitales. Car si la révolution des technologies de l’information et de la communication (ou TIC) s’est avant tout traduite par un surcroît de consommation énergétique (représentant environ 10\% de la consommation mondiale d’électricité, l’avenir pourrait être différent avec la digitalisation généralisée, le big data et l’Internet des objets.\\


Une nouvelle révolution industrielle, celle de l’application des TIC à la gestion du monde matériel, pourrait avoir un impact particulièrement marqué dans le secteur énergétique avec la gestion intégrée de la demande d’énergie et de l’offre renouvelable dans les réseaux intelligents (les smart grids).
Dans les pays industrialisés, ceux-ci peuvent gérer, aux marges des réseaux, les interactions entre bâtiments autonomes en énergie, production solaire décentralisée, stockage, véhicules électriques… Dans les pays en développement, et singulièrement en Afrique, les mini-réseaux basés sur l’énergie solaire pourraient assurer l’accès à l’énergie dans les zones aujourd’hui non connectées.\\



\textbf{Le défi des « émissions négatives » et de l’adaptation}\\

Les politiques climatiques sont prises en étau : d’un côté, les communautés scientifiques ne cessent de confirmer la nécessité d’agir très vite pour répondre à l’urgence climatique ; de l’autre, l’examen attentif des transitions énergétiques passées ou en cours révèle le caractère très progressif des transformations socio-techniques ; dans ce domaine, impossible de « passer en force ».\\


Les transitions bas carbone devraient permettre de réduire le flux annuel des émissions, mais le feront-elles assez rapidement ? En outre, elles ne réduiront pas le stock de CO2 et des autres gaz à effet de serre qui s’accumulent dans l’atmosphère, à un rythme particulièrement soutenu ces dernières décennies.\\


Pour obtenir des trajectoires compatibles avec un réchauffement inférieur à 2 °C, comme le veut l’Accord de Paris, il faudrait en effet ramener à zéro les émissions peu après 2050, puis assurer un développement massif des « émissions négatives ». Ce concept a été introduit dans les scénarios du GIEC, dont un rapport spécial à paraître en 2018 contiendra sans aucun doute des développements sur ce thème. \\

\newpage
\textbf{Comment mettre en œuvre ces émissions négatives ? Différentes pistes sont ouvertes.}\\

On peut d’abord appliquer sur une grande échelle des techniques de capture et stockage du carbone, non plus seulement en les associant aux installations utilisant des énergies fossiles, mais aussi à des centrales énergétiques fonctionnant à la biomasse. On peut également augmenter l’absorption et le stockage du carbone dans les sols. Plus exotique, on peut imaginer pomper du CO2 contenu dans l’atmosphère en le stockant sous forme de carbonates ou en le recyclant.\\

Ou encore, et cela suscite une inquiétude grandissante, en intervenant volontairement sur les grands cycles géochimiques, par dispersion d’aérosols dans l’atmosphère, ensemencement des océans ou déploiement de satellites « parasols ». C’est ce que l’on nomme la « géo-ingénierie », cet ensemble de manipulations à grande échelle de notre environnement, encore toutes hypothétiques et qui pourraient faire peser des risques directs sur les grands équilibres planétaires.
Aucune de ces technologies n’est pour l’heure prête à l’emploi et toutes posent de redoutables problèmes en termes de connaissances scientifiques, financement, éthique et gouvernance.\\


L’avenir est donc ouvert. Il y a des utopies possibles, en tout cas des mouvements de longue période à l’image de cette profonde transformation de la consommation d’énergie et des énergies renouvelables. Il s’agit aujourd’hui d’accélérer cette transformation, par tous les moyens (raisonnables).
Mais dans le brouhaha du monde, cela n’empêchera pas de devoir affronter un autre défi immense – et tout particulièrement pour les pays et les communautés les plus vulnérables : celui de l’adaptation à des modifications climatiques d’origine humaine, que l’on peut tenter de limiter mais qui sont désormais inéluctables.

\subsection{Énergies renouvelables : l'essor sera lent}
\begin{itemize}
	\item \textbf{Lien : }  \url{https://www.pourlascience.fr/sd/energie/energies-renouvelables-lessor-sera-lent-7953.php} 
	\item \textbf{Auteur : } Vaclav Smil est Professeur émérite à la Faculté de l'environnement de l'université du Manitoba (Canada).
	\item \textbf{Date : } 20 Juin 2014
	\item \textbf{Source : } Pour La Science est la version française du mensuel Scientific American. C'est une revue de vulgarisation scientifique dans toutes les disciplines, dont les articles sont signés par les chercheurs eux-mêmes.
\end{itemize}


\includepdf[pages=-]{PLS441EnergiesRenouvelable.pdf}

\end{document}


