\documentclass[8pt]{article}

\usepackage[T1]{fontenc}
\usepackage[utf8]{inputenc}
\usepackage{graphicx}
\usepackage{lmodern}
\usepackage{amsmath}
\usepackage{xfrac}
\usepackage{amsthm}
\usepackage{listings}
\usepackage{enumerate}
\usepackage{amssymb}
\usepackage{cancel}
\usepackage{amsfonts}
\usepackage{float}
\usepackage{fullpage}

\DeclareUnicodeCharacter{200A}{ }
\DeclareUnicodeCharacter{2009}{ } 
\renewcommand*\contentsname{Table des matières}

\PassOptionsToPackage{hyphens}{url}\usepackage{hyperref}

\usepackage{listings}
\author{ControverSciences}
\title{Projet AREN : l'eugénisme}
\date{11 janvier 2017}

\begin{document}
\maketitle

\tableofcontents
\newpage
\section{Textes à débattre}
\subsection{Le CRISPR-Cas9 ou la nouvelle baguette magique des apprentis-sorciers}
\begin{itemize}
	\item \textbf{Lien : }  \url{http://www.lesechos.fr/02/01/2016/lesechos.fr/021591969631_revolution-dans-la-manipulation-des-genes.htm} 
	\item \textbf{Auteur : } Yann Verdo
	\item \textbf{Date : } 02 janvier 2016
	\item \textbf{Description : } Un article très alarmiste qui soulève tout de même des questionnements éthiques de CRISPR-Cas9.
	\item \textbf{Source : } Les échos est un quotidien français d'information économique et financière.
\end{itemize}


Une nouvelle technique de modification du génome, CRISPR-Cas9, s'est répandue comme une traînée de poudre dans les laboratoires. Et soulève de graves questions éthiques. \\

CRISPR-Cas9. Retenez bien ce sigle imprononçable, car vous n'allez pas tarder à en entendre parler, et de plus en plus souvent. Derrière cet acronyme anglais à rallonge, si rébarbatif qu'il semble avoir été inventé exprès pour décourager le profane d'aller plus loin, se cache un nouvel outil d'ingénierie génétique qui, depuis sa mise au point en 2012, s'est répandu comme une traînée de poudre dans les laboratoires du monde entier, suscitant une déferlante de publications scientifiques et de colloques internationaux réunissant tous les pontes de la recherche génomique... Un outil, ­surtout, qui peut changer le cours de notre civilisation et faire basculer l'humanité dans une nouvelle ère qui ressemble à s'y méprendre à celle dans le film de science-fiction « Bienvenue à Gattaca ». \\

La biologiste américaine Jennifer Doudna de l'université de Berkeley, l'une des deux scientifiques à qui revient le re­doutable honneur d'avoir inventé cet outil révolutionnaire, a elle-même appelé au printemps, dans un plaidoyer publié par la revue « Science », pour un moratoire temporaire sur son utilisation, s'agissant du moins des expériences réalisables sur le génome de l'embryon humain. Tel le Dr Franken­stein effrayé par sa propre créature. \\

\textbf{Accessible au niveau master }

« Introduire une modification génétique peut désormais se faire en une semaine, contre plusieurs mois auparavant, et pour un coût dérisoire », observe David Bikard, directeur du laboratoire de biologie de synthèse de l'Institut Pasteur. Il note qu'avec cette technique, les manipulations sur le patrimoine génétique de cellules humaines se sont démocratisées au point d'être désormais effectuées en travaux pratiques par des étudiants en master de biologie ! « Qui plus est, poursuit-il, il est devenu possible d'introduire plusieurs modifications à la fois, ce qui était irréalisable auparavant. »\\

\textbf{Bébés génétiquement modifiés} 

Sur le front de la lutte contre les maladies génétiques, l'édition du génome constitue une arme redoutable dans la main des médecins. Le succès obtenu en mars dernier par une équipe du MIT sur des souris, qui ont été guéries d'une maladie du foie jusque-là réputée incurable, est venu concrétiser le potentiel médical de CRISPR-Cas9. Et les applications sur l'homme se feront d'autant moins attendre que la nouvelle technique est d'une grande fiabilité. Le risque de provoquer des modifications génétiques non désirées - ce que les spécialistes appellent les « off-targets effects » - est faible. « Il a encore diminué depuis que l'équipe de Feng Zhang au MIT [qui dispute au duo Emmanuelle Charpentier-Jennifer Doudna l'antériorité de la découverte de CRISPR-Cas9, NDLR] a publié début décembre dans "Science" une étude présentant un variant de la protéine Cas9 ciblant encore mieux le (ou les) gène(s) visé(s) », note David Bikard. Comme Emmanuelle Charpentier l'écrivait elle-même en octobre dans la revue « Pour la science » : « une chirurgie précise et universelle des génomes [peut] commencer ». \\

Au-delà des aspects médicaux, c'est bien la recherche génétique qui a le plus à gagner de l'avènement de CRISPR-Cas9. Alors que le séquençage à haut débit a permis de décrypter les génomes de milliers d'espèces et d'encore plus d'individus, et que les études d'association établissent des corrélations entre génotypes et phénotypes, mais sans pouvoir les expliquer, CRISPR-Cas9 offre la possibilité de connaître la base moléculaire de ces corrélations. En supprimant un gène donné, on peut en effet déterminer précisément quelle était sa fonction. \\

Naturellement, les problèmes éthiques soulevés par cette révolution technique sont à la mesure des nouvelles possibilités qu'elle offre - c'est-à-dire gigantesques. Vouloir débarrasser son futur bébé d'une maladie génétique grave comme la myopathie semble parfaitement légitime ; mais cela l'est déjà moins s'il s'agit, disons, d'un léger strabisme ; et que dire des parents qui désireront se servir de cet outil pour choisir la couleur des yeux de leur progéniture ? \\

La question éthique se pose avec d'autant plus d'acuité que CRISPR-Cas9 peut être ­utilisé sur deux types de cellules : les cellules différenciées d'un organisme, auquel cas les modifications introduites ne se transmettent pas à sa descendance ; mais aussi les cellules reproductrices, pour lesquelles ces modifications sont transmissibles, et donc en mesure d'orienter l'évolution d'une espèce. Conscient de ce danger, le gratin mondial de la géno­mique, qui s'est réuni début décembre à Washington à l'invitation de l'Académie américaine des sciences pour discuter des enjeux de CRISPR-Cas9, a appelé à la plus grande vigilance. Cela suffira-t-il à éviter les dérives ?

\subsection{Le Crispr-Cas9, outil génétique révolutionnaire et… dangereux}
\begin{itemize}
	\item \textbf{Lien : }  \url{http://www.la-croix.com/Sciences/Sciences-et-ethique/Le-Crispr-Cas9-outil-genetique-revolutionnaire-dangereux-2016-02-09-1200738624} 
	\item \textbf{Auteur : } Denis Sergent
	\item \textbf{Date : } 09 février 2016
	\item \textbf{Description : } Un article aussi très alarmiste qui soulève tout de même des questionnements éthiques de CRISPR-Cas9.
	\item \textbf{Source : } La croix quotidien français, qui se réclame ouvertement chrétien et catholique.
\end{itemize}

Cette nouvelle méthode, qui permet de modifier rapidement le génome, ouvre des perspectives immenses mais soulève de nombreuses questions éthiques.

Depuis la naissance de la biologie moléculaire et du génie génétique dans les années 1970, les chercheurs ont toujours cherché à mettre au point des outils microscopiques de plus en plus précis afin, tel un chirurgien, de réaliser des opérations sur les gènes, ces fragments du ruban génomique (ADN) qui commandent la fabrication des protéines.\\

Couper pour enlever un « gène malade » chez l’homme, le remplacer par un gène correcteur, inactiver un gène ou encore ajouter un gène étranger à des fins thérapeutiques, industrielles ou environnementales chez la bactérie, la plante ou l’animal.\\

Depuis décembre 2015, on parle beaucoup d’une nouvelle technique de transformation du génome, portant un sigle barbare pour le néophyte : Crispr-Cas9. Cet outil est issu de la nature, puisqu’il s’agit d’une adaptation d’un mécanisme de défense des bactéries.\\

En pratique, c’est une trousse à outils, un couteau suisse du génie génétique, permettant de supprimer et d’insérer des gènes (ADN), à un endroit bien précis du chromosome, au sein du génome de n’importe quelle cellule (1). \\

\textbf{Une technique simple et rapide}

« Simple, cette technique est déjà utilisée par les étudiants en master de biologie », observe David Bikard, chercheur à l’Institut Pasteur et spécialiste de cette technique. Elle connaît un essor extrêmement rapide et, aujourd’hui, on estime qu’elle est utilisée dans 3 000 laboratoires dans le monde », poursuit-il. Du fait de sa simplicité, sa rapidité d’action et son faible coût, la technique Crispr-Cas9 a commencé à être exploitée pour toutes sortes d’applications. Certaines nobles, d’autres beaucoup moins.\\


La première qui vient à l’esprit est la thérapie génique, c’est-à-dire la correction de maladies génétiques simples, dues au défaut d’un seul gène. Elle a déjà été entreprise avec succès chez la souris atteinte de maladies génétiques comme la tyrosinémie entraînant une atteinte du foie et des reins, ou la myopathie de Duchenne. Pour les maladies multigéniques ou multifactorielles comme le cancer, cela semble plus compliqué. \\

\textbf{Pas encore d’applications médicales probantes}

En Chine, en avril 2015, l’équipe de Junjiu Huang de l’Université Sun Yat-Sen de Canton, est même allée plus loin. Elle a tenté, à titre expérimental, de traiter des embryons humains porteurs d’un gène anormal entraînant une maladie du sang potentiellement mortelle, la bêta-thalassémie, avant de les détruire.\\

Les biologistes chinois ont eux-mêmes indiqué avoir « eu de grandes difficultés » (apparition de mutations hors cible) et affirmé que leurs travaux « montraient la nécessité urgente d’améliorer cette technique pour des applications médicales ». En clair, que cette nouvelle technique n’était pas encore maîtrisée pour une telle application.\\

Autorisée en Chine, cette recherche sur les embryons humains avait été validée par le Comité national d’éthique médicale de Chine. Toutefois ce travail, que certains considèrent comme eugénique, a fait l’objet d’un vif débat à l’étranger.\\

\textbf{Stériliser les moustiques contre le paludisme}

Une autre application de cette technique est d’inactiver un ou plusieurs gènes. C’est ce qu’a réalisé Eric Marois, biologiste moléculaire à l’Université de Strasbourg, avec une équipe de l’Imperial College de Londres. Son objectif ? Stériliser les moustiques vecteurs du parasite du paludisme.\\

« Après avoir identifié trois gènes nécessaires à la fertilité des moustiques, nous les avons inactivés par cette technique, explique Eric Marois. Dans plus de 90\% des cas, ces insectes ont transmis leur modification à leur progéniture, ce qui a entraîné une baisse de la population d’insectes. »\\

Prometteuse en santé publique, cette technique pose un problème éthique du point de vue de l’environnement. Elle pourrait également modifier les tiques vectrices de la maladie de Lyme (borréliose).\\

\textbf{Entre le bioterrorisme et l’homme augmenté}

Autre application plus inattendue : la reconstitution d’animaux disparus. C’est en effet le projet de George Church, professeur de génétique à Harvard, qui souhaite faire renaître un mammouth laineux, disparu il y a environ 4 000 ans dans les sols gelés de la toundra sibérienne, en récupérant son génome et le réinsérant, au moyen de la technique Crispr-Cas9, au sein du génome d’un éléphant d’Asie. « Un objectif qui pourrait être atteint d’ici sept à dix ans » selon lui.\\

La technique Crispr, si elle se vulgarise, peut aussi être détournée à des fins criminelles. Ce serait le cas, par exemple, en transformant génétiquement des bactéries ou virus neutres en micro-organismes pathogènes capables de secréter des toxines mortelles. Une éventualité à laquelle pensent sans doute déjà les services de biosécurité des pays occidentaux.\\

Enfin, « l’édition du génome » pourrait aussi, en théorie, être appliquée directement sur les cellules reproductrices. Ce qui permettrait de transmettre de nouveaux caractères génétiques (« bébé sur mesure ») à la descendance et modifierait l’évolution de l’espèce humaine. Une manière d’aller vers « l’homme augmenté » cher aux idéologies transhumanistes par exemple.\\

\newpage
\subsection{Bientôt des "bébés à la carte" ?}
\begin{itemize}
	\item \textbf{Lien : }  \url{http://www.sciencesetavenir.fr/sante/bientot-des-bebes-a-la-carte_26012} 
	\item \textbf{Auteur : } Par L'Obs avec AFP
	\item \textbf{Date : } 03 octobre 2013
	\item \textbf{Description : } Taille, sexe, couleur des yeux : une méthode proposant aux parents de sélectionner des traits spécifiques pour leurs enfants a été brevetée aux Etats-Unis.
	\item \textbf{Source : } Sciences et Avenir est un magazine mensuel français de vulgarisation scientifique.
\end{itemize}

Une méthode proposant aux parents de choisir certains traits spécifiques chez leurs enfants à naître a été brevetée en septembre aux Etats-Unis, suscitant une réaction irritée de plusieurs chercheurs européens. "Il est clair que sélectionner des enfants de la manière préconisée par la méthode brevetée par la société 23andMe est hautement discutable sur le plan éthique", écrivent les quatre auteurs d'un commentaire publié jeudi 3 octobre par la revue "Genetics in Medicine".\\

Après cinq ans d'attente, la société 23andMe a réussi à faire breveter une méthode de sélection des gamètes des donneurs basée sur des calculs génétiques réalisés par ordinateur. La méthode décrite dans la demande de brevet prévoyait de sélectionner les donneurs d'ovules ou de spermes de manière à améliorer les chances de produire un bébé ressemblant aux caractéristiques souhaitées par le couple bénéficiaire.\\

Parmi celles-ci, peuvent figurer la taille, le sexe, la couleur des yeux, certains traits de personnalité ou encore les risques de développer une dégénérescence maculaire liée à l'age (DMLA) et certains types de cancers. La société 23andMe reconnaissait toutefois que la méthode n'était pas infaillible et qu'il s'agissait seulement de faire en sorte que le bébé aient des chances "accrues" d'avoir les traits souhaités dans le cadre d'une procréation médicalement assistée (PMA).\\


"L'utilisation du diagnostic préimplantatoire pour éviter l'implantation d'embryons porteurs de graves anomalies génétiques est en passe de devenir une pratique courante, mais l'utilisation d'un programme informatique pour sélectionner les donneurs de gamètes afin d'aboutir à un bébé ayant les traits souhaités par ses parents semble avoir des implications beaucoup plus vastes, car ce processus implique la sélection de traits qui n'ont aucun lien avec une maladie" écrivent les chercheurs, dirigés par Sigrid Sterckx, de l'Institut de bioéthique de Gand, en Belgique.\\

Ils rappellent que la société avait déjà suscité une polémique en 2012, lorsqu'elle avait réussi à breveter un test évaluant la propension à développer la maladie de Parkinson.\\

Réagissant à l'attribution de son dernier brevet, la société 23andMe souligne sur son site internet que la situation "a évolué" depuis sa demande sur la question de la procréation médicalement assistée, mais qu'elle "respectera son engagement à faciliter l'accès des gens à leurs données génétiques". \\

\newpage
\section{Corpus de ressources}
\subsection{Quelle éthique pour les ciseaux génétiques ?}
 
\begin{itemize}
	\item \textbf{Lien : }  \url{https://lejournal.cnrs.fr/articles/quelle-ethique-pour-les-ciseaux-genetiques} 
	\item \textbf{Auteur : } Léa Galanopoulo
	\item \textbf{Date : } 20 juin 2016
	\item \textbf{Description : } En permettant de modifier l’ADN avec une facilité déconcertante, les outils d’ingénierie génomique comme CRISPR-Cas9 ouvrent la voie à des questionnements éthiques et législatifs complexes.
	\item \textbf{Source : } CNRS Le journal, l'objectif de ce site est de partager largement avec les amateurs de science, les professeurs et leurs élèves, les étudiants et tous les citoyens curieux, des contenus destinés jusque-là à la communauté des agents du CNRS, chercheurs, ingénieurs et techniciens, ceux des labos comme ceux des bureaux.
\end{itemize}

Peut-on toucher au génome humain ? La question ne cesse d’être posée depuis les années 1990 avec l’arrivée des thérapies géniques. Pourtant, elle est en passe de prendre aujourd’hui une dimension totalement nouvelle, avec l’émergence, en 2013, de l’édition génomique. Les outils TALEN, ZFN et bien sûr CRISPR-Cas9 permettent de couper, à la base près, des séquences d’ADN choisies. Grâce à cette chirurgie du génome, il ­devient possible d’inactiver ou de supprimer n’importe quel gène, voire de le remplacer par un autre. \\

Parmi ces trois outils d’ingénierie génomique, CRISPR est roi. Alors que les TALENs et les ZFNs font intervenir une protéine reconnaissant l’ADN à couper, CRISPR, lui, identifie directement l’ADN, grâce à un ARN guide. À titre d’exemple, il faut plusieurs mois pour développer une ZFN spécifique à un gène, contre quelques heures pour un CRISPR… « C’est beaucoup plus simple ! Il suffit de synthétiser un ARN complémentaire à l’ADN que l’on souhaite cibler, c’est facile à obtenir et à utiliser », précise Carine Giovannangeli, directrice de recherche en biologie moléculaire au laboratoire Structure et instabilité des génomes. \\

\textbf{Une technique au succès fulgurant}

Plus précis, plus rapides et surtout moins chers que les thérapies géniques classiques, les CRISPR se sont répandus comme une traînée de poudre dans les laboratoires du monde entier. « Désormais, les chercheurs utilisent presque exclusivement le système CRISPR-Cas9 », précise Carine Giovannangeli, qui développe des outils d’ingénierie génomique au sein de l’équipe TACGENE. En à peine deux ans d’existence, les CRISPR comptabilisent déjà un millier d’études publiées. Et des millions d’euros investis dans des start-up ou des projets de recherche. « La technique va plus vite que la recherche fondamentale », ajoute Patrick Gaudray, directeur de recherche en génomique au sein de l’unité Génétique, immunothérapie, chimie et cancer et membre, de 2008 à 2016, du Comité consultatif national d’éthique. \\

En avril 2015, la question cesse d’être théorique avec la publication d’une étude chinoise expérimentale. Les chercheurs ont utilisé ces ciseaux moléculaires pour modifier l’ADN d’embryons humains atteints d’une maladie monogénique : la bêta-thalassémie. Dès lors, impossible d’ignorer les interrogations éthiques soulevées par l’ingénierie du génome. A-t-on le droit de modifier le génome d’embryons humains et ainsi celui de générations futures ? Comment contrôler ces nouveaux organismes génétiquement modifiés par cette technique ? Autant de questions auxquelles la société civile et le monde de la recherche vont devoir répondre rapidement. « Il n’y a pas que l’aspect bénéfice-risque qui compte, il faut aussi voir ce qui est compatible avec les valeurs sociales sur lesquelles est fondée notre société », indique Anne Cambon-Thomsen, médecin, directrice de recherche émérite et experte auprès du comité d’éthique du CNRS. Elle préside également la Société française de génétique humaine, société qui a mis en place un groupe de réflexion interdisciplinaire sur ces questions. \\

Fondamentalement, l’ingénierie génomique ne pose aucune question nouvelle par rapport la thérapie génique, qui consiste à introduire un gène « médicament » dans un organisme. Et pourtant : « CRISPR permet un véritable saut technologique, un changement d’échelle. Parce qu’il est simple et peu cher, il ouvre à tous des capacités encore inexplorées », ajoute Anne Cambon-Thomsen, qui analyse les enjeux sociétaux des biotechnologies appliquées à la santé. « CRISPR fait un peu caisse de résonance. Mais, en fin de compte, il n’existe pas de rupture quant aux possibilités d’applications thérapeutiques », précise Patrick Gaudray. \\

\textbf{Crispations autour de la modification d’embryons}

Les inquiétudes médiatiques et sociétales autour des CRISPR se cristallisent principalement sur la modification d’embryons humains et la transmission à la descendance, « car c’est irréversible », souligne Anne Cambon-Thomsen. En mai dernier, un sondage commandé par l’association conservatrice et anti-avortement Alliance Vita présentait 76\% des Français comme défavorables à cette modification génétique. « Mais on n’a pas attendu CRISPR pour se poser ces questions », informe Patrick Gaudray. Et effet, en la matière, les règles sont claires depuis une vingtaine d’années. En 1997, la France signe, avec 28 autres pays, la convention d’Oviedo, ratifiée en 2011. Celle-ci interdit, entre autres, de pratiquer des modifications génétiques transmissibles à la descendance. Actuellement, les thérapies géniques, tout comme les CRISPR, ne peuvent être utilisées que pour une application thérapeutique sur les cellules somatiques, du foie ou des muscles par exemple. \\


« On nage dans les ambiguïtés », souligne Patrick Gaudray. Alors que les chercheurs ont la possibilité d’uti­liser des CRISPR pour provoquer des mutations dans les cellules germinales comme les spermatozoïdes ou les ovules, « ils n’ont pas le droit de fabriquer un embryon pour vérifier ce qu’ils ont fait », ajoute Anne Cambon-Thomsen. La recherche utilise en général des embryons issus de projets parentaux abandonnés. « En éthique, il faut savoir s’adapter. Il pourrait être envisageable de reposer la question de la modification génétique des embryons dans un but thérapeutique, ou au moins d’envisager quelques ­exceptions », suggère le médecin. Une chose est sûre : les applications thérapeutiques des CRISPR soulèvent d’immenses espoirs. « Les patients poussent pour qu’on les utilise, car ils sont motivés par le bénéfice possible sur leur maladie », indique Anne Cambon-Thomsen. \\

\textbf{Un procédé sûr et efficace ?}

L’édition génomique est en passe de bouleverser le monde de la santé, et de nombreuses applications thérapeutiques sont en cours de développement. En particulier contre les maladies dont les gènes sont bien identifiés, comme la myopathie de Duchenne. Cette dystrophie musculaire est en effet liée à une anomalie du gène DMD. En décembre dernier, plusieurs équipes de recherche ont démontré, dans Science, qu’il était possible, chez la souris, d’introduire un CRISPR dans le muscle pour venir modifier le gène déclenchant de la maladie. Résultats : les souris retrouvaient de la force musculaire ! VIH, leucémie ou encore mucoviscidose, les CRISPR sont expérimentés sur de nombreuses autres maladies. \\

Il reste cependant une poignée d’obstacles techniques à franchir avant d’envisager la commercialisation de traitements chez l’homme. Les CRISPR peuvent par exemple manquer leur cible et couper un autre gène que celui désiré. Cependant, ces effets dits « off-target » ­deviennent de mieux en mieux prévisibles informatiquement. « Des outils bio-informatiques sont disponibles aujourd’hui pour identifier les séquences qui ressemblent beaucoup à la cible et pour prédire, limiter, voire éviter, ces cassures indésirables. De plus, des améliorations du système CRISPR-Cas9 ont permis récemment de diminuer fortement ces effets hors cible », explique Carine Giovannangeli. Si, en recherche fondamentale, les effets off-target ne présentent finalement que peu de risque, en clinique, les enjeux sont tout autres… Que faire si, au lieu de supprimer un gène « malade », le traitement élimine un gène indispensable au fonctionnement du système immunitaire par exemple ? « La sécurité doit être assurée, mais ce n’est sans doute qu’une question de temps avant d’y arriver », ajoute Patrick Gaudray. Depuis 2013, des protéines Cas9 de plus en plus performantes et spécifiques sont régulièrement mises sur le marché. \\

« Une autre question à régler, en particulier dans le cadre d’applications thérapeutiques, est celle de la délivrance du système CRISPR-Cas9. En général, on va cibler des organes que l’on sait atteindre, comme les yeux, le foie ou les muscles », précise Carine Giovannangeli. Il pourrait également être possible d’administrer ce bistouri génétique à l’aide d’un virus, ou directement en sortant les cellules de l’organisme pour les réimplanter une fois modifiées… Une utilisation sûre et efficace des CRISPR chez l’homme reste donc encore à développer. Par ailleurs, « il ne faut pas exagérer les conséquences que peuvent avoir les gènes sur certaines pathologies multifactorielles. Ce serait une dérive scientifiquement non justifiée qui pourrait être utilisée par des compagnies privées notamment pour faire du profit, sans résultats pertinents », met en garde Anne Cambon-Thomsen. \\

\textbf{Le génome, un patrimoine sacré ?}

En définitive, derrière la question de l’embryon humain se cache celle de la valeur qu’accorde notre société au génome. « Est-ce qu’on considère que le gène est une valeur en soi, sacrée ?, s’interroge Anne Cambon-Thomsen. Dans ce cas, on n’y touche pas. » \\

Sur ce point, deux camps s’opposent. D’un côté, pour une majorité, le génome doit être considéré comme le fruit d’un héritage commun à préserver. Ce patrimoine, résultat d’années d’évolution humaine, ne doit donc en aucun cas être corrompu par l’intervention d’outils biotechnologiques dont nous contrôlons encore mal la portée. D’un autre côté, les partisans de l’homme augmenté considèrent que la nature a parfois besoin d’un coup de pouce pour supprimer ou corriger ses imperfections. \\

Cette approche transhumaniste recèle cependant des dérives potentiellement graves. « Le risque serait ensuite de vouloir modifier des gènes qui codent pour un trait n’ayant rien à voir avec une maladie, comme la couleur des yeux », avertit Anne Cambon-Thomsen. Les CRISPR deviendraient alors des outils privilégiés d’une forme d’eugénisme, allant à l’encontre des principes fondamentaux de justice et d’égalité entre les hommes. \\

\textbf{Des outils capables de détruire une espèce}

Comme toute découverte révolutionnaire, les CRISPR brassent derrière eux leur lot de polémiques, à commencer par la question de la propriété intellectuelle. D’un côté, les chercheuses Jennifer Doudna et Emmanuelle Charpentier revendiquent depuis 2012 la paternité de la découverte du mécanisme spécifique de CRISPR-Cas9. De l’autre, une seconde demande de brevet, déposée par Feng Zhang, a été déposée en 2013, briguant l’application des CRISPR sur les cellules mammifères. « Mais, derrière ces figures médiatiques, il ne faut pas oublier toute la démarche scientifique en amont impliquant des chercheurs de tous horizons », rappelle Patrick Gaudray. Depuis 2013, une dizaine d’entreprises spécialisées dans les CRISPR ont été créées. \\

Selon Patrick Gaudray, la question cruciale autour des CRISPR reste celle des applications non humaines. « Nous avons désormais des outils à portée de main qui permettraient de faire disparaître une espèce », souligne-t-il, faisant référence à une expérience de 2015 qui avait rendu stérile le moustique tigre, à l’aide des CRISPR, pour limiter la transmission de la dengue. « L’éradication du moustique est une réalité totalement envisageable aujourd’hui ! Les approches techno-centrées sont-elles justifiées, sachant qu’on affecte de manière irréversible la biodiversité ? », s’interroge le chercheur. Cette vision utilitariste est, en tout cas, déjà de mise aux États-Unis, où la Food and Drug Administration a autorisé la commercialisation de champignons de Paris modifiés, via des CRISPR, pour ne jamais noircir. A-t-on réellement besoin de modifier durablement ces organismes vivants dans un tel but ? \\

\textbf{Quand l’éthique se confronte au marché} 

OGM ? Thérapie innovante ? La définition des objets issus des CRISPR n’est pas encore établie. Seule certitude : « Les grandes entreprises vont breveter à tout-va le moindre gène modifié, sans que cela ne leur coûte beaucoup d’argent », indique Patrick Gaudray. Les CRISPR amplifient ainsi le débat sur la brevetabilité du vivant. « Est-ce qu’un agriculteur devra payer une firme s’il voit la mutation apparaître spontanément dans son champ ? », ironise Patrick Gaudray. Si le débat éthique est indispensable, il pâtit de pressions industrielles conséquentes. « La pression induite par la peur de perdre des parts de marché ne doit pas être un obstacle à l’évaluation éthique des applications de la méthode », écrit ainsi l’Inserm dans une saisine sur la question. \\

« Nous sommes face à un modèle économique et social dans lequel les décisions sont prises en majorité par les lobbies qui ne représentent pas la société dans son ensemble », déplore Patrick Gaudray, encourageant à « un vrai débat de société intelligent autour de l’ingénierie du génome, qui intègre scientifiques, société civile, citoyens, mais sans oppositions de principe ». Des groupes de travail dirigés par deux sociétés savantes ont d’ailleurs été formés dans ce but. Carine Giovannangeli évoque, quant à elle, la mise en place d’« un cahier des charges et de standards à adapter en fonction des applications ». Concernant les CRISPR, la décision finale sera, dans tous les cas, tranchée au niveau législatif. \\



\newpage
\subsection{CRISPR-Cas9: des ciseaux génétiques pour le cerveau}
 
\begin{itemize}
	\item \textbf{Lien : }  \url{https://lejournal.cnrs.fr/articles/crispr-cas9-des-ciseaux-genetiques-pour-le-cerveau} 
	\item \textbf{Auteur : } Léa Galanopoulo
	\item \textbf{Date : } 3 mai 2016
	\item \textbf{Description : } En permettant d’intervenir sur l’ADN de manière chirurgicale, les ciseaux génétiques CRISPR-Cas9 ne font pas que révolutionner les techniques d’édition du génome, ils ouvrent aussi des opportunités pour l’étude du cerveau.
	\item \textbf{Source : } CNRS Le journal, l'objectif de ce site est de partager largement avec les amateurs de science, les professeurs et leurs élèves, les étudiants et tous les citoyens curieux, des contenus destinés jusque-là à la communauté des agents du CNRS, chercheurs, ingénieurs et techniciens, ceux des labos comme ceux des bureaux.
\end{itemize}

En permettant d’intervenir sur l’ADN de manière chirurgicale, les ciseaux génétiques CRISPR-Cas9 ne font pas que révolutionner les techniques d’édition du génome, ils ouvrent aussi des opportunités enthousiasmantes pour l’étude du cerveau.
CRISPR-Cas9… Derrière cet acronyme barbare se cache une innovation révolutionnaire : remplacer un gène par un autre ou le modifier. La méthode paraît presque trop simple, et pourtant elle est le fruit de près de trente ans de recherche. CRISPR-Cas9 (prononcez « crispère ») fonctionne comme des ciseaux génétiques : il cible une zone spécifique de l’ADN, la coupe et y insère la séquence que l’on souhaite. \\

CRISPR-Cas9 est un complexe formé de deux éléments : d’un côté, un brin d’ARN, de séquence homologue à celle de l’ADN que l’on veut exciser, et de l’autre, une endonucléase, le Cas9. Dans la cellule, le brin d’ARN va reconnaître la séquence homologue sur l’ADN et s’y placer. L’enzyme Cas9 se charge alors de couper la chaîne ADN complémentaire à ce brin ARN. Le trou laissé par le passage du CRISPR-Cas9 pourra alors être comblé par n’importe quel nouveau fragment d’ADN.  \\


« C’est une technique révolutionnaire, certainement l’innovation majeure du XXIe siècle en biotechnologie ! », s’enthousiasme Jean-Stéphane Joly, directeur de recherche à l’Institut des neurosciences Paris-Saclay. Si CRISPR-Cas9 met le monde scientifique en émoi, c’est qu’il étend les possibilités de retouche génétique à l’infini : supprimer un gène malade, le remplacer par une séquence saine ou encore étudier la fonction précise d’un brin d’ADN, à la molécule près… Aucun secteur de la biologie n’y échappe, et de nouvelles applications sont publiées quotidiennement.  \\

\textbf{De la recherche fondamentale à la biotechnologie}

Les origines de la découverte du CRISPR-Cas9 remontent à 1987, lorsque des chercheurs japonais découvrent chez la bactérie Escherichia coli des séquences d’ADN dont l’enchaînement des bases (A, C, T, G) se lit de la même manière dans les deux sens : à l’instar des mots « radar » ou « kayak », on parle de palindromes… Le rôle de ces fragments, baptisés CRISPR, pour « courtes répétitions en palindrome regroupées et régulièrement espacées », ne sera finalement mis en lumière que vingt ans plus tard : d’une part quand on constatera que les morceaux d’ADN intercalés entre les palindromes sont souvent des séquences d’ADN de virus, d’autre part quand on montrera que les bactéries porteuses de ces séquences résistent mieux aux infections. L’ARN d’un complexe CRISPR-Cas9 lui permet ainsi de reconnaître et se lier à l’ADN viral présent dans la bactérie, pour ensuite le détruire.  Dès 2012, les chercheuses Emmanuelle Charpentier et Jennifer Doudna vont s’inspirer de cette réaction immunitaire bactérienne et la détourner pour en faire un véritable outil biotechnologique à la portée de tous.  \\

\textbf{Supprimer et remplacer l’ADN pour comprendre la logique du cerveau}

Cela suscite de nombreux espoirs, et notamment celui d’élucider le fonctionnement du cerveau. L’utilisation des CRISPR démultiplie en effet les possibilités de recherche fondamentale en neurosciences : en coupant un gène précis sur un modèle animal, on peut déterminer plus précisément son rôle, dans le développement du cerveau par exemple. De plus, elle ouvre la voie à de nombreuses applications thérapeutiques. Par exemple, si un gène est incriminé dans une maladie mentale, il devient envisageable, à terme, de l’éliminer, le corriger ou le remplacer avec notre bistouri génétique.  \\

On ne s’étonnera donc pas qu’en neurosciences plusieurs recherches utilisent déjà le CRISPR. C’est notamment le cas des travaux de Jean-Stéphane Joly. « Grâce à l’édition génomique, nous caractérisons avec précision des mutations responsables de la microcéphalie chez le poisson zèbre », explique-t-il. Le chercheur coordonne également le Réseau d’études fonctionnelles chez les organismes modèles (Efor), qui s’appuie entre autres sur l’édition du génome pour construire la carte génétique et phénotypique de nombreux modèles animaux et végétaux.  \\

\textbf{Dresser la carte d’identité génétique des maladies cérébrales}

Tumeurs cérébrales, étude du développement des neurones, autisme… En coupant de simples fragments d’ADN, il devient possible d’établir le profil génétique d’un nombre immense de maladies mentales. D’autant plus qu’elles sont souvent multigéniques. Pour l’autisme, par exemple, plus de 300 variations génétiques ont déjà été identifiées. Seulement, elles ne s’expriment pas toutes de la même façon au niveau des différents neurones. Avec CRISPR-Cas9, il devient possible d'étudier localement ces expressions génétiques. D’ailleurs, les gènes codants pour une protéine ne sont pas les seules parties de l’ADN impliquées dans le développement de troubles. Entre ces parties, il existe des séquences d’ADN régulatrices, que l’on a considérées pendant longtemps comme des déchets génétiques. Or ces parties non codantes jouent en réalité un rôle essentiel dans de nombreuses pathologies. À l’avenir, CRISPR-Cas9 pourrait servir à mieux caractériser le rôle encore largement méconnu de ces fragments.  \\

\textbf{Une technique simple, rapide et bon marché}

Si CRISPR-Cas9 constitue une avancée révolutionnaire, d’autres techniques d’édition du génome existaient avant elle. Frédéric Causeret, chercheur en neurosciences, les utilisait auparavant sur ses souris, pour étudier notamment leur développement cérébral . Pourtant, il envisage désormais d’avoir recours à CRISPR-Cas9 dans les mois qui viennent. « Pour modifier le génome de la souris, nous disposons déjà de la recombinaison homologue. Mais le CRISPR est nettement plus rapide ! », précise-t-il.  \\

Alors qu’avec l’ancienne technique il faut près d’un an pour obtenir une souris avec la mutation souhaitée, le CRISPR permet d’obtenir ces modifications en deux mois à peine. Et à moindre coût ! La manipulation revient au total à quelques milliers d’euros, alors que certaines souris mutantes valent parfois jusqu’à 50 000 euros. « La technique est très simple à mettre en œuvre, elle nous épargne certaines étapes fastidieuses », explique Frédéric Causeret.  \\

Le CRISPR-Cas9 suscite les rêves les plus fous notamment parce qu’il permet de supprimer et de remplacer plusieurs gènes en même temps, contrairement aux autres techniques d’édition génétique. « On va pouvoir générer des modèles de pathologies mentales multigéniques de manière simple et rapide », prédit Frédéric Causeret. Certaines recherches utilisant CRISPR-Cas9 sont mêmes parvenues à moduler l’expression d’un gène donné, sans avoir à l’éditer lui-même.  \\

\textbf{L’édition du génome pour tous !}

« Le CRISPR ne va pas changer nos routines, mais nous ouvre des perspectives que l’on n’envisageait même pas car c’était beaucoup trop long », indique Frédéric Causeret. La technique permet en effet de générer des pertes de fonction génétique extrêmement rapides, sur de nombreuses lignées, ou encore de tracer n’importe quelle protéine dans le cerveau. Pour Jean-Stéphane Joly, cette avancée va révolutionner la biologie ; à l’instar de la méthode d’amplification génique PCR, qui a révolutionné la biologie moléculaire et l’analyse génomique il y a une trentaine d’années. « Le CRISPR-Cas9, c’est l’iPhone de la biotechnologie », sourit-il, précisant que l’on verra certainement apparaître dans les années à suivre des centaines de nouveaux brevets.  \\


Alors qu’il y a quelques années, l’édition génomique n’était disponible que pour les souris, le CRISPR-Cas9 permet désormais d’éditer le génome de tous les êtres vivants : poisson zèbre, primate, insectes sociaux…. Si les recherches se multiplient, leurs résultats ne sont pour l’instant pas extrapolables à l’homme.  \\

\textbf{Quels freins à l’application humaine ?}

Envisager une application thérapeutique avec le CRISPR-Cas9 pose en premier lieu la question de l’administration. Si, in vitro, il suffit d’injecter le mélange dans le noyau, in vivo, la tâche reste ardue. En particulier, comment le délivrer jusqu’au cerveau ? « Chez la souris, nous transférons in utero le matériel génétique dans l’embryon par la méthode dite d’électroporation », explique Frédéric Causeret.  \\

Pour certaines maladies neurodégénératives, on pourrait envisager une édition du génome avant l’apparition des premiers symptômes. Mais cette méthode semble difficilement applicable à l’homme. La voie d’entrée des CRISPR pourrait donc se faire grâce à un vecteur viral, à l’instar des thérapies géniques classiques. Enfin, l’inoculation du matériel génétique pourrait aussi avoir lieu après le stade embryonnaire. « Dans le cadre de certaines maladies neurodégénératives par exemple, on pourrait envisager une édition du génome chez l’adulte avant l’apparition des premiers symptômes », imagine Frédéric Causeret.  \\

Néanmoins, pour ces applications humaines, les problèmes éthiques et de sécurité prédominent. Les risques que les CRISPR manquent leur cible sont faibles, mais existent. « Il suffirait que le CRISPR-Cas9 modifie une séquence qui ressemble beaucoup à celle que l’on cible pour déclencher une catastrophe. Comme si, par erreur, il découpait un gène suppresseur de tumeur par exemple », précise Frédéric Causeret.  \\

Agronomie, écologie, neurologie… Les applications des CRISPR semblent sans limites. « Certains pensent par exemple utiliser l’édition génomique pour rendre le moustique stérile et ainsi éradiquer le paludisme. Ou même à empêcher les pommes de terre de noircir ! », explique Jean-Stéphane Joly, ajoutant qu’il faudra strictement contrôler ces modifications génétiques et leur effet sur l’environnement. Une chose est sûre, le CRISPR-Cas9 est loin d’avoir dévoilé toute l’étendue de ses applications. « À l’évidence, les découvertes sur la réécriture du génome méritent le prix Nobel ! », assure même Jean-Stéphane Joly.

\newpage
\subsection{Les 9 questions clés de la médecine prédictive }
\begin{itemize}
	\item \textbf{Lien : }  \url{https://lejournal.cnrs.fr/articles/les-9-questions-cles-de-la-medecine-predictive-partie-1} et \url{https://lejournal.cnrs.fr/articles/les-9-questions-cles-de-la-medecine-predictive-partie-2} 
	\item \textbf{Auteur : } Charline Zeitoun
	\item \textbf{Date : } 15 juillet 2013
	\item \textbf{Description : } Que nous apprennent les tests génétiques ? Peut-on personnaliser les traitements ? Qui peut faire un test génétique ? 
	\item \textbf{Source : } CNRS Le journal, l'objectif de ce site est de partager largement avec les amateurs de science, les professeurs et leurs élèves, les étudiants et tous les citoyens curieux, des contenus destinés jusque-là à la communauté des agents du CNRS, chercheurs, ingénieurs et techniciens, ceux des labos comme ceux des bureaux.
\end{itemize}

\textbf{Que nous apprennent les tests génétiques ?}

L’analyse de nos gènes (dans le sang, la salive, etc.) permet le diagnostic de 1 000 maladies génétiques (sur les 5 000 répertoriées) et l’estimation de la susceptibilité à des centaines d’autres influencées par la génétique. Ces tests indiquent des facteurs de risque sous forme de pourcentages. C’est un calcul de probabilité, celle de développer une maladie si nous avons telle variation d’un gène, par comparaison à la population qui ne l’a pas. « Le calcul, réalisé à partir de statistiques d’épidémiologie, tient aussi compte des variabilités selon les groupes géographiques, l’âge, etc. », explique Anne Cambon-Thomsen, généticienne au CNRS. Les chiffres peuvent donc varier pour une même pathologie. Par exemple, selon l’Institut national du cancer, une mutation du gène BRCA1 implique de 51 à 75\% de risque de développer un cancer du sein héréditaire avant 70 ans, contre environ 10\% dans la population générale (tous cancers du sein confondus, héréditaires ou non). Et le risque d’être atteint de dégénérescence maculaire liée à l’âge (DMLA) est 2,5 fois plus élevé quand on est porteur d’une variation du gène CFH. Ce type de risque devient une certitude pour certaines maladies génétiques comme la maladie de Huntington, dégénérescence du système nerveux central inéluctable vers l’âge de 40 ans si on a le gène HD muté. Mais la plupart des maladies sont multifactorielles. Les gènes, lorsqu’ils sont impliqués, n’entraînent alors que des risques faibles qui se mêlent à de nombreux autres facteurs comme l’alimentation, la rencontre avec un virus, une bactérie, etc. « Ainsi, posséder le variant génétique qui rend moins résistant à la lèpre ne vous fera en réalité pas courir plus de risques que votre voisin de pallier si vous vous trouvez dans une région préservée de l’agent infectieux de cette maladie… », commente Anne Cambon-Thomsen. Autre exemple : le diabète de type 2 est autant lié à des gènes qu’à la consommation de sucre et à un mode de vie trop sédentaire. \\

\textbf{Prédire permettra-t-il de guérir ?}

Ce n’est pas si simple. En ce qui concerne les maladies génétiques comme Huntington, certains cancers héréditaires ou la mucoviscidose, la thérapie génique pourrait permettre, en théorie, de guérir avant même d’être malade. « Mais cette technique n’a encore donné que peu de résultats probants », rappelle Hervé Chneiweiss, directeur de recherche au CNRS, directeur de l’unité Plasticité gliale et président du comité d’éthique de l’Inserm. En revanche, la chirurgie préventive permet bel et bien de supprimer un risque s’il concerne un organe précis et amputable, comme le sein ou les ovaires. Bien entendu, de telles opérations confrontent médecins et patients à des décisions extrêmement difficiles, dans la mesure où une part non négligeable des individus peut ne jamais développer la maladie. Autre approche : prescrire une surveillance accrue en cas de forte prédisposition. « À condition que le dépistage soit inoffensif. La question se pose par exemple pour les rayons X des mammographies : ceux-ci sont susceptibles de provoquer des lésions dans l’ADN et de favoriser justement les cancers du sein héréditaires que l’on guette », pointe Hervé Chneiweiss. Enfin, du côté des maladies multifactorielles, associées à de faibles risques, « la prédiction par les gènes n’a aujourd’hui que peu d’intérêt médical », résume Anne Cambon-Thomsen. Sauf peut-être à pousser les gens à faire du sport, à manger moins de sucre et à prendre d’autres bonnes résolutions en cas de prédisposition au diabète, à des maladies cardiovasculaires, etc. « Reste à prouver que l’information génétique influence davantage les gens que les campagnes de prévention destinées à l’ensemble de la population, ce que les études de comportement actuelles peinent à démontrer », commente la chercheuse.  \\

\textbf{Peut-on personnaliser les traitements en fonction de l’ADN des patients ?}

Oui, car on peut aujourd’hui prédire l’efficacité de certains traitements en fonction de l’ADN du patient. « Cette branche de la médecine, la pharmacogénétique, est en plein essor », informe Anne Cambon-Thomsen. L’intérêt est inestimable : éviter d’administrer des médicaments inefficaces et supprimer des effets secondaires désagréables, voire mortels. De nombreux exemples affichent déjà des résultats tangibles. L’Agence européenne du médicament recommande ainsi, depuis 2008, un test génétique avant la prise d’Abacavir (un anti-VIH) afin d’identifier les patients porteurs d’un variant d’un gène HLA. Selon les études épidémiologiques, la moitié d’entre eux tolèrent en effet très mal ce médicament, voire en meurent. « Mais c’est probablement en cancérologie que la génomique suscite aujourd’hui le plus d’applications », poursuit Anne Cambon-Thomsen. Celle-ci permet en effet d’améliorer le diagnostic et de décider de la chimiothérapie la plus adéquate en fonction des altérations génétiques identifiées dans la tumeur, notamment pour les cancers du sein.  \\

\textbf{Quels sont les grands progrès attendus de la génomique ?}

En plus de faciliter le choix des traitements, la génomique sert aussi à en mettre au point de nouveaux. « Cette approche permet en effet d’identifier de nouvelles cibles liées à des mutations génétiques sur lesquelles pourront agir de nouvelles molécules médicamenteuses à développer », commente Anne Cambon-Thomsen. Souvent, ces mutations ne concernent qu’une partie seulement des patients, d’où une médecine de plus en plus personnalisée. Une mutation du gène CFTR vient ainsi de permettre la mise au point du premier médicament contre la mucoviscidose, l’Ivacaftor, très efficace, mais chez 5\% des malades seulement. « Et, dans le domaine du cancer, des recherches visent maintenant à comparer l’ensemble du génome de la tumeur avec celui de tissu sain de l’individu. Le but serait à l’avenir de traiter cette pathologie selon le profil génétique de la tumeur et du patient », explique la chercheuse. Et pour l’avenir plus lointain ? « La médecine passera par la recherche des interactions entre les gènes, encore très peu étudiées, et des interactions des gènes avec l’environnement, pour lesquelles il faudra développer les cohortes avec des données sur les modes de vie des individus », conclut Anne Cambon-Thomsen.  \\

\textbf{Qui peut faire un test génétique ?}

En pratique, tout le monde. Il suffit en effet d’expédier sous pli un peu de salive à un laboratoire étranger accessible sur Internet. Le kit vendu par 23andMe, par exemple, permettait de connaître sa susceptibilité génétique à développer 162 maladies et à réagir à 20 médicaments, le tout pour une centaine de dollars. « Mais en France, obtenir ainsi un tel test est illégal », commente Sonia Desmoulin-Canselier, chercheuse à l’Unité mixte de recherche de droit comparé, à Paris. La loi relative à la bioéthique fixe un cadre très strict : un test ne doit être entrepris que dans un but judiciaire, pour la recherche scientifique ou dans un but médical. Il peut alors s’agir de confirmer le diagnostic d’un patient présentant les symptômes d’une maladie génétique, de rechercher une maladie héréditaire en l’absence de symptômes, de permettre un choix éclairé en matière de procréation, etc. Les praticiens qui font les analyses doivent avoir un agrément de l’Agence de la biomédecine. Les tests doivent être prescrits par un médecin qui doit recueillir le consentement signé du patient, ou bien de ses parents s’il s’agit d’une personne mineure. « Dans ce dernier cas, il faut impérativement que le patient ou sa famille tire un bénéfice médical du test, c’est-à-dire des mesures préventives ou curatives immédiates », souligne Simone Bateman, sociologue au Cermes3. L’accompagnement du patient par un conseiller en génétique et des psychologues lors du protocole est capital étant donné l’angoisse et les mauvaises interprétations que les résultats peuvent causer. Les tests Internet, outre une fiabilité parfois controversée, n’offrent souvent aucun suivi de ce type.  \\

\textbf{Fait-on des découvertes fortuites en génétique ?}

Oui, cela arrive. Dans le cadre médical, on ne cherche pourtant qu’un seul ou que quelques gènes précis, liés à l’objet de la consultation. « Mais, quand la cause génétique n’est pas clairement connue, ou lorsqu’une première exploration n’a pas abouti à un diagnostic, procéder au séquençage complet de l’ensemble des gènes pour poursuivre les recherches est devenu une pratique courante, sauf en anténatal », explique Anne Cambon-Thomsen. Problème : cela génère d’autres informations et risque d’occasionner des découvertes fortuites sur d’autres risques de maladies. « Or la question se pose aujourd’hui de l’opportunité d’opter d’emblée pour le séquençage complet, car son coût diminue et deviendra sous peu plus avantageux que celui d’une suite d’examens ciblés en cascade », indique la généticienne. Il pourrait alors y avoir de plus en plus de découvertes fortuites.  \\
 
 
\textbf{Gardera-t-on le droit de ne pas connaître le résultat des tests génétiques ?}

En France, on peut refuser de connaître le résultat de son test. « Dans tous les cas, si l’on découvre une anomalie génétique grave faisant aussi courir des risques aux membres de la famille, et si une prévention ou des soins sont possibles, alors les choses se compliquent », informe Sonia Desmoulin-Canselier. Parce que le patient peut refuser que quiconque soit averti. Le devoir d’assistance du médecin à la famille se heurte donc au secret médical… « Si le consentement du patient est obtenu, le médecin invite par courrier les autres membres de la famille à se rendre à une consultation de génétique, sans dévoiler d’informations, afin que les destinataires aient eux aussi le droit de ne pas savoir, poursuit la juriste, mais cette invitation est déjà une information pesante, me semble-t-il… » Toutes ces questions éthiques pourraient devenir fréquentes et cornéliennes si les tests génétiques portant sur tout le génome se généralisent. Que faire, en particulier, des découvertes fortuites ? « Il y a actuellement un débat international très animé, et les avis divergent, commente Anne Cambon-Thomsen. Par exemple, l’American College of Medical Genetics and Genomics a recommandé aux médecins de regarder systématiquement 57 gènes lors de tout test et d’informer le patient sans qu’il puisse refuser. » Une position qui ne fait pas l’unanimité aux États-Unis et tranche avec les avis du Comité consultatif national d’éthique (CCNE) qui, en France, s’en remet aux praticiens pour agir dans l’intérêt du patient, en proposant de l’informer tout en respectant son droit de ne pas savoir.  \\
 
 
\textbf{Comment préserver notre intimité génétique ?}

« En France, aux États-Unis et dans de nombreux pays, la loi sanctionne l’utilisation discriminatoire d’informations génétiques, notamment dans les domaines de l’emploi et des assurances », explique Sonia Desmoulin-Canselier. Mais, quand les tests sont autorisés à titre indicatif, comme c’est le cas pour les employeurs américains, « reste la difficulté de prouver que c’est bien l’information génétique qui est à l’origine de la mesure jugée discriminatoire », souligne la juriste. En Angleterre, le gouvernement a ouvert une brèche : il autorise les assureurs à tenir compte du résultat d’un test sur la maladie de Huntington pour contracter une assurance-vie excédant 500 000 livres sterling. Il est vrai que des informations aussi lourdes de conséquences, si elles deviennent largement disponibles et fiables, remettent forcément en cause le fonctionnement même des assurances, qui repose par définition sur l’incertitude. Quant à l’anonymat promis aux participants à la recherche scientifique, dont l’ADN est parfois recueilli par le biais de firmes privées sur Internet (le client donne son accord en cochant une case), il a récemment été mis à mal. En janvier 2013, Yaniv Erlich, généticien à Cambridge, aux États-Unis, a joué les hackers du génome et a réussi à trouver l’identité de 50 personnes qui avaient donné leur ADN anonymement pour un projet scientifique.  \\
 

\textbf{Trop de diagnostic va-t-il tuer le diagnostic ?}

« Si, à l’avenir, on se met à chercher tout et n’importe quoi chez des individus bien portants, on va créer de plus en plus de catégories de personnes à risque », fait remarquer Ilana Löwy. Et accentuer un travers de la médecine d’aujourd’hui, certes de plus en plus efficace, qui consiste à maintenir de plus en plus de personnes « suspendues entre bonne santé et suspicion de maladie, parfois pendant des années, alors qu’elles ne présentent aucun symptôme », poursuit la biologiste et historienne de la médecine. Outre le stress provoqué et d’éventuelles chirurgies inutiles, cela coûte cher à la collectivité et au patient. « Que se passera-t-il quand un banal test donnera 300 facteurs de risque : fera-t-on des examens complémentaires pour chacun d’eux ? », interroge la chercheuse. Dès lors, prévenir vaudra-t-il vraiment mieux que guérir ? Ou bien cela précipitera-t-il le système d’assurance-maladie vers la faillite complète ? \\

\subsection{Connaître son ADN : espoir ou menace ?}

\begin{itemize}
	\item \textbf{Lien : }  \url{http://lemonde.fr/sciences/article/2015/05/04/connaitre-son-adn-espoir-ou-menace_4627192_1650684.html} 
	\item \textbf{Auteur : } Florence Rosier
	\item \textbf{Date : } 04 mai 2015
	\item \textbf{Description : } Quels peuvent être l’intérêt potentiel et les risques – médicaux ou éthiques – des analyses du génome, commercialisées en ligne mais interdites en France ? Deux généticiens en débattent.
	\item \textbf{Source : } Le monde
\end{itemize}

Le 19 février, la Food and Drug Administration (FDA) autorisait un premier test génétique en vente libre. Il permet de dépister le risque de transmettre à sa descendance une maladie héréditaire rare, le syndrome de Bloom, qui entraîne un retard de croissance et un risque élevé de cancer. Avec cette autorisation, certains redoutent un premier pas vers la libéralisation des analyses des génomes personnels proposées sur Internet. D’autres espèrent peut-être une extension du territoire de la génétique, notamment au vu du marché considérable que représentent ces tests. \\

Et vous, seriez-vous prêts à tout connaître de votre ADN... et de ses risques ?

Depuis 2007, ces analyses du génome pour convenance personnelle - interdites en France  - rencontrent un succès croissant.En mars 2015, la société 23andMe annonçait que plus de 900 000 « clients » avaient fait appel à ses tests de génotypage en libre accès. A partir d’un simple prélèvement salivaire, cette start-up californienne, créée en 2006 et en partie financée par Google, propose aux particuliers, moyennant 99 dollars, une analyse des données de leur génome. \\

\textbf{Veto au volet « médical »}

Mais en novembre 2013, la FDA a mis son veto au volet « médical » de cette activité, interdisant à 23andMe de livrer des prédictions sur les prédispositions à des maladies ou les réponses à des médicaments. Avec deux conséquences immédiates : aux Etats-Unis, 23andMe s’est recentrée sur les analyses « récréatives », notamment généalogiques, des données du génome. Et elle a créé deux succursales, au Canada et au Royaume-Uni, qui proposent quant à elles une interprétation médicale des génomes. \\

Il était impossible d’imaginer la prodigieuse ampleur de ces progrès techniques : entre 2003 et 2013, le temps et le coût de la lecture complète du génome humain ont été divisés par 2 millions ! Ce fulgurant essor technique a conduit à remettre en cause certaines notions fondamentales de génétique, dont le concept même de gène. \\

Ces avancées rendent aussi ces analyses de génomes de plus en plus accessibles. Mais ces tests en libre accès sont-ils une menace ? Une promesse ? Quelle est leur fiabilité ? Peuvent-ils avoir une valeur médicale prédictive et préventive ? Ou bien la santé des gens n’est-elle qu’un prétexte ? Un garde-fou par un encadrement international de ces pratiques est-il encore envisageable  ? \\

\textbf{Notre nature même d’être humain est interrogée}

A ces questions plutôt techniques et scientifiques s’en ajoutent d’autres, philosophiques et éthiques. Car bien au-delà des interrogations et des doutes légitimes sur le bien-fondé et les bénéfices médicaux de ces analyses, c’est notre nature même d’être humain qui est interrogée. La génétique va-t-elle nier nos dimensions psychologiques, émotionnelles et sociales ? Les spectres de l’« enfant parfait » et des dérives eugéniques vont-ils ressurgir ? \\

Pour y voir plus clair, deux généticiens français de renom confrontent leurs points de vue. Le premier, Jean-Louis Mandel, professeur au Collège de France, défend une position iconoclaste voire provocatrice : il est l’un des seuls biologistes français à plaider pour la liberté individuelle de connaître les données de son propre génome, notamment pour éviter la transmission de maladies génétiques graves. Bravant l’interdit français, il a testé les apports et les limites de ces analyses en faisant faire, il y a quelques années, le décryptage de son propre génome. Face à lui, Patrick Gaudray, membre du Comité consultatif national d’éthique, dénonce le consumérisme lié à la commercialisation de ces tests, ainsi que les dangers de la sacralisation de l’information génétique. \\

\textbf{Lexique}

\textit{Test génétique} : Recherche, dans un ou plusieurs gènes, de mutations à l’origine d’une maladie bien identifiée. \\

\textit{Test de génotypage} : Caractérisation, chez une personne donnée, d’un très grand nombre de « variants génomiques ». Ce sont de courtes séquences d’ADN qui présentent de fréquentes variations d’un individu à un autre. Ces variations peuvent être corrélées à un certain nombre de traits, comme la prédisposition à des maladies ou des caractères morphologiques ou ethniques. \\

\textit{SNP (« polymorphismes d’un seul nucléotide »)} : Séquences du génome qui varient beaucoup d’un individu à l’autre, ces variations portant sur une seule lettre, ou nucléotide, de l’ADN. En 2010, la société 23andMe caractérisait 500 000 SNP pour 300 à 400 dollars (270 à 360 euros). En 2012, elle en caractérisait deux fois plus, pour un coût réduit à 99 dollars. \\

\textit{Séquençage complet} : Consiste à « lire » la totalité du message inscrit dans les chromosomes de chaque personne. Mais on est loin de savoir interpréter toute cette masse de données. \\

\subsection{Débat sur la génomique personnelle : « Pourquoi juger a priori abominable cette démarche ? »}

\begin{itemize}
	\item \textbf{Lien : }  \url{ http://lemonde.fr/sciences/article/2015/05/04/debat-sur-la-genomique-personnelle-pourquoi-juger-a-priori-abominable-cette-demarche_4627194_1650684.html} 
	\item \textbf{Auteur : } Florence Rosier
	\item \textbf{Date : } 04 mai 2015
	\item \textbf{Description : } Entretien. Le généticien Jean-Louis Mandel est l’un des seuls biologistes français à plaider pour la liberté individuelle de connaître les données de son propre génome.
	\item \textbf{Source : } Le monde. Le professeur Jean-Louis Mandel ne déclare aucun lien d’intérêt avec la société 23andMe ni avec aucune autre firme développant des tests diagnostiques ou génomiques.
\end{itemize}

Jean-Louis Mandel, généticien, est professeur au Collège de France. En 1991, son équipe a découvert un mécanisme de mutations inédit : une répétition instable de courtes séquences dans l’ADN, responsable d’un retard mental lié au chromosome X. Ce type de mutations est en cause dans plus de quinze maladies neurologiques comme l’ataxie de Friedreich ou la maladie de Huntington. \\


\textbf{En 2010, vous avez fait faire l’analyse de votre génotype par la société 23andMe. Quelles étaient vos motivations ?}

En tant que généticien, je voulais me rendre compte par moi-même de la qualité et des limites des tests réalisés, des données fournies et des explications qui les accompagnaient. Je ne comprenais pas la violente opposition de l’immense majorité de mes collègues à leur égard, ni d’ailleurs certaines positions du Comité consultatif national d’éthique (CCNE). Pourquoi fallait-il juger a priori abominable cette démarche ? En France, on refuse un débat sur le sujet. Il faudrait s’interroger sans tabou : pour l’individu, le système de santé et la société, ces tests sont-ils scandaleux, dangereux, ou sont-ils une liberté ? Les informations qu’ils fournissent sont-elles utiles ? Quels sont les éventuels avantages – ou les risques – d’une recherche médicale fondée sur la participation de patients volontaires et informés ?\\

\textbf{Que vous a appris l’analyse de votre génotype ?}

J’ai d’abord découvert que j’étais un porteur sain de la mutation la plus fréquente du gène de la mucoviscidose. Personne dans ma famille n’avait jamais été atteint. Etant porteur d’une seule copie mutée, j’avais une « chance » sur deux de transmettre cette mutation à mes enfants ; si mon épouse avait elle-même été porteuse d’une copie mutée, nos enfants auraient eu une probabilité de 25\% d’être touchés. Ma période reproductive est terminée, mais ma fille était désireuse d’avoir des enfants. Je lui ai conseillé de se faire tester : elle n’était pas porteuse de la mutation. Si elle l’avait été, sans doute son conjoint aurait-il pu faire ce test. La mucoviscidose reste une maladie très grave, même si de nombreux progrès ont été réalisés.\\

\textbf{Quid de vos prédispositions à des maladies communes ?}

Pour au moins une maladie, connaître mes prédispositions aurait pu avoir des vertus préventives. Par exemple, j’ai découvert que j’étais porteur de plusieurs variants prédisposant à une maladie oculaire : la dégénérescence maculaire liée à l’âge (DMLA), première cause de cécité chez les plus de 60 ans dans les pays industrialisés. Je suis allé consulter mon ophtalmologue : elle m’a fait un examen de fond d’œil, qui était normal. Par ailleurs, on sait que le tabac augmente notablement le risque de DMLA. Je ne suis pas fumeur, mais si je l’avais été, connaître cette prédisposition aurait été un argument supplémentaire pour m’inciter à arrêter de fumer. Ce n’est déjà pas si mal ! Et je fais plus attention à limiter mon exposition à des lumières trop intenses, à l’effet délétère. Enfin, deux médicaments (Avastin et Lucentis) sont désormais disponibles contre certaines formes de DMLA.\\

\textbf{Et pour d’autres affections communes comme le diabète ?}

Il faut bien l’admettre : l’intérêt médical de la plupart des risques de maladies communes calculés par 23andMe était très faible. Pourquoi ? Parce que le déterminisme de ces maladies est généralement très complexe. Pour chaque maladie, l’effet de chaque variant génétique est généralement très faible. Chacun d’eux interagit avec de nombreux autres facteurs génétiques et environnementaux : modes de vie, alimentation, tabac, infections, pollution, stress… De plus, votre prédiction pouvait varier selon le nombre de variants testés. Et la liste des variants pris en compte, pour chaque maladie, n’était pas exhaustive. Pour le diabète de type 2, la valeur prédictive des analyses de 23andMe restait très faible. En réalité, ces prédictions n’ont de valeur que pour deux ou trois maladies où ne jouent qu’un petit nombre de facteurs génétiques au rôle majeur. C’est le cas pour la DMLA et la maladie d’Alzheimer.\\

\textbf{Dans la maladie d’Alzheimer, quel est l’intérêt de disposer de cette information ?}

On a découvert en 1993 un variant génétique, ApoE4, qui prédispose à cette maladie. Environ 25\% des gens en possèdent une copie : leur risque est multiplié par 3 ou 4. Pour les 2\% de gens qui ont les deux copies de ce variant, le risque est multiplié par 9 à 12. Sur son site, la société 23andMe délivrait ce message : « Ne regardez pas tout de suite vos résultats. Lisez d’abord nos informations : vous déciderez après si vous voulez connaître votre prédisposition. » J’avoue que j’ai eu deux minutes d’hésitation. Il se trouve que je n’ai aucune copie de ce variant, ce qui ne veut pas dire que je ne développerai pas la maladie. A quoi peut servir cette information, dans la mesure où la prévention reste limitée – même si l’activité intellectuelle ou physique semble bénéfique ? Je pense qu’une personne âgée de 20 à 30 ans n’a pas intérêt à connaître cette prédisposition. Mais pour une personne de 60 ans, pourquoi pas ?\\

\textbf{En novembre 2013, la FDA a mis son veto à l’interprétation médicale des données de génotypage des nouveaux clients de 23andMe. Ses motifs vous semblent-ils fondés ?}

La Food and Drug Administration invoque deux motifs. D’une part, elle juge que ces tests sont des dispositifs médicaux : 23andMe aurait donc dû demander une autorisation préalable à leur mise sur le marché. Mais on peut soupçonner une intervention corporatiste des sociétés qui réalisent des tests génétiques. D’autre part, la FDA a estimé que 23andMe n’avait apporté aucune donnée clinique permettant de garantir la sécurité, l’efficacité et la fiabilité de ces analyses. Pour ce qui est de la fiabilité prédictive, elle reste en effet très limitée pour la plupart des maladies communes. Par ailleurs, ces tests portent sur un million de variants : des erreurs sont statistiquement possibles, avec des faux positifs ou des faux négatifs sur quelques dizaines de variants. Pour autant, 23andMe réalisait ces tests dans un laboratoire de génétique accrédité. La FDA reproche aussi à 23andMe de n’avoir pas démontré la validité médicale de ces tests. Mais lorsque certains médecins prescrivent des tests dont l’utilité médicale n’est pas démontrée, nul n’y trouve à redire ! Enfin, la FDA s’inquiétait des risques sanitaires : elle craignait que des clients de ces tests ne prennent des décisions hors de tout encadrement médical, parce qu’ils auraient été effrayés des résultats. Mais aux Etats-Unis, plusieurs études ont montré que les clients de 23andMe n’ont pas eu ces craintes. Quand vous avez une longue liste de risques potentiels, l’effet se dilue. Ces tests ne me semblent pas dangereux.\\

\textbf{Selon vous, les informations en ligne données par 23andMe avaient des vertus pédagogiques. Mais en tant que généticien, ne représentez-vous pas un public très averti ?}

De 2009 à fin 2013, la société 23andMe fournissait des informations très bien documentées. Pour chaque caractère ou chaque prédisposition calculée, elle livrait non seulement des renseignements sur les maladies concernées, leurs facteurs de risque et les éventuels moyens préventifs, mais elle hiérarchisait la confiance que l’on pouvait avoir dans les résultats. Elle analysait de façon critique la littérature scientifique sur laquelle se fondaient ses interprétations. Comme pour le Guide Michelin, elle attribuait des étoiles : une étoile pour les études peu fiables, et jusqu’à quatre étoiles si l’on disposait de nombreuses études concordantes et très fiables. De plus, elle fournissait deux types d’explications : l’un pour un public non averti mais cultivé ; l’autre pour un lectorat de généticiens.\\

\textbf{Quid du volet « pharmacogénomique », qui analyse les réponses individuelles médicaments ?}

J’ai appris que j’avais une sensibilité accrue à un médicament très utilisé en cardiologie : la warfarine, un anticoagulant oral de la famille des « antivitamines K » (AVK). On sait que la réponse à la warfarine varie beaucoup selon les patients, ce qui a d’importantes répercussions sur l’efficacité et la sécurité de ce traitement – les risques d’hémorragies liés aux AVK sont en effet très importants. Le fait de savoir que je réponds mieux à cet AVK n’est pas anodin : si je devais suivre un tel traitement, il faudrait sans doute en diminuer le dosage initial.\\

\textbf{Qu’avez-vous pensé des analyses plus « récréatives » ?}

Je me suis beaucoup amusé ! De façon anecdotique, 23andMe a découvert un variant qui explique pourquoi certaines personnes ne peuvent comprendre une page d’A la recherche du temps perdu : Proust y explique comment transformer le « pot de chambre en vase de parfum ». Car ce variant contribue au fait que certaines personnes, après avoir mangé des asperges, détectent une odeur dans leurs urines tandis que d’autres non. D’autres analyses sont contestables. Par exemple, 23andMe suggère une prédiction de l’influence que peut avoir l’allaitement maternel sur votre QI. Ou encore, analyse un variant censé prédire votre capacité à apprendre de vos propres erreurs. Mais la littérature scientifique sur laquelle se fondent ces prédictions est très peu fiable (comme l’indique d’ailleurs 23andMe) – même si certaines études ont été publiées dans PNAS ou Science !\\

\textbf{Le volet sur les origines ethniques vous semble-t-il intéressant ?}

L’analyse de mon génotype a indiqué que 96\% de mon génome était d’origine juive ashkénaze. C’est parfaitement exact : originaires de Pologne, mes quatre grands-parents sont juifs ashkénazes. Tous mes « cousins potentiels » qui avaient accepté de donner leur origine ou leur nom de famille ont la même origine ; certains avaient comme moi des ancêtres dans la ville de Lodz. Par ailleurs, un de mes post-doctorants a pu retrouver ainsi trois de ses cousins au cinquième degré, issus d’une branche de la famille ayant émigré aux Etats-Unis au XIXe siècle. \\

Le 19 février, la FDA a autorisé la vente libre d’un premier test génétique pour dépister le risque de transmettre une maladie très rare, le syndrome de Bloom. Certains y voient un premier pas vers la généralisation de ces tests en libre accès. Qu’en pensez-vous ?\\

Prenons l’exemple de la mucoviscidose. En France, lorsqu’un premier enfant est atteint, on juge éthique de faire dépister ses frères et sœurs et ses cousins. Et lorsque ceux-ci sont porteurs d’une mutation, on juge éthique que leurs conjoints se fassent dépister. On juge aussi éthique de faire un diagnostic prénatal lors d’une grossesse suivante, donc de prévenir un second cas. Mais on ne juge pas éthique de faire un dépistage des « porteurs sains » du gène de cette maladie chez tous les couples qui ont un projet parental – c’est-à-dire de prévenir un premier cas. J’ai du mal à comprendre. Aux Etats-Unis, le dépistage des mutations du gène de la mucoviscidose est proposé à tous les couples qui ont un projet parental. En Israël, c’est une politique d’Etat qui a évidemment un coût. Faut-il ou non laisser la liberté aux couples de connaître leurs risques de transmettre des maladies très graves à leurs enfants ? Il faudrait faire des études supplémentaires. Mais par peur du risque – certes réel – de dérive eugénique, on évacue le problème. On n’en parle même pas.

\subsection{Débat sur la génomique personnelle : « Nous ne sommes pas le produit de nos gènes »}

\begin{itemize}
	\item \textbf{Lien : }  \url{ http://lemonde.fr/sciences/article/2015/05/04/debat-sur-la-genomique-personnelle-la-valeur-predictive-et-clinique-des-tests-est-surestimee_4627189_1650684.html} 
	\item \textbf{Auteur : } Florence Rosier
	\item \textbf{Date : } 04 mai 2015
	\item \textbf{Description : } Entretien. Patrick Gaudray, membre du Comité consultatif national d’éthique, dénonce le consumérisme lié à la commercialisation de ces tests, ainsi que les dangers de la sacralisation de l’information génétique.
	\item \textbf{Source : } Le monde. Chercheur en génétique, le professeur Patrick GAUDRAY est membre du Comité national consultatif d’éthique (CCNE). Directeur de recherche au CNRS (université de Tours), il a participé à la grande aventure du projet Génome humain au cours des années 1990. De 2001 à 2006, il a été directeur scientifique adjoint des sciences de la vie du CNRS. 
\end{itemize}

\textbf{En mars, la société 23andMe annonçait avoir totalisé plus de 900 000 « clients » pour ses tests de génotypage en vente libre. Quels sont les problèmes posés par ces tests, interdits en France ?}

Le premier problème tient au terme même de customers (« clients »). Nous sommes là dans une dynamique de consumérisme : c’est une démarche choquante dans un pays comme le nôtre, où les médecins préfèrent parler de leur « patientèle » que de leur « clientèle ». Il s’agit d’un système entièrement guidé par le business, où la santé des gens n’est qu’un prétexte. D’ailleurs, 23andMe ne monnaye-t-elle pas les données de ses clients auprès de groupes pharmaceutiques ? En janvier, cette société a revendu à Genentech, pour 60 millions de dollars, le profil génétique d’une cohorte de 14 000 clients atteints de la maladie de Parkinson ou de personnes de leurs familles – pour la recherche des causes génétiques de cette maladie. En 2014, elle signait un accord avec Pfizer sur une cohorte de 10 000 clients atteints de la maladie de Crohn. Mais quelle est la nature du consentement donné par ces clients ? \\

\textbf{Fin 2013, la FDA a invoqué le manque de fiabilité des prédictions fournies par ces tests pour en interdire l’interprétation médicale…}

C’est le second défaut majeur de ces tests : leur valeur prédictive et clinique est surestimée. La Food and Drug Administration leur reproche de prétendre être bénéfique pour la santé des gens, alors qu’ils ne font que mettre en évidence des mutations dans des gènes donnés, corrélées à l’apparition de certaines maladies – mais sans liens démontrés de cause à effet. De plus, ces tests vous vendent des prédictions sans certitudes. Or en médecine, les messages probabilistes sont difficiles à comprendre. Et cette incertitude peut entraîner une détresse. Même en cas de mutation classique du gène BRCA1, qui expose à un risque accru de cancer du sein, on peut seulement dire à une femme : « Vous avez un risque de 60\% de développer un cancer du sein avant l’âge de 70 ans ». De plus, être indemne de cette mutation ne protège pas de ce cancer. La prédiction peut même aller à l’encontre de la prévention : il y a un risque de démobilisation vis-à-vis de pratiques à visée préventive, comme le dépistage par mammographie. \\

\textbf{Même si ces prédictions étaient fiables, la fréquente absence de solutions préventives ne compromet-elle pas leur intérêt ?}

Apprendre que vous êtes porteur d’une prédisposition forte à une maladie donnée n’est pas forcément bienfaisant, s’il n’existe aucun traitement satisfaisant ni aucun moyen préventif. Si l’on parvenait à affiner la prédiction et à mettre en place une prévention efficace, alors oui, ces tests pourraient être un bien. Par ailleurs, les tests de prédisposition au cancer du sein montrent les dilemmes préventifs auxquels ils exposent. Toutes les porteuses de la mutation du gène BRCA1 ou BRCA2 doivent-elles se faire enlever de façon préventive les deux seins, les deux ovaires et les trompes de Fallope ? La lecture de ces données génétiques doit rester intelligente, modulée et respectueuse de la personne et de son histoire familiale. \\

\textbf{Que pensez-vous du point de vue de Jean-Louis Mandel, qui défend l’intérêt de certaines informations délivrées par ces tests ?}

Je respecte le point de vue des généticiens cliniciens comme Jean-Louis Mandel, qui sont au contact quotidien des patients. Voir un enfant mourir de la mucoviscidose est extrêmement difficile ! Mais ces tests en vente libre soulèvent des questions éthiques majeures que l’on ne peut pas ignorer. Même si l’on parvenait à lever les incertitudes sur le déterminisme des maladies, ce questionnement éthique subsisterait. Il doit obtenir une réponse sociétale. \\

\textbf{Vous pointez aussi ce danger : ces tests nous réduisent à notre dimension génétique…}

Ils gomment notre dimension globale, oubliant notre dimension psychologique, affective et sociale ! Mais nous ne sommes pas le produit de nos gènes. Notre vie doit-elle être entièrement guidée par notre dimension génétique ? Devons-nous mener une vie d’ascète, suivre à la lettre les recommandations des nutritionnistes, courir 15 kilomètres par jour, faire tester notre ADN ? Si c’est le libre choix de la personne, pourquoi pas. Mais si c’est dicté par l’oukase d’une compagnie d’assurance, d’un employeur ou de la société, non ! En 2008, les Etats-Unis ont voté une loi, le Genetic Information Non-discrimination Act (GINA). Elle interdit aux assureurs et aux employeurs l’usage inapproprié d’informations génétiques de particuliers. Mais cette loi est déjà en partie contournée par les assureurs. Certaines personnes, par ailleurs, ont fait séquencer leur génome pour montrer à leur assureur qu’elles ne sont pas prédisposées à telle maladie et que leurs primes d’assurance doivent baisser ! Ce n’est pas un monde où j’aurais envie de vivre. Je redoute aussi qu’on escamote notre consentement libre et informé. N’allons-nous pas nous faire imposer le fardeau de notre carte d’identité génétique, ainsi que tous les comportements associés à cette connaissance, censés prévenir nos risques de maladies ? Ce serait une remise en cause du droit des patients à ne pas savoir. \\

\textbf{Vous dites que nous ne nous construisons pas sur des bases médicales et génétiques, mais sur un imaginaire…}

Il y a plus de vingt ans, j’ai été frappé par le cas d’une jeune femme de 25 ans, dont le père était atteint de la maladie de Huntington, une affection neurologique héréditaire très grave. Elle avait un risque sur deux de développer cette maladie entre 35 et 45 ans. Un test génétique est apparu, qui lui permettait de savoir si elle était porteuse de la mutation qui déclenche la maladie. Elle s’est laissée convaincre de le faire, mais elle a attendu des mois avant de rechercher le résultat. Finalement, elle n’était pas porteuse de la mutation. Quelques semaines plus tard, elle se suicidait. Elle avait bâti sa vie autour du fait qu’elle serait malade. \\

\textbf{Ces tests ne posent-ils pas aussi la question de « la normalité » ?}

Nous sommes tous porteurs de mutations qui peuvent conduire à des maladies plus ou moins graves. A partir de quand doit-on considérer ces mutations comme délétères ? Il est illusoire, scientifiquement insensé et sans doute criminel de penser pouvoir définir ce qu’est un « ADN normal ». Aucune référence n’est ici valide ! Avec ces tests, on est dans la tyrannie des normes dénoncée par le philosophe et médecin Georges Canguilhem. Même en cas de mutation avérée dans un gène qui expose à un risque de maladie sanguine, par exemple, on peut s’interroger : cette mutation se suffit-elle en elle-même ? Ou bien faut-il la conjonction de plusieurs événements génétiques pour prédisposer à cette affection ? \\

\textbf{Le CCNE mène actuellement une réflexion sur les questions posées par le séquençage à très haut débit du génome. Quels en sont les enjeux ?}

Le séquençage complet du génome pourrait se démocratiser, car les fulgurants progrès des techniques de séquençage s’accompagnent d’une prodigieuse diminution de leurs coûts. Lorsque le projet Génome humain (HUGO) a été lancé, en 1990, il visait à établir la séquence complète du génome humain. Nous savions ce but accessible, et de plus en plus rapidement. Mais nous n’avions pas prévu l’ampleur de cette accélération ! Imaginer qu’en dix ans, nous aurions divisé le temps et le coût de la lecture d’une séquence complète d’ADN humain par deux millions, c’était inconcevable ! Le projet HUGO a été terminé en 2003 : il a nécessité treize ans de collaborations internationales pour établir la première séquence complète du génome humain, avec un coût total de 3 milliards de dollars. En 2013, un génome humain complet pouvait être séquencé en quelques heures, pour un peu plus d’un millier de dollars ! \\

\textbf{Séquencer le génome d’une personne ne signifie pas en fournir une interprétation médicale…}

Certainement non. Pour cela, la qualité du séquençage doit être bien plus poussée que celle qui ne demande que quelques heures : il faut lever de nombreuses ambiguïtés. « La séquence à 1 000 dollars, l’interprétation à 100 000 dollars », écrivait Elaine Mardis en 2010 dans la revue Genome Medicine. Les progrès des connaissances issues du séquençage des génomes ont entraîné de nombreuses remises en cause des notions de génétique fondamentale. Nous ne sommes pas au bout de nos surprises ! Qu’est-ce qu’un gène ? On nage dans le flou, plus proche d’une réalité biologique. Il n’existe pas de continuité entre la séquence de l’ADN, celle de l’ARN et celle des protéines. Les gènes sont morcelés, certains ne codent pas pour des protéines mais pour des ARN. Et les gènes ne font pas tout, comme le montre l’épigénétique, qui s’intéresse aux processus de régulation de l’activité des gènes par des facteurs de l’environnement. Autre exemple : quand j’étais jeune chercheur, on parlait « d’ADN poubelle ». Des années plus tard, cette notion est battue en brèche. On découvre un sens à tout ce qui en semblait dénué. Tout ce que nous ne comprenons pas tend à être négligé, y compris par les sociétés qui font ces prédictions génétiques. Leurs conclusions sont biaisées. \\

\textbf{Quid du mythe de « l’enfant parfait » ?}

La démocratisation du séquençage du génome à un stade préconceptionnel [chez les couples ayant un projet parental] reviendrait à un « mariage de fonds génétiques », et non plus à un mariage de personnes. Cela réactiverait le mythe de réaliser un enfant le plus proche possible de la perfection. Mais quelle perfection ? Sur la base de quelles normes, définies par qui ? \\

\textbf{Pourquoi le dépistage du gène de la mucoviscidose n’est-il pas pratiqué en population générale en France ?}

Rechercher le gène de la mucoviscidose chez toutes les femmes ayant un projet de grossesse coûterait très cher. Au-delà de la question économique, les questions éthiques sont nombreuses. Tous les dépistages préconceptionnels ou prénataux posent un risque de dérive eugénique. On sait que 96\% des grossesses au cours desquelles on détecte un fœtus porteur de la trisomie 21 aboutissent à une interruption médicale de grossesse. C’est par définition un acte eugénique. Jean-Louis Mandel a raison de pointer le fait qu’en France, on juge éthique de pratiquer un dépistage de la mucoviscidose pour prévenir la naissance d’un second enfant, après un premier cas familial. Mais on ne juge pas éthique de le faire pour prévenir la naissance d’un premier enfant. C’est le critère qui a été choisi en France. Il aurait pu y en avoir d’autres, comme l’appartenance à des communautés ethniques où la maladie est plus fréquente. Mais cela poserait d’autres questions éthiques que Jean-Louis Mandel semble ne pas voir. Car admettons que nous fassions un dépistage ciblé de la maladie de Tay-Sachs chez les populations juives ashkénazes, par exemple. Cela voudrait dire que les seuls malades qui passeraient au travers de ce crible seraient ceux qui n’appartiennent pas à ces populations. Et cela pourrait engendrer un sentiment d’injustice. En réalité, il n’existe pas de réponse simple à ces questions. Il faut continuer à s’interroger. \\

\textbf{Pourrions-nous demain renoncer à ces nouvelles techniques ?}

Il est illusoire de penser que nous pourrions revenir à un état d’avant le séquençage de l’ADN. Il est probablement déjà plus simple et moins onéreux de séquencer la totalité du génome d’une personne que de rechercher une mutation dans un gène donné : c’est donc cela qui risque d’être pratiqué demain ! Mais le CCNE s’interroge : quelle forme de régulation pourrions-nous mettre en place ? Je reste émerveillé par les extraordinaires progrès de nos connaissances sur les génomes, eux-mêmes permis grâce aux foudroyants progrès des techniques de séquençage. Mais je refuse de me laisser fasciner : ces progrès sont une lumière qui peut brûler.

\end{document}


