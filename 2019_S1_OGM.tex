\documentclass[8pt]{article}

\usepackage[T1]{fontenc}
\usepackage[utf8]{inputenc}
\usepackage{graphicx}
\usepackage{lmodern}
\usepackage{amsmath}
\usepackage{xfrac}
\usepackage{amsthm}
\usepackage{listings}
\usepackage{enumerate}
\usepackage{amssymb}
\usepackage{cancel}
\usepackage{amsfonts}
\usepackage{float}
\usepackage{fullpage}
\usepackage{pdfpages}

\DeclareUnicodeCharacter{200A}{ } 
\renewcommand*\contentsname{Table des matières}

\PassOptionsToPackage{hyphens}{url}\usepackage{hyperref}

\usepackage{listings}
\author{ControverSciences\textit{ et al} }
\title{Projet AREN - Corpus de ressources \\  OGM dans les cultures et dans l'alimentation.}
\date{15 Janvier 2019}

\begin{document}
\maketitle

Les sujet des OGM dans les cultures et dans l'alimentation est abordé par les définitions d'un OGM ainsi que son utilisation actuelle des le monde, en Europe et en France (\ref{sec:definition}). Par ailleurs les différentes catégories d'OGM (herbicides, insecticides et fongicides) sont explicités (\ref{sec:resistance}). Une fois les bases et définitions posées, plusieurs aspects sont abordés.
Les effets des OGMs sur la santé et la controverse qui en a découlé est synthétisé dans un article récent de Sylvestre Huet, journaliste scientifique pour \textit{Le Monde} (\ref{sec:sante}). 
Ensuite un article du journal \textit{Pour La Science}, décrit les enjeux et débats actuelles sur l'encadrement et la définition des OGM et l'impact de cette encadrement sur la biodiversité et sur la concentration des industries (\ref{sec:brevets}). 
Enfin un article de \textit{CNRS Le Journal} s'attache à l'impact de l'utilisation des pesticides (herbicides, insecticides et fongicides) sur les écosystèmes (\ref{sec:saturation}).\\

Ce corpus de préparation permet d'aborder les différentes facettes (définitions, santé, biodiversité, pesticides et enjeux économique) du débat sur utilisation des OGM Bt (\ref{sec:Bt}). Les OMGs Bt sont interdits en Europe (et en France) et permettent l'utilisation de moins d'insecticides, il y a donc l'effet bénéfique de moins de pesticides déversé, mais de l'autre coté l'effet négatif de monocultures brevetés qui diminue la biodiversité des cultures et favorise la concentration des industries.

\tableofcontents
\newpage
\section{Corpus de ressources}
\subsection{Organismes génétiquement modifiés (OGM)}
\label{sec:definition}

\begin{itemize}
	\item \textbf{Lien : }  \url{https://www.ecologique-solidaire.gouv.fr/organismes-genetiquement-modifies-ogm} 
	\item \textbf{Date : }  25 novembre 2016 
	\item \textbf{Source : } Le \textit{Ministère de la Transition écologique et solidaire} a pour mission générale de préparer et mettre en œuvre la politique du Gouvernement dans tous les domaines liés à l’écologie, la transition énergétique et à la protection de la biodiversité.
	
\end{itemize}

Depuis que l’homme cultive des plantes et élève des animaux pour se nourrir, il a toujours sélectionné ceux qui présentaient des caractéristiques bénéfiques afin d’améliorer les générations suivantes. Ces caractéristiques reflétaient des variations génétiques naturelles et ont résulté, par exemple, en un rendement accru ou une résistance particulière à des maladies ou à des pressions environnementales. Un organisme génétiquement modifié (OGM) est un organisme (animal, végétal, champignon, micro-organisme) dont on a modifié le matériel génétique d’une manière qui ne s’effectue pas naturellement, pour lui conférer une ou plusieurs caractéristiques recherchées. \\

\textbf{Les différents domaines d’utilisation des OGM }\\

L’utilisation d’OGM la plus connue est celle qui est faite en agriculture pour conférer de nouveaux caractères aux plantes pour leur utilisation en alimentation animale et humaine. Toutefois, ils sont également utilisés dans d’autres domaines tels que :
\begin{itemize}
	\item la recherche fondamentale pour répondre à des questions de compréhension du vivant ;
	\item le milieu industriel, où des micro-organismes peuvent être utilisés pour produire des molécules ;
	\item la médecine humaine, où ils peuvent servir à de vecteurs en thérapie génique ou pour la production de vaccins ou de médicaments.
\end{itemize}


Plusieurs applications du génie génétique apparaissent dans le domaine animal. C’est le cas par exemple aux États-Unis, où un saumon transgénique dont la croissance est multipliée par deux par rapport à un saumon non modifié a été autorisé pour la mise sur le marché en 2016.\\

D’autres applications apparaissent, comme l’utilisation de moustiques génétiquement modifiés. Ils ont été autorisés à la dissémination dans l’environnement au Brésil. Ce moustique transgénique stérile est censé permettre de lutter contre la dengue, une maladie qui fait de nombreuses victimes dans les pays tropicaux. Cette technique serait un moyen de combattre d’autres maladies parasitaires telles que le chikungunya, le virus Zika ou encore le paludisme.\\

Des protéines d’intérêt thérapeutique sont produites en France, en milieu confiné, à partir de bactéries, de plantes ou d’animaux génétiquement modifiés, comme l’insuline ou des hormones de croissance.\\

\textbf{Les plantes génétiquement modifiées}\\

Les espèces végétales génétiquement modifiées les plus cultivées dans le monde sont le soja, le maïs, le coton et le colza. Le soja et le maïs occupent à eux seuls plus de 80 \% des surfaces cultivées d’OGM. Le riz, la papaye, l’aubergine, la pomme de terre ou la betterave font aussi régulièrement l’objet de modifications génétiques.\\

Les modifications génétiques portent essentiellement sur l’introduction de deux caractères dans les cultures : tolérance à un ou plusieurs herbicides et résistance aux ravageurs (insectes nuisibles pour les cultures agricoles) par production d’une molécule insecticide, ou une combinaison de ces deux caractères. D’autres caractéristiques sont en développement (autres résistances aux maladies, efficience de l’utilisation de l’azote, tolérance à la sécheresse, développement de qualités organoleptiques), mais ne font pas encore l’objet de mise en culture importante.\\

En 2015, les organisations internationales ont recensé 180 millions d’hectares cultivés dans 28 pays, par 18 millions d’agriculteurs. Après 20 ans de croissance continue, les surfaces des cultures OGM ont diminué de 1\% en 2015 par rapport à 2014 (chiffres issus du rapport  annuel de l’ISAAA ).\\

\textbf{Les pays producteurs de plantes génétiquement modifiées}\\

Les principaux producteurs sont les États-Unis (70,9 millions d’hectares), le Brésil (44,2 millions d’hectares), l’Argentine (24,5 millions d’hectares), l’Inde (11,6 millions d’hectares) et le Canada (11 millions d’hectares). Le Vietnam a cultivé des plantes génétiquement modifiées pour la première fois en 2015.\\

En Europe, où seule la culture du maïs Bt génétiquement modifié MON810 est autorisée (maïs génétiquement modifié pour produire une toxine qui cible les insectes nuisibles , la pyrale et la sésamie), les surfaces cultivées en 2015 ont diminué de 18\% par rapport à 2014 et représentent moins de 0,1\% des surfaces mondiales d’OGM cultivées. L’Espagne, le Portugal, la République tchèque, la Slovaquie et la Roumanie ont cultivé 116 870 hectares de maïs Bt génétiquement modifié en 2015, dont 92\% des surfaces pour l’Espagne.\\

L’Europe importe des OGM principalement pour l’alimentation des animaux. En effet, l’Union européenne importe 70\% des protéines végétales qu’elle incorpore dans l’alimentation animale. On estime que 80\% des importations d’aliments pour animaux est OGM (principalement les tourteaux de soja, à 85\% OGM). Très peu d’OGM sont destinés directement à l’alimentation humaine.\\

\textbf{La situation de la France}\\

En France, la culture du maïs transgénique MON810 (seule plante autorisée à la culture en Europe) est interdite depuis 2008. Avant cette interdiction, les cultures de plantes transgéniques sur le territoire n’ont jamais été très importantes (22 000 ha de maïs génétiquement modifié en 2007). Cette interdiction passait par un moratoire sur la culture du maïs MON810, renouvelé à deux reprises en 2012 et 2014. En 2015, une directive européenne  a donné la possibilité aux États membres de ne pas autoriser la culture d’un OGM sur leur territoire respectif pour des motifs socio-économiques. Le gouvernement français a activé cette possibilité ainsi que 18 autres États membres.\\

La France a aussi cultivé des plantes génétiquement modifiées à titre expérimental pour des essais en plein champ (betterave, blé, colza, luzerne, maïs, peuplier, vigne…). Depuis 2013, date de la dernière culture expérimentale de peuplier, il n’y a plus aucune plante OGM cultivée en France.\\

En dehors de la culture, la France importe toutefois des matières premières dont elle dispose en quantité insuffisante sur le territoire pour la fabrication d’aliments pour animaux, notamment du soja génétiquement modifié.\\

En France, les OGM sont donc principalement utilisés à des fins de recherche, en milieu confiné (sans contact avec la population ou l’environnement), pour diverses applications telles que la production de protéines thérapeutiques (l’albumine par exemple) ou pour l’alimentation animale.

\newpage
\subsection{Résistance aux insectes et tolérance aux herbicides}
\label{sec:resistance}

\begin{itemize}
	\item \textbf{Lien : }  \url{http://www.ogm.gouv.qc.ca/utilisation_actuelle/tolerance_herbicides/tolerance.html}
	\item \textbf{Source : } Le gouvernement du Québec propose le site ogm.gouv.qc.ca qui rend compte de l’état actuel des connaissances générales et scientifiques sur les OGM : la conception, les processus d’approbation, les questions éthiques, les effets sur la santé et l’environnement, etc. Le site a pour objet non pas de présenter tous les points de vue connus sur les OGM, mais plutôt d’offrir une information factuelle et accessible concernant l’utilisation des OGM, notamment en matière d’agriculture intensive ou industrielle. Des mises à jour sont effectuées régulièrement, afin de présenter l’information la plus complète et la plus précise possible (\url{http://www.ogm.gouv.qc.ca/a_propos.html}).
\end{itemize}

\textbf{Résistance aux insectes}\\

La lutte contre les insectes nuisibles constitue l’une des principales préoccupations des agriculteurs. Ces derniers ont donc recours aux insecticides pour protéger leurs cultures. Notamment, de la poudre contenant des bactéries Bt (pour Bacillus thuringiensis, une bactérie du sol présente à l’état naturel) est grandement utilisée en agriculture traditionnelle ou biologique depuis de nombreuses années. Cette bactérie produit des protéines appelées delta-endotoxines, lesquelles sont toxiques pour les insectes qui les ingèrent. En effet, le système digestif de l’insecte transforme la protéine naturelle non toxique en une forme plus petite et très toxique qui s’attaque aux intestins, tuant éventuellement l’insecte. \\

Depuis quelques années, de nombreuses cultures ont été modifiées génétiquement pour produire leur propre toxine Bt et les rendre ainsi résistantes à des insectes. Il existe en effet plusieurs variantes de toxines naturellement produites par les bactéries Bt et qui nuisent à différents groupes d’insectes. La toxine Cry1Ab, l’une des toxines les plus utilisées en génie génétique, nuit aux lépidoptères, des papillons de jour et de nuit, mais pas aux insectes des autres familles. \\

La toxine Bt est considérée sans danger pour l’être , car elle est rapidement détruite dans l’estomac et les parois des intestins des mammifères n’activent pas l’activité toxique de la protéine Bt.\\

\textbf{Tolérance aux herbicides}\\

Les mauvaises herbes peuvent réduire grandement le rendement des cultures. En effet, elles entrent en concurrence avec les plantes agricoles pour les nutriments du sol, l’eau et la lumière. Afin d’éliminer les plantes indésirables, les cultivateurs peuvent vaporiser leurs champs avec des herbicides chimiques. Il existe plusieurs types d’herbicides, différenciés selon les familles de végétaux qu’ils détruisent.\\

Les herbicides à spectre large détruisent la majorité des plantes. Ils ne peuvent donc pas être appliqués au champ quand les plants cultivés ont commencé à pousser. Ce type d’herbicide est le plus souvent vaporisé en préémergence, c’est-à-dire avant que les plantes cultivées ne sortent de terre. Les mauvaises herbes qui apparaissent après la sortie des cultures sont éliminées d’une autre façon, souvent mécaniquement.\\

Depuis peu, il existe de nouveaux herbicides postémergents avec un spectre plus restreint qui permettent aux fermiers d’éliminer les mauvaises herbes en vaporisant directement les cultures. Comme ce type d’herbicide détruit moins de variétés de plantes, les agriculteurs doivent traiter leurs champs plus souvent afin de détruire toutes les mauvaises herbes.\\

Les chercheurs ont développé des plantes transgéniques tolérantes aux herbicides à spectre large. Ainsi, les agriculteurs peuvent vaporiser leurs champs n’importe quand pour détruire toutes les mauvaises herbes, sans nuire aux cultures. La culture de ce type d’OGM devrait permettre, théoriquement, de réduire le nombre d’applications d’herbicides.\\

Contrairement à une opinion répandue, les plantes GM tolérantes à un herbicide ne synthétisent pas ou ne contiennent pas un herbicide.\\

Bien qu’il existe plusieurs types de tolérance à un herbicide (tolérance à l’herbicide glyphosate, glufosinate, imidazoline, etc.), les plantes GM utilisent habituellement l’une des stratégies suivantes :
\begin{itemize}
	\item la plante transgénique produit une nouvelle protéine qui annule l’effet toxique de l’herbicide;
	\item dans la plante GM, la protéine normalement ciblée par l’herbicide est remplacée par une nouvelle protéine non sensible à l’herbicide. 
\end{itemize}

Une analyse des différents types de plantes tolérantes aux herbicides et des effets agronomiques, environnementaux et socio-économiques de leur utilisation a été réalisée par l’Institut national de Recherche agronomique (INRA) en France et l’Institut des sciences végétales du Conseil national de Recherche scientifique (CNRS). Cette analyse couvre les plantes tolérantes aux herbicides obtenues par génie génétique, par croisement standard ou par mutagénèse. Selon leur rapport, ces types de plantes peuvent être un outil complémentaire intéressant face à certaines situations de désherbage difficile. Cependant, une mauvaise utilisation de cette technologie et une utilisation répétée sans rotation des cultures peuvent les rendre inefficaces.

\newpage
\subsection{OGM-poisons ? La vraie fin de l’affaire Séralini.}
\label{sec:sante}

\begin{itemize}
	\item \textbf{Lien : }  \url{http://huet.blog.lemonde.fr/2018/12/11/ogm-poisons-la-vraie-fin-de-laffaire-seralini/} 
	\item \textbf{Auteur : } Sylvestre Huet
	\item \textbf{Date : }  27 Février 2018 
	\item \textbf{Source : } Blog \{Sciences$^2$\}
\end{itemize}

Vous souvenez-vous ? Ces images spectaculaires de rats atteints de cancers envahissants, si gros qu’en en voit les boules sous le poil. Exhibés à la télévision. Diffusés en film, livre, articles retentissants. Et de cette formidable campagne de presse lancé par le titre choc de l’Obs : «Oui, les OGM sont des poisons». \\

Oui, vous vous souvenez. Mais savez-vous que le 10 décembre, la revue Toxicology Sciences a publié l’un des articles de recherche montrant qu’il s’agissait d’une infox ? Certainement pas.

Revenons à ce jour de septembre 2012. L’hebdomadaire publie alors un épais dossier à l’appui de son titre. Mais un dossier étrange : ses seules sources d’information sont l’équipe du professeur Gilles-Eric Séralini, auteur principal d’une expérience publiée le même jour et des militants opposés à l’utilisation des plantes transgéniques. Comme si l’équipe de journalistes du Nouvel Observateur mobilisée pour ce coup de presse n’avait besoin de personne, en particulier d’autres experts du sujet, pour juger de la solidité de la thèse présentée par l’équipe du biologiste. Étrange puisque cette thèse s’oppose frontalement à nombre d’études déjà publiées. En affirmant que les rats nourris au maïs génétiquement modifié pour tolérer le glyphosate – principe actif des herbicides les plus utilisés dans le monde par les agriculteurs, dont le fameux Round Up inventé par Monsanto – en souffrent jusqu’à la mort.

Radios et télés enchaînent, sans plus d’enquête critique – mais c’est difficile à ce rythme – au point que le gouvernement, par la voix de son ministre de l’agriculture, Stéphane Le Foll, annonce le soir même qu’il va demander une modification des procédures européennes destinées à expertiser les risques des plantes transgéniques avant leur mise sur le marché.\\

\textbf{Données brutes}\\

Quelques mois plus tard, les deux agences publiques d’expertise concernées – ANSES (Agence nationale de sécurité sanitaire de l'alimentation, de l'environnement et du travail) et HCB (Haut Conseil des Biotechnologies) – publiaient une analyse complète de l’article de Gilles-Eric Séralini et al. et concluaient toutes deux à son incapacité à démontrer quoi que ce soit. Les données brutes de l’expérience montrent que sa mauvaise réalisation, en particulier par le trop faible effectif des groupes contrôles, interdisait de tirer une quelconque conclusion des observations faites sur la santé des rats au bout de deux ans de régime au maïs modifié génétiquement.\\

Toutefois, l’ANSES recommandait de conduire une expérience « vie entière » – deux ans pour les rats – afin de répondre à la question posée par Séralini : « manger ce maïs transgénique rend-il malade à long terme, en particulier cela provoque t-il des cancers ? ». De son côté, le comité scientifique du HCB ne le recommandait pas vraiment, mais disait en substance : si cela peut rendre confiance aux citoyens et aux consommateurs, pourquoi pas ?\\

Cela a t-il été fait ? Oui. Au prix d’environ 15 millions d’euros dépensés par la Commission Européenne et la France et de milliers de rats de laboratoire. Par trois expériences différentes et indépendantes. Beaucoup mieux préparées et conduites que celle de Gilles-Eric Séralini. Et pour quel résultat ? Allons droit au but, comme à l’Olympique de Marseille : RAS. Rien à signaler côté santé des rats qu’ils soient nourris 90 jours, un an ou deux ans, avec des maïs transgéniques (tant pour le maïs tolérant au glyphosate que pour celui produisant son propre insecticide). Il y a certes quelques signaux dans l’expérience française, mais plus liés à des différences entre variétés de grains utilisés, pas vraiment entre maïs transgéniques et non transgéniques.\\

\textbf{Rêvons un peu}\\

Avant d’en venir à ces expériences et de leurs résultats, rêvons un peu. Rêvons que les journaux, radios, télévisions, journalistes et ONG ou responsables politiques qui ont en chœur assuré à leurs publics, lecteurs, électeurs et militants que Gilles-Eric Séralini avait « prouvé » que « les OGM » sont des « poisons » mortels, vont consacrer autant d’efforts, de temps de paroles, de longueur d’articles et de propos publics à annoncer cette nouvelle  désormais bien établie.\\

Ce rêve n’a aucune chance de se réaliser. Ces actions ne sont susceptibles de rapporter aucune voix lors d’une élection, aucun soutien d’une opinion publique à des candidats aux postes électifs plus motivés par leurs conquêtes que par la qualité du débat public. Côté presse non plus : ce type d’information normale, a-t-on appris dans les écoles de journalisme, « ne fait pas vendre ». L’homme qui mord un chien, c’est une info, mais si c’est un chien qui mord un homme, c’est une info seulement s’il en meurt. Une plante transgénique qui tue, c’est une information; elle se contente de nourrir, ce n’en est pas une. Et les près de 98\% des journalistes qui ont écrit sur cette affaire sans lire l’article originel de G-E Séralini ne vont pas plus lire les résultats de ces expériences ni se voir incité à les présenter par des rédactions en chef qui n’y verront pas le motif d’un titre bien saignant.\\

\textbf{Donc, cessons de rêver. Et informons.}\\

Quatre expériences ont été conduites. Trois européennes et une française.
\begin{itemize}
	\item \textbf{Marlon} qui a étudié l’état de santé des animaux d’élevage nourris avec des plantes transgéniques comparé avec celui d’animaux n’en consommant pas.
	\item \textbf{GRACE} (GMO risk assessment and communication of evidence) dans un cadre toxicologique réglementaire avec du maïs MON 810 (maïs modifié pour produire la toxine insecticide Bt) avec des études à 90 jours et à un an dans l’objectif de vérifier si les protocoles à 90 jours ne ratent pas des processus plus lents.
	\item \textbf{G-TwYST} (GM plants two years safety testing) qui réalise notamment l’expérience vie entière avec du maïs tolérant au glyphosate et visant l’apparition de cancers à long terme que G-E Séralini prétendait faire… mais avec des rats mieux choisis pour ce type d’étude et en nombre suffisant (50 dans chacun des groupes testés et groupes contrôles contre les dix de Séralini) permettant d’obtenir des statistiques significatives.
	\item \textbf{GMO 90+}, c’est l’expérience française, proposée par Bernard Salles, le dernier auteur de l’article de Toxicological Sciences. Elle était destinée à étudier si l’on peut tirer d’une expérience sur six mois, des informations sur des « précurseurs » biologiques susceptibles d’indiquer de futurs problèmes de santé chez les rats testés. L’expérience est conduite avec les deux types de maïs transgéniques (tolérant au glyphosate et Bt). Elle fait appel à des technologies dite « omiques » (protéomique, etc) pour traquer des signaux faibles dans le métabolisme susceptibles d’être précurseurs de maladies survenant à plus long terme. Elle fut financée par le ministère de la Transition écologique et solidaire.
\end{itemize}


Ces expériences sont terminées, les résultats publiés ou en cours de publication (mais déjà connus des spécialistes car exposés en séminaires). L’expérience GMO90+ vient ainsi d’être publiée dans Toxicological Sciences. Elles doivent donner lieu à des analyses croisées complètes permises par une transparence totale sur les données brutes de chacune d’entre elles. Les informations disponibles vont toutes dans le même sens : pour un rat, avaler du maïs rendu tolérant au glyphosate, ou producteur de la toxine Bt (issue d’une bactérie commune) ou un maïs standard, c’est kif kif pour sa santé. L’étude GMO90+, très minutieuse, conclut à l’absence d’effets (clinique, physiopathologique, dans les analyses d’urine…) d’une nourriture avec les maïs génétiquement modifiés. L’étude à deux ans ne montre en particulier aucun effet sur la survenue de cancers.\\

\textbf{Quelques remarques :}
\begin{itemize}
	\item Dire que ces expériences prouvent que « Les OGM ne sont pas des poisons » serait une ânerie de même calibre que l’affirmation inverse du Nouvel Observateur en septembre 2012. Elles montrent seulement que les plantes transgéniques testées, et uniquement celles-là, ne sont pas des poisons.
	\item Ces expériences donnent raison une fois de plus aux biologistes qui estiment qu’il faut « une raison » (biochimique, biologique) de se demander si telle ou telle plante transgénique pose un problème de santé ou non et non supposer a priori que l’introduction d’un gène (ou sa manipulation à l’aide des nouvelles techniques disponibles comme CRISPR) représente un risque sanitaire plus élevé que, par exemple, un croisement artificiel utilisé en sélection de semences traditionnel. En l’occurrence, il n’y avait pas de « raison » de penser que le gène de tolérance au glyphosate ou celui permettant la production de la toxine Bt et les protéines qu’ils codent constituaient un risque sanitaire pour la consommation humaine.
	\item Les technologies de manipulation génétique progressent, notamment avec CRISPR. La perspective de voir des plantes modifiées pour les cultures grandit. La réponse militante consistant à vouloir à toute force suspecter a priori ces plantes modifiées et voulant interdire ces techniques de manière générique pourrait bien se terminer par une défaite généralisée et le recul de la vigilance. Les résultats de ces trois expériences sont ainsi agités par les semenciers utilisant la transgenèse et leurs partisans pour réclamer… que l’on ne fasse plus du tout d’études toxicologiques à 90 jours sur les plantes transgéniques. C’est le retour de bâton qu’il fallait craindre, un retour de bâton d’autant plus dangereux avec les nouvelles techniques d’édition du génome. Les décisions d’encadrement réglementaires ont en effet été prises sur la base des «spasmes de l’opinion publique», note un sociologue, et non sur des analyses scientifiques montrant la nécessité de prendre des précautions avec les produits d’une technologie nouvelle.
	\item Si ces expériences démontrent l’innocuité sanitaire de ces deux plantes transgéniques, elles ne disent rien de leur (in)utilité ou de leurs effets sociaux, économiques, agronomiques et environnementaux.
	\item Comme il est très peu probable que les résultats conclusifs de ces expériences réalisées avec un grand luxe de précautions seront autant diffusées auprès des citoyens et consommateurs, comme d’ailleurs de « décideurs » (élus notamment), il est regrettable que l’affaire Séralini soit celle d’un lanceur de fausse alerte, puisque toute fausse alerte occupe une part de la citoyenneté et de l’expertise publique disponible pour une vraie alerte sanitaire ou environnementale. Certes, il vaut mieux se tromper de temps en temps et traiter une fausse alerte que de passer à côté d’une vraie mais ne pas se noyer dans les fausses alertes est indispensable. Sinon, c’est l’histoire du petit garçon qui criait toujours au loup et qui n’a pas été cru lorsque le vrai loup est arrivé qui risque de survenir.
\end{itemize}

\newpage
\subsection{Les brevets sur les plantes mettent en danger le modèle agricole européen }
\label{sec:brevets}

\begin{itemize}
	\item \textbf{Lien : }  \url{https://www.pourlascience.fr/sd/agronomie/les-brevets-sur-les-plantes-mettent-en-danger-le-modele-agricole-europeen-9545.php} 
	\item \textbf{Auteur : } Frédéric Thomas est historien des sciences
	et des techniques, chargé de recherche à l’Institut de recherche pour le développement (IRD), au sein de l’UMR Patrimoines locaux, au	Muséum national d’histoire naturelle, à Paris.
	
	\item \textbf{Date : }  24 Février 2017
	\item \textbf{Source : } Pour La Science est la version française du mensuel Scientific American. C'est une revue de vulgarisation scientifique dans toutes les disciplines, dont les articles sont signés par les chercheurs eux-mêmes.
	\item \textbf{Résumé : }
	Si l’Europe décide que les plantes issues des nouvelles biotechnologies ne sont pas des OGM,
	elle va ouvrir grand ses marchés aux plantes brevetées, favorisant la concentration
	des industries semencières et les monocultures intensives.
\end{itemize}

Il y a une dizaine d'années, la Commission européenne a nommé un groupe d'experts chargé d'évaluer si les plantes issues d'une nouvelle génération de biotechnologies sont des organismes génétiquement modifiés (OGM) et s'il faut donc encadrer leur mise sur le marché comme telles. Depuis, d'autres techniques, telle CRISPR-Cas, ont rejoint cette nouvelle génération , mais la question, elle, n'est toujours pas tranchée.\\

Derrière ce débat très technique pointe un enjeu social, économique et culturel de taille : sommes-nous prêts à ouvrir nos marchés sans aucun garde-fou aux plantes issues de ces nouvelles technologies qui seront, comme les OGM, des plantes protégées par brevet, produisant les mêmes effets de concentration des marchés et de renforcement des systèmes de monoculture ?\\

Dans l'état actuel des débats, l'Europe devrait reconnaître que la plupart des plantes issues de ces nouvelles biotechnologies sont des OGM, mais en plaçant ces techniques à l'annexe 1B de la directive européenne 2001/18/CE. Ces plantes seraient ainsi exclues du champ d'application de la directive et seraient mises sur le marché non seulement sans aucune évaluation de leurs risques pour l'environnement et la santé humaine, mais aussi sans aucune régulation juridique. Si l'Europe et la France poursuivent dans cette direction, les plantes protégées par brevets vont déferler sur nos marchés. Et le système de protection intellectuelle jusqu'ici préféré – l'UPOV, l'Union internationale pour la protection des obtentions végétales – va vite devenir obsolète. Autant de mauvaises nouvelles tant pour la diversité de nos systèmes d'innovation que pour l'accès aux ressources génétiques, la diversité des types de variétés cultivées et la diversité de nos systèmes agraires et alimentaires européens.\\

Que connaît-on déjà des effets des plantes brevetées sur l'économie des semences ? L'amélioration des plantes a longtemps été un secteur économique très peu concentré avec des petits marchés segmentés, occupés par de nombreuses petites et moyennes entreprises (PME) développant de multiples formes de partenariats avec les instituts de recherche publique et mettant sur le marché des types variétaux relativement diversifiés. Il en a résulté un maintien de la pluralité des systèmes d'innovation et, par conséquent, une relative diversité des variétés proposées sur des marchés de taille moyenne correspondant aux différentes unités agroécologiques et climatiques de chaque pays, de chaque région...\\

L’arrivée des OGM brevetés a bouleversé cette structuration, car les brevets sont de formidables instruments de concentration. 
En 2013, une étude du Commissariat général à la stratégie et à la prospective en France a montré que seulement dix grandes firmes multinationales contrôlent 60\% du marché mondial des semences. 
À l’exception du Français Limagrain, de l’Allemand KWS, du Japonais Sakata et du Danois DLF Trifolium, ces groupes sont tous de l’industrie chimique : les Américains Monsanto, DuPont Pioneer,Dow, Windfield Solution, et les Allemands Bayer et BASF.\\

Cette récente concentration est le résultat direct du recours aux brevets. Par exemple, tout élément qui constitue une innovation, au sens du droit des brevets, dans une séquence d’ADN ou d’ARN peut être breveté. En pratique, les brevets portent sur des gènes d’intérêt qui sont a priori insérables dans n’importe quelle variété d’une même espèce, voire dans plusieurs espèces, surtout lorsqu’il s’agit d’un gène de résistance à un herbicide dont l’efficacité est universelle. Les traits protégés par des brevets ont donc vocation à conquérir des marchés de dimension mondiale. 
Cette concentration croissante et accélérée a été
identifiée en 2013 comme préoccupante par
le Comité économique, éthique et social du
Haut Conseil des biotechnologies (HCB), car
elle met en péril ce qui reste de la diversité
de nos systèmes d’innovation et, in fine, de
la diversité des variétés cultivées.\\

Jusqu’à présent, les marchés européens
des semences ont peu senti les effets de
cette concentration, car les plantes protégées
par brevets sont essentiellement des OGM. Or
les OGM sont peu implantés en Europe. Seul
l’événement MON810 de Monsanto (rendant
les maïs résistants au RoundUp) est autorisé
par l’Union européenne, et encore 17 pays
sur 28, dont la France, interdisent la culture
des variétés de maïs possédant cette modification. C’est précisément ce que les plantes
brevetées issues des nouvelles biotechnologies risquent de changer si elles sont mises
sur le marché sans aucune réglementation.\\

D’après l’Office européen des brevets (OEB) ,
1 927 brevets sur des plantes génétiquement
modifiées ont été déposés en 2014 à l’ OEB ,
contre seulement 170 sur des plantes non
génétiquement modifiées. Quelles sont ces
plantes non génétiquement modifiées brevetées et déjà présentes sur les marchés européens ? Ils’agit principalement de deux types de
variétés : des variétés mutées tolérantes à
des herbicides et des plantes obtenues par
des techniques tout à fait classiques, mais
qu’un certain dévoiement de la directive
européenne 98/44/CE sur les biotechnologies
permet de breveter.\\

Les variétés mutées tolérantes aux herbicides se sont vite diffusées sur les marchés
européens pour un certain nombre d’espèces.
BASF et DuPont, par exemple, possèdent de
nombreux brevets sur des traits de résistance
à leurs herbicides pour des espèces de grande
culture telles que le colza et le tournesol. Les
inventions brevetées consistent essentiellement en des résistances aux herbicides de ces
compagnies,comme dans un OGM classique. Les
brevets décrivent la façon dont ces résistances
sont obtenues. Les revendications des brevets
sont ensuite très larges et portent toujours
sur la séquence d’ADN mutée. Les multinationales de l’agrochimie intègrent ensuite ces
traits brevetés soit dans les variétés qu’elles
sélectionnent elles-mêmes, soit dans celles
des PME de sélection semencière.\\

C’est ce que fait BASF en France avec sa
technologie Clearfield ). Les grandes firmes de l’agrochimie ne
sont pas agressives – BASF percevrait moins de 10\% sur les ventes des variétés Caussade
portant le trait Clearfield. Leurs stratégies consistent avant tout à faire les yeux doux
aux sélectionneurs afin qu’ils insèrent les
traits brevetés dans leurs variétés. C’est
ainsi que Monsanto a conquis le marché
brésilien du soja, non pas en vendant ses
variétés d’OGM, mais en intégrant ses traits
brevetés dans les variétés des semenciers
brésiliens... avant d’en prendre le contrôle.
En somme, les firmes multinationales de
l’agrochimie sont en train de réussir avec les
variétés mutées tolérantes aux herbicides
ce qu’elles n’ont pas pu faire avec les OGM
sur les marchés européens.\\

À côté de ces variétés mutées, de plus en
plus de plantes obtenues par des méthodes
classiques de sélection sont désormais brevetées. La pratique consiste à breveter une
plante par un brevet de produit en décrivant
son procédé d’obtention, même si ce procédé,
essentiellement biologique, est non brevetable
en théorie, selon la directive européenne 98/44.
En d’autres termes, le brevet protège la plante
en tant que telle, décrite par son procédé d’obtention. L’argument consiste à dire que ce n’est
pas parce que la loi interdit de breveter « les
procédés essentiellement biologiques » que
« les produits » qui en sont issus ne sont pas
brevetables, y compris s’il s’agit de variétés
au sens de l’UPOV.\\

Si fallacieux qu’il paraisse, ce raisonnement
est celui que la Grande chambre des recours,
l’organe judiciaire le plus élevé de l’Office européen des brevets, a tenu, le 25 mars 2015,
dans les affaires Brocoli II et Tomate ridée II.
Ces variétés ont été brevetées alors qu’elles
avaient été obtenues par des méthodes classiques, et ce malgré les multiples pressions
des mouvements anti-brevets, rejoints, pour
une fois, par le HCB.\\

Cette expansion du champ d’application
des brevets sur les innovations variétales
est le résultat d’une volonté délibérée de
l’Office européen des brevets de vider de
leur contenu les différentes exclusions des
organismes vivants du champ des brevets
– exclusions initialement inscrites dans les
accords internationaux sur la protection des
droits intellectuels et transcrites dans la
directive européenne 98/44/CE.
La Commission européenne et l’Office européen des brevets présentent souvent ces
évolutions comme inéluctables. Il s’agirait
de suivre la politique américaine en matière
de brevet pour que l’Europe ne décroche pas
complètement en matière de biotechnologies. En fait, l’Office européen des brevets
est aujourd’hui un organisme hors de tout
contrôle démocratique et fortement pénétré
par les lobbies industriels via son Comité
consultatif permanent, essentiellement
composé de représentants de l’industrie.
Il affiche de nombreuses exceptions à la
brevetabilité, notamment d’ordre éthique,
tout en s’évertuant ensuite à les vider de
leur contenu. Il a ainsi « créé dans la société
civile un sentiment de manipulation par le
politique qui, tout en revendiquant une façade
éthique, s’affranchit en pratique de la plus
grande partie des contraintes », constate
Marie-Angèle Hermitte, juriste spécialiste
de ces questions, dans un récent ouvrage.\\

S’orienter vers une déréglementation de
la mise sur le marché des plantes issues
des nouvelles biotechnologies, c’est renforcer encore l’impact de cette économie
des brevets et, par conséquent, accélérer
la remise en cause de notre modèle agricole
européen fondé sur une diversité des types
de variétés présents sur nos marchés. Ceci
relie la question du statut juridique de ces nouvelles biotechnologies à la nouvelle directive
européenne 2015/412. Cette directive permet
désormais aux États membres de l’Union européenne d’interdire la culture de plantes
génétiquement modifiées pour des motifs
autres que sanitaires et environnementaux,
comme les impacts socio-économiques de
ce type de plantes sur les orientations ou les
modèles agricoles de chaque Nation, mais
à condition de montrer que ces mesures ne
sont pas discriminatoires contre les OGM. Or
en pratique, il sera très difficile d’interdire
ces plantes sans avoir de solides preuves de
leurs impacts négatifs sur des orientations
agricoles spécifiques telles que l’agroécologie,
comme l’a montré le colloque international
sur le sujet organisé en octobre 2016 par
le HCB. Si les plantes issues des nouvelles
biotechnologies n’entrent pas dans le champ
d’application de cette nouvelle directive, il
deviendra définitivement illusoire de protéger
nos modèles agricoles contre ces technologies favorisant au contraire les modèles
productivistes.\\


En 2013, le groupe de travail du HCB sur
les biotechnologies végétales et la propriété
industrielle avait alerté les pouvoirs publics
sur les risques de renforcement de l’oligopole de l’agrochimie pour nos orientations
agricoles. À sa suite, le Conseil économique
éthique et social du HCB avait souligné les
dangers des brevets pour la recherche en
amélioration des plantes et l’innovation
variétale, s’inquiétant notamment que « les
programmes de sélection ne [puissent] être
entrepris sans risque, pour le sélectionneur,
d’intégrer à son insu des éléments brevetés ».
Et dans son rapport de 2013, le Commissariat
général à la stratégie cité plus haut avait à son
tour affirmé, en des termes politiques plus
forts encore, que « l’Europe doit lutter contre
l’instrumentalisation du brevet comme outils
de guerre juridique pour bloquer l’innovation,
et ce faisant, la liberté de production ». À la
fin du printemps, le HCB devrait rendre son
rapport final sur les nouvelles techniques
d’obtention des plantes. Souhaitons qu’il se
souvienne de ces conclusions. On ne peut,
d’un côté, mettre en garde contre les brevets
sur les plantes et, de l’autre, prétendre que les
plantes issues des nouvelles biotechnologies
doivent échapper aux réglementations sur
les OGM. Et souhaitons aussi que la société
civile s’empare vite de ces questions : il est
plus que temps de remettre en cause l’emprise
des brevets sur le vivant.

\newpage
\subsection{Quand les écosystèmes saturent}
\label{sec:saturation}

\begin{itemize}
	\item \textbf{Lien : }  \url{https://lejournal.cnrs.fr/articles/le-boom-de-lagriculture-urbaine} 
	\item \textbf{Auteur : } Grégory Fléchet
	\item \textbf{Date : }  19 Octobre 2017
	\item \textbf{Source : } CNRS Le journal, l'objectif de ce site est de partager largement avec les amateurs de science, les professeurs et leurs élèves, les étudiants et tous les citoyens curieux, des contenus destinés jusque-là à la communauté des agents du CNRS, chercheurs, ingénieurs et techniciens, ceux des labos comme ceux des bureaux.
	\item \textbf{Résumé : }
	 La prolifération des algues vertes sur certaines plages bretonnes témoigne d’un phénomène de pollution en pleine recrudescence sur la planète : l’eutrophisation. Une expertise scientifique fait la lumière sur les origines et les conséquences de ce syndrome d’indigestion des milieux naturels. 
\end{itemize}


Le trou dans la couche d’ozone et le dérèglement climatique actuel sont deux illustrations de la capacité de notre espèce à modifier de façon drastique le fonctionnement du système Terre. L’eutrophisation des écosystèmes en est une troisième. S’il reste encore peu familier du grand public, ce phénomène associé au déversement dans les cours d’eau et les nappes phréatiques d’importantes quantités de nitrates et de phosphates n’en témoigne pas moins de l’influence à large échelle des activités humaines sur l’environnement.\\

Du grec eu (bien) et trophein (engraisser, nourrir), l’eutrophisation se caractérise par la perturbation d'un écosystème aquatique dûe à un apport excessif de nutriments. « Ce syndrome peut être assimilé à l’indigestion d’un écosystème ayant emmagasiné tellement de nutriments qu’il n’est plus en mesure de les décomposer par lui-même », résume Gilles Pinay, directeur de l’Observatoire des sciences de l’Univers de Rennes et rapporteur de l’expertise scientifique collective Eutrophisation. Commandée par le ministère de l’Agriculture et de l’Alimentation et le ministère de la Transition écologique et solidaire, cette expertise menée par 45 chercheurs a été rendue publique le 19 septembre 2017. Ses conclusions s’appuient sur l’analyse de plus de 4000 publications scientifiques dans des domaines qui vont de l’écologie à l’hydrologie en passant par la biogéochimie, les sciences sociales, le droit et l’économie.\\

\textbf{Un phénomène mondial en pleine accélération}\\

Si les effets les plus visibles de cette fertilisation non consentie que sont les proliférations ponctuelles d’algues sont décrits depuis l’Antiquité, le phénomène prend véritablement une dimension planétaire vers la fin du XIXe siècle, lorsque commencent à se multiplier les grandes agglomérations et leur cortège de zones industrielles. Directement liée aux rejets dans les fleuves des eaux usées issues de ces vastes aires urbaines, cette pollution sera jugulée de manière efficace dès lors que les villes décideront de s’équiper de stations d’épuration. « La première crise moderne associée à l’eutrophisation remonte aux années 1970, lorsque des milieux naturels emblématiques tels que les Grands Lacs d’Amérique du Nord ou le lac Léman ont vu leur concentration en oxygène diminuer de manière drastique », complète le biogéochimiste du CNRS. La réduction puis l’interdiction des phosphates dans les lessives va à son tour permettre d’enrayer cette première alerte d’eutrophisation.\\


Le répit sera néanmoins de courte durée. Depuis le début du XXIe siècle, une vague d’eutrophisation plus insidieuse se répand à travers le monde : proliférations végétales parfois toxiques, perte de biodiversité, diminution de la concentration d’oxygène pouvant engendrer la mort massive d’organismes aquatiques, comptent parmi les symptômes de cette fertilisation plus diffuse. Le milieu marin semble tout particulièrement affecté par ce nouvel épisode. En l’espace d’une quarantaine d’années, le nombre et l’emprise des zones hypoxiques (à faible concentration d’oxygène) et anoxiques (sans oxygène du tout) y a en effet triplé à l’échelle du globe. Un recensement mené en 2010 a permis d’estimer leur nombre à près de 500 pour une superficie totale au niveau mondial avoisinant la moitié du territoire français.\\

Outre une augmentation de la fréquence et de l’importance du phénomène, on constate par ailleurs une extension des effets de l'eutrophisation à des zones géographiques qui étaient jusqu'ici épargnées comme certains grands lacs d'Afrique de l'Est ou les lagunes méditerranéennes. En France, les proliférations d'algues vertes dans certaines baies de Bretagne sont un problème récurrent depuis les années 1970/1980. « Bien que des scientifiques spécialistes des milieux côtiers et les associations environnementalistes alertent sur la question de l' enrichissement excessif de ces milieux naturels en azote et en phosphore depuis plus de 30 ans, il faudra attendre le milieu des années 2000 et le basculement du traitement médiatique vers les dangers sanitaires associés aux marées vertes (les émanations de sulfure d'hydrogène qui en résultent ont déjà provoqué le décès de plusieurs personnes, NDLR) pour que la sensibilité sociale au problème de l'eutrophisation s’accroisse considérablement », souligne Alix Levain, chargée de recherche en ethnologie et sociologie au Laboratoire interdisciplinaire sciences innovations sociétés.\\

\textbf{L’agriculture intensive pointée du doigt}\\

En raison des engrais chimiques qu’il utilise en abondance pour fertiliser les cultures et des grands volumes d’effluents provenant des élevages industriels, le modèle agricole intensif actuel est régulièrement pointé du doigt lorsqu’on évoque le problème de l’eutrophisation. Les récents épisodes de marées vertes en zone littorale ou les proliférations de microalgues toxiques qui touchent certains lacs de basse altitude en période estivale ont toutefois des origines anciennes, comme le rappelle Gilles Pinay : « Nous avons un legs de près d’un siècle de relargage d’azote et de phosphore dans l’environnement, que les écosystèmes aquatiques ne sont toujours pas parvenus à épurer. » Ces dernières années, la limitation des épandages de lisier en plein champ, la réduction de l’érosion des sols via la plantation de cultures hivernales ou la promotion de pratiques agricoles moins gourmandes en engrais chimiques furent autant de mesures prises au niveau européen dans le but de réduire l’impact de l’eutrophisation sur les écosystèmes aquatiques. En dépit de ces efforts, les bénéfices pour ces milieux naturels demeurent malheureusement limités. « En tenant compte de ces différentes mesures, nous avions prédit que les relargages d’azote dans l’environnement diminueraient de manière drastique à partir de 2012, indique Gilles Pinay. Or, on sait aujourd’hui que le temps de résidence de cet élément chimique dans certains milieux comme les petits cours d’eau avait été largement sous-estimé par les études scientifiques. » Un travail expérimental au long cours auquel a contribué le chercheur du CNRS a depuis démontré que les sols mettaient non pas trois ou quatre années pour digérer l’azote issu des engrais agricoles, comme on le supposait jusqu’ici, mais plusieurs décennies.

\textbf{Des sites plus sensibles que d’autres}\\

La sensibilité à l’eutrophisation varie en outre beaucoup selon les écosystèmes. En ne laissant pas le temps à la matière organique de se transformer sur place, les torrents sont par exemple peu sensibles au phénomène, à la différence des lacs situés à l’aval de ces mêmes rivières, au sein desquels le temps de séjour des eaux qui les alimentent est parfois très long.\\

La nature des sols et leurs usages, la structure à la fois géographique et géologique d’un bassin-versant qui alimente un fleuve sont autant de facteurs qui entrent également en ligne de compte. La présence de puissants tourbillons au large de certaines zones côtières y favorise quant à elle l’accumulation des éléments nutritifs.\\

En Bretagne, cela explique notamment pourquoi les marées vertes se manifestent toujours dans les mêmes baies d’une année sur l’autre. Des milieux naturels a priori semblables étant ainsi susceptibles de réagir différemment à un apport excessif de nutriments, une analyse au cas par cas est bien souvent nécessaire pour déterminer à partir de quel niveau de concentration un écosystème a de fortes chances d’être affecté par l’eutrophisation. « La réglementation européenne qui définit les niveaux maximums de nitrate et de phosphore autorisés dans l’eau potable (respectivement fixés à  50 mg/litre et 5 mg/litre par Bruxelles, NDLR) n’est pas pertinente lorsqu’il s’agit de faire face au problème de l’eutrophisation, certains biotopes subissant les effets négatifs du phénomène pour des concentrations qui peuvent être dix fois inférieures à ces seuils réglementaires », précise Gilles Pinay.\\

\textbf{Des symptômes aggravés par le réchauffement climatique}\\

Dans le contexte du changement climatique global, parvenir à identifier les écosystèmes aquatiques les plus sensibles à l’accroissement de ces flux d’éléments nutritifs s’avère plus que jamais primordial pour lutter contre l’eutrophisation. Parce qu’elle devrait stimuler la production de biomasse végétale tout en diminuant la concentration d’oxygène dissous dans l’eau, l’élévation progressive des températures risque en effet d’amplifier les symptômes actuels de l’eutrophisation des milieux aquatiques. En raison d’épisodes pluvieux dont l’intensité devrait aller crescendo dans les prochaines décennies, une plus grande prévalence des crues et donc des phénomènes d’érosion risque en outre d’accentuer l’exportation de sédiments riches en azote et en phosphore vers ces mêmes écosystèmes. De récentes modélisations de ces flux de nutriments menées à l’échelon mondial montrent d’ailleurs que les quantités parvenant jusqu’aux océans ont déjà presque doublé au cours du XXe siècle, passant de 34 à 64 millions de tonnes par an pour l’azote et de 5 à 9 millions de tonnes par an pour le phosphore. Or il faut savoir que l’agriculture contribue désormais à plus de la moitié de ces flux de matières nutritives. « L'enjeu est désormais de quantifier le temps de réponse des écosystèmes aux apports d'azote et de phosphore afin de mieux déterminer des objectifs réalistes de reconquête de la qualité de l'eau, explique Gilles Pinay. Pour cela il va nous falloir intensifier fortement le suivi de la qualité de l’eau pour en décrypter les variations. 



\newpage
\section{Textes à débattre}
\subsection{Les OGM Bt c'est moins d'insecticides, c'est bio ! }
\label{sec:Bt}

\begin{itemize}
	\item \textbf{Lien : }  \url{https://blogs.mediapart.fr/pierre-yves-morvan/blog/080119/les-ogm-bt-cest-moins-dinsecticides-cest-bio} 
	\item \textbf{Auteur : } Pierre Yves Morvan
	\item \textbf{Date : }  8 janvier 2019
	\item \textbf{Source : } Blog Mediapart : Pour une écologie réaliste - Pour vraiment nourrir toute l'humanité à venir, pour vraiment sortir du réchauffement climatique
\end{itemize}

La communauté scientifique s’accorde pour dire que l’utilisation de cultures transgéniques Bt résistantes aux insectes contribue à réduire le volume et la fréquence de l’utilisation d’insecticides sur les cultures de maïs, de coton et de soja (CIUS). Ce résultat a été particulièrement significatif pour la culture du coton en Afrique du Sud, en Australie, en Chine, aux États-Unis et au Mexique.\\

Les OGM Bt diminuent l'utilisation des pesticides, mais ne les suppriment pas totalement, en raison même de l'une de leurs qualités, leur extrême sélectivité. Contrairement au pyrèthre bio qui tire sur tout ce qui bouge – même les abeilles – chaque OGM Bt cible son principal ravageur, celui-là seul, sans nuire aux insectes utiles ni à d’autres ravageurs éventuels moins dangereux, des seconds couteaux en quelque sorte. C’est pourquoi on utilise encore des insecticides classiques dans les cultures OGM, pour contrôler les seconds couteaux. Mais le résultat fondamental est que les OGM résistant aux insectes permettent de réduire globalement l’utilisation des insecticides. D'autant que la technique évolue rapidement, on produit maintenant des OGM résistant à plus d'un agresseur – et toujours seulement aux agresseurs, jamais aux insectes utiles. \\

L'adoption à grande échelle d’une culture Bt, en favorisant indirectement l'abondance des prédateurs généralistes dans les champs Bt (via un usage moins intensif d’insecticides), pourrait aider à restaurer un service écosystémique clé pour une agriculture durable, la régulation biologique naturelle par la faune auxiliaire.\\

Les OGM Bt, c'est moins d'insecticides, c'est bio !
Paradoxalement, le bio qui veut moins d'insecticides refuse les OGM Bt...\\

Les chercheurs ne débattent plus aujourd’hui sur le point de savoir si les OGM Bt résistant aux insectes permettent de réduire l’utilisation des insecticides ; ils le permettent.\\

Mais quelques militants ont du mal à avouer que l'on puisse accorder des bienfaits écologiques aux OGM. Ils ont trouvé l'astuce : "OK, disent-ils, les agriculteurs utilisent moins d'insecticides sur les cultures Bt ; mais il faut tout compter, et donc compter aussi les insecticides fabriqués par la plante OGM Bt". C'est astucieux, mais c'est oublier que les OGM Bt ne font que copier ce que la nature fait depuis des millénaires – en mieux : mettre des pesticides dans les plantes. Si donc ont fait les comptes de cette façon, il faut alors considérer aussi qu'un champ de pommes de terre bio est un champ gorgé de pesticide, en raison de la solanine produite naturellement par les pommes de terre. La différence importante est que la solanine de la pomme de terre bio est un glycoalcaloïde poison pour l'homme, alors que la protéine Bt du maïs OGM est inoffensive pour l'homme.\\

Limiter la pollution diffuse des phytosanitaires est une bonne chose.
Pour cela, le meilleur moyen est de ne pas limiter l'utilisation des OGM Bt.
En plus, l'agriculture bio en profitera (et on ne la taxera pas pour ce bénéfice).


\end{document}



