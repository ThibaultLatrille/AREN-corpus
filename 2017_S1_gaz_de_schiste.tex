\documentclass[8pt]{article}

\usepackage[T1]{fontenc}
\usepackage[utf8]{inputenc}
\usepackage{graphicx}
\usepackage{lmodern}
\usepackage{amsmath}
\usepackage{xfrac}
\usepackage{amsthm}
\usepackage{listings}
\usepackage{enumerate}
\usepackage{amssymb}
\usepackage{cancel}
\usepackage{amsfonts}
\usepackage{float}
\usepackage{fullpage}

 
\DeclareUnicodeCharacter{200A}{ } 
\renewcommand*\contentsname{Table des matières}

\PassOptionsToPackage{hyphens}{url}\usepackage{hyperref}

\usepackage{listings}
\author{ControverSciences}
\title{Projet AREN : gaz de schiste}
\date{11 janvier 2017}

\begin{document}
\maketitle

\tableofcontents

\newpage
\section{Textes à débattre}
\subsection{Gaz de schiste : oui mais pas chez nous !}
\begin{itemize}
	\item \textbf{Lien : }  \url{http://www.desideespourdemain.fr/index.php/post/2016/01/21/Gaz-de-schiste-\%3A-oui-mais-pas-chez-nous-!} 
	\item \textbf{Auteur : }  Marc-Antoine Authier
	\item \textbf{Date : } 21 janvier 2016 
	\item \textbf{Source : }  L'Institut Montaigne se reclame comme un think tank indépendant. Sa vocation est d'élaborer des propositions concrètes dans les domaines de l'action publique, de la cohésion sociale, de la compétitivité et des finances publiques.

\end{itemize}
Depuis 40 ans, aucun méthanier n’a quitté les côtes américaines pour exporter la production nationale de gaz. Jusqu’à aujourd’hui, le gaz était en effet considéré comme ressource stratégique par la Maison Blanche, et donc réservé au marché domestique, depuis le choc pétrolier de 1973 et la forte hausse du prix du pétrole. Cependant,  au milieu des années 2000, le gaz de schiste a fait son apparition dans le mix énergétique américain et la production gazière est rapidement devenue excédentaire, à tel point que le gouvernement vient de rétablir  l’autorisation d’exporter. Les méthaniers peuvent donc reprendre la mer.\\

L’Europe est la destination privilégiée de ces exportations. Et pour cause : le prix spot*, soit le prix basé sur les transactions actuelles et non déterminés par des contrats de long terme, est deux fois supérieur sur le Vieux Continent. Son niveau structurellement inférieur outre-Atlantique est dû à l’absence de coûts de liquéfaction, de transport et de regazéification dans la chaîne d’approvisionnement. Ainsi, tant que l’Europe importera du gaz en provenance des Etats-Unis, les prix de cette ressource seront deux à trois supérieurs à ceux que pourraient offrir une production nationale.\\

Dans son rapport Gaz de schiste : comment avancer, l’Institut Montaigne propose d’envisager les opportunités liées à l’exploitation du gaz de schiste en France. En effet, le statu quo n’est pas une option de politique énergétique, puisque cette ressource est déjà une réalité au niveau international. Sur le plan économique, cela revient à se priver d’une précieuse ressource nationale qui permettrait d’améliorer notre balance commerciale. Sur le plan environnemental, cela permet une moindre dépendance au pétrole et donc une baisse de nos émissions de CO2. Pourtant, le Tribunal Administratif de Cergy-Pontoise a récemment confirmé l’abrogation des permis d’exploitation du gaz de schiste. Il est donc peu probable que la France avance sur le sujet. Paradoxalement, il est fort possible qu’elle importe du gaz de schiste en provenance des Etats-Unis.

\newpage
\section{Textes à débattre}
\subsection{Oui, les gaz de schiste peuvent contribuer au redressement productif !}
\begin{itemize}
	\item \textbf{Lien : }  \url{http://www.euro-petrole.com/oui-les-gaz-de-schiste-peuvent-contribuer-au-redressement-productif-!-n-f-6583} 
	\item \textbf{Auteur : } Amicale des Foreurs
	\item \textbf{Date : } 14/09/2012
	\item \textbf{Source : } Amicale des Foreurs se réclame comme indépendante financièrement car gérée uniquement par des bénévoles entièrement dévoués à la cause de ses adhérents, qui la font vivre, et à leur place dans l’industrie pétrolière.
\end{itemize}


Réindustrialisation et redressement productif : deux mots magiques déclamés par le nouveau gouvernement à cor et à cri face à une situation économique française sinistrée et une industrie exsangue, dans des domaines aussi stratégiques que sont la sidérurgie, la chimie, la pétrochimie, le bois, le cuir… Pire encore, des fleurons de l’industrie française comme les chantiers navals, l’automobile et les forages pétroliers se retrouvent fragilisés par la crise et certains choix stratégiques hasardeux.\\

Le débat – ou plutôt le non débat – autour des gaz de schiste est selon nous symptomatique de cette situation.\\

En effet, comment comprendre l’inflexibilité du gouvernement sur la question des gaz de schiste alors que le déficit commercial atteint près de 70 milliards d’euros sur les 12 derniers mois (dont 90\% liés à l’importation d’hydrocarbures), que le prix du gaz et le prix à la pompe ne cessent d’augmenter et que le chômage concerne désormais plus de 3 millions de personnes ?

On ne présente plus l’exemple américain, où la production des hydrocarbures de schiste a permis la création de centaines de milliers d’emplois, a divisé le prix du gaz par deux et a redynamisé toute une industrie moteur de croissance. En pleine campagne électorale, Barack Obama ne se prive d’ailleurs pas de rappeler cette « success story » industrielle lors de ses meetings… A une échelle certes différente, peut-on réellement se permettre de fermer les yeux sur un tel potentiel ? Comme le disait récemment Claude Allègre, c’est comme si on avait un trésor dans son jardin et qu’on refusait de le déterrer.\\

Nous avions l’année dernière regretté le vote d’une loi par l’ancienne majorité qui interdisait la technique de fracturation hydraulique pour les forages pétroliers, tout en l’autorisant pour la géothermie. Cette curiosité législative, constitutionnellement douteuse, avait conduit à l’annulation de trois permis d’exploration dans le sud (au mépris même du principe de sécurité juridique indispensable à l’attractivité d’un pays vis-à-vis des investisseurs) et avait surtout fermé la porte à toute réflexion sur l’opportunité qu’il y aurait à produire, proprement, les hydrocarbures de schiste.\\

Les réserves en gaz et en pétrole sont pourtant potentiellement élevées : l’Agence Internationale de l’Energie estime à cinq mille milliards de m3 les réserves en gaz de schiste, soit 90 ans de notre consommation annuelle, tandis que l’Institut Français du Pétrole estime à environ sept milliards de m3 les ressources en pétrole de roche-mère du Bassin Parisien.\\

Certes, et tous les foreurs le savent bien, des réserves potentielles ne sont pas forcément des réserves avérées : la recherche et l’exploration aboutissent parfois à des mauvaises surprises, comme elles peuvent également déboucher sur de bonnes. Quel risque prendrait-on à autoriser la recherche et l’évaluation des réserves disponibles, notamment dans le bassin parisien où la géologie est déjà connue en raison d’une histoire pétrolière de plus de 50 ans ? S’il s’avérait que les ressources effectivement présentes dans notre sous-sol n’étaient pas aussi importantes que prévues, au moins serions-nous fixés !\\

La question de l’exploitation se pose différemment. Elle exige en effet un débat et une analyse scientifique des techniques disponibles, au premier rang desquels la désormais interdite fracturation hydraulique, utilisée dans l’industrie pétrolière depuis plus de 30 ans ! Ne fermons pas la porte à cette technique sous la pression de certains lobbies alors même qu’il a été démontré que les accidents industriels liés à l’exploitation des gaz de schiste aux Etats-Unis n’étaient pas dus à l’utilisation de cette technique, mais bien à une mauvaise cimentation des puits et à des fuites de conteneurs en surface.\\

Symptomatique de notre situation économique et industrielle, cette interdiction témoigne aussi du nouveau rapport au risque que nous entretenons. On entend encore la ministre de l’écologie affirmer son opposition à la fracturation hydraulique au motif que « nulle part au monde il n’a été prouvé qu’elle était sans risque pour l’environnement ». Mais quelle activité industrielle – et même humaine, tout simplement – peut se targuer d’être absolument inoffensive ? Doit-on par exemple interdire la finance parce que certaines pratiques entraînent des risques pour l’économie réelle, comme l’a prouvée la crise des subprimes de 2008 ? Évidemment non, car tout est question de contrôle et de régulation. C’est le chemin choisi par l’Afrique du Sud qui vient récemment de remettre en cause l’interdiction de la fracturation hydraulique et de créer un « comité de contrôle » qui sera chargé de superviser les opérations de forage.\\

Il ne s’agit pas de sacrifier l’environnement sur l’autel de l’économie et de la croissance, mais bien de concilier les deux exigences. \\

« Choisissons la voie de la régulation », écrivions-nous l’année dernière dans une tribune du journal « Le Monde ». Peut-être avions-nous été trop ambitieux et pas assez pragmatiques. Si le gouvernement (et toutes les parties prenantes de ce dossier) acceptait d’autoriser la recherche et l’évaluation des réserves, ce serait déjà un formidable pas en avant – il sera possible ensuite d’en tirer les conséquences et de décider de l’autorisation ou non de la production des hydrocarbures de schiste.

\newpage
\section{Corpus de ressources}
\subsection{Le gaz, énergie de la transition ?}

\begin{itemize}
	\item \textbf{Lien : }  \url{https://lejournal.cnrs.fr/articles/le-gaz-energie-de-la-transition} 
	\item \textbf{Auteur : } Yaroslav Pigenetr 
	\item \textbf{Date : } 29 juillet 2014
	\item \textbf{Description : } À l’heure de la transition énergétique, il est devenu nécessaire de disposer d’une source d’énergie stockable assurant le relais des énergies renouvelables intermittentes. Ainsi, certains présentent le XXIe siècle comme l’âge d’or du gaz, le considérant comme la moins nocive des énergies fossiles.
	\item \textbf{Source : } CNRS Le journal, l'objectif de ce site est de partager largement avec les amateurs de science, les professeurs et leurs élèves, les étudiants et tous les citoyens curieux, des contenus destinés jusque-là à la communauté des agents du CNRS, chercheurs, ingénieurs et techniciens, ceux des labos comme ceux des bureaux.
\end{itemize}

Que peuvent avoir en commun le P-DG de l’équipementier Vallourec, Greenpeace Allemagne et l’Agence internationale de l’énergie (AIE) ? Tous voient dans le méthane, plus connu sous le nom de gaz naturel, si ce n’est la panacée, du moins la source d’énergie incontournable de la transition énergétique. Pourtant, ce gaz partage avec le pétrole ou le charbon le double défaut des énergies fossiles, être épuisable et générateur de gaz à effet de serre. « Il est important de resituer le problème du développement de nouvelles énergies dans le contexte de la double contrainte du développement énergétique durable à long terme, lié à la lutte contre le changement climatique, et de la raréfaction des sources d’énergie conventionnelles, affirme l’économiste Patrick Criqui, du laboratoire Pacte-Edden. Or, malgré la montée en puissance des sources renouvelables telles que l’éolien, le solaire ou la biomasse, il n’existe à court et à moyen terme aucune solution de substitution totale aux énergies fossiles. » Dès lors, de nombreux spécialistes s’accordent à considérer le gaz, qu’il soit d’origine conventionnelle ou non, comme l’énergie de la transition. Reste à savoir quelle place il occupera dans la loi de transition énergétique qui a été présentée par Ségolène Royal au Conseil des ministres le 30 juillet et qui sera débattue au Parlement à l’automne. \\

\textbf{L’énergie fossile la moins polluante}

Parce qu’il est relativement facile à stocker et à distribuer, le gaz est une source d’énergie polyvalente et mobilisable à tout moment, ce qui en fait aussi le relais idéal des énergies éoliennes et solaires, par nature intermittentes, et qu’il peut pallier, notamment lors des pics de consommation électrique. « Actuellement, dans les grands pays industriels, le solaire photovoltaïque et l’éolien ne peuvent à eux seuls assurer une production électrique répondant de manière satisfaisante à la stabilité ou, au contraire, aux fluctuations de la demande, précise Alain Dollet, directeur adjoint scientifique de l’Institut des sciences de l’ingénierie et des systèmes, chargé de la cellule Énergie du CNRS. Tant que des procédés efficients de stockage de l’électricité ne seront pas déployés au sein de réseaux plus “intelligents”, on aura recours aux centrales thermiques classiques, et, si l’on n’y brûle pas du gaz, ce sera du pétrole, ou pire, du charbon. » Or le gaz naturel est compa­rativement la moins polluante des énergies fossiles : tandis que 1 kWh produit avec du méthane n’émet que 400 grammes de CO2, le pétrole et le charbon en émettent respectivement 600 et 800 grammes, en plus des suies et des particules fines. \\

\textbf{De nouveaux gisements découverts}

Par ailleurs, même si les réserves de gaz naturel fossile sont par essence limitées, la découverte et l’exploitation aux États-Unis de gisements non conventionnels – ce que l’on appelle improprement le gaz « de schiste » – a repoussé l’échéance de l’épuisement des réserves. « Du point de vue de l’utilisateur final, rien ne distingue un gaz conventionnel d’un gaz non conventionnel : il s’agit du même gaz naturel, insiste Alain Dollet. C’est l’endroit où réside le gaz et les techniques spécifiques employées pour l’en extraire qui déterminent sa nature conventionnelle ou pas. » \\

En effet, conventionnel ou non, ce qui sort du puits est toujours du méthane. Un gaz produit naturellement par transformation progressive de la matière organique emprisonnée dans une roche-mère. La particularité des gaz non conventionnels est qu’ils nécessitent le recours à des procédés plus agressifs, plus élaborés, et donc plus coûteux, pour aller récupérer le gaz présent dans des zones jusque-là inexploitées car non rentables. Les spécialistes identifient ainsi trois types de gaz non conventionnels en fonction de la couche géologique dont ils sont extraits. D’une part, le gaz de roche-mère, habituellement, et abusivement, appelé gaz « de schiste », qui, comme son nom l’indique, est resté prisonnier de sa roche-mère minérale et que l’on va libérer en fracturant cette dernière. Ensuite, le gaz de réservoir compact, un gaz qui a certes migré depuis sa roche-mère mais qui est ensuite retenu prisonnier dans une roche-réservoir qu’il faut fracturer ou forer horizontalement pour l’en extraire. Enfin, le gaz de charbon, ou gaz de couche, improprement appelé gaz « de houille », qui est un gaz de roche-mère naturellement présent dans tous les filons de houille, la roche-mère étant cette fois du charbon. C’est ce même gaz qui, lorsqu’il s’échappe spontanément du charbon et s’accumule dans les galeries de mines, provoque les coups de grisou. \\


La tranformation en méthane commence donc par le dépôt et l’accumulation de déchets organiques (végétaux terrestres, plancton marin, etc.) qui vont former une boue sur laquelle vont s’empiler de nouvelles couches géologiques. « Durant l’enfouissement, cette matière organique est piégée dans ce qu’on appelle la roche-mère, explique Bruno Goffé, du Cerege. Sous l’effet de la pression et de la température, elle va progressivement se transformer en charbon, s’il s’agit de sédiments terrestres ou, s’ils sont d’origine marine, en un résidu organique, le kérogène, puis en huile et enfin en gaz. » À ce stade, une partie du gaz va s’en échapper et remonter par poussée d’Archimède jusqu’à être retenue dans un piège géologique, la roche-réservoir, où le méthane va alors s’accumuler. \\

C’est de ces gisements situés dans des roches-réservoirs qu’on extrait le gaz naturel conventionnel à l’aide de forages verticaux. Le gaz non conventionnel résidant dans la roche-mère ou dans des roches particulièrement difficiles d’accès. Les géologues ont calculé que de 10 à 40\% des hydrocarbures générés dans la roche-mère en demeurent prisonniers. En outre, seule une petite fraction des hydrocarbures expulsés finit piégée dans des roches-réservoirs. On sait donc que les roches-mères recèlent un gisement potentiel du même ordre de grandeur que celui de tous les gisements conventionnels exploités depuis les débuts de l’industrie gazière et restant à découvrir ! Au point que, dans un de ses documents publié en 2011, intitulé « Entrons-nous dans l’âge d’or du gaz ? », l’AIE estimait probable un scénario où la consommation de gaz augmenterait de 50\% d’ici à 2035, représentant alors plus d’un quart de notre consommation d’énergie totale. \\

\textbf{Une évaluation discutable des réserves}

L’optimisme de l’AIE est toutefois loin de faire l’unanimité, notamment en ce qui concerne l’estimation des réserves de gaz exploitables à un coût économiquement, socialement et environnementalement tolérable. Selon Bruno Goffé, « l’évaluation actuelle des réserves fait débat, car, en fait, on ne connaît pas grand-chose de cellec-ci : en Pologne, par exemple, où elles étaient estimées entre 30 et 440 ans de consommation nationale, certains opérateurs, comme Exxon, se sont retirés, signifiant sans doute que la ressource est moins abondante que prévu ou plus difficile à extraire ». \\

Il n’en demeure pas moins que, dès les années 1970, face à l’épuisement rapide des réserves conventionnelles et motivée par l’envol des prix, l’industrie gazière a adapté et développé des techniques lui permettant de libérer et de puiser ne serait-ce qu’une infime fraction de cette ressource non conventionnelle de gaz. C’est ainsi que, dans les années 2000, la combinaison de la maîtrise des forages horizontaux et de l’amélioration de la technique de la fracturation hydraulique a permis la première exploitation économique du gaz de roche-mère.

\textbf{Les sales débuts du fracking}

« La fracturation hydraulique a été inventée en 1949. Plus d’un million de puits ont déjà été fracturés pour toutes sortes d’usages, pas seulement pour les gaz non conventionnels, mais aussi pour la géothermie, pour le pétrole conventionnel, pour l’eau », rappelle Bruno Goffé. La mise en œuvre de cette méthode supposée éprouvée s’est toutefois faite dans la douleur. « Il est vrai que la première vague de forages, qui a conduit au boom des gaz “de schiste” aux États-Unis, s’est effectuée la plupart du temps sans réelle évaluation des risques, sans concertation avec la population et hors de toute régulation », regrette Patrick Criqui. \\


Résultat : elle s’est soldée par des dégâts environnementaux et humains importants, qui expliquent aujourd’hui l’intense méfiance ressentie, notamment en Europe, à l’encontre de l’exploitation des gaz de roche-mère. Et ce d’autant plus que, contrairement aux États-Unis où les propriétaires de terrains possédant également les ressources de leur sous-sol et retirent donc un profit des forages, dans la plupart des pays, l’État est propriétaire du sous-sol, et donc seul bénéficiaire de son exploitation, via les concessions. « On se retrouve à exploiter des gisements qui sont au pied des citoyens, ces derniers constatant bien qu’ils sont exposés aux désagréments de l’exploitation sans en retirer aucun bénéfice financier immédiat », relève Normand Mousseau, professeur au département de physique de l’université de Montréal et auteur de plusieurs livres et articles sur les gaz non conventionnels.\\

\textbf{Les principaux risques identifiés}

« Il faut reconnaître que, grâce aux déboires des exploitants américains, on sait maintenant ce qu’il ne faut pas faire », remarque Bruno Goffé. Car, quelle que soit la position de chacun sur l’opportunité d’exploiter les gaz non conventionnels, un relatif consensus existe désormais quant aux principaux risques et impacts socio-environnementaux liés à cette activité : contamination des nappes d’eau souterraines par les hydrocarbures et les additifs chimiques des boues de fracturation en raison de fuites dans les puits ; pollution des sols à la suite de mauvais retraitements, voire de l’épandage illégal des liquides de fracturation usés ; emprise sur les paysages due au ballet des camions de chantier et à la multiplication des puits dont les rendements chutent rapidement ; consommation excessive d’eau durant la phase de fracturation, au détriment des usages résidentiels et agricoles ; émission de gaz à effet de serre, notamment de ­méthane, lors de l’exploitation. « On sait comment prévenir ces risques ou y remédier par la mise en place de bonnes pratiques comme le retraitement et le recyclage des liquides de fracturation, note Alain Dollet. Bien sûr, ces bonnes pratiques ont un coût parfois élevé, qui pourrait s’avérer dissuasif pour d’éventuels exploitants, en particulier en Europe. » \\


\textbf{La nécessaire diversification des sources d’énergie}

De fait, la plupart des experts estiment que, même si les Européens se lançaient dans l’exploitation de gaz de roche-mère, le coût de production de ce gaz serait équivalent, voire supérieur, aux cours internationaux. « En ce qui concerne l’exploitation des gaz non conventionnels, pour des raisons liées autant aux conditions géologiques qu’aux spécificités de l’industrie pétrolière américaine, la transposition du modèle états-unien au reste du monde est peu probable, précise Patrick Criqui. Il n’empêche que, avec des ressources conventionnelles en voie d’épuisement et un gaz payé ­aujourd’hui deux à trois fois plus cher qu’aux États-Unis, la question se pose d’un approvisionnement gazier de l’Europe permettant de concilier sécurité de l’approvisionnement et compétitivité de l’industrie. » Car, selon cet économiste, l’Europe aura de toute façon besoin du gaz pour satisfaire la demande électrique sans faire la part trop belle à la plus sale mais la moins chère des énergies fossiles : le charbon. La récente décision du gouvernement allemand d’autoriser la prospection des gaz de roche-mère en vue d’une exploitation s’inscrit dans ce compromis. « Le gaz n’est ni la panacée ni le cauchemar de la transition énergétique, mais il est appelé à jouer un rôle important dans la diversification des sources d’énergie qui permettront d’assurer cette transition », résume Alain Dollet. \\



\newpage
\subsection{Du gaz en France}
 
\begin{itemize}
	\item \textbf{Lien : }  \url{https://lejournal.cnrs.fr/articles/du-gaz-en-france} 
	\item \textbf{Auteur : } Laure Cailloce
	\item \textbf{Date : } 23 juillet 2014
	\item \textbf{Description : } Il n’y pas que du gaz de schiste, aujourd’hui interdit d’exploitation, dans le sous-sol français. Le gaz de charbon, présent dans les anciens bassins miniers de Lorraine et du Nord-Pas de-Calais, est une piste prometteuse.
	\item \textbf{Source : } CNRS Le journal, l'objectif de ce site est de partager largement avec les amateurs de science, les professeurs et leurs élèves, les étudiants et tous les citoyens curieux, des contenus destinés jusque-là à la communauté des agents du CNRS, chercheurs, ingénieurs et techniciens, ceux des labos comme ceux des bureaux.
\end{itemize}

L’équivalent d’une dizaine d’années de consommation nationale de gaz, soit 370 milliards de mètres cubes : c’est la quantité de gaz de charbon que recèleraient les sous-sols de Lorraine et du Nord-Pas-de-Calais, selon des estimations confirmées par l’Institut français du pétrole. Plus que ce qu’a fourni le gisement de gaz naturel de Lacq, définitivement fermé en novembre 2013, en cinquante ans d’exploitation. Ce gaz made in France, composé à plus de 90\% de méthane, fait aujourd’hui l’objet de toutes les attentions. Il représente la dernière chance de produire du gaz en France depuis que la fracturation hydraulique y a été interdite. De quoi contribuer – à la marge – à l’indépendance énergétique de la France, qui a dû importer la quasi-totalité de son gaz en 2013, et assurer la transition énergétique vers un modèle intégrant davantage d’énergies renouvelables. Un permis d’exploration a été accordé par l’État à l’entreprise australienne EGL (European Gas Limited) afin de déterminer la faisabilité et les conditions d’une possible exploitation. \\

\textbf{Stimuler plutôt que fracturer}

Le gaz de charbon n’a pourtant rien d’une nouveauté. « Les anciens mineurs de fond le connaissent bien, témoigne Raymond Michels, géochimiste au laboratoire Géoressources. Et pour cause : c’est le grisou tant redouté du temps où l’on exploitait le charbon. Le gaz, emprisonné dans la structure même du charbon, se libérait de façon inopinée lors du creusement des galeries… » Aujourd’hui que les mines sont fermées, l’idée est d’aller chercher de façon systématique cette ressource naturelle dans les couches de charbon les plus profondes, qui n’ont jamais été exploitées par les sociétés minières du fait de leur difficulté d’accès : généralement plus d’un kilomètre sous terre.  \\

Pour ce faire, une technique – inédite en Europe – a été mise au point en Amérique du Nord et en Australie, où le gaz de charbon fait déjà l’objet d’une exploitation : la stimulation. Rien à voir avec la fracturation hydraulique utilisée pour l’extraction du gaz de schiste, assurent les chercheurs. Dans le cas de la fracturation, on envoie de grandes quantités d’eau (et les additifs appropriés) afin de créer une surpression et de fracturer la roche dans laquelle le gaz est emprisonné. Dans le cas du gaz de charbon, il s’agit au contraire de créer une dépressurisation. « On pompe l’eau naturellement présente dans la roche, et le déficit de pression ainsi créé force le gaz hors des microfissures du charbon », explique Raymond Michels.  \\


Ce n’est pas la seule innovation. Une technique de forage directement issue de l’industrie pétrolière – le forage horizontal – devrait également être utilisée. « À partir d’un puits vertical, on creuse en étoile des forages horizontaux qui suivent les couches de charbon », précise Raymond Michels. Avantage de la technique : exploiter au mieux la ressource et de limiter le nombre de puits en surface – donc les nuisances liées à l’activité. Au total, la société EGL, qui a déjà creusé cinq puits de reconnaissance en Lorraine, estime à trente le nombre de sites de production qui fonctionneront à terme dans la région pour un début d’exploitation envisagé d’ici à trois ans. Aucun forage n’a été à ce jour réalisé par EGL dans le Nord-Pas-de-Calais, où l’on estime le gisement à deux années de consommation nationale de gaz.  \\

\textbf{Pas d’exploitation avant dix ans}

« On est encore dans la phase exploratoire, nuance Yann Gunzburger, chercheur au laboratoire Géoressources et coordinateur du projet GazHouille, un groupement pluridisciplinaire de chercheurs (géologues, économistes, juristes, psychosociologues…) chargé d’évaluer les risques et les enjeux d’une exploitation du gaz de charbon en Lorraine. Il ne faut pas préjuger de la décision qui sera prise in fine par les pouvoirs publics. En tout état de cause, l’exploitation, si elle est autorisée et réalisable, ne devrait pas commencer avant cinq à dix ans. » Les premières enquêtes menées auprès de la population montrent de la curiosité pour le gaz de charbon et peu de réticence a priori, d’autant que le scénario aujourd’hui privilégié pour sa commercialisation serait en faveur de l’économie lorraine. « Au lieu d’injecter le gaz dans le réseau national, où il serait vendu au prix du marché, il s’agirait de le commercialiser à un coût moindre aux industriels installés localement », explique Yann Gunzburger. De quoi attirer de nouvelles entreprises dans une région fortement touchée par le chômage, espèrent les plus optimistes.



\newpage
\subsection{Le gaz de schiste, risque ou opportunité ?}
 
\begin{itemize}
	\item \textbf{Lien : }  \url{https://www.youtube.com/watch?v=OPOWHjl9flY} 
	\item \textbf{Auteur : } Commission européenne
	\item \textbf{Date : } 21 janvier 2014
	\item \textbf{Source : } Commission européenne
\end{itemize}
\textbf{Environnement: la Commission européenne préconise des principes minimaux applicables au gaz de schiste}

La Commission européenne a adopté ce jour une recommandation visant à garantir la mise en place de mesures appropriées en matière de protection de l'environnement et du climat en ce qui concerne la technique de fracturation hydraulique à grand volume (fracking) utilisée notamment dans l'exploitation du gaz de schiste. Cette recommandation devrait aider tous les États membres désireux de recourir à cette technique à gérer les risques environnementaux et sanitaires et à accroître la transparence à l'égard des citoyens. Elle introduit également des règles du jeu équitables pour le secteur et offre un cadre plus clair aux investisseurs. \\

La recommandation s'accompagne d’une communication qui examine les possibilités et les problèmes associés à la fracturation hydraulique à grand volume appliquée à l’extraction des hydrocarbures. Ces deux documents s'inscrivent dans le cadre plus global d’une initiative de la Commission visant à mettre en place un cadre d'action dans les domaines du climat et de l'énergie à l'horizon 2030. \\

M. Janez Potočnik, commissaire européen chargé de l'environnement, a déclaré à ce propos: «Le gaz de schiste suscite des espoirs dans certaines régions d'Europe, mais également des inquiétudes. La Commission répond aux demandes d’action en formulant des principes minimaux que les États membres sont invités à suivre afin de tenir compte des aspects environnementaux et sanitaires et de donner aux exploitants et aux investisseurs la prévisibilité nécessaire.» \\

La recommandation adoptée, qui se fonde sur la législation en vigueur de l’Union européenne et qui la complète en tant que de besoin, invite en particulier les États membres: 

- à planifier les projets et à évaluer les possibles effets cumulatifs avant de délivrer des autorisations;\\
- à évaluer rigoureusement les incidences sur l’environnement et les risques associés;
- à veiller à ce que l’intégrité du puits corresponde aux meilleures pratiques;
- à contrôler la qualité de l’eau, de l’air, des sols au niveau local avant le début des activités, afin de détecter d'éventuels changements et de parer aux risques émergents;
- à limiter les émissions atmosphériques, y compris les émissions de gaz à effet de serre, par le captage du gaz;
- à informer le public des produits chimiques utilisés dans les différents puits, et
- à veiller à ce que les exploitants appliquent les bonnes pratiques pendant toute la durée du projet.
La Commission continuera à faciliter les échanges d'informations avec les États membres, l’industrie et les organisations de la société civile en ce qui concerne la performance environnementale des projets relatifs au gaz de schiste. \\

\textbf{Étapes suivantes}
Les États membres de l’Union sont invités à appliquer les principes formulés dans un délai de six mois et, à compter de décembre 2014, à informer chaque année la Commission des mesures qu’ils auront mises en place. La Commission assurera le suivi de l'application de la recommandation au moyen d'un tableau de bord accessible au public, qui permettra de comparer la situation dans les différents États membres, et elle examinera dans dix-huit mois l'efficacité de cette approche. \\

\textbf{Contexte}
Le gaz naturel conventionnel est piégé dans des réservoirs souterrains. Le gaz de schiste, qui est aussi un gaz naturel, diffère en ce sens qu'il est piégé dans de la roche et ne peut être libéré que par fracturation de cette dernière. L'Union européenne n'a pour l'instant qu'une expérience limitée de l'application à grande échelle et intensive de la fracturation hydraulique à grand volume. La technique consiste à injecter d’importants volumes d’eau, de sable et de substances chimiques dans un puits afin de fracturer la roche et de faciliter ainsi l’extraction du gaz. Jusqu'à présent, l'Europe s'était essentiellement intéressée à la fracturation hydraulique à faible volume, appliquée dans des réservoirs de gaz compact conventionnel et le plus souvent dans des puits verticaux, qui ne représentait qu'une petite partie des activités d'exploitation de pétrole et de gaz de l'Union européenne. \\

Les incidences sur l’environnement et les risques associés devront être gérés correctement. Étant donné qu'il faut forer davantage de puits sur une plus grande superficie pour obtenir le même volume de gaz qu'avec des puits conventionnels, une évaluation rigoureuse et une atténuation des effets cumulés s'imposent.\\

La majeure partie de la législation de l'Union en matière d’environnement est antérieure à la pratique de la fracturation hydraulique à grand volume. Pour cette raison, certains aspects environnementaux ne sont pas traités de manière exhaustive dans la législation actuelle de l’Union, d'où les inquiétudes du public et la nécessité d'une action au niveau de l’Union.

\newpage
\subsection{Les gaz de schiste, une énergie controversée}
 
\begin{itemize}
	\item \textbf{Lien : }  \url{http://www.sciencespo.fr/ceri/fr/content/dossiersduceri/les-gaz-de-schiste-une-energie-controversee-0} 
	\item \textbf{Auteur : } François Gemenne
	\item \textbf{Date : } Janvier 2014
	\item \textbf{Source : } Centre de Recherche Internationale de Sciences Po 
	\item \textbf{Qualité de la source : } François Gemenne est chercheur en science politique à l'université de Liège (CEDEM) et à l'université de Versailles Saint-Quentin-en-Yvelines (CEARC). Il est expert associé au CERI et   vient de publier « Les négociations internationales sur le climat, une histoire sans fin ?
\end{itemize}

Le gaz de schiste peut-il entraîner une révolution énergétique ? Depuis ses débuts aux Etats-Unis dans les années 1970, l’extraction de ce gaz naturel donne lieu à de nombreuses controverses, notamment en raison des impacts des forages sur l’environnement. Alors que la Commission européenne a donné en janvier 2014 un timide feu vert à l’exploitation de ce gaz sur le sol européen, la France continue de refuser les permis de recherche, et donc l’exploitation de ce gaz sur le sol français. Ce Dossier du CERI tente de déchiffrer les enjeux de l’exploitation du gaz de schiste, et les raisons de la controverse.  \\

Le gaz de schiste est un gaz naturel contenu dans une roche argileuse, et dont les ressources sont inégalement réparties sur la planète. Contrairement au gaz naturel traditionnel, dont l’exploitation reste relativement aisée, celle du gaz de schiste se révèle particulièrement périlleuse, et lourde d’atteintes à l’environnement. La première méthode d’extraction, le forage horizontal qui permettait de drainer le gaz de la roche poreuse, a été abandonné dans les années 1990 au profit d’une autre technique, la fracturation hydraulique, qui consiste à provoquer dans la roche des micro-fractures, à l’aide d’eau injectée à haute pression, ce qui permet de libérer le gaz et de le récupérer ensuite.  \\

L’exploitation massive des gaz de schiste aux Etats-Unis, à partir des années 2000, a entraîné une chute spectaculaire des prix de l’énergie et la relance, tout aussi spectaculaire, du secteur de la pétrochimie. Ce gaz représente aujourd’hui 35\% de la production de gaz aux Etats-Unis, ce qui leur a permis de devenir un pays exportateur de gaz naturel, et sans doute de contester, à terme, la place de la Russie comme premier exportateur de gaz naturel.  \\

Mais cet essor sans précédent, qui n’est pas sans rappeler la ruée vers l’or noir, s’est aussi accompagné de nombreuses controverses quant à ses conséquences environnementales. Ces controverses ont culminé avec la sortie du film documentaire Gasland, sorti en 2010, qui a mis en lumière les risques et impacts associés à la technique de fracturation hydraulique. Cette technique, qui mobilise de très importantes quantités d’eau, peut provoquer de sérieuses pollutions des nappes d’eau souterraines à cause notamment des produits chimiques employés pour la fracturation, mais présente également un risque pour l’activité sismique et rejette dans l’atmosphère de grandes quantités de méthane, un gaz à effet de serre particulièrement nocif.  \\

Malgré ce bilan environnemental peu reluisant, certains voient néanmoins dans le gaz de schiste un potentiel de relance économique considérable, ainsi qu’un maillon essentiel dans la lutte contre le changement climatique. La combustion des gaz de schiste rejette en effet moins de dioxyde de carbone (le principal gaz à effet de serre anthropique) que la combustion du charbon ou du pétrole. Et des voix s’élèvent, en France et en Europe, pour que soient délivrées des autorisations de prospection et d’exploitation des gaz de schiste, au nom de la relance économique, des prix de l’énergie et parfois de la lutte contre le changement climatique. \\ 

On estime les réserves mondiales de gaz de schiste à plus de 200 billions de m3. Ces réserves sont inégalement réparties à travers le monde, puisque la Chine, l’Argentine, l’Algérie et les Etats-Unis s’en partagent environ la moitié. En Europe, ce sont la Pologne et la France qui possèdent les plus importantes réserves, avec environ 4 billions de m3 chacun. Après le feu vert de la Commission européenne, donné en janvier 2014 au grand dam des associations écologistes, la Pologne se lancera en 2014 dans l’exploitation de ses réserves. En France, le ministre de l’Environnement Philippe Martin a pourtant confirmé en novembre 2013 l’interdiction des permis de recherche sur le sol français, au grand dam de certains de ses collègues du gouvernement, dont le ministre du Redressement productif Arnaud Montebourg.  \\

La France peut-elle et doit-elle rester à l’écart de cette ruée sur le gaz de schiste ? La réponse à cette question mobilise un grand nombre d’enjeux et d’intérêts, que ce Dossier du CERI ambitionne de décrypter. Mathilde Mathieu, Oliver Sartor et Thomas Spencer, chercheurs à l’Institut du Développement durable et des Relations internationales (Iddri), se penchent d’abord sur le modèle américain, premier pays à s’être lancé dans l’exploitation massive des gaz de schiste, et s’interrogent sur ses possibles implications pour la stratégie énergétique européenne. Kari De Pryck, doctorante au CERI, analysent ensuite la raisons, acteurs et stratégies engagés dans les controverses sur le gaz de schiste : quels sont les ressorts qui sous-tendent la contestation, et quels en sont ses impacts politiques ? François Damerval, assistant parlementaire au Parlement européen, nous livre ensuite un témoignage éclairant sur le rôle des lobbies dans ce débat, et tire plusieurs réflexions issues d’un récent voyage d’étude aux Etats-Unis. Enfin, Maxime Combes, économiste à ATTAC, replace la question des gaz de schiste dans le contexte plus large de la lutte contre le changement climatique, et voit dans ces hydrocarbures non-conventionnels un moyen de retarder encore davantage la transition énergétique vers une économie pauvre en carbone. L’urgence est pourtant là. 

\end{document}


