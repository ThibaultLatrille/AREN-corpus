\documentclass[8pt]{article}

\usepackage[T1]{fontenc}
\usepackage[utf8]{inputenc}
\usepackage{graphicx}
\usepackage{lmodern}
\usepackage{amsmath}
\usepackage{xfrac}
\usepackage{amsthm}
\usepackage{listings}
\usepackage{enumerate}
\usepackage{amssymb}
\usepackage{cancel}
\usepackage{amsfonts}
\usepackage{float}
\usepackage{fullpage}
 
\DeclareUnicodeCharacter{200A}{ } 
\renewcommand*\contentsname{Table des matières}

\PassOptionsToPackage{hyphens}{url}\usepackage{hyperref}

\usepackage{listings}
\author{ControverSciences}
\title{Projet AREN : la voiture électrique}
\date{11 janvier 2017}

\begin{document}
\maketitle

\tableofcontents
\newpage

\section{Textes à débattre}
\subsection{La voiture électrique en alternative à l’air thermique}
\begin{itemize}
	\item \textbf{Lien : }  \url{http://www.voiture-electrique-populaire.fr/paradigme-mobilite-individuelle} 
	\item \textbf{Auteur : } Non spécifié
	\item \textbf{Date : } Non spécifié
	\item \textbf{Description : } Ce texte est très biaisé en faveur de la voiture éléctrique.
	\item \textbf{Source : }  Voiture Electrique Populaire se revendique comme  une structure totalement indépendante de tout acteur économique, personne morale ou groupe de pression, et notamment de tous constructeurs, fabriquant ou distributeur de voitures électriques, batteries, bornes de recharge,... L’équipe de Voiture Electrique Populaire est composée d’amateurs passionnés par cette nouvelle forme de mobilité.
\end{itemize}


Actuellement, les transports dépendent à 97\% du pétrole (IFP, 2006). Le trafic automobile a de grandes implications géopolitiques et nourrit certains conflits et désordres planétaires. Par ailleurs, l’automobile représente 13\% des émissions de CO2 dans le monde et 20\% en Europe. Les grandes agglomérations sont saturées par les monoxydes et dioxydes de carbone (CO, CO2), les oxydes d’azote (NOx) et autres fines particules résultant de la combustion des carburants des voitures thermiques. Les conséquences sont importantes sur la qualité de vie.\\

trafic automobile pollutionAinsi, au rythme des chocs pétroliers et au fur et à mesure de la prise en compte des enjeux environnementaux par les consommateurs et gouvernements, la voiture écologique est apparue comme une solution partielle mais crédible à certains troubles générés par le trafic routier. De nombreux programmes de développement d’automobiles propres ont été menés et des tests concluants réalisés.\\

Pourtant, force est de constater que la voiture économique et électrique, qui a souvent été présenté depuis un siècle comme l’innovation de la décennie, n’a toujours pas rencontré le moindre succès. La voiture écologique s’est toujours terminée par des échecs techniques ou commerciaux. Est-elle une innovation vouée à des désillusions perpétuelles ? Le regain de popularité actuel qu’elle connaît n’est-il qu’un feu de paille ?\\

Nous connaissons aujourd’hui un paradigme historique dans l’intégration des enjeux environnementaux, tant dans les politiques gouvernementales que dans les comportements des consommateurs. Les travaux du GIEC et la mobilisation autour de la conférence de Copenhague sont là pour en témoigner. Par ailleurs, les programmes de vehicules électriques des constructeurs automobiles semblent intégrer des plans de commercialisation de masse, ce qui est une première. La voiture électrique sera-t-elle donc à l’origine d’un changement de paradigme de la mobilité individuelle ? Au contraire, le véhicule électrique est-il voué à un échec perpétuel ?
\newpage

\subsection{La voiture électrique, ou l’arbre qui cache la forêt…}
\begin{itemize}
	\item \textbf{Lien : }  \url{https://greenwatchers.org/tag/voiture-electrique/} 
	\item \textbf{Auteur : } Pierre Japhet (Président fondateur de GreenWatchers)
	\item \textbf{Date : } 8 octobre 2010
	\item \textbf{Description : } Ce texte est très militant, contre la voiture éléctrique et en particulier la voiture individuelle.
	\item \textbf{Source : }  GreenWatchers est une association à but non lucratif, qui selon ses dires, contribue, à  son échelle, à la transformation écologique de notre société. 
\end{itemize}


C’est la vedette du Mondial de l’Automobile cette année : la voiture 100\% électrique. Peu importe son autonomie limitée, son coût prohibitif. Pourquoi cet engouement ? Peut-être parce que la voiture électrique, ce serait l’écologie souriante, le mariage impossible entre société de consommation et développement durable ? Malheureusement, ce n’est pas si simple…car le rêve de la voiture électrique risque de se fracasser contre deux réalités :\\

1) La production d’électricité, indispensable pour faire rouler ces petits engins, est elle-même polluante, surtout dans les pays où elle est produite à base de charbon, c’est à dire presque tous en dehors de la France…sans parler de la fabrication des engins eux-mêmes. Au final, le taux d’émission de CO2 au kilomètre serait d’une centaine de grammes, c’est à dire plus que la dernière Toyota Prius, qui n’est « que » hybride…\\

2) Ces véhicules ont besoin de batteries performantes, au lithium. Et, comme la demande en véhicules explose dans les pays émergents, il va falloir beaucoup, mais beaucoup de lithium. Or les réserves de lithium ne sont pas inépuisables, et leur exploitation pose également des problèmes écologiques.\\

La solution, ce n’est donc pas de changer le moteur, mais de changer nos comportements. La voiture de demain, électrique ou non, devra occuper une place bien différente de celle d’aujourd’hui dans les villes et les campagnes : en clair, elle sera partagée entre plusieurs utilisateurs et connectée à des réseaux de transport en commun qui assureront les liaisons grandes distances.\\

Mais c’est surtout dans nos têtes que la voiture doit changer de place ! Aujourd’hui elle est devenue selon les cas un symbole de réussite sociale, un défouloir, un instrument de violence ou la compensation d’une vie médiocre. Et parfois, un moyen de transport…\\

La voiture est un formidable symbole de la difficulté à adopter un modèle de développement durable, car elle cristallise toutes les contradictions entre :

– les limites physiques de la planète, et nos réels besoins de mobilité, d’une part,

– une société de consommation qui encourage l’individualisme, le consumérisme, et flatte notre égo d’autre part…et dont la voiture électrique est le dernier avatar…\\

Oui, la voiture de demain sera électrique. Mais elle sera surtout utilitaire et rare. Et ce sera çà, la vraie révolution !

 \newpage
\section{Corpus de ressources}
\subsection{Quelle voiture pour demain ?}
 
\begin{itemize}
	\item \textbf{Lien : }  \url{https://lejournal.cnrs.fr/articles/quelle-voiture-pour-demain} 
	\item \textbf{Auteur : } Meryem Tizniti
	\item \textbf{Date : } 30 mars 2015
	\item \textbf{Description : } Des voitures moins gourmandes en énergie, moins polluantes, connectées et complètement autonomes. C’est ce que nous promettent chercheurs et industriels pour demain.
	\item \textbf{Source : } CNRS Le journal, l'objectif de ce site est de partager largement avec les amateurs de science, les professeurs et leurs élèves, les étudiants et tous les citoyens curieux, des contenus destinés jusque-là à la communauté des agents du CNRS, chercheurs, ingénieurs et techniciens, ceux des labos comme ceux des bureaux.
\end{itemize}

\textbf{Le défi de l'autonomie}

Quelle sera la conduite de demain ? Serons-nous toujours sollicités ou complètement débarrassés de la « corvée de conduite » dans les situations répétitives et ennuyeuses ? Philippe Bonnifait, professeur à l’université de technologie de Compiègne et directeur du GDR Robotique, travaille sur les voitures autonomes dans lesquelles le conducteur, toujours présent derrière le volant, décide de laisser la main à un robot, « partenaire » de conduite. Ce dernier se présente sous la forme d’un ordinateur de bord capable de conduire le véhicule, c’est-à-dire de prendre des décisions après traitement des données recueillies par des logiciels de perception de l’environnement, de localisation, de planification..., embarqués dans le véhicule. Prudent, le robot partenaire de conduite est capable de rendre la main au conducteur quand la situation de conduite devient trop complexe par rapport à ses capacités...\\


Des chercheurs de l’Institut Pascal, à Clermont-Ferrand, en collaboration avec le constructeur Ligier, ont mis au point en 2013 une navette totalement autonome, qui répond au nom de EZ10. Cette navette d’une capacité de 10 personnes roule seule sur des sites protégés (sites industriels, parcs d’attraction, parkings). Testé au CHU d’Estaing, le véhicule assurait la liaison entre le parking et le parvis, distants de 350 mètres. « Cette navette se déplaçait entre une borne située sur le parvis et une autre sur le parking, à la manière d’un ascenseur horizontal », indique Michel Dhome, directeur de l’Institut Pascal. Actuellement, un test sur le site R\&D Michelin de Ladoux évalue les capacités du véhicule pour fonctionner de deux façons : en « mode tram » aux heures de pointe pour assurer la liaison entre le parking et le site industriel ; en « mode taxi » aux heures creuses pour permettre aux personnes de se déplacer d’un bâtiment à l’autre plus aisément. Après une première phase d’apprentissage, la navette se fraye un chemin à l’aide de son ordinateur de bord capable d’analyser en temps réel les informations qui sont collectées par les différents systèmes embarqués : deux caméras vidéo et des télémètres laser. « Nous avons développé un système économiquement bon en choisissant des capteurs à bas coût., continue Michel Dhome. Contrairement à la Google Car dont le capteur principal, situé sur le toit, vaut plus cher que le véhicule lui-même. »\\
 

\textbf{Des matériaux plus légers}

« Pour consommer moins, il faut des voitures plus légères », explique Sabine Denis, scientifique de l'Institut Jean Lamour. Or les métaux représentent à l'heure actuelle 70\% du poids d'un véhicule. Pour réduire cette part, deux méthodes sont utilisées : rendre les matériaux et procédés existants plus performants (aciers à très haute résistance, nouveaux alliages d’aluminium, composites…) ou en développer de nouveaux. On peut citer les procédés de mise en forme par fabrication additive, qui permettent de fabriquer des pièces plus légères sur mesure, ainsi que les matériaux architecturés, dans lesquels il s’agit d’associer plusieurs matériaux et de les architecturer dans l’espace pour obtenir différentes propriétés physiques et chimiques. Dans toutes ces démarches, la modélisation à différentes échelles est indispensable, en étroite relation avec l’expérimentation. Toujours aidé par la modélisation, il est possible, grâce à la métallurgie dite combinatoire, de trouver de nouveaux matériaux ou de modifier un matériau connu pour améliorer certaines de ses propriétés. Cependant, les constructeurs étant peu enclins au changement, Yannick Champion, chercheur à l’Institut de Chimie des Matériaux de Paris-Est, reconnaît avoir « des cartons remplis de matériaux aux coûts raisonnables, très performants mais non commercialisés car ils ne répondent pas totalement au cahier des charges des entreprises. » \\


\textbf{Des alternatives au pétrole}

Le XXIe siècle verra-t-il la fin des véhicules thermiques ? En Europe, la norme Euro VI imposera à l’horizon 2020 de limiter les émissions de CO2 à 95 g/km par véhicule contre une moyenne de 115 g/km pour une voiture neuve produite aujourd’hui. Les véhicules tout électriques se présentent comme LA solution, avec deux pistes privilégiées : les batteries nouvelle génération et les piles à combustible.\\

Les moteurs des véhicules électriques en circulation aujourd’hui sont principalement alimentés par des batteries de type Li-ion. Problème : ces dernières présentent une autonomie assez faible - 200 km au mieux. Le but de la recherche dans le secteur des accumulateurs est de stocker de l’énergie en masse et à bas coût. « La recherche tend vers le développement d’accumulateurs verts grâce à la synthèse à basse température, car elle demande peu d’énergie, ou à la biominéralisation en synthétisant des produits issus de la biomasse » indique Jean-Marie Tarascon, professeur au Collège de France et directeur du Réseau sur le stockage électrochimique de l’énergie (RS2E). La batterie lithium-soufre (ou Li-S), dont les prototypes donnent de bons résultats en laboratoire, présente une densité d’énergie massique (la quantité d'énergie qu'une batterie peut restituer par rapport à sa masse) de 400 Wh/kg, soit deux fois celle des batteries Li-ion. « J’ai beaucoup d’espoir pour cette batterie qui devrait arriver sur le marché dans les prochaines années », indique Patrice Simon, directeur adjoint du RS2E.\\

Autre technologie prometteuse : la batterie Li-air, dans laquelle une électrode au lithium est couplée à une électrode de pile à combustible. Cette batterie permet de concentrer une densité d’énergie massique de l’ordre de 1 000 Wh/kg, mais elle présente un verrou technologique majeur : la formation de superoxydes qui dégradent rapidement la batterie. Si ces nouvelles technologies laissent augurer des véhicules plus « propres », il ne faut pas oublier la production de l’énergie nécessaire pour charger ces batteries. Par exemple, pour produire une batterie Li-ion d'une énergie de 1 kWh, il faut 400 kWh d'énergie, ce qui génère une émission de 75 kg de CO2. Difficile dans ces conditions de parler de véhicule « zéro émission »…\\

La pile à combustible, elle, présente un atout majeur par rapport aux batteries : la possibilité de parcourir de longues distances. La Toyota Mirai, lancée au Japon cette année et en Europe l’année prochaine, annonce une autonomie de 500 km. Une pile à combustible utilise l’hydrogène (H2) et l’oxygène (O2) comme carburants. Problème : l’hydrogène n’existe pas à l’état naturel. Il faut donc le produire de façon industrielle. Soit par électrolyse de l'eau, qui décompose la molécule d’eau en hydrogène et en oxygène (ce qui demande d'utiliser de l’électricité produite à ce jour majoritairement dans des centrales thermiques ou nucléaires...), soit en l’extrayant de composés hydrogénés comme le méthanol (un alcool), le gaz naturel, l’essence ou même le charbon... « L'autre frein pour l’usage de ces piles est leur coût », indique Gérald Pourcelly, professeur à l’Institut européen des membranes et chargé de mission à l’Institut de chimie du CNRS pour l’Alliance Ancre (Énergie). En effet, le platine (Pt), métal noble et donc cher, y est utilisé comme catalyseur. Les recherches s’intensifient pour en diminuer la quantité. La mise au point de matériaux à l’échelle nanométrique, la nanostructuration, se présente comme la voie à suivre pour relever ce défi. Elle permet d’utiliser une moindre quantité de Pt sans pour autant diminuer les performances de la pile.\\

Un autre souci posé par l’usage de cette technologie est le stockage du combustible, H2. Ce dernier est comprimé à des pressions élevées, de l’ordre de 700 bars, dans des réservoirs faits d’un matériau à la fois très léger et très résistant. L’hydrogène est un gaz très volatil et inflammable, voire explosif sous certaines conditions de pression et de températures. Les fuites sont à éviter absolument. Qu'on se rassure : les réservoirs sont capables de supporter des pressions très élevées, allant jusqu’à 3 fois la pression de remplissage soit 2100 bars. « Cette technologie est plus sûre que celle du gaz pétrole liquéfié,» rappelle Gérald Pourcelly, qui précise :« Aujourd’hui, il est possible de comprimer jusqu’à 6 kg de H2 pour un réservoir de 100 kg ». Sachant qu’il faut environ 1 kg d’hydrogène pour 100 km, il est possible de parcourir environ 600 km avec un tel remplissage. Mais, dans ce cas, la moitié du volume du coffre de la voiture est occupé, ce qui ne correspond pas au cahier des charges des constructeurs...\\

\textbf{Routes et voitures connectées}

À Grenoble, Carlos Canudas de Wit et son équipe NeCS du Gipsa-Lab travaillent sur des algorithmes de régulation et de prédiction du trafic à l’aide d’un dispositif expérimental installé sur la rocade intérieure sud de la ville, le Grenoble Traffic Lab. Cette portion de route, longue de 10,5 km, connaît tous les jours des engorgements aux heures de pointe qui engendrent pollution et bouchons. Les chercheurs ont déployé un réseau de mini-capteurs, enfouis sous l’asphalte, qui mesurent la vitesse du véhicule et le flux. L’équipe intègre ensuite ces informations. L’algorithme développé pourra permettre, dans les années à venir, de réguler le trafic en temps réel en appliquant des commandes aux actionneurs installés à l’entrée (feux) et tout au long (panneaux de signalisation de vitesse variable) de la rocade. Carlos Canudas de Wit reste cependant prudent, les algorithmes de régulation permettant une optimisation du trafic mais pas une élimination totale des problèmes de circulation, qui dépendent fortement du flux de véhicules. Enfin, un conducteur pourra ou non suivre les suggestions affichées sur le bord de la route. « Avec l’arrivée de la voiture connectée, il sera possible de collecter les informations issues d’un véhicule précis et de suggérer à son conducteur des consignes personnalisées », précise le chercheur.\\


Le véhicule du futur et sa connectivité sont au cœur des travaux de Giovanni Pau et de son équipe au sein du laboratoire d’informatique de Paris-6. À travers la chaire « Smart \& connected mobility », cofinancée par Renault et Atos, il étudie la connectivité des voitures et explore plus particulièrement deux voies : la communication entre les véhicules, de type communication peer-to-peer et la communication avec l’environnement. Les études réalisées dans ce cadre s’intéressent aux voitures qui seront sur le marché dans les quinze à vingt prochaines années et qui (très probablement !) seront toutes connectées entre elles.\\

Une chose est sûre : si les voitures connectées présentent de nombreux avantages, elles posent aussi d’importantes questions de sécurité. En février dernier, un sénateur américain a publié un rapport assez alarmant : des données collectées auprès de 16 grands constructeurs automobiles ont montré que la quasi-totalité des véhicules connectés sur le marché présentaient des failles de sécurité. Sécuriser les voitures contre les piratages informatiques représente donc un défi de taille pour les chercheurs du monde entier.

\newpage
\subsection{Ne voiture rien venir ?}
 
\begin{itemize}
	\item \textbf{Lien : }  \url{https://www.youtube.com/watch?v=E9-mqOil5dY} 
	\item \textbf{Auteur : } Julien Goetz
	\item \textbf{Date : } 21 novembre 2016
	\item \textbf{Description : } Cette vidéo questionne la place de la voiture dans nos cités.
	\item \textbf{Source : } DataGueule, chaque épisode de cette émission hebdomadaire produite par france4 tente de révéler et décrypter les mécanismes de la société et leurs aspects méconnus.
\end{itemize}

\newpage
\subsection{Les batteries de véhicules électriques}
 
\begin{itemize}
	\item \textbf{Lien : }  \url{http://future.arte.tv/fr/voiture-electrique-quels-defis/les-batteries-de-vehicules-electriques} 
	\item \textbf{Auteur :} Arte
	\item \textbf{Date : } 23 décembre 2015
	\item \textbf{Description : } 3 vidéos d'1 minute pour tout savoir sur le moteur électrique, comprendre ses avantages et ses inconvénients face au moteur thermique, ainsi que le fonctionnement des batteries au lithium qui le composent.
	\item \textbf{Source : } Arte
\end{itemize}

\newpage
\subsection{La voiture électrique n’est pas écologique, vraiment ?}
 
\begin{itemize}
	\item \textbf{Lien : }  \url{https://mrmondialisation.org/la-voiture-electrique-nest-pas-ecologique-vraiment/} 
	\item \textbf{Auteur :} Mr. mondialisation
	\item \textbf{Date : } 7 octobre 2015
	\item \textbf{Description : } Depuis quelques années, les voitures électriques ne cessent de gagner en popularité dans le monde, tout particulièrement en Europe, tout en restant marginales. Si certains pays s’engagent en faveur du développement durable avec des politiques d’incitation à l’achat de véhicules électriques, une vague d’arguments s’opposant à elles a déferlé dans les médias. À juste titre ?
	\item \textbf{Source : } Mr. mondialisation, un think tank citoyen francophone et international. Il met en avant les initiatives qui s'inscrivent dans une démarche à la fois sociale et solidaire, et dont l'objectif est d'établir des alternatives au modèle de consommation dominant. Il a donc une démarche militante tout est étant assez rigoureux, sans être non plus à la pointe de la rigueur.
\end{itemize}

La Norvège, le Japon et le Royaume-Uni font figure d’exemple en matière d’investissements dans la voiture électrique. Objectif : faire baisser les émissions de gaz à effet de serre et assainir l’air localement. Cependant, son appellation de “voiture écologique” fait polémique depuis sa (re)naissance. En effet, si elle n’émet pas de dioxyde de carbone à la conduite, elle a, comme tout objet, un impact écologique non négligeable avant et après utilisation. Le jeu en vaut-il alors la chandelle ? \\

Tout d’abord, rappelons que la seule voiture vraiment écologique est celle qui n’existe pas. Il n’y pas à ce jour d’alternative 100\% propre, y compris en matière d’hydrogène. Une voiture reste globalement une boite d’acier, de plastique et de ressources polluantes à l’extraction. Ainsi, la question doit être abordée de manière comparative aux systèmes actuels : qu’elle est la solution « la moins grave » ? la technologie a-t-elle été pleinement exploitée ? que peut-on faire évoluer aujourd’hui ? Si l’idéal reste le vélo et les véhicules collectifs, les consommateurs ne sont pas prêt à abandonner leur voiture, ce qui impose de développer une alternative dans ce domaine. Que reproche-t-on alors à cette voiture électrique tant aimée à ses débuts et aujourd’hui tant détestée des plus conservateurs ? Pourquoi un tel acharnement ? \\

\textbf{Une presse française unanime}

Courant 2013, l’ensemble de la presse française titrait « la voiture électrique n’est pas écologique ». Un consensus médiatique soudain basé notamment sur une étude de l’Ademe qui relativise son côté « vert » avec raison. Tout d’abord, la fabrication de batteries (modèles actuels) nécessite de nombreux produits chimiques, tels que le plomb, le lithium (rare) ou encore le cadmium. Leur recyclage demande également des dépenses énergétiques élevées. Sans surprise, la production, le transport et l’assemblage de ses pièces représenteraient un niveau de pollution à la fabrication semblable à celui d’un véhicule classique. Enfin, on reproche à la voiture de consommer beaucoup d’électricité. Or, si 15\% de l’électricité de la planète provient de centrales nucléaires, qui ne produisent pas directement de CO2, 40\% de celle-ci est toujours produite depuis des centrales à charbon et 20\% par des centrales à gaz, qui sont très polluantes. Les énergies renouvelables restent marginales mais se développent. \\

En réalité, le débat n’est pas si tranché. Si la plupart des médias ont focalisé leur attention sur le négatif de l’étude, peu ont relevé les côtés positifs qui font du véhicule électrique une alternative bien plus enviable que le modèle pétrolier actuel. Le plus important de ces points, c’est de purifier l’air localement. Et c’est un avantage non négligeable. La pollution de l’air coute 100 milliards à la France chaque année, dévoilait un rapport gouvernemental publié en 2015. L’air vicié des villes, c’est aussi 45 000 décès prématurés chaque année dans l’hexagone. Par ailleurs, l’étude de l’Ademe montre qu’en matière de rejet de CO2, le véhicule électrique l’emporte malgré le coût écologique de sa production. De plus, le moteur électrique a une fiabilité mécanique supérieure du fait même de sa simplicité de fonctionnement. Moins de pannes pour moins de pièces à modifier et une plus grande durabilité (en dehors de la batterie), le véhicule électrique garde une nette longueur d’avance. D’un point de vue humain, les études montrent également qu’une voiture électrique diminue le stress de son conducteur. La conduite est moins agressive et le véhicule moins bruyant. Enfin, le « plein » d’énergie est bien moins couteux (de 30 à 60 fois) qu’un plein d’essence tout en offrant la possibilité à l’utilisateur de produire chez lui sa propre énergie et donc de vivre de manière bien plus autonome. \\

Pour finir, peut-on vraiment reprocher à la voiture électrique d’utiliser de l’électricité en prenant uniquement la France en référence et non pas les spécificités du monde réel et local ? N’est-ce pas déplacer un problème général (la production d’énergie) pour l’imputer à la voiture elle-même ? Cette critique n’est-elle pas implicitement le souhait de faire perdurer toutes voies de productions à base d’énergies fossiles ? Placez un véhicule électrique dans un pays comme le Costa Rica (en bonne route vers du 100\% d’électricité renouvelable) où chez un habitant qui possède une éolienne ou un fournisseur d’énergie durable (ce qu’un militant écologiste fait généralement), son bilan écologique est soudainement beaucoup plus positif. La responsabilité de l’évolution lente du mix énergétique est-elle a imputer au véhicule ou aux décisions politiques ? \\

\textbf{Le point obscure du stockage de l’énergie}


Après avoir décortiqué les faits reprochés à la voiture électrique, on constate qu’il ne reste qu’un élément vraiment à charge contre elle : la batterie, depuis sa production jusqu’à son recyclage. C’est en effet cette simple batterie qui plonge dans le rouge les indicateurs. Alors que les technologies fossiles ont été exploitées sous toutes les coutures, la voiture électrique demeure une île pratiquement vierge à explorer, avec des améliorations attendues en matière de batteries propres. À titre d’exemple, l’entreprise Tesla promet de grandes avancées en matière de stockage d’énergie. Elle offre un recyclage à hauteur de 60\% à ce jour et envisage d’atteindre les 90\% à terme. La marque annonce pour 2018 des batteries « Ryden Dual Carbon Battery » sans métaux lourds et biodégradables.\\

En 2015, l’université de Stanford annonçait à son tour la mise au point d’une batterie écologique et durable sans lithium. Enfin, l’avenir semble être aux nanotubes de carbone et aux supercondensateurs, si on en croit les scientifiques de l’université Rice et l’université de technologie du Queensland. En Espagne, la société Graphenano a débuté la production de super-batteries au graphène, adaptées aux véhicules électriques, et jusqu’à quatre fois plus efficaces. Janvier 2016, c’est le CNRS et le CEA français qui annonçaient le lancement d’une batterie révolutionnaire au sodium-ion. Des technologies nouvelles et pleines de promesses mais qui risquent rapidement de faire de l’ombre aux mastodontes des énergies fossiles qui, eux, ont déjà joué toutes leurs cartes technologiques.\\

Même si certains aspects cités précédemment ont tendance à ternir l’image idéalisée de la voiture électrique, l’espoir d’une amélioration de la technologie en fait une alternative intéressante d’un point de vue écologique, sans être l’unique solution à la mobilité. Certes, la qualifier de véhicule écologique à part entière serait erroné à ce jour, mais elle dispose d’un moteur plus propre dont l’impact local sur l’air et la santé est indéniable. Elle est capable d’effectuer des performances satisfaisantes sans produire de pollution locale et son utilisation est gage d’un grand pas dans la lutte contre la dégradation de l’environnement. Les pays scandinaves osent d’ailleurs le pari de la transition où le véhicule électrique semble jouer un rôle important. La Suède vise, par exemple, à se débarrasser à 100 \% des énergies fossiles dans un futur proche. Dans cette optique, le passage des vieux véhicules polluants vers l’électrique, si possible avec des batteries de dernière génération, jouera un rôle déterminant.\\

Si vous êtes de ceux qui hésitent encore à passer à l’électrique, il existe des portails qui éclaircissent la situation. Pour obtenir de plus amples informations sur le marché actuel des voitures électriques en Europe, ainsi que les aides à obtenir, nous vous invitons à consulter cette infographie très complète : Les voitures électriques en Europe, 7 infos à retenir en 2015. Elle présente, entre autres, les principales aides fiscales européennes pour l’achat d’un véhicule électrique, les chiffres-clés à connaître et les modèles les plus populaires. Une base pertinente pour ceux qui envisagent de vendre leur voiture d’occasion afin de faire l’acquisition d’un modèle électrique ou hybride. Mais si la voiture électrique s’avère en définitive une fausse alternative, peut-être emportera-t-elle dans sa tombe le mythe de l’automobile individuelle.

\newpage
\subsection{Émissions de CO2  : l’impasse de la voiture électrique}
 
\begin{itemize}
	\item \textbf{Lien : }  \url{http://www.lemonde.fr/economie/article/2015/10/23/emissions-de-co2-l-impasse-de-la-voiture-electrique_4795636_3234.html#cDo9AT3Gy6AMSgzE.99} 
	\item \textbf{Auteur :} Stéphane Lhomme
	\item \textbf{Date : } 23 novembre 2015
	\item \textbf{Description : } Le cycle de vie d’un véhicule électrique le rend aussi polluant qu’un véhicule thermique. Le subventionner n’a pas de sens, explique le directeur de l’Observatoire du nucléaire, Stéphane Lhomme.
	\item \textbf{Source : } Le monde. Stéphane Lhomme reste un peu biaisé sachant qu'il est directeur de l’Observatoire du nucléaire, et donc a des conflits d'intérêts.
\end{itemize}

A l’approche de la COP21, le gouvernement français intensifie sa croisade en faveur de la voiture électrique, probablement parce qu’il s’agit de la seule action pouvant faire croire que le pays hôte se préoccupe du climat… \\

Or, contrairement à ce que croient la plupart des gens, soumis à une propagande continuelle des politiques et des industriels, la voiture électrique n’est pas plus vertueuse pour le climat que la voiture thermique, essence ou diesel. \\

Ce sont là les conclusions d’une étude, déjà ancienne, de l’Agence de l’environnement et de la maîtrise de l’énergie (Ademe), ignorées délibérément par le gouvernement (Elaboration selon les principes des ACV des bilans énergétiques, des émissions de gaz à effet de serre et des autres impacts environnementaux induits par l’ensemble des filières de véhicules électriques et de véhicules thermiques à l’horizon 2012 et 2020, novembre 2013). \\

La donnée la plus cruciale est que la fabrication des batteries est tellement émettrice de CO2 qu’il faut avoir parcouru de 50 000 à 100 000 km en voiture électrique pour commencer à être moins producteur de CO2 qu’une voiture thermique. Soit 15 à 30 km par jour, 365 jours par an, pendant 10 ans ! \\

Sachant que ces voitures servent essentiellement à des trajets courts, il est probable que le kilométrage nécessaire pour s’estimer « vertueux » ne sera jamais atteint. De plus, tout le CO2 émis par une voiture électrique est envoyé dans l’atmosphère avant même que ne soit parcouru le moindre kilomètre, alors que la voiture thermique émet son CO2 au fil des ans… \\

Par ailleurs, il est partout prétendu que la voiture électrique n’émet pas de particules fines. Mais comme le signale le magazine Science et Vie (janvier 2015), « les pneus, les freins et l’usure des routes émettent presque autant de microparticules que le diesel ». La voiture électrique émet certes moins de particules que la voiture thermique, puisqu’elle ne dispose pas d’un pot d’échappement, mais elle possède bien des freins, des pneus, et roule sur le goudron ! \\

\textbf{A 80 \% nucléaire}

Au final, la voiture électrique n’est pas plus écologique que la voiture thermique. L’argent public consacré à son développement est donc totalement injustifié. Or, il s’agit de sommes astronomiques : 

– le gouvernement a lancé un plan d’installation de 7 millions de bornes de rechargement à environ 10 000 euros pièce, soit un coût d’environ 70 milliards d’euros. Il est d’ailleurs poignant de voir les élus de petites communes, croyant faire un geste pour l’environnement, casser la tirelire municipale pour s’offrir une borne ;

– le bonus « écologique » à l’achat d’une voiture électrique dépasse 10 000 euros par véhicule, souvent complété par une prime de la région. La quasi-totalité des acheteurs sont des ménages aisés : une fois de plus, l’argent de tous est offert aux plus privilégiés. \\

En réalité, au pays de l’atome, tous les moyens sont bons pour « booster » la consommation d’électricité, en baisse continue depuis des années. Car la voiture électrique en France peut être considérée comme une « voiture nucléaire » : la quasi-totalité des bornes de rechargement installées sont branchées sur le réseau électrique ordinaire, à 80 \% nucléaire. \\

Il ne faut pas se laisser abuser par les certificats mis en avant par M. Bolloré et ses Autolib (Paris), Bluecub (Bordeaux) et Bluely (Lyon), assurant qu’elles sont rechargées aux énergies renouvelables : il ne s’agit que de jeux d’écriture ; l’électricité utilisée est la même qu’ailleurs. \\

Nous ne faisons pas ici la promotion de la voiture thermique, elle-même une calamité environnementale. Mais, justement, personne n’aurait l’idée d’offrir 10 000 euros à l’achat d’une voiture diesel, de lui réserver des places de stationnement et de remplir son réservoir à prix cassé…


\newpage
\subsection{L’avenir de l’automobile sera électrique}
 
\begin{itemize}
	\item \textbf{Lien : }  \url{http://lemonde.fr/automobile/article/2016/07/04/voiture-electrique-la-belle-prise_4962979_1654940.html} 
	\item \textbf{Auteur :} Jean-Michel Normand
	\item \textbf{Date : } 4 juillet 2015
	\item \textbf{Description : } Ils s’y mettent tous ! Français, allemands, américains ou chinois, les constructeurs axent une bonne part de leur stratégie sur l’électrique.
	\item \textbf{Source : } Le monde
\end{itemize}

Ces dernières semaines se sont succédé les actes d’allégeance à l’électrique, en particulier de la part des constructeurs allemands qui, jusqu’alors, apparaissaient plutôt réticents. Il faut croire que, désormais, tout le monde y croit dur comme fer. Le plus vibrant des plaidoyers est venu de Volkswagen, touché par la grâce de la fée électricité après avoir été atteint de plein fouet par le scandale des moteurs diesel truqués. Selon Matthias Müller, le nouveau patron du groupe, le « dieselgate » constitue « un catalyseur ». La stratégie Volkswagen à l’horizon 2025, présentée mi-juin, fera donc la part belle aux véhicules électriques, avec pas moins de trente modèles destinés à représenter jusqu’à 25\% des ventes. \\

Le groupe Daimler (Mercedes, Smart) n’est pas en reste. Il prépare un plan stratégique basé sur « l’électromobilité », qui pourrait prévoir la création d’une marque consacrée aux modèles « zéro émission ». Mercedes dévoilera au Mondial de l’automobile de Paris (du 1er au 16 octobre) une voiture tout-électrique dont l’autonomie atteindra 500 km. Une réponse à Tesla mais aussi à Porsche, qui s’est fraîchement converti, et à BMW, alors que l’Allemagne a institué, le 2 juillet, un système de primes (4 000 euros) en faveur des modèles électriques.  \\

\textbf{Electromania}

De son côté, PSA Peugeot-Citroën vient de signer deux accords avec son actionnaire chinois Dongfeng, prévoyant la fabrication, d’ici trois ans, de modèles conçus en commun. Ford, soucieux de ne pas laisser le champ libre à la Chevrolet Bolt, que General Motors s’apprête à lancer, vient de faire savoir qu’un modèle électrique de grande diffusion est en approche, sans doute pour 2019. Quant aux marques qui se sont positionnées de longue date sur ce marché, elles continuent de peaufiner leur gamme. Renault améliore régulièrement l’autonomie de la Zoé, qui, de 200 km aujourd’hui, devrait être portée à 300 km d’ici à 2020.  \\

Pendant ce temps, le marché frémit. Même si la part des véhicules électriques (1,16\%) demeure marginale, la France, deuxième marché européen (22 000 immatriculations annuelles) après la Norvège (27 000), voit ses ventes croître rapidement, soutenues par d’importantes subventions publiques (bonus de 6 300 euros). En mai, le rythme de progression atteignait 57\% sur un an.  \\

Au-delà du « greenwashing », destiné à redorer le blason d’une industrie automobile ayant pris de coupables libertés avec la réglementation antipollution, cette électromania recouvre une réalité technologique. « L’affaire Volkswagen n’a pas seulement écorné l’image du diesel. Elle a convaincu les constructeurs que les améliorations apportées aux moteurs essence et aux véhicules hybrides ne suffiront pas ; le développement de modèles 100\% électriques s’impose comme un passage obligé pour satisfaire aux futures normes européennes », estime Jamel Taganza, consultant chez Inovev, un cabinet spécialiste de l’automobile.  \\

La multiplication de mesures interdisant aux véhicules les plus polluants l’accès au centre-ville – à Paris, mais aussi dans certaines grandes métropoles – constitue un autre levier favorable à la diffusion de modèles électriques. « En particulier en Chine, premier marché mondial, où les autorités mettent en place un vaste plan destiné à favoriser ce type de véhicule », insiste Jamel Taganza, qui rappelle que « lors du dernier Salon de l’automobile de Pékin, plus de la moitié des nouvelles voitures électriques étaient présentées par des constructeurs nationaux ». Comme en France, la part de la fée électricité dépasse à peine 1\% des immatriculations dans l’empire du Milieu, soit, tout de même, quelque 250 000 voitures par an.  \\

\textbf{Une voiture plus désirable}

Plus crédible et (un peu) plus performante en termes d’autonomie, l’automobile électrique progresse aussi sur un autre plan : elle est en train de devenir plus désirable. L’aura acquise par Tesla, dont le futur Model 3, qui doit élargir la clientèle de la marque à partir de 2018, a déjà reçu quelque 325 000 précommandes, y a contribué. L’engagement des constructeurs premium allemands confirme que ces voitures ne sont pas seulement destinées à des fonctions utilitaires.  \\

Enfin, à mesure que les ventes décollent, on en sait davantage sur la clientèle : plus aisée, plus féminine, mais aussi plus périurbaine que vraiment citadine. Des utilisateurs qui rechargent leur véhicule à la maison ou sur leur lieu de travail, rarement sur les bornes publiques. Ce qui devrait inciter à mettre davantage l’accent sur l’installation de bornes dans les logements collectifs.

\newpage
\subsection{Le bilan écologique mitigé de la voiture électrique}
 
\begin{itemize}
	\item \textbf{Lien : }  \url{http://lemonde.fr/economie/article/2016/09/26/le-bilan-ecologique-mitige-de-la-voiture-electrique_5003498_3234.html} 
	\item \textbf{Auteur :} Laetitia Van Eeckhout
	\item \textbf{Date : } 26 septembre 2016
	\item \textbf{Description : } Si une voiture à 100\% électrique ne rejette aucun CO2 dans l’air, au stade de la fabrication, elle est près de deux fois plus énergivore qu’un véhicule thermique.
	\item \textbf{Source : } Le monde
\end{itemize}

A l’heure où le diesel et le fonctionnement lacunaire des systèmes antipollution de nombreux constructeurs n’en finissent pas de défrayer la chronique, la voiture électrique pourrait être le compromis idéal pour garder une certaine liberté de déplacement tout en préservant la planète. Et de fait, lorsqu’elle circule, une voiture à 100\% électrique ne rejette aucun CO2 dans l’air, ce qui n’est pas le cas de ses semblables diesel comme essence.  \\

Sauf que le débat voiture thermique/voiture électrique n’est pas aussi simple. Pour réellement comprendre l’impact écologique d’un modèle, il faut raisonner du « berceau à la tombe », c’est-à-dire en prenant en compte l’ensemble des étapes de la vie d’un véhicule, depuis les matériaux utilisés pour le fabriquer jusqu’au traitement en fin de vie.  \\

\textbf{Substances acidifiantes dans l’atmosphère}

Or, au stade de la fabrication, une voiture électrique est près de deux fois plus énergivore qu’un véhicule thermique, comme le montre une étude de l’Agence de l’environnement et de la maîtrise de l’énergie (Ademe) comparant les bilans environnementaux des véhicules électriques et des véhicules thermiques essence et diesel.  \\

La voiture électrique est en effet pénalisée par la production de l’énergie qui lui est nécessaire (surtout si elle est à base de charbon) et son transfert jusqu’à la prise, et par la fabrication des batteries. Nécessitant l’extraction de nombreuses matières premières, comme le cobalt ou le lithium, le processus de fabrication des batteries consomme beaucoup d’énergie et émet des gaz à effet de serre. Sur l’ensemble de sa construction, un véhicule électrique émet 6,6 tonnes équivalent CO2 – dont près de la moitié pour les batteries –, quand un véhicule thermique en émet 3,8 tonnes.  \\

Autre point noir de l’empreinte écologique du véhicule électrique, sa forte contribution à l’augmentation de la teneur en substances acidifiantes dans l’atmosphère, à l’origine des « pluies » acides. Les émissions de dioxyde de soufre pendant la phase d’extraction des métaux nécessaires à l’élaboration de la batterie sont en effet importantes.  \\

\textbf{Un bilan carbone meilleur}

Même à l’usage, l’empreinte écologique de la voiture électrique n’est pas totalement neutre contribuant un tant soit peu à la pollution atmosphérique. « Si la voiture électrique n’émet pas de polluants à l’échappement, le freinage ou l’abrasion des pneus engendrent comme sur tout autre véhicule des particules fines », explique Maxime Pasquier, du service transport et mobilité de l’Ademe. « Toutefois, tient à ajouter celui-ci, la voiture électrique use moins les freins. »  \\

Au total, malgré le coût écologique de sa production, le véhicule électrique affiche, utilisation comprise, un bilan carbone meilleur que celui d’un véhicule thermique (9 tonnes équivalent CO2 contre 22 tonnes) « Et le véhicule électrique, insiste M. Pasquier, favorise la diversification énergétique et réduit notre dépendance au pétrole. S’il n’existe qu’une façon de produire le carburant des véhicules thermiques, l’électricité peut être produite de différentes façons, et notamment en utilisant des énergies renouvelables. » \\

\newpage
\subsection{Voitures et batteries Tesla : la révolution du tout électrique ?}
 
\begin{itemize}
	\item \textbf{Lien : }  \url{http://bit.ly/1WSYEGN}  ou sinon le lien non raccourci ci-dessous \\  \url{http://www.cite-sciences.fr/fr/ressources/science-actualites/detail/news/voitures-et-batteries-tesla-la-revolution-du-tout-electrique/?tx_news_pi1[controller]=News&tx_news_pi1[action]=detail&cHash=4ce8728b3d6a00a063712c82ecac0a2d
}
	\item \textbf{Auteur : Adrien Denèle}
	\item \textbf{Date : } 26/11/2015
	\item \textbf{Description : }  Tesla Motors parie sur la révolution du tout électrique. Avec ses voitures sans carburant et bientôt ses batteries domestiques de stockage, la firme californienne compte jouer un rôle majeur dans cette transition, quitte à l’accélérer afin de la dominer.
	\item \textbf{Source : } Très bonne enquête récente sur l'entreprise Tesla par La cité des Sciences et de l'Industrie. 
\end{itemize}

En pleine présentation du Model X, le SUV électrique de Tesla Motors, son PDG Elon Musk s’autorise une petite pique à l’adresse de ses concurrents automobiles. Nous sommes en effet le 29 septembre 2015, quelques semaines après l’énorme scandale des logiciels tricheurs de contrôle pollution de Volkswagen. Toute l’industrie automobile polluante est embarrassée : Elon Musk sait qu’un coup est à jouer. Une partie entière de sa présentation porte donc sur… la propreté de l’air. « C’est un sujet un peu polémique, s’amuse-t-il innocemment, mais nous avions prévu ces fonctionnalités bien avant les récents événements ! » Les rires dans la salle montrent que le doute est permis, mais qu’importe, le coup médiatique est réussi. Tesla va jusqu’à proposer un Bioweapon defense mode, un bouton tout à fait réel qui filtrera l’air à l’intérieur du véhicule, même en cas d’apocalypse chimique ou d’attaque biologique ! Le message est limpide : Tesla Motors n’émet pas d’émissions polluantes et vous protège même contre celles de vos concurrents !\\

Elon Musk présente le Model X, SUV électrique connu pour ses « Falcon Wings ». Le milliardaire semble toujours aussi mal à l’aise dans ce type d’exercice, mais son public lui reste conquis. Un effet Apple ?
Le message a toutes les chances d’être bien reçu en cette fin d’année 2015 placée sous le signe du climat. Prévue début décembre à Paris, la COP21 affiche ses promesses et l’intérêt du public s’aiguise petit à petit. L’entreprise Tesla Motors occupe une place de choix dans ce secteur : tout comme l’illustre scientifique dont elle s’inspire, Nikola Tesla, elle mise tout sur l’électrique. Avec ses voitures électriques et ses batteries domestiques, Tesla compte en effet dominer le marché énergétique d’un avenir pas si lointain. Un futur dans lequel chaque voiture serait électrique, sans conducteur, et où les réseaux devront maîtriser le stockage d'énergie. Un scénario de science-fiction ? Loin de là, car Tesla mise sur sa mise en œuvre d’ici à 2030...\\

\textbf{Modèles S et X, électrons libres de l’automobile}

Si vous êtes passé par la gare de Lyon, à Paris, au mois d’août 2015, vous êtes certainement tombé sur une présentation de Tesla. Leur voiture la plus célèbre, la berline modèle S, y était présentée au public – qui pouvait bien sûr en profiter pour passer commande. L’œil des badauds y était attiré par un détail étrange : le capot avant s’ouvrait sur un vaste espace vide, sans moteur. Naturellement, on se dirige vers le coffre afin de voir si le moteur s’y trouve, mais là encore, un espace de rangement complètement vide. Une surprise de taille pour les conducteurs de diesels et autres véhicules à essence, qui préfigure de la révolution à venir dans notre conception des automobiles. \\

« Tout cela fait partie de notre “plan secret” en trois phases », confie Charles Delaville, communicant chez Tesla Motors France. Un plan dicté par Elon Musk, le PDG de Tesla, sur une note de blog à l’entreprise en 2006. « L’idée était d’abord de prouver notre savoir-faire électrique », poursuit le responsable. Ainsi en 2008 naît la Roadster, une voiture de sport électrique très performante, produite en petit nombre. Très chère, aussi, mais l’objectif n’était pas tant de la vendre que de prouver la capacité de Tesla à la concevoir. \\

Du haut de gamme au grand public

« Nous sommes désormais dans la deuxième étape, celle des modèles haut de gamme X et S », indique M. Delaville. Le modèle S, sorti en 2012, est une berline haut de gamme ayant pour particularité la présence interne d’un écran tactile de 17 pouces. Sur l’écran s’affichent les nombreuses options (caméra de recul, GPS, navigateur internet). Sans oublier la suspension adaptative, les portières qui s’ouvrent lorsque le conducteur approche, ou encore les poignées rétractables... \\

Bref, une flopée de gadgets, dont on devine l’origine chez l’excentrique Elon Musk. Une petite folie qui n’empêche pas les résultats. Le dernier « sous-modèle » en date, le P85D, s’est vu décerner une grande quantité d’éloges et de vidéos flatteuses, comme celle de l’émission Turbo en France. Vient ensuite le Model X, un SUV qui arrivera sur le marché dans les prochains mois, avec cependant deux ans de retard sur la date prévue. \\

\textbf{Les modèles S et X}

Une fois encore, le véhicule mise sur le haut de gamme avec un coût élevé (130 000 dollars environ), mais avec une volonté résolument avant-gardiste, comme le prouve la présence des Falcon Wings : des portes à ouverture verticale capables de repérer l’environnement pour ne rien cogner et de s’ouvrir y compris à proximité de voitures garées. « La phase 3 de notre plan consiste à apporter un véhicule électrique performant au grand public », précise M. Delaville. Ce Model 3 visera la barre des 35 000 dollars, le prix moyen d’un véhicule aux États-Unis. Celui-ci sera le porte-étendard de Tesla Motors pour s’imposer à une plus large échelle et sortir de son image de constructeur pour riches. \\

\textbf{Une victoire sans résistance ?}

Le grand public est-il prêt à investir dans une voiture électrique ? De nombreux clichés, interrogations légitimes et peurs subsistent, en partie promus par l’industrie pétrolière qui a tout intérêt à retarder l’arrivée de ces modèles. A contrario, nombre d’observateurs jugent la victoire des voitures électriques inévitable. \\

En effet, elles surpassent largement leurs ancêtres sur de nombreux points. Tout d’abord, la performance. Les moteurs électriques ont un rendement nettement supérieur à celui d’un moteur à combustion limité par les lois de la thermodynamique et ses cycles définis : l’électrique l’emporte à plus de 80 \% face à 20 \% de rendement. Aussi – et c’est lié – le coût de l’énergie requise pour un « plein » est-il largement inférieur, puisque l’électricité est moins onéreuse que le pétrole. Sans oublier un « détail » environnemental : le pétrole est forcément une énergie fossile, alors que l’électricité peut être renouvelable. Vient ensuite le plaisir de la conduite, point sensible pour de nombreux conducteurs. Ils apprécieront sans aucun doute les accélérations imbattables des moteurs électriques, qui sont directement reliés aux roues et n’attendent pas une combustion lente pour démarrer. Les voitures Tesla vont même plus loin avec des modes d’accélération Insane (« fou furieux ») qui permettent de passer de 0 à 60 km/h en trois secondes. Les vidéos d’utilisateurs montrent leur surprise face à cette pointe de vitesse aussi immédiate que silencieuse. \\

L’accélération Insane des Tesla permet aux conducteurs de ressentir une poussée d’adrénaline d’ordinaire réservée aux propriétaires de voitures de luxe. Les voitures électriques devraient offrir cette sensation au grand public d’ici dix ans.
Reste le point noir des voitures électriques, qui inquiète encore les consommateurs : les recharges longues et impossibles dans des zones reculées. Étant donné le manque de motivation des autorités publiques, Tesla a pris les devants avec son réseau de superchargeurs. Ces bornes sont réparties stratégiquement du nord au sud des États-Unis ainsi qu’aux quatre coins de l’Europe. La recharge est bien plus rapide que sur une prise domestique, le Model S peut ainsi se recharger à 80 \% (soit 200 km d’autonomie) en 40 minutes. Et cela... gratuitement ! Tesla vous propose donc d’amortir le coût de ses modèles par une énergie potentiellement gratuite. Une prime de lancement ? « Non, nous comptons réellement maintenir cette gratuité sur le long terme ! » promet le communicant de Tesla, M. Delaville. De nombreux usagers ont déjà transmis les « romans photos » de leurs voyages gratuits en Europe ou d’une frontière à l’autre. \\

\textbf{Un avenir sans conducteur ?}

Tesla innove également sur les technologies de bord. « Les propriétaires d’un Model S depuis septembre 2014 ont automatiquement reçu le pilotage automatique de leurs véhicules », explique Charles Delaville. Automatiquement ? Oui, car Tesla intègre à ses véhicules un système d’exploitation semblable à celui qui équipe les smartphones Apple ou Android. La mise à jour 7.0 permet d’ajouter la fonctionnalité d’une conduite sans volant sur certaines routes et autoroutes. Elon Musk vient d’ailleurs de promettre que la 7.1 permettra à la voiture d’entrer et sortir du garage seule. Cette introduction du software (les logiciels) dans le hardware (les composants physiques) est l’une des plus profondes modifications actuelles de l’automobile. \\

Plus importante même que le passage à l’électrique ? C’est la thèse de Tony Seba. Ce conférencier de la Silicon Valley fait état dans son livre Clean Disruption of Energy and Transportation d’une théorie qui aurait semblé folle il y a encore dix ans. D’ici à 2030, explique-t-il, le parc automobile sera non seulement électrique mais encore totalement autonome. Bref, des voitures sans chauffeurs. Tony Seba décrit un avenir qui laisse rêveur : fin de la propriété privée des voitures, réduction du parc automobile de 80 \% et remplacement des parkings en ville par des parcs. Mais sa thèse de départ est-elle fondée ? \\

Voyons les nouveaux venus de l’industrie automobile. Tesla donc, qui s’est créé avec l’objectif de prendre la tête de la transition. En France, Carlos Ghosn, pour Renault, s’est engagé à commercialiser un modèle d’ici 2018. Mais les poids lourds – que bien peu de monde imaginait sur ce marché voilà dix ans – seront Google et Apple. Car si les géants actuels maîtrisent la technologie rutilante des véhicules à moteur, la conception d’un logiciel pour permettre à une voiture de rouler sans conducteur est une tout autre histoire. Selon Tony Seba, « l’avenir automobile appartient donc aux entreprises issues du monde de l’informatique […] ; les géants du secteur actuel n’ont d’autre choix que de s’adapter pour survivre ». \\

\textbf{Une Gigafactory pour produire les batteries en rythme}

Reste le critère essentiel pour convaincre les familles les plus modestes : le prix. Les modèles actuels restent très chers (plus de 70 000 euros pour la Tesla Model S et 130 000 euros pour le Model X). Si le prix devient similaire à celui des modèles à pétrole, pour des performances comparables et un coût de revient inférieur (un moteur électrique a très peu de pièces), à quoi bon acheter une voiture conventionnelle ? \\

Le coût repose en grande partie sur les batteries : ces pièces représentent 30 à 40 \% du prix final du véhicule. Tesla utilise les modèles lithium-ion qui, avec un très bon rapport énergétique pour un faible volume, sont idéaux pour les véhicules. Leur coût baisse d’environ 20 \% tous les deux ans. \\

Les innovations technologiques et les fabrications à la chaîne pourraient suffire à réduire les coûts jusqu’à un niveau convenable d’ici cinq à dix ans. Mais Elon Musk semble trouver le temps long. Le PDG de Tesla a donc lancé un projet pharaonique (comme d’habitude) : la Gigafactory. Une giga-usine dont la superficie initiale, un kilomètre de long sur 430 mètres de large (imaginez six tours Eiffel mises à plat dans un rectangle) pourra être doublée, voire triplée, par l’ajout d’étages et de bâtiments. Cette usine sera entièrement dédiée à la production en masse de batteries. \\


Avec un investissement de 5 milliards de dollars (dont 2 investis par Panasonic et 1,3 de réduction de taxes par l’État du Nevada où l’usine est installée), Tesla mise gros. Si le développement des véhicules électriques tarde, la concurrence plus puissante aura le temps de s’adapter et pourrait couler Tesla Motors. Imaginez simplement le marketing redoutable d’Apple lorsque leur voiture électrique sortira... L’entreprise a donc pris les devants. Avec la Gigafactory, le coût des batteries pourrait chuter de 13 \% ! Selon Tesla, « d’ici 2020, elle atteindra sa capacité maximale et produira plus de batteries au lithium-ion par an qu’il n’en était produit dans le monde en 2013 ». L’usine débutera sa production dès 2017, à moins d’un retard imprévu auquel sont par ailleurs habituées les sociétés d’Elon Musk (SpaceX, sa société spatiale, a souffert de nombreux lancements en retard). \\

\textbf{PowerWall, batterie révolutionnaire ?}

Bâtir une gigantesque usine uniquement pour fabriquer des batteries automobiles vous semble-t-il démesuré ? De fait, la Gigafactory aura également un autre objectif : commercialiser des batteries à usage domestique et professionnel. Deux modèles de batteries lithium-ion à fixer sur un mur seront produits : le PowerPack, destiné aux entreprises qui souhaitent stocker et revendre leur énergie ; et le PowerWall, réservé aux particuliers, annoncé au prix d’environ 3 700 euros. Ces deux nouvelles cordes à l’arc (électrique) déjà très étoffé d’Elon Musk ont une grande ambition : révolutionner les réseaux électriques. Mais est-ce vraiment une si bonne idée que d’inonder le marché de batteries de stockage ? \\

Pour un spécialiste du secteur électrique européen (qui a préféré rester anonyme afin de ne pas engager la parole de sa société), l’idée n’est pas mauvaise, mais ne suffira pas : « Ce stockage distribué n’est pas compétitif aujourd’hui sur le réseau européen, sauf s’il répond en même temps à un autre besoin comme la mobilité ». \\

Pour comprendre son point de vue, il faut partir du fonctionnement de nos réseaux électriques. « La première chose à savoir, c’est que les besoins, pour ce type de batterie, varieront selon les continents, explique-t-il. Aux États-Unis, secteur visé par Tesla, le réseau peut être moins dense qu’en Europe. Certaines zones reculées, comme de petites villes en Alaska, peuvent trouver un intérêt à stocker l’énergie, pour s’alimenter en cas de panne par exemple. » Le réseau européen, lui, n’a rien à envier à son homologue états-unien : « Au contraire, il est l’un des plus robustes au monde ! » De ce fait, les batteries PowerWall ne présenteront que peu d’intérêt pour pallier des pannes trop rares, ou aux impacts déjà très réduits. \\


\textbf{Stockage et solaire}

Quid du stockage d’énergie ? Tesla met en avant la capacité du PowerWall à soulager le réseau face aux « pics » de consommation, le soir. Mais à cet égard aussi, l’Europe est bien placée. « Au Japon, en revanche, la rapidité des montées de charge et le souci de disposer de ressources bien réparties – vu la fréquence des séismes – ont motivé le déploiement de batteries. En France et en Europe, le réseau électrique est très maillé et l’intérêt du PowerWall reste limité », estime le spécialiste. \\

Par ailleurs, l’avenir des réseaux électriques est désormais placé sous la lumière du photovoltaïque, avec la pénétration fracassante des réseaux européens par l’énergie solaire. « Avec ses 35 GW de panneaux solaires installés, contre 5 GW pour la France, l’Allemagne est largement en avance sur nous », confie l'expert. La baisse du coût des panneaux solaires laisse présager leur développement massif en Europe. Et avec des batteries couplées aux panneaux solaires, a priori plus besoin de craindre les pics de production solaire qui seront amortis en bonne partie.  \\

« S’il y a un effet d’emballement dans le développement des panneaux solaires, il vaudrait mieux que quelques personnes s’équipent de batteries, avant qu’elles ne se généralisent à tout le réseau ». Des batteries Tesla ? « Pourquoi pas ? mais si les batteries lithium-ion de Tesla sont très compactes, idéales pour des voitures, d’autres technologies existent pour le stockage stationnaire », confie l’ingénieur. Les lithium-ion sont en effet désavantagées par rapport aux batteries plus imposantes, mais moins chères, comme celles au sodium-soufre ou encore les nouveaux volants cinétiques. « Pour un immeuble, il vaut clairement mieux avoir une grosse batterie située en sous-sol plutôt qu’un PowerWall par appartement ». \\

Tesla miserait donc sur l’emballement du solaire avec son PowerWall. N’oublions pas un dernier détail : Elon Musk est également l’un des présidents de SolarCity, une entreprise conceptrice de… panneaux solaires. Le plan de Tesla apparaît donc en pleine lumière : la firme mise sur une révolution rapide de l’énergie, qu’elle tente d’accélérer. Le destin de Tesla peut ainsi être rapproché de celui de ses batteries : aujourd’hui chargées à bloc, elles doivent être employées rapidement sous peine d’être déchargées et remplacées. 
\end{document}


