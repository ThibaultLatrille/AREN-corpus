\documentclass[8pt]{article}

\usepackage[T1]{fontenc}
\usepackage[utf8]{inputenc}
\usepackage{graphicx}
\usepackage{lmodern}
\usepackage{amsmath}
\usepackage{xfrac}
\usepackage{amsthm}
\usepackage{listings}
\usepackage{enumerate}
\usepackage{amssymb}
\usepackage{cancel}
\usepackage{amsfonts}
\usepackage{float}
\usepackage{fullpage}
\usepackage{pdfpages}
\usepackage{tcolorbox}

\DeclareUnicodeCharacter{200A}{ } 
\renewcommand*\contentsname{Table des matières}

\PassOptionsToPackage{hyphens}{url}\usepackage{hyperref}

\usepackage{listings}
\author{ControverSciences\textit{ et al} }
\title{Projet AREN - Corpus de ressources \\   Sortie du glyphosate.}
\date{14 Avril 2019}

\begin{document}
\maketitle

Ce corpus de ressources et de textes à débattre étudiera la sortie et l'interdiction du glyphosate. Les sujet est abordé par la définition et l'utilisation actuelle du glyphosate, ainsi que ses effets connus et le plan de sortie prévu par le gouvernement (\ref{sec:glyphosate}). Ensuite, une tribune du journal \textit{Le Monde}, publiée par le sociologue David Demortain, aborde le fonctionnement des agences d’expertise officielles et leurs non-indépendances (\ref{sec:expertise}). Enfin un article complet et long rédigé par \textit{L'Institut National de la Recherche Agronomique} permet à celles et ceux qui souhaitent d'approfondir les enjeux actuels de la sortie du glyphosate. (\ref{sec:pesticide}).\\


Ce corpus de préparation permet d'aborder les différentes facettes du débat sur la sortie et l'interdiction du glyphosate en France. Deux textes sont proposés pour le débat, et peuvent être coupés afin d'extraire les paragraphes suffisants au débat. Le premier est en article du journal \textit{Le Monde} sur le renoncement du gouvernement à sortir du glyphosate en 3 ans (\ref{sec:interdiction}). Le second est un article du journal \textit{Midi Libre} sur la présence de glyphosate dans l'urine d'habitants de l’Hérault (\ref{sec:pisse}). 

\tableofcontents
\newpage
\section{Corpus de ressources}
\subsection{Glyphosate~: le plan de sortie}
\label{sec:glyphosate}

\begin{itemize}
	\item \textbf{Lien~: }  \url{https://agriculture.gouv.fr/glyphosate-le-plan-de-sortie} 
	\item \textbf{Auteur~: } agriculture.gouv.fr
	\item \textbf{Date~: } 13 Mars 2019 
	\item \textbf{Source~: } agriculture.gouv.fr est produit, édité et géré par le ministère de l’Agriculture et de l'Alimentation. Il s'agit de l’administration chargée de la politique agricole, halieutique, alimentaire et forestière. Il organise aussi l’enseignement et la recherche dans ces domaines.
	\item \textbf{Résumé~: } Le glyphosate est un désherbant non sélectif, c'est-à-dire agissant sur les différentes adventices des espèces cultivées. Produit uniquement par Monsanto dans un premier temps, sous le nom commercial de Roundup, il est depuis 2000 tombé dans le domaine public, et est désormais fabriqué et commercialisé par un grand nombre de sociétés.
\end{itemize}

\textbf{Qu'est-ce que le glyphosate~?}\\

Le glyphosate est un désherbant utilisé pour détruire ce que l'on appelle communément les mauvaises herbes, ou des plantes qui poussent dans un endroit ou elle n'ont pas été installées. Le glyphosate agit en bloquant la chaîne de synthèse des précurseurs d’acides aminés essentiels pour le fonctionnement de la plante, notamment pour la photosynthèse.
Cet herbicide à la propriété d’être total (il agit sur un mécanisme que tous les végétaux possèdent pour fonctionner) et systémique (il peut se déplacer dans la totalité du système de la plante, des tissus, jusqu’aux racines).
Produit uniquement par Monsanto dans un premier temps, sous le nom commercial de Roundup, il est depuis 2000 tombé dans le domaine public. Il est désormais fabriqué et commercialisé par un grand nombre de sociétés.\\

En matière d'usages non agricoles, l'utilisation du glyphosate par les collectivités dans les espaces ouverts au public est interdite en France depuis le 1er janvier 2017.\\

\textbf{Comment est utilisé le glyphosate en agriculture~?}\\

Avec plus de 9 100 tonnes de matière active consommées en France métropolitaine en 2016 (données Base Nationale des Ventes des Distributeurs), tous usages confondus, le glyphosate est utilisé, en agriculture, pour éliminer les végétaux des parcelles avant semis et sans travailler le sol, ou pour détruire la flore difficile~: plantes vivaces, invasives, allergènes ou toxiques.
En grandes cultures, le glyphosate est utilisé majoritairement entre la récolte d’une culture et le semis de la culture suivante.\\

\textbf{Les études disponibles sur le glyphosate}\\

Le 10 mars 2015, le Centre international de recherche sur le cancer (CIRC), placé auprès de l’Organisation mondiale de la santé, a publié un rapport concluant que le glyphosate devait désormais être classé cancérogène probable pour l’homme (classement 2A du CIRC). L’agence estimait notamment qu'il existe des preuves démontrant une association entre l'exposition au glyphosate et le développement de cancers tels que le lymphome non hodgkinien et le cancer du poumon.

La Commission européenne a proposé de renouveler l’autorisation du glyphosate dans l’Union Européenne pour une durée de 5 ans. La France s’est prononcée contre cette proposition, avec d’autres États membres.

L’Anses a été saisie par les ministères en charge de la Consommation, de la Santé, du Travail, de l’Écologie et de l’Agriculture, ainsi que par les associations de consommateurs UFC-Que Choisir et CLCV, afin de proposer au niveau européen des mesures appropriées lors de l’examen du projet de décision de renouvellement de l’approbation.\\

L’Anses a confié l’instruction de cette saisine à un groupe d’experts spécialisés en toxicologie, cancérogenèse, mutagenèse et épidémiologie, qui indique que~:
\begin{itemize}
	\setlength\itemsep{-0.25em}
	\item le niveau de preuve de cancérogénicité chez l’animal et chez l’homme ne permet pas de proposer un classement 1A ou 1B (cancérogène avéré ou présumé pour l'être humain) dans le cadre de l’application des critères du règlement (CE) n°1272/2008 (CLP)~;
	\item la classification en catégorie 2 (substances suspectées d'être cancérogènes pour l'homme, CLP) peut se discuter, sans que l’Agence ne puisse se prononcer en l’absence d’une analyse détaillée de l’ensemble des études.
\end{itemize}


Compte tenu de ces résultats, l’Agence estime nécessaire que le classement du glyphosate soit rapidement revu par l’Agence européenne des produits chimiques (ECHA).\\

En 2017, l’Institut national de la recherche  agronomique (INRA) a remis aux ministres concernés son rapport sur les usages et les alternatives au glyphosate dans l’agriculture française. Ce rapport répond à plusieurs objectifs~: analyser les usages du glyphosate, identifier les alternatives possibles avec leurs incidences économiques et organisationnelles ainsi les difficultés spécifiques à certaines filières ou modes de production.  Il propose enfin des mesures d’accompagnement de la transition vers des systèmes de production sans glyphosate.\\

\textbf{Plan de sortie du glyphosate~: le dispositif}\\

Le Gouvernement a engagé un plan d’action global pour la réduction de l’utilisation des produits phytosanitaires avec un objectif de -25\% en 2020 et -50\% en 2025, et a décidé de mettre fin aux principaux usages du glyphosate d’ici trois ans au plus tard et d’ici cinq ans pour l’ensemble des usages, tout en précisant que les agriculteurs ne seraient pas laissés dans une impasse.  \\

Les mesures du plan de sortie~:
\begin{itemize}
	\setlength\itemsep{-0.25em}
	\item la création d’un centre de ressources pour rendre accessible à l’ensemble de la profession agricole les solutions existantes pour sortir du glyphosate. La plate-forme est en ligne depuis le 1er février 2019~;
	\item le renforcement des actions d’accompagnement dans le cadre du programme Écophyto pour diffuser les solutions et trouver de nouvelles alternatives pour les usages pour lesquels il demeurerait des impasses~;
	\item la mobilisation des réseaux territoriaux des chambres d’agriculture, et de l’enseignement agricole pour faire connaître et promouvoir les alternatives au glyphosate sur l’ensemble des territoires avec l’appui des CIVAM et des coopératives agricoles~;
	\item le suivi des quantités vendues et utilisées des produits contenant du glyphosate afin de faire toute la transparence sur les usages en publiant régulièrement les données et en les mettant à disposition du public~;
	\item la valorisation de ce travail au niveau européen avec les pays volontaires pour s’engager comme la France dans une sortie rapide du glyphosate. Une première réunion avec ces pays se tiendra en marge du prochain Conseil Agriculture en juillet.
\end{itemize}

L’animation et le suivi de ce plan d’actions sont confiés à une  «~Task Force~» regroupant les deux ministères, l’INRA, l’Acta et de l’APCA. Elle est présidée par le Préfet Pierre-Étienne Bisch, nommé le 1er décembre 2018, coordinateur interministériel du plan de sortie du glyphosate, ainsi que du plan d'actions sur les produits phytopharmaceutiques et une  agriculture moins dépendante aux pesticides rendu public le 25 avril 2018.\\

Le coordinateur rendra compte des actions engagées et des progrès accomplis tous les trois mois aux ministres et aux parlementaires.\\


\textbf{Révision des autorisations de mise sur le marché par l'Anses}\\

La Commission européenne a renouvelé l'approbation du glyphosate pour 5 ans. À la suite de cette décision, l'Anses va réexaminer les demandes de renouvellement des autorisations de mise sur le marché (AMM) des produits phytopharmaceutiques contenant du glyphosate en prêtant une attention particulière au potentiel génotoxique de chacune des préparations, de même que la conformité des coformulants utilisés.

Les objectifs de protection des eaux souterraines dans les zones vulnérables, notamment en ce qui concerne les utilisations non agricoles, de protection des opérateurs et de protection de la biodiversité pourront conduire le cas échéant à des restrictions d'utilisation.

L'Anses, avec l'appui de l'INRA, réalisera également une évaluation comparative des produits restant autorisés d'ici la fin de l'année 2020. Elle sera basée sur les alternatives constituées par des produits de biocontrôle ou des produits à faible risque, et sur les méthodes non chimiques de prévention et de lutte d’usage courant.

À l'issue de cette évaluation comparative, ne pourront être maintenues que les utilisations sans alternative, en situation d’impasse.\\

Conformément au règlement d'approbation du glyphosate, l'Anses s'apprête également à restreindre dans toutes les AMM les usages en pré-récolte aux traitements en tâche, c'est-à-dire localisés, et donc à retirer l’ensemble des usages herbicides généraux avant récolte.

\newpage
\subsection{Glyphosate~: «~L’expertise scientifique n’a pas été décisive~»}
\label{sec:expertise}

\begin{itemize}
	\item \textbf{Lien~: }  \url{https://www.lemonde.fr/sciences/article/2018/02/06/le-glyphosate-ou-le-piege-de-la-privatisationde-l-expertise-publique_5252331_1650684.html} 
	\item \textbf{Auteur~: }  David Demortain; sociologue à l’INRA, au sein du Laboratoire Interdisciplinaire Sciences Innovations Sociétés.
	\item \textbf{Date~: } 6 février 2018
	\item \textbf{Source~: } Le Monde est un journal  français à la ligne éditoriale parfois présentée comme étant de centre gauche, bien que cette affirmation soit récusée par le journal lui-même, qui revendique un traitement non partisan. Le journal est édité par le groupe Le Monde, détenu à 72,5\% par la société Le Monde libre, elle-même contrôlée à parité par les hommes d'affaires Xavier Niel et Matthieu Pigasse. Il bénéficie de subventions de la part de l'État français. 
	\item \textbf{Résumé~: } Les agences d’expertise officielles sont prisonnières d’une économie de la connaissance façonnée par les firmes agrochimiques.
\end{itemize}

La saga sur la réautorisation du glyphosate, principe actif de nombreux désherbants, s’est terminée en novembre 2017 de la plus imprévue des manières. Le revirement de deux Etats membres a permis d’avaliser la proposition de la Commission de réautoriser pour cinq ans le pesticide le plus utilisé au monde.\\

La fin de la saga est inattendue parce que l’affaire du glyphosate s’était nouée depuis 2015 sur le terrain de l’évaluation de la toxicité de la substance, et notamment de son caractère cancérogène – probable, selon le Centre international de recherche sur le cancer (CIRC). Mais elle ne s’est pas dénouée sur ce terrain scientifique.\\

Comme on le sait, les agences du circuit d’expertise officielle, European Food Safety Authority (EFSA) et European Chemicals Agency (ECHA), n’ont pas abondé dans le sens du CIRC, ce qui leur a valu leur lot d’accusations de complaisance vis-à-vis des données des fabricants de Roundup. C’est ce que les députés européens veulent ­tirer au clair~: ils viennent de créer une commission spéciale sur le rôle des agences européennes et les failles ­potentielles dans l’évaluation scientifique des pesticides.\\

L’avis des agences n’a pourtant pas emporté la décision de la Commission européenne et des Etats membres. Les durées d’autorisation proposées par la Commission (quinze ans, puis dix, puis cinq) n’ont rien à voir avec une mesure de la toxicité du produit ou sa durée de persistance dans les sols. Aucune étude scientifique décisive n’a motivé le changement de position de la Pologne ou de l’Allemagne – plus liée, semble-t-il, au rachat de Monsanto par Bayer qu’à un calcul de risque cancérogène. La contre-expertise de l’avis du CIRC par les agences ne semble pas non plus avoir empêché les gouvernements ­opposés à la réautorisation du glyphosate de continuer à penser que le produit était plus néfaste que bénéfique.\\

Contre toute attente, l’expertise scientifique n’a donc pas été décisive. Pour le comprendre, il faut regarder de plus près l’origine de la connaissance que les agences mobilisent, et la ­manière dont elle est produite. Les ­experts scientifiques ne peuvent ­expertiser que ce que la connaissance scientifique disponible documente, et qu’ils choisissent de prendre en considération. Or il est probablement crucial de reconnaître aujourd’hui que la connaissance sur les produits chimiques et leurs risques a son économie.\\

Cette connaissance est consignée dans des études produites par les ­départements de recherche \& développement des firmes agrochimiques, ou dans des laboratoires privés sous contrat, qui réalisent des études aux protocoles contrôlés, interprétables par les agences. Ce ne sont pas des institutions de recherche publique, avec le loisir d’examiner des hypothèses alternatives ou d’engager des études sur des protocoles nouveaux, ­coûteux ou longs. Les agences ne donnent pas accès à ces études, notamment parce que les firmes ont réussi à faire admettre depuis une dizaine d’années environ qu’elles constituent une propriété intellectuelle.\\

La recherche publique, elle-même, a son économie, très contrainte, que les budgets publics peinent à financer. Que des experts issus d’universités soient en situation de conflit d’intérêts parce qu’ils mènent des études pour des firmes, travaillent avec elles dans des réseaux européens, voire qu’ils publient des articles prérédigés par les firmes, est le symptôme de la marchandisation de la recherche universitaire, et de la rareté relative des recherches de plus long terme, d’intérêt public.\\

Lors même que ces études sont disponibles, elles ne rentrent pas forcément dans le lot des connaissances considérées par les experts pour évaluer les produits. L’évaluation des risques qu’ils pratiquent est une discipline en soi, avec ses critères de fiabilité, de qualité et d’applicabilité des études. Elle a ses protocoles préférés, ses manières de mesurer la toxicité, ses formules de calcul.\\

\textbf{Des organismes internationaux discrets}\\

Ces méthodes sont le plus souvent forgées dans des organismes internationaux discrets, où sont sur-représentés les scientifiques des firmes et les experts habitués à l’évaluation de leurs produits. C’est grâce à ces méthodes que les études peuvent être répliquées, et que les évaluations sont rendues in fine plus ­robustes. Mais c’est aussi en leur nom que des études peuvent être déclarées non pertinentes, et la base des ­connaissances prises en compte dans l’expertise réduite d’autant. \\

Les agences d’expertise européennes sont un élément de cette économie de la connaissance. Elles en ­recueillent les produits, mais n’influencent que marginalement son fonctionnement. On ne leur en a pas donné la mission ni les moyens. Si ­elles restreignent leur travail à la validation d’études préformatées – tel le directeur de l’EFSA assimilant dans un entretien le travail des experts à la ­vérification d’informations d’état civil pour délivrer un passeport – et qu’elles disqualifient dans le même temps la pertinence scientifique des avis d’autres acteurs, y compris des ONG, elles seront prisonnières de cette économie de la connaissance.\\

Il y a fort à parier alors que leurs avis continueront de créer la controverse autant que d’éclairer les décisions politiques.

\newpage
\subsection{Le glyphosate, un pesticide parmi les autres~?}
\label{sec:pesticide}

\begin{itemize}
	\item \textbf{Lien~: }  \url{http://www.inra.fr/Chercheurs-etudiants/Systemes-agricoles/Tous-les-dossiers/Le-glyphosate-un-pesticide-parmi-les-autres} 
	\item \textbf{Auteur~: }  Pascale Mollier, rédactrice en chef canal scientifique Web INRA.
	\item \textbf{Date~: } 28 mai 2018
	\item \textbf{Source~: } L'Institut national de la recherche agronomique (INRA) est un organisme français de recherche en agronomie fondé en 1946, sous la double tutelle du ministère chargé de la Recherche et du ministère chargé de l’Agriculture.
	Premier institut de recherche agronomique en Europe et deuxième dans le monde en nombre de publications en sciences agricoles et en sciences de la plante et de l'animal, l'INRA mène des recherches finalisées pour une alimentation saine et de qualité, pour une agriculture durable, et pour un environnement préservé et valorisé. 
	\item \textbf{Résumé~: } Ce dossier donne des éléments informatifs sur les recherches menées à l’INRA concernant différents aspects de la problématique du glyphosate.
\end{itemize}

\subsubsection{Introduction}

Le glyphosate est un herbicide total foliaire systémique, c’est-à-dire un herbicide non sélectif absorbé par les feuilles et ayant une action généralisée. Il a été homologué en 1974, sous brevet Monsanto et sous la marque Roundup. Il est passé dans le domaine public en 2000 et est produit désormais également par d’autres firmes. C’est actuellement l’herbicide le plus employé dans le monde. En France, il représente 9100 tonnes produites en 2016.\\

Pour améliorer sa pénétration dans les feuilles, le glyphosate est formulé avec des adjuvants tensioactifs, qui permettent de stabiliser ses performances tout en restant à des doses réduites. Au niveau réglementaire, il existe des procédures d’autorisation de mise sur le marché différentes pour la substance active seule et pour les produits formulés. La première est ressort de l’UE, la deuxième est du ressort des États membres qui utilisent les produits.\\

L’autorisation précédente du glyphosate a pris fin le 30 juin 2016. Le 27 novembre 2017, l’UE a voté une prolongation de 5 ans de la substance active. La France pour sa part s’est prononcée pour une prolongation de seulement trois ans des produits commerciaux. De plus en France, le glyphosate est interdit dans les espaces publics depuis le 1er janvier 2017 et le sera pour les particuliers le 1er janvier 2019.\\

Dans ce contexte, et à la demande des ministères en charge de l’Agriculture et de l’Environnement, l’INRA a piloté une étude qui analyse les leviers et freins spécifiques à la suppression du glyphosate en agriculture et le replace dans la perspective plus globale de diminution des pesticides du plan Ecophyto.

\subsubsection{Glyphosate et santé humaine~: pas de certitude}
C’est un véritable procès, avec des témoins à charge et à décharge, qui est instruit contre le glyphosate par rapport à la santé humaine. Comment s’y retrouver dans cet imbroglio dont les médias se font l’écho~? Trois questions à Gérard Pascal, expert en santé de l’alimentation humaine.\\

Au niveau européen, les substances actives des pesticides sont autorisées pour des périodes de 10 ou 15 ans, renouvelables, sur la base d’un dossier constitué par un Etat membre désigné comme rapporteur. C’est l’Allemagne qui a rapporté le dossier du glyphosate pour la période d’autorisation de 1998 à 2016. C’est aussi l’Allemagne qui a préparé un rapport en 2013, puis un addendum en 2015 en vue du renouvellement de cette autorisation.\\

L’agence européenne EFSA a rendu en octobre 2015 un avis sur la base de ces rapports. Huit autres agences ont rendu des avis convergents~:«~Le glyphosate est peu susceptible de présenter un risque cancérigène pour l’homme au travers de la chaine alimentaire~». L’Anses estime «~ne pas pouvoir proposer un classement 1B dans le cadre de l’application du règlement… et ne pas pouvoir se prononcer sur un classement en 2A~». \\

Au contraire, le CIRC publie en mars 2015 une monographie sur cinq molécules, dont le glyphosate, qui estime que le glyphosate est «~probably carcinogenic to humans~» et préconise le classement dans le groupe 2A.
Les instances d’évaluation semblent ainsi se partager en deux camps opposés. Comment expliquer une telle divergence~? Nous avons posé la question à Gérard Pascal.\\

\textbf{Comment s’y retrouver dans ces évaluations divergentes~?}\\

Gérard Pascal~: On ne peut pas conclure la même chose quand on ne regarde pas les mêmes éléments…Sur la quantité d’études qui existent sur le glyphosate, les deux parties n’ont pas retenu les mêmes (voir encadré 1). Il y a d’abord les modèles testés~: pour une estimation du risque génotoxique pour l’homme, les agences n’ont considéré que les études menées in vivo sur les mammifères, alors que le CIRC a pris aussi en compte des études sur des non mammifères. Ensuite, il y a le produit testé~: la substance active seule pour les agences, les produits commerciaux pour le CIRC, c’est-à-dire le glyphosate combiné à un adjuvant. C’est certes plus proche de la réalité, mais cela complique les choses, car les effets observés peuvent provenir de l’adjuvant. C’est le cas par exemple de la POE-tallowamine, qui, à forte dose, est suspectée d’être responsable de cas d’empoisonnement chez des agriculteurs. La France a d’ailleurs interdit 126 préparations de glyphosate contenant cet adjuvant. Troisième point~: la qualité des études. Les agences ont écarté certaines études positives minoritaires car elles n’étaient pas réalisées selon les standards internationaux. Enfin, et surtout, la démarche est différente~: le CIRC, et c’est son rôle, estime le danger, tandis que les agences, qui statuent en vue de la réglementation, estime le risque, c’est-à-dire le danger combiné à l’exposition.\\

Au final, s’il n’est pas contestable que le glyphosate est un produit dangereux, par essence même, puisque c’est un produit qui tue les «~mauvaises herbes~», le risque qu’il présente pour l’homme, qui dépend des conditions de son utilisation, n’est, à mon sens, pas établi scientifiquement à ce jour.\\

\textbf{Il y a eu beaucoup de suspicion de conflits d’intérêt dans cette affaire. Comment s’y retrouver~?}\\

G. P.~: Le problème majeur que soulève cette affaire, c’est la perte de confiance dans l’expertise scientifique. Les agences ont été accusées en particulier d’utiliser des articles écrits par des scientifiques payés par Monsanto, les fameux «~Monsanto papers~». En fait, il s’agit de revues qui recensent des études, et non pas d’études primaires qui rapporteraient des résultats d’expériences. De plus, les scientifiques qui signent ces revues, même rétribués, mettent en jeu leur crédibilité et ont donc intérêt à fournir un travail de qualité. Cependant, le principe est contestable en soi et ce type de revues n’est plus accepté par les éditeurs dès lors qu’il y a suspicion de conflits d’intérêt. Ce que l’on pourrait reprocher aux agences, c’est un certain manque de transparence. Par exemple, on n’a pas facilement accès à la liste des experts, qui sont nommés par les Etats membres rapporteurs dans le cas de l’évaluation initiale, ni à leur déclaration d’intérêt. Sur ce point, c’est plus rigoureux dans le domaine des additifs alimentaires, qui était le mien, que dans celui des pesticides.\\

Quant au CIRC, il n’est pas exempt non plus de suspicion. Le président de son comité a, par exemple, occulté une étude épidémiologique américaine très attendue, mais qui n’était pas encore publiée, appliquant en cela strictement les règles en vigueur au CIRC. Cependant, les résultats de cette étude étaient connus et ne montraient aucune relation entre cancer et glyphosate sur plus de 54 000 agriculteurs suivis pendant 20 ans (9). Plus préoccupant, un scientifique influent a initié une pétition signée par une soixantaine de collègues pour protester contre l’avis de l’EFSA. Or, ce scientifique est accusé d’avoir reçu des subsides de la part d’avocats en procès contre Monsanto, ce qu’il n’a pas nié.\\

\textbf{Puisqu’il y a désaccord sur l’évaluation du glyphosate, ne doit-on pas le retirer en application du principe de précaution~?}\\

G. P.~: A mon sens, une application intelligente du principe de précaution se fait à partir d’une analyse de risques, et non pas sur l’existence d’un danger. Sinon, on serait conduit à interdire beaucoup de choses, y compris l’automobile… Le glyphosate, de par ses avantages, a été très (trop~?) utilisé dans le monde. De plus, dans l’opinion publique, il est lié aux OGM résistants aux herbicides. Je pense donc qu’il est condamné à terme, non pas pour des raisons scientifiques en lien avec la santé humaine, mais à cause de la perception de la société, qui rejette ainsi un certain modèle d’agriculture. Ce rejet reflète une inquiétude générale par rapport à un environnement imprégné de produits chimiques dont on ne connaît pas tous les effets, comme les effets chroniques, à long terme ou cocktails.\\


\subsubsection{Glyphosate et environnement~: encore matière à recherche}

Dans l’environnement, les pesticides sont dégradés par des microorganismes, générant des produits de transformation dont certains peuvent poser problème. Comment gérer ce risque qui apparaît a posteriori~? Trois questions à Fabrice Martin-Laurent, écotoxicologue à l’INRA.\\

Expert auprès de l’AFNOR et de l’ISO pour l’écotoxicologie terrestre, Fabrice Martin-Laurent travaille depuis des années sur le devenir des pesticides, dont le glyphosate, dans l’environnement, en particulier le sol et l’eau. Avec ses collaborateurs, il a montré le lien qui existe entre la vitesse de dégradation des pesticides dans un milieu et l’abondance et la composition de la communauté microbienne qui les dégradent. Une partie des microorganismes présents dans le sol et les sédiments ont en effet une activité «~épuratrice~», grâce à leur équipement enzymatique qui dégrade et minéralise les pesticides afin de les utiliser comme une source de nutriments et d’énergie pour leur croissance.\\

Les travaux de Fabrice Martin-Laurent mettent en évidence l’importance de préserver les structures paysagères qui favorisent cette action microbienne (bandes enherbées pour les sols, zones humides pour les sédiments).\\

\textbf{Comment étudiez-vous le devenir du glyphosate dans l’environnement~?}\\

Fabrice Martin-Laurent~: Nous étudions comment les pesticides se transforment dans le sol et dans l’eau sous l’action des microorganismes. Nous avons trois principaux modèles de pesticides représentatifs, dont le glyphosate. Nous mesurons le taux de transformation du glyphosate marqué au carbone 14, c’est-à-dire son degré de minéralisation par les microorganismes du sol.\\

Nous avons observé que, dans de nombreux sols agricoles, le taux de minéralisation pouvait atteindre 75\% de la quantité initialement ajoutée. Alors que dans l’eau, la biodégradation est souvent incomplète, conduisant à l’accumulation d’un métabolite intermédiaire, l’AMPA. L’AMPA est le métabolite le plus fréquemment détecté dans les cours d’eau français. A priori, il n’est pas toxique pour les microorganismes du sol, mais nous n’avons pas beaucoup de données à ce sujet. Son accumulation pose, sinon une question environnementale, du moins une question de science~: pourquoi n’est-il pas dégradé totalement alors que de nombreux travaux scientifiques montrent que ce composé est théoriquement biodégradable~? Pour une majorité de produits phytosanitaires, on a souvent une succession de microorganismes dégradants qui se relaient pour assurer la minéralisation complète d’un composé. Nous ne savons pas s’il en est de même pour le glyphosate.\\
\newpage
Le cas du glyphosate est un exemple qui illustre plus généralement notre difficulté à suivre le devenir et à évaluer la toxicité des produits de transformation (= métabolites) qui se forment dans l’environnement à partir des pesticides et des autres composés organiques produits par les activités humaines.\\

\textbf{Comment pourrait-on améliorer l'évaluation environnementale des pesticides~?}\\

F. M-L~: Le problème est que l’on ne connaît pas les produits de transformation d’un pesticide avant qu’il ne soit commercialisé et appliqué sur les parcelles agricoles. C’est pourquoi, dès la mise en marché, il faudrait commencer à identifier ces produits et à étudier leurs propriétés toxicologiques. En effet, pour certains produits, les effets négatifs sur l’environnement ne sont mis en évidence que 20 ou 30 ans après leur commercialisation. Ces impacts sur l’environnement constituent une des causes principales du retrait des pesticides, et, en général, ce sont des travaux issus de la recherche publique qui donnent l’alerte.\\

Il faudrait donc que l’évaluation «~post-commercialisation~» soit considérée comme aussi importante que l’évaluation avant commercialisation, qui génère déjà des coûts très importants pour les industriels. La question est de savoir qui doit prendre en charge cette évaluation a posteriori.\\

\textbf{Finalement, a-t-on vraiment les moyens d’évaluer correctement les risques des pesticides~?}\\

F. M-L~: Il est certain qu’il est difficile d’évaluer tous les effets écotoxicologiques potentiels des produits phytosanitaires et de leurs produits de transformation~: tous les effets chroniques, tous les effets cocktails, tous les effets des adjuvants, etc. En ce qui concerne les adjuvants, qui sont associés à la substance active pour formuler le produit commercial, le choix est fait à partir des listes de la réglementation REACH, qui définit le niveau de risque d’un grand nombre de produits. Mais une fois associé à la substance active dans une formulation commerciale, on ne connait pas leur devenir, dans le sol, l’eau ou les organismes. Pour une meilleure harmonisation européenne, il faudrait que l’évaluation des produits formulés soit pilotée par l’EFSA et non par les Etats membres, comme c'est le cas pour les substances actives.\\

Ces limites de l’évaluation posent question~: jusqu’à quel point doit-on accepter le risque écotoxicologique et environnemental lié à l’utilisation des produits phytosanitaires~? Si la procédure d’homologation était modifiée comment pourrait-on la financer~? Faudrait-il répercuter ce surcoût sur tous les maillons de la chaîne, du producteur au consommateur~? Si on estime que certains produits phytosanitaires sont indispensables pour l’agriculture, alors il faut se donner les moyens pour que ces produits soient sûrs pour préserver leur utilisation. Ces questions nécessitent de repenser de fond en comble le type d’agriculture que la société souhaite avoir.\\

Il y a néanmoins un point important qui pourrait être amélioré à moindre coût, c’est la transparence des évaluations. Il faudrait, d’une part que le choix des laboratoires évaluateurs soit arbitré par l’EFSA, et non par les industriels demandeurs, et d’autre part que ces dossiers d’évaluation soient facilement accessibles. Cela assoirait la crédibilité de l’ensemble de la procédure qui implique les autorités européennes, les firmes, les laboratoires sous-traitants et de nombreux experts scientifiques.

\subsubsection{ Glyphosate en agriculture~: pas irremplaçable, sauf… }

Une étude de l’INRA publiée en 2017 analyse les alternatives pour sortir du glyphosate en France selon les systèmes de culture. Les combinaisons de techniques nécessaires peuvent amener à reconcevoir plus globalement certains de ces systèmes, tandis que d’autres risquent de se trouver, au moins momentanément, dans une impasse technique. Trois questions à Xavier Reboud, coordinateur de l’étude.\\

En novembre 2017, l’INRA publie un rapport intitulé «~Usages et alternatives au glyphosate dans l’agriculture française~», à la demande des quatre ministères en charge de l’Agriculture, de l’Environnement, de la Santé et de la Recherche. Après avoir enquêté sur les pratiques culturales actuelles, les auteurs du rapport précisent qu’il existe déjà une agriculture sans glyphosate en France~: 2/3 des parcelles n’en reçoivent pas, même en grandes cultures. Le rapport analyse les leviers qui permettent de se passer de cet herbicide très efficient et identifie des situations où son interdiction resterait problématique, voire susceptible de conduire à des impasses techniques.\\

\textbf{Quelle est votre vision de cette future interdiction du glyphosate et de ses conséquences pour l’agriculture~?}\\

Xavier Reboud~: L’interdiction du glyphosate s’inscrit dans un mouvement plus global pour une agriculture moins consommatrice de pesticides. C’est un choix de société. Mais justement, c’est toute la société qui est concernée, y compris le consommateur qui, même citadin, doit comprendre les logiques de production. Un exemple très simple~: une des solutions pour se passer d’herbicide en verger consiste à entretenir de l’herbe entre les arbres pour contrôler les adventices. Conséquence~: on aura des fruits plus petits, car l’arbre profitera de moins de ressources lors du remplissage des fruits, du fait de la présence de l’herbe. Or, le marché français déclasse souvent les pommes de moindre calibre. De même, on aura des fruits moins parfaits si on utilise des méthodes biologiques contre les ravageurs, parce qu’elles sont moins infaillibles que les pesticides. Donc, des fruits plus petits, avec des défauts, parfois plus chers aussi.\\

Autre exemple, on a pu calculer pour le vin que, sans glyphosate, le surcoût pour le viticulteur est d’environ 23 centimes d’euros par bouteille. Pour un vin de pays, ce surcoût, à l’échelle d’hectolitres, peut faire perdre le marché au producteur par rapport à la concurrence espagnole ou chilienne. C’est donc un gros enjeu pour le producteur, alors que le surcoût pour le consommateur est modeste.\\

\textbf{L’impulsion doit-t-elle venir des pouvoirs publics~?}\\

X. R.~: Le rôle des pouvoirs publics est de restaurer la confiance et de promouvoir la traçabilité. Le consommateur pourrait accepter de payer un surcoût à condition qu’il soit informé sur le produit. Il pourrait donc y avoir, à l’instar du label Bio, une forme de label «~sans glyphosate~», au moins pendant la période de transition. D’autres mesures incitatives sont déjà en place en France pour diminuer l’usage des pesticides~: les Certificats d’Economie de Produits Phytopharmaceutiques (CEPP), qui incitent les vendeurs de pesticides à développer des conseils et des matériaux pour en utiliser moins, sous peine d’amendes. Il y a aussi la redevance pour pollution diffuse, acquittée par les distributeurs de pesticides et qui finance en partie le plan Ecophyto.\\

Au niveau de l’Europe également, il existe un pilier, dit de «~verdissement~» de la PAC, avec des aides augmentées pour les infrastructures écologiques (haies, murets, bandes enherbées) et pour le maintien des prairies. Il y a aussi des aides pour encourager la diversification, dont le montant serait indexé sur le nombre de cultures dans la rotation.\\

\textbf{Que peut apporter la recherche dans ce contexte~?}\\

X. R.~: La recherche doit être pourvoyeuse de connaissances, d’outils, d’indicateurs opérationnels. L’INRA ne reste pas dans une posture d’expertise distante, mais, par son implication dans Ecophyto et le réseau Dephy, l’Institut se trouve souvent au cœur des expérimentations, en contact avec les agriculteurs. Les questions de recherche sont nombreuses et sont travaillées depuis longtemps à l’INRA~: conception de systèmes agricoles innovants économes en pesticides et évaluation de ces systèmes.\\

Plus spécifiquement, par rapport à l’interdiction du glyphosate, les recherches pourront se focaliser sur les leviers agronomiques, en particulier les couverts d’interculture (couvrants et faciles à détruire, par exemple sensibles au gel) et les méthodes de désherbage mécanique. Un grand programme de l’ANR est consacré en 2018 à la robotique, pour concevoir des robots «~intelligents~» capables de détecter et de désherber sélectivement les zones infestées de mauvaises herbes.Le biocontrôle est aussi un axe à développer, d’autant qu’il existe à ce jour peu d’alternatives biologiques aux herbicides.
Enfin, il faut approfondir le champ socioéconomique et analyser les impacts de la disparition du glyphosate, même si ce n’est pas facile. Pour faciliter le transfert, il y aura à rédiger des fiches-action CEPP dédiées à la réduction du glyphosate.

\subsubsection{  Se passer de pesticides~: des solutions à conjuguer  }
L’interdiction à venir du glyphosate est-elle en rupture avec la stratégie gouvernementale de réduction progressive des pesticides~? Trois questions à Christian Huyghe, directeur scientifique Agriculture à l’INRA.\\

L’interdiction à venir d’un pesticide emblématique tel que le glyphosate est-elle en rupture avec la stratégie de réduction progressive des pesticides suivie par le gouvernement depuis 2008, dans le cadre des plans Ecophyto 1, puis Ecophyto 2~? Nous avons posé la question à Christian Huyghe, directeur scientifique Agriculture à l’INRA.\\

\textbf{Interdire les pesticides, à l’instar de ce qui se profile pour le glyphosate, est-ce une bonne stratégie~?}\\

Christian Huyghe~: On peut appréhender l’interdiction du glyphosate comme un déclencheur de changement. Pour tous les pesticides de synthèse, qui ont, de par leur efficacité, un impact fort sur le milieu, la problématique est la même~: tant qu’on ne les interdit pas, ou qu’on les substitue entre eux, on n’induit pas de changement profond. Le glyphosate, molécule exemplaire d’efficacité, a contribué à l’installation de systèmes de culture plus simples, avec moins d’espèces cultivées, des rotations plus courtes, des exploitations plus grandes, des productions bien valorisées à l’aval. Toute la chaîne, du champ au produit final, s’est construite dans cette logique, selon un processus que l’on qualifie parfois de «~verrouillage~», ce qui sous-entend qu’il y a une clé… Je parlerais plutôt de situation d’équilibre que les acteurs de la chaîne maintiennent entre eux. Pour changer, il faut que tous ces acteurs bougent en même temps. Il n’y a donc pas une seule clé. Pour remplacer une molécule comme le glyphosate, il faut combiner plusieurs leviers d’efficacité partielle~: on va forcément vers plus de complexité.\\

\textbf{Quels sont les principaux leviers pour se passer de pesticides~?}\\

C. H.~: Il faut d’abord favoriser la prophylaxie, très active en élevage, mais un peu oubliée pour les cultures, c’est-à-dire faire baisser la pression des ravageurs et faire baisser le stock de graines d’adventices dans le sol. Pour cela, il faut favoriser les régulations biologiques à l’échelle des paysages, selon les principes de l’agroécologie.\\

Un des grands leviers est le biocontrôle. On sait déjà utiliser certains macroorganismes alliés~: les carabes par exemple, ou les alouettes, qui consomment les graines du sol. Les trichogrammes, les coccinelles et les parasitoïdes relèvent également de cette catégorie. Autre exemple, on utilise en Chine des poulets pour éliminer les criquets avant leur envol. En matière de biocontrôle, il y a à mon sens deux fronts de science à explorer~: premièrement, l’environnement microscopique de la plante, présent dans le sol et sur les parties aériennes de la plante, ce que l’on appelle le «~phytobiome~», qu’il faut mieux connaître et utiliser. Deuxièmement, l’écologie chimique, c’est-à-dire d’une part les phéromones et kairomones, ces substances que perçoivent les insectes pour localiser leurs congénères ou leurs proies~; et d’autre part l’ensemble des composés volatils émis par les plantes qui permettent aux insectes ravageurs d’identifier et atteindre leurs cibles alimentaires. On peut jouer sur ces molécules et cet environnement pour contrer les ravageurs. Ce type d’action est sans doute plus efficace à grande échelle, ce qui implique une coordination entre exploitations, comme c’est le cas dans certains territoires viticoles qui ont mis en place la confusion sexuelle.\\

Un deuxième levier majeur est l’utilisation de variétés résistantes développées grâce à la génétique. On le voit bien en vigne avec les cépages résistants au mildiou et à l’oïdium qui changent fondamentalement le système en réduisant de 80\% l’usage de fongicides.\\

Troisième levier~: l’agriculture de précision, particulièrement importante pour les questions de désherbage. Un désherbage mécanique assisté par caméra et géolocalisation est précis à 2 cm près ! Même si aujourd’hui le désherbage mécanique est utilisé sur des parcelles entières, il pourrait être ciblé spécifiquement sur les «~tâches~» d’adventices vivaces géolocalisées.\\

On peut aussi éviter de désherber en combinant différentes cultures. On a de très beaux exemples de cultures sous couvert vivant~: l’association de colza avec des légumineuses gélives ou des associations blé-luzerne ou maïs-trèfle blanc.\\

\textbf{Quelles seraient les meilleures mesures incitatives pour impulser le changement~?}\\

C. H.~: Le retrait des pesticides est évidemment une mesure plus drastique qu’une invitation à les réduire.
Les taxer peut introduire des distorsions de concurrence entre pays.
Les Certificats d’Economie de Produits Phytopharmaceutiques (CEPP), qui établissent des «~fiches action~» pour proposer des solutions alternatives à des usages de pesticide, sont un très bon moyen de valoriser les leviers alternatifs et d’orienter la recherche. Ils sont également une incitation forte puisque chaque distributeur de pesticides, qualifié d’«~obligé~», doit acquérir un nombre de certificats proportionnel à ses volumes historiques de vente.\\

Côté consommateur, favoriser le consentement à payer passe par une bonne information. Comme on n’a pas de capteurs pour doser rapidement les résidus de pesticides dans les produits, il faudrait donner l’information sous forme d’IFT (indice de fréquence de traitements). Une étude dans le cas de la vigne montre que le consommateur serait prêt à payer un surcoût pour une baisse d’IFT.

\newpage
\section{Textes à débattre}
\subsection{Emmanuel Macron renonce à sa promesse d’interdire le glyphosate en 2021}
\label{sec:interdiction}

\begin{itemize}
	\item \textbf{Lien~: }  \url{https://www.lemonde.fr/politique/article/2019/01/25/le-president-renonce-a-sa-promesse-d-interdire-le-glyphosate-en-2021_5414363_823448.html} 
	\item \textbf{Auteur~: }  Rémi Barroux, journalist.
	\item \textbf{Date~: }25 janvier 2019
	\item \textbf{Source~: } Le Monde est un journal  français à la ligne éditoriale parfois présentée comme étant de centre gauche, bien que cette affirmation soit récusée par le journal lui-même, qui revendique un traitement non partisan. Le journal est édité par le groupe Le Monde, détenu à 72,5\% par la société Le Monde libre, elle-même contrôlée à parité par les hommes d'affaires Xavier Niel et Matthieu Pigasse. Il bénéficie de subventions de la part de l'État français. 
	\item \textbf{Résumé~: } En affirmant, jeudi, que l’objectif de sortie d’ici à trois ans n’était «~pas faisable~», le chef de l’Etat a fait un geste en direction de la FNSEA. 
\end{itemize}

En quelques mots prononcés lors d’un débat citoyen auquel il s’était invité, Emmanuel Macron a rouvert le délicat dossier de l’interdiction du glyphosate. Jeudi 24 janvier, en fin de journée, à Bourg-de-Péage (Drôme), le président de la République a déclaré que la France ne parviendrait pas à se passer totalement de cet herbicide controversé d’ici trois ans, un engagement qu’il avait pris personnellement.\\

«~Je sais qu’il y en a qui voudraient qu’on interdise tout du jour au lendemain. Je vous dis~: un, pas faisable, et ça tuerait notre agriculture. Et même en trois ans on ne fera pas 100\%, on n’y arrivera, je pense, pas~», a-t-il déclaré, tout en encourageant les «~productions alternatives~» pour ne plus utiliser cet herbicide. Répondant à l’interrogation d’un apiculteur qui évoquait la mort de ses abeilles, M. Macron a rappelé que la France s’était battue pour que l’homologation de cet herbicide, commercialisé en particulier par la firme Monsanto avec le Roundup, ne soit renouvelée que pour cinq ans, quand l’Union européenne en proposait quinze.\\

Indiquant que certains «~ne voulaient pas bouger du tout~», le chef de l’Etat a rappelé qu’un «~contrat de confiance~» allait être signé et qu’il fallait «~aider ceux qui bougent~». «~Il a été montré qu’il y avait des doutes. Il n’y a aucun rapport indépendant ou pas indépendant qui a montré que c’était mortel~», a-t-il aussi déclaré.\\

Alors que cette substance avait été jugée «~probablement~» cancérigène par le Centre international de recherche sur le cancer (CIRC), qui dépend de l’Organisation mondiale de la santé (OMS), en juillet 2015, l’OMS et la FAO (l’Organisation des Nations unies pour l’alimentation et l’agriculture) concluaient, un an plus tard~: «~Le glyphosate est peu susceptible de présenter un risque cancérogène pour l’homme à travers le régime alimentaire.~»\\

Avec ces déclarations, Emmanuel Macron fait un geste en direction de ceux qui critiquaient la future interdiction de l’herbicide – la Fédération nationale des syndicats d’exploitants agricoles (FNSEA) en particulier. Profitant de la crise des «~gilets jaunes~», le principal syndicat agricole avait dénoncé, le 23 novembre 2018, «~l’agribashing~». Dans un communiqué commun avec les Jeunes Agriculteurs, la FNSEA avait alors remis en question «~l’augmentation de la redevance pollution diffuse, les charges supplémentaires induites par la séparation du conseil et de la vente des produits phytosanitaires, les interdictions de produits de traitements sans solutions ni alternatives~».\\

\newpage
\textbf{«~Renoncement~»}\\

C’est «~le “L’environnement, ça commence à bien faire” d’Emmanuel Macron~», a commenté sur Twitter l’eurodéputé écologiste Yannick Jadot, en référence à la phrase prononcée par l’ancien président Nicolas Sarkozy au Salon de l’agriculture en 2011, remettant en cause les objectifs de réduction de consommation des pesticides décidés lors du Grenelle de l’environnement.\\

En 2018, le débat parlementaire autour du glyphosate, dans le cadre de la discussion de la loi agriculture et alimentation, avait été tendu. Certains députés, y compris LRM, comme Matthieu Orphelin (Maine-et-Loire), s’étaient battus, en vain, pour que l’objectif de l’interdiction du glyphosate soit inscrit dans la loi. Le gouvernement et sa majorité parlementaire avaient rejeté l’amendement, arguant que l’objectif de sortie d’ici à 2021, affirmé par le chef de l’Etat, était clair et qu’il n’était nul besoin de l’inscrire dans un texte.\\

La possibilité que cet engagement ne soit finalement pas respecté est un «~renoncement~», a réagi l’association de défense de l’environnement Générations Futures dans un communiqué publié dans la soirée~: «~Après le refus de l’interdiction dans la loi, cette déclaration sonne comme un renoncement à un réel objectif de sortie du glyphosate qui n’est pas acceptable.~»\\

«~Un rapport de l’INRA, en novembre 2017, indiquait que des alternatives au glyphosate existent déjà pour 90\% des surfaces agricoles. Dans ces conditions, il est bizarre d’affirmer, trois ans avant l’objectif, qu’on ne pourra pas l’atteindre~», explique François Veillerette, le directeur de l’association.\\

Les propos d’Emmanuel Macron sont d’autant plus incompréhensibles, selon lui, que le tribunal administratif de Lyon vient d’annuler, le 15 janvier, en application du principe de précaution, la décision d’autorisation de mise sur le marché du Roundup Pro 360, prise en mars 2017 par l’Agence nationale de sécurité sanitaire de l’alimentation, de l’environnement et du travail (Anses). «~Le président de la République doit revenir sur ses déclarations et réaffirmer l’importance de l’objectif de sortie du glyphosate~», conclut M. Veillerette.

\newpage
\subsection{Campagne héraultaise “Je pisse du glyphosate, et vous ?” : les résultats des tests sont tombés}
\label{sec:pisse}

\begin{itemize}
	\item \textbf{Lien~: }  \url{https://www.midilibre.fr/2019/04/12/campagne-heraultaise-je-pisse-du-glyphosate-et-vous-les-resultats-des-tests-sont-tombes,8125811.php} 
	\item \textbf{Auteur~: } Juliette Moreau Alvarez
	\item \textbf{Date~: } 12 avril 2019
	\item \textbf{Source~: } Midi libre est un journal quotidien régional français, fondé en 1944 dont le siège se trouve à Saint-Jean-de-Védas, près de Montpellier. Il est diffusé dans la moitié est de la région Occitanie. Il appartient depuis 2015 au Groupe La Dépêche basé à Toulouse. 
	\item \textbf{Résumé~: } Organisé à l’initiative du collectif Glypho 34, soixante-quatre habitants de l’Hérault ont testé leur urine pour y rechercher des traces de glyphosate. Verdict.
\end{itemize}

Après presque un mois d’attente, les résultats des tests d’urine de soixante-quatre habitants de l’Hérault sont tombés. Verdict~: leur taux de glyphosate est supérieur au seuil autorisé.\\

L’initiative menée par une dizaine de collectifs en France a été portée dans le département par Glypho 34, dans le cadre de la campagne nationale “Je pisse du glyphosate, et vous ?”. Les résultats s’inscrivent dans la moyenne nationale, soit 1 nanogramme par millilitre, le décuple du seul autorisé de 0,1 nanogramme par millilitre dans l’eau potable. Le taux le plus élevé des inscrits héraultais est tout de même de 3,29 nanogrammes par millilitre.\\

\textbf{Cas unique en France}\\

Sur la soixantaine d’inscrits, un retraité se distingue. Ses analyses affichent un taux de 0,08 nanogramme par millilitre. Il est le seul cas testé en France à se trouver en dessous du seuil autorisé dans l’eau potable. "C’est un habitant de Murviel, explique Gérard Pinsard, l’un des représentants du collectif citoyen Glypho 34. Nous allons le rencontrer et étudier son cas pour essayer de comprendre ce résultat étonnant." L’hypothèse du jeûne avant les tests est notamment avancée.\\

\textbf{Porter plainte}\\

L’objectif de soixante-deux des inscrits est de porter plainte contre les laboratoires et les présidents des organismes français et européens qui ont participé à la mise en vente du glyphosate. "Fin 2017, ils connaissaient le danger mais ils s’en sont fichés, dénonce Gérard Pinsard. On travaille avec un huissier qui collecte chaque plainte individuelle. On espère pouvoir les déposer le 17 mai."


\end{document}



