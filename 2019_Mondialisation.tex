\documentclass[8pt]{article}

\usepackage[T1]{fontenc}
\usepackage[utf8]{inputenc}
\usepackage{graphicx}
\usepackage{lmodern}
\usepackage{amsmath}
\usepackage{xfrac}
\usepackage{amsthm}
\usepackage{listings}
\usepackage{enumerate}
\usepackage{amssymb}
\usepackage{cancel}
\usepackage{amsfonts}
\usepackage{float}
\usepackage{fullpage}
\usepackage{pdfpages}
\usepackage{tcolorbox}

\DeclareUnicodeCharacter{200A}{ } 
\renewcommand*\contentsname{Table des matières}

\PassOptionsToPackage{hyphens}{url}\usepackage{hyperref}

\usepackage{listings}
\author{ControverSciences\textit{ et al} }
\title{Projet AREN - Corpus de ressources \\  Mondialisation et environnement .}
\date{24 Mars 2019}

\begin{document}
\section{Corpus de ressources}
\subsection{Environnement : la France deuxième pays le plus performant du monde}
\label{sec:definition}

\begin{itemize}
	\setlength\itemsep{-0.1em}
	\item \textbf{Lien~: }  \url{https://www.lesechos.fr/2018/01/environnement-la-france-deuxieme-pays-le-plus-performant-du-monde-982955} 
	\item \textbf{Auteur~: } Paul Boulben, Journaliste chez Ouest-France.
	\item \textbf{Date~: } 27 Janvier 18
	\item \textbf{Source~: } Les Échos est un quotidien français d’information économique et financière. Le quotidien est d'orientation libérale. Il revendique une ligne éditoriale indépendante, non partisane, favorable à l'économie de marché, ouverte sur le monde et notamment le monde européen. 
\end{itemize}

L'université de Yale a dévoilé mardi son classement bi-annuel de l'Indice de performance environnementale, basé sur 24 critères. La France est deuxième derrière la Suisse, dans un classement qui fait la part belle aux Européens.
La qualité de l'air est l'enjeu principal des politiques de santé publique. C'est la principale conclusion du  rapport 2018 de l'Indice de performance environnementale (IPE), réalisée par l'Université de Yale, qui évalue l'efficacité des politiques environnementales et qui place la Suisse comme pays le plus performant du monde.\\

L'étude bi-annuelle, réalisée par l'Université de Yale aux États-Unis, classe 180 pays en fonction de 24 critères environnementaux : on y trouve la qualité de l'air, des eaux, la préservation des ressources naturelles animales, végétales et minérales, de la biodiversité etc. Ces critères sont eux-mêmes rangés dans deux objectifs principaux de politique environnementale : la santé environnementale (40\%) et la vitalité de l'écosystème (60\%).
Dans son classement 2018, la Suisse est classée «\ pays le plus performant\ », grâce entre autres à ses «\ engagements de longue date envers la protection de la santé publique et de ses ressources naturelles\ ». La France, dauphine de la Suisse, se distingue sur l'objectif «\ vitalité de l'écosystème\ », en particulier sur la qualité de l'air et sur la protection des espaces naturels et marins.\\

«\ La richesse est un déterminant majeur\ » du succès des politiques environnementales, relèvent les chercheurs de l'étude. Sur les vingt pays les plus performants, tous sont des pays industrialisés et quasiment tous européens : le Danemark (3e), le Royaume-Uni (6e), l'Allemagne (13e) et l'Italie (16e). La bonne qualité de l'air est l'un des critères déterminants de leur performance environnementale. Toutefois parmi les pays «\ riches\ », les États-Unis (27e) et le Canada (25e) pâtissent de leurs émissions importantes de gaz à effet de serre ainsi que de leur production de bois, qui limite les effets du reboisement.
Au contraire, très peu de pays en développement sont considérés «\ performants\ » sur les enjeux environnementaux. La pression démographique en Chine (120e) et en Inde (177e) a un impact certain, en particulier sur la mauvaise qualité de l'air.
De manière surprenante, l'archipel des Seychelles (39e) est l'État qui a progressé le plus sur les dix dernières années, notamment grâce à son implication dans la réduction des émissions de gaz à effet de serre. Le rapport cite aussi la politique environnementale colombienne : après les accords de paix avec les Farc, le pays a mené des actions de développement durable dans les zones autrefois victimes des conflits armés.\\

Ce classement confirme deux idées relativement répandues.
D'abord, ce sont les pays riches qui ont les moyens de financer et d'atteindre leurs objectifs en termes de politique environnementale. «\ Atteindre des objectifs de soutenabilité nécessite une certaine richesse matérielle afin d'investir dans les infrastructures, nécessaires à la protection de la santé humaine et des écosystèmes\ », expliquent les chercheurs.
Ensuite, les États en développement satisfont difficilement aux objectifs écologiques. L'urbanisation et l'industrialisation menacent la bonne «\ vitalité de l'écosystème\ » : «\ trop souvent, la croissance économique nuit à l'environnement, particulièrement à cause de l'exploitation des ressources naturelles\ », note le rapport.\\

L'analyse des facteurs politiques qui sous-tendent les classements montre clairement que le revenu est un facteur déterminant du succès environnemental, ont déclaré les chercheurs, notant que les investissements dans l'eau potable et l'assainissement moderne, en particulier, se traduisaient rapidement par de meilleurs résultats en matière de santé environnementale. Pourtant, à tous les niveaux de développement, certains pays obtiennent des scores dépassant ceux de pays comparables dont la situation économique est similaire, démontrant ainsi qu'une bonne gouvernance et des choix politiques judicieux affectent également les résultats, ajoutent-ils.\\

Les décideurs doivent savoir qui dirige et qui a du retard sur les défis énergétiques et environnementaux, afin de mettre en place les politiques permettant de répondre aux objectifs environnementaux de l'Accord de Paris sur le climat (2015).
Cependant, l'efficacité des politiques mises en place dépendra de facteurs extérieurs incertains, qui sont autant de défis à relever\ : 
les risques géopolitiques, avec notamment des tensions accrues entre Chine et États-Unis; 
l’apurement du passé, avec notamment la gestion encore très incertaine de la sortie des politiques monétaires exceptionnelles actionnées partout dans le monde\ ; 
la soutenabilité environnementale de cette promesse de croissance mondiale\ ; 
les atermoiements de la gouvernance mondiale, qui voient des accords commerciaux bilatéraux ou régionaux se substituer à un multilatéralisme bloqué\ ; 
la fragilité de la construction d’ensembles plurinationaux et notamment de l’édifice européen, ...

\newpage
\subsection{Sondage : les Français sans illusion face aux effets de la mondialisation}
\begin{itemize}
	\setlength\itemsep{-0.1em}
	\item \textbf{Lien~: }  \url{https://theconversation.com/sondage-les-francais-sans-illusion-face-aux-effets-de-la-mondialisation-93042} 
	\item \textbf{Auteur~: } Jézabel Couppey-Soubeyran est Maître de conférences en économie à l’Université Paris 1 Panthéon-Sorbonne et conseillère éditoriale.
	\item \textbf{Date~: } 15 mars 2018
	\item \textbf{Source~: }  The Conversation France est un média en ligne d'information et d'analyse de l'actualité indépendant, qui publie des articles grand public écrits par les chercheurs et les universitaires. 
\end{itemize}

Pour sa 6e édition, le Printemps de l’économie a fait réaliser un sondage exclusif OpinionWay auprès de 1 000 personnes sur le thème «\ Les Français et la mondialisation\ ».
Ce sondage révèle assez nettement la perception négative et la désillusion que la mondialisation nourrit. Mauvaise opinion pour 60\% des Français interrogés, inquiétude pour l’avenir des prochaines générations à 71\%… la mondialisation est pour les sondés synomyme de pauvreté, de chômage, d’inégalités et d’uniformisation culturelle, selon le sondage que vient de publier OpinionWay pour le printemps de l’économie.\\ 

Argent, commerce, économie, uniformisation et pauvreté arrivent en tête des mots que la mondialisation évoque et les termes négatifs (complot, arnaque, conflit, destruction…) l’emportent très largement sur ceux positifs (développement, chance, opportunité). 60\% des personnes interrogées ont une «\ mauvaise opinion de la mondialisation\ ».\\

Au niveau de la France, un tiers seulement des sondés perçoivent des effets positifs sur la croissance. Pour la majorité d’entre eux, la mondialisation a eu des effets négatifs sur le pouvoir d’achat, la préservation de l’environnement, l’emploi et les salaires. \\

Les personnes interrogées ne croient guère en la possibilité d’une mondialisation plus harmonieuse et perçoivent avec pessimisme les effets de la mondialisation sur l’avenir.\\

Aux gouvernants de prendre la mesure d’une telle désillusion, et d’œuvrer au maintien d’un monde ouvert où les gains de la mondialisation puissent être mieux partagés. Car, à n’en pas douter, la mondialisation en intensifiant les échanges produit des gains. Le problème est que ces gains sont fortement concentrés quand, au contraire, nombreux sont les perdants pour qui la mondialisation rime avec perte d’emploi, concurrence des bas salaires, effacement du local, perte d’identité culturelle.\\

Pour les gagnants de la mondialisation, le monde n’a pas de frontière et les opportunités sont vastes. Pour les perdants, l’espace se réduit à un territoire en berne, la mobilité est impossible et l’horizon bouché. C’est ce fossé que l’action publique doit combler en favorisant la mobilité des personnes au moins autant que celle des marchandises et en faisant vivre le local autant que le global. Sans cela continueront de fermenter les velléités de fermeture et de repli d’où viennent les heures les plus sombres de l’histoire.

\newpage
\subsection{Progrès social : «\ Trouver la formule respectant équité, liberté et durabilité environnementale\ »}
\begin{itemize}
	 \setlength\itemsep{-0.1em}
	\item \textbf{Lien~: }  \url{https://www.lemonde.fr/idees/article/2018/11/02/progres-social-trouver-la-formule-respectant-equite-liberte-et-durabilite-environnementale_5378197_3232.html} 
	\item \textbf{Auteur~: } Panel international sur le progrès social (Marc Fleurbaey et collaborateurs)
	\item \textbf{Date~: } 2 novembre 2018
	\item \textbf{Source~: }   Le Monde est un journal  français à la ligne éditoriale parfois présentée comme étant de centre gauche, bien que cette affirmation soit récusée par le journal lui-même, qui revendique un traitement non partisan. Le journal est édité par le groupe Le Monde, détenu à 72,5\% par la société Le Monde libre, elle-même contrôlée à parité par les hommes d'affaires Xavier Niel et Matthieu Pigasse. Il bénéficie de subventions de la part de l'État français.
\end{itemize}

L’idée occidentale selon laquelle les institutions de nos sociétés libérales, démocratiques et capitalistes ont atteint leurs formes finales, et représentent le but de toute nation (la “fin de l’histoire”), doit être fermement rejetée. Les acquis sociaux et démocratiques peuvent être balayés en une élection et remplacés par des politiques autoritaires et destructrices sur les plans social et environnemental. L’histoire est toujours en marche, et nous devons repenser nos institutions si nous voulons rendre notre modèle durable. Des idées intéressantes et des innovations venant de chaque continent peuvent mener à de nouvelles formes de participation populaire, une plus grande harmonie avec la nature ou une meilleure manière de gérer les conflits. Partout dans le monde, une grande diversité d’avancées économiques, politiques ou sociales montre le pouvoir de l’imagination et le nombre impressionnant d’idées promettant de mener vers une société meilleure.\\

Le défi de notre époque est de trouver une ou plusieurs formules respectant à la fois l’équité (que personne, à l’échelle nationale comme internationale, ne soit laissé pour compte et que soit réalisée une société inclusive), la liberté (économique et politique, ce qui implique le respect de l’Etat de droit, des droits de l’homme et des droits démocratiques au sens large) et la durabilité environnementale (préserver les écosystèmes non seulement pour les générations à venir, mais aussi pour eux-mêmes, pour respecter toutes les formes de vie).\\

La mondialisation et les innovations technologiques sont les deux causes principales des transformations socio-économiques actuelles. Si les vertus et les dangers de la mondialisation sont bien connus des experts (bien que pas toujours des décideurs, malheureusement), il existe bien plus d’incertitudes sur les effets qu’auront les innovations technologiques sur notre qualité de vie et sur les inégalités sociales. Il est important cependant d’avoir conscience que la mondialisation et les innovations technologiques ne sont pas des évolutions naturelles que les sociétés ne pourraient que subir ou arrêter brutalement. Au contraire, la façon dont la mondialisation et l’innovation se déploient à travers le temps peut être largement influencée par les choix politiques. Il importe donc de les guider pour en faire des forces d’inclusion sociale. Aider les perdants de la mondialisation et des bouleversements technologiques, faciliter leur transition vers les nouvelles opportunités offertes par ces évolutions est donc nécessaire, mais pas suffisant. Il faut aussi, et surtout, faire en sorte que les changements eux-mêmes dans ces domaines aient lieu d’une manière qui engendre plus de bénéfices et moins de pertes pour tous.\\

Comment imaginer de meilleures institutions et de meilleures politiques publiques ? Penser que le progrès social peut passer simplement par la prise du pouvoir central pour imposer des réformes économiques et sociales serait dramatiquement insuffisant. C’est au niveau de chaque institution et organisation, de la famille à l’entreprise transnationale, de la communauté locale au groupe régional d’États, de la petite ONG au parti politique, qu’il faut s’attaquer aux inégalités en termes de ressources, mais aussi, et surtout, en termes de pouvoir et de statuts. Réformer toutes ces institutions et toutes ces organisations dans les champs économique, politique et social ne se fera pas simplement par l’arrivée de partis plus «\ progressistes\ » au pouvoir, mais nécessitera des initiatives venant de la base, des changements dans la gouvernance de nombreuses organisations, et plus particulièrement dans les principales institutions économiques, à tous les niveaux, de la petite entreprise aux organisations internationales.

\end{document}



