\documentclass[8pt]{article}

\usepackage[T1]{fontenc}
\usepackage[utf8]{inputenc}
\usepackage{lmodern}
\usepackage{float}
\usepackage{fullpage}

 
\DeclareUnicodeCharacter{200A}{ } 
\DeclareUnicodeCharacter{2009}{ } 
\renewcommand*\contentsname{Table des matières}

\PassOptionsToPackage{hyphens}{url}\usepackage{hyperref}

\usepackage{listings}
\author{ControverSciences\textit{ et al} }
\title{Projet AREN - Corpus de ressource \\  Nourrir les humains.}
\date{15 novembre 2017}

\begin{document}
\maketitle

\tableofcontents
\newpage
\section{Textes à débattre}
\subsection{Pourquoi nous allons tous devenir des mangeurs d’insectes}
\begin{itemize}
	\item \textbf{Lien : }  \url{http://www.lavoixdunord.fr/235723/article/2017-10-07/pourquoi-nous-allons-tous-devenir-des-mangeurs-d-insectes} 
	\item \textbf{Auteur : } Bérangère Barret
	\item \textbf{Date : } 7 Octobre 2017
	\item \textbf{Source : } La Voix du Nord est un quotidien régional du Nord de la France.
\end{itemize}

\textbf{Parce que c’est plutôt bon.}\\

Les vers de farine, également nommés ténébrions, ont un petit goût de noisette pas désagréable. Grillés par les bons soins de Virginie, ils croquent sous la dent et agrémenteront sans problème un apéro des plus futuristes. Les cookies vers-pépites de chocolat sont moelleux et chocolatés. Et s’il n’y avait pas cette petite partie de l’abdomen de l’insecte qui dépasse du gâteau, on n’y verrait que du feu à cet apport protéinique très bon pour la santé.\\

\textbf{C’est riche et nourrissant.}\\

Car c’est là l’un des principaux avantages à devenir entomophage* : les apports nutritifs des vers, grillons et autres criquets sont extraordinaires. Surtout le grillon. «  C’est vraiment un champion  », sourit Virginie Mixe, couvant des yeux ses petites bêtes massées dans un grand tube à épices. Ces petites bêtes contiennent jusqu’à 60 \% de protéines. Et ce n’est pas tout. Un grillon contient aussi 5,46 mg de fer, 6,5 mg de vitamine A, 13,8 mg de vitamine B3… Mieux qu’un complément alimentaire.\\

\textbf{C’est bon pour la planète.} \\

«  Les insectes, c’est pas forcément pour les gens qui veulent faire Koh Lanta, mais plutôt pour ceux qui développent une sensibilité à l’écologie.  » Virginie Mixe s’est d’ailleurs lancée dans l’élevage pour cette raison. «  Je n’avais pas spécialement d’appétence pour ça à la base !  » Mais elle se préoccupe de la planète, de ces productions de viande énergivores et de ces surfaces envahies de culture de soja, alternative pas si écolo à la viande… Les insectes prennent peu de place, s’élèvent relativement facilement et, les phobiques ne le savent que trop bien, se reproduisent à vitesse grand V. « Cerise » sur le gâteau, ils sont dépourvus de système nerveux. Ils ne souffrent donc pas au moment de passer à la casserole.

\newpage
\subsection{Les burgers aux insectes arrivent en Europe : faut-il se laisser tenter ?}
\begin{itemize}
	\item \textbf{Lien : }  \url{https://www.ouest-france.fr/insolite/les-burgers-aux-insectes-arrivent-en-europe-faut-il-se-laisser-tenter-5198715} 
	\item \textbf{Auteur : } Arnaud Bélier
	\item \textbf{Date : } 21 août 2017
	\item \textbf{Source : } Ouest-France est un quotidien régional français, édité à Rennes et vendu dans les régions de l'ouest de la France, ainsi qu'à Paris. Les principes revendiqués par la devise du journal « Justice et Liberté », répondent à la trinité : humanisme, démocratie chrétienne et social libéralisme.
\end{itemize}


Depuis ce midi, des burgers et des boulettes confectionnés avec des insectes sont en vente en Suisse. Une première en Europe. Les Nations Unies approuvent la démarche, jugée bonne pour le développement durable et de nature à nourrir 9 milliards d’humains d’ici 2050. Pas si simple, tempèrent les services sanitaires français.\\

Oubliez les cheese-burgers, les Long giant bacon et autre Big-Mac. Depuis ce lundi 21 août, une chaîne de supermarchés en Suisse, propose des burgers aux insectes dans sept de ses magasins : à Genève, Lausanne, Bâle, Berne, Winterthour, Zürich et Lugano.\\

Les burgers contiennent des vers de farine, mais aussi du riz et des légumes (carottes, céleri, poireaux) et des épices (origan, chili). Coop distribue également des boulettes à base de vers de farine, mélangés à des pois chiches, des oignons, de l’ail et des épices.\\

\textbf{Une première en Europe}\\

Le 1er mai, la Suisse a été le premier pays du Vieux Continent à autoriser la vente de produits alimentaires contenant des insectes, en l’espèce les vers de farine (tenebrio molitor), les grillons domestiques (acheta domesticus) et les criquets migrateurs (locusta migratoria). L’Office fédéral de la sécurité alimentaire et des affaires vétérinaires (Osav) a posé quelques restrictions : les insectes doivent provenir d’un élevage, avoir subi un traitement contre les germes végétatifs et avoir été surgelés. Le but étant d’écarter toute contamination potentielle par des pesticides, des parasites, des staphylocoques, des ténias nains ou des spores du botulisme.\\

\textbf{Bons pour le développement durable}\\

« L’utilisation d’insectes en tant que denrée alimentaire a beaucoup d’avantages : leur intérêt culinaire est élevé, leur production économise les ressources de la planète et leur profil nutritif est riche », assure un fabricant. Selon une étude de 2012, « les vers de farine participent de manière beaucoup moins importante que d’autres aliments au réchauffement climatique : 2 fois moins que le lait, 2,5 fois moins que la viande de porc, 1,8 fois moins que le poulet et 8,5 fois moins que la viande de bœuf. »\\


\textbf{Un tiers des humains en mange déjà}\\

Alors prêts à sauter le pas ? Pas si sûr. Au printemps, 9\% des Suisses se déclaraient prêts à imiter les candidats de l’émission de téléréalité Koh Lanta. « Mangez des insectes. Les insectes sont en abondance, ils sont une source riche en protéines et en minéraux », encourageait pourtant l’Organisation des Nations unies pour l’alimentation et l’agriculture (FAO) lors de la présentation d’un rapport consacré au sujet en 2013.
\\

Un tiers de la population mondiale en consomme déjà, du Congo à la Thaïlande, en passant par le Chiapas, au Mexique. La FAO concluait : « D’ici 2030, plus de 9 milliards de personnes devront être nourries, tout comme les milliards d’animaux élevés chaque année », alors même que « la pollution des sols et de l’eau dus à la production animale intensive et le surpâturage conduisent à la dégradation des forêts ».\\

\textbf{Réticences en France}\\

À quand les steaks de grillons en France ? Pas tout de suite, si l'on en croit, toutefois, l'Agence nationale de sécurité sanitaire de l’alimentation (Anses). Dans une étude réalisée en avril 2015, en collaboration avec l’unité d’entomologie fonctionnelle et évolutive de l’université de Liège (Belgique), l’Anses pointait des riques d’ordre chimique (venins), allergènes, biologiques (virus, parasites) et physiques (risque d’avaler les parties dures de l’insecte). L’Anses recommandait enfin d’encadrer les « conditions d’élevage et de production [afin de] garantir la maîtrise des risques sanitaires ».

\newpage
\section{Corpus de ressources}
\subsection{Vidéo. Biodiversité, l'essentielle différence}

\begin{itemize}
	\item \textbf{Lien : }  \url{https://www.youtube.com/watch?v=1F6JGk51_l0} 
	\item \textbf{Auteur : } Julien Goetz
	\item \textbf{Date : } 7 février 2015
	\item \textbf{Description : } L'homme, par ses activités mondialisées, transforme l'ensemble de l'équilibre planétaire. En bâtissant des villes, en ouvrant de nouvelles terres cultivables, en transformant ses cultures en industrie, nous modifions l'habitat de nombreuses espèce vivantes, végétales ou animales. Quand nous ne les éradiquons pas purement et simplement. Or, aucune espèce n'existe de manière isolée. Elles sont toutes un maillon dans une mécanique du vivant plus large. C'est cette biodiversité qui fait la richesse du monde que nous habitons aujourd'hui. La diversité est même ce qui préserve le mieux chaque espèce. Irons-nous consciemment jusqu'à la sixième exctinction massive d'espèces sur la planète ?
	\item \textbf{Source : } DataGueule, chaque épisode de cette émission hebdomadaire produite par france4 tente de révéler et décrypter les mécanismes de la société et leurs aspects méconnus.
\end{itemize}

\newpage
\subsection{Vidéo. Insectes comestibles : une industrie à inventer}

\begin{itemize}
	\item \textbf{Lien : }  \url{https://lejournal.cnrs.fr/videos/insectes-comestibles-une-industrie-a-inventer} 
	\item \textbf{Durée : } 5 min 53
	\item \textbf{Date : } 17 Avril 2015
	\item \textbf{Réalisateur : } Pierre de Parscau
	\item \textbf{Description : } Les insectes comestibles sont aujourd'hui considérés comme une source alternative de protéines pour une population humaine qui devrait gagner 2 milliards d'individus d'ici à 2050. Encore faut-il pouvoir fabriquer ces protéines à un coût compétitif. Dans cette vidéo à voir aussi sur le site de notre partenaire Le Monde.fr, découvrez comment chercheurs et industriels tentent de lever les verrous liés à cette nouvelle forme d'élevage, et de mettre en place une véritable filière française et européenne.
	\item \textbf{Source : } CNRS Le journal, l'objectif de ce site est de partager largement avec les amateurs de science, les professeurs et leurs élèves, les étudiants et tous les citoyens curieux, des contenus destinés jusque-là à la communauté des agents du CNRS, chercheurs, ingénieurs et techniciens, ceux des labos comme ceux des bureaux.
\end{itemize}



\newpage
\subsection{Le boom de l’agriculture urbaine}

\begin{itemize}
	\item \textbf{Lien : }  \url{https://lejournal.cnrs.fr/articles/le-boom-de-lagriculture-urbaine} 
	\item \textbf{Auteur : } Aurélie Sobocinski
	\item \textbf{Date : }  24 Février 2015 
	\item \textbf{Source : } CNRS Le journal, l'objectif de ce site est de partager largement avec les amateurs de science, les professeurs et leurs élèves, les étudiants et tous les citoyens curieux, des contenus destinés jusque-là à la communauté des agents du CNRS, chercheurs, ingénieurs et techniciens, ceux des labos comme ceux des bureaux.
\end{itemize}

Toits cultivés, jardins partagés, friches exploitées… Une déferlante verte aux formats nouveaux gagne aujourd’hui le cœur des villes de l’Hexagone et d’Europe, après avoir déjà conquis l’Amérique du Nord. En Île-de-France, de premiers recensements font ainsi apparaître que la surface totale des jardins associatifs pourrait atteindre celle de la surface de maraîchage professionnel ! Et à Marseille, on compte un millier de petites parcelles où sont cultivés des légumes potagers sur une trentaine d’hectares. Loin des canons agricoles classiques, lové dans les plus petits interstices de parfois quelques mètres carrés, le phénomène intrigue les scientifiques. Que cache cette multiplication d’expérimentations entre béton et bitume ? Effet de mode ou mouvement durable ?\\

\textbf{Jassur, un programme dédié à l’agriculture urbaine}\\

Pour essayer de mieux saisir le phénomène, le programme Jassur (link is external) (Jardins associatifs urbains et villes durables), financé par l’Agence nationale de la recherche, a été lancé en janvier 2013. Car l’agriculture urbaine, dont la définition même varie selon les continents (lire En coulisses), est un sujet plus complexe qu’il n’y paraît : « Toutes les expériences de jardins productifs urbains ne répondent pas à la même dynamique. Cela peut aller du simple loisir à une réelle activité commerciale en passant par un projet visant à restaurer du lien social. Côté recherche, un de nos premiers défis est donc de définir un cadre pour analyser le phénomène sous toutes ses facettes », expliquent la sociologue Laurence Granchamp et la géographe Sandrine Glatron, organisatrices de la première école thématique française sur ce sujet, qui s’est tenue en juin 2013 à Strasbourg.\\

Autre défi pour les scientifiques : mieux cerner les clés du succès des jardins potagers urbains. « Ils se sont développés à un rythme accéléré ces dernières années, à partir du moment où les préoccupations concernant le changement climatique et la succession de différents scandales alimentaires – vache folle, poulet à la dioxine – ont remis en question les formes de production alimentaire ainsi que leur localisation », explique Laurence Granchamp. En témoignent notamment le succès des circuits courts comme les Amap, le retour des marchés paysans dans lesquels des producteurs se rassemblent en un lieu donné pour vendre en direct leurs produits, ou encore des labels AOC, qui répondent à l’exigence des consommateurs d’une plus grande proximité et qualité dans l’origine des produits. L’agriculture urbaine serait en quelque sorte le signe le plus récent de cette remise en question.\\

\textbf{Les citadins en demande d’une nature nourricière}\\

Autre raison du boom des jardins productifs : l’évolution récente du rapport des citadins à la ville et à la nature. « L’essor des jardins productifs va de pair avec celui de la diffusion d’une réelle culture du développement durable, de la prise de conscience des limites de l’environnement et de la nécessité de pratiques plus respectueuses, affirme l’urbaniste Jean-Noël Consales, cocoordinateur du projet Jassur. Il est l’un des vecteurs bien concrets de ce changement de rapport des habitants à la ville et à l’urbain. » Aujourd’hui, la demande de nature des habitants urbains ne se limite plus à une nature paysagère et esthétique, mais aussi… nourricière. « La crise économique de 2008 est passée par là : quelle que soit la taille des parcelles, la fonction alimentaire revient quasi systématiquement dans la bouche des porteurs de projet, poursuit le chercheur. Or cette dimension avait disparu chez les utilisateurs des jardins ouvriers à la fin des années 1990, davantage axés sur le loisir. » Une chose est sûre : « Ces espaces de production dans la ville modifient réellement les comportements alimentaires de leurs usagers dans le sens d’une plus grande qualité, y compris pour ceux issus de milieux très populaires », comme a pu l’observer l’urbaniste à Marseille.\\

\textbf{L’agriculture urbaine, un modèle alimentaire alternatif ?}\\

L’un des principaux objectifs du programme Jassur consiste aujourd’hui à quantifier précisément la valeur productive de ces jardins urbains. « Selon de premières indications, le phénomène représenterait une aide substantielle à l’alimentation de nombreux foyers urbains », annonce d’ores et déjà Jean-Noël Consales. De là à imaginer que ces dispositifs pourraient nourrir une ville entière… « Au-delà des scientifiques, une grande partie des acteurs sociaux, économiques et politiques se posent aujourd’hui la question de la viabilité d’un tel modèle d’agriculture et du développement d’une économie verte », souligne le chercheur, citant la professionnalisation de certaines parcelles associatives et le lotissement par des propriétaires fonciers de terrains urbains en… jardins. « Ce timide élan reste néanmoins incomparable avec l’ampleur des projets de fermes urbaines qui se déploient et se professionnalisent sur la base de techniques très intensives en Amérique du Nord », relativise Laurence Granchamp.\\

L’utopie d’une production agricole urbaine aux vastes surfaces se heurte dans nos frontières à la dure réalité de la pression foncière, estime de son côté Sandrine Glatron. « Il y a un réel conflit d’usage de l’espace, en tout cas en ce qui concerne les villes d’Europe », explique la géographe. À cela s’ajoute une réalité économique, complète-t-elle : « Il reste beaucoup moins cher de manger un chou de Bretagne même quand vous habitez à Strasbourg que de le faire pousser juste à côté dans une terre vouée à la construction d’immeubles en raison du prix du foncier et du mode d’exploitation – sans pesticide ou bio – moins productif a priori. »\\


\textbf{Le succès inattendu des jardins potagers urbains}\\

Alors comment expliquer le succès des jardins urbains ? « Si les collectivités publiques, dont les surfaces d’agriculture urbaines dépendent essentiellement, s’y intéressent de plus en plus, c’est pour tous ces services, bien au-delà donc de la seule production alimentaire, qu’elles peuvent remplir au cœur de la ville, de sociabilité, de solidarité, de lien au vivant, de lutte contre l’obésité, d’éducation environnementale, de sauvegarde de la biodiversité, ou encore de gestion des déchets », observe Sandrine Glatron. « Ce type de projet ne peut résoudre à lui seul la misère sociale d’un quartier, tempère Jean-Noël Consales. Pour autant, il constitue un outil précieux, un vecteur d’amélioration, d’appropriation et même de production commune du cadre de vie par le mélange de publics, la création d’emplois et la garantie d’une certaine équité écologique qu’il apporte à tous. » Assez pour contribuer, sans l’être officiellement annoncé, au maintien de la paix sociale… Résultat : « Il n’y a pas une agglomération aujourd’hui, pas un Scop qui n’intègre pas la question agricole », conclut le chercheur. À défaut de nourrir des villes entières, l’agriculture urbaine a visiblement encore de belles saisons devant elle.\\


\newpage
\subsection{Il est possible de nourrir l’humanité de façon durable}

\begin{itemize}
	\item \textbf{Lien : }  \url{http://fr.unesco.org/news/il-est-possible-nourrir-humanite-facon-durable-conseil-consultatif-scientifique-du-secretaire} 
	\item \textbf{Auteur : } Conseil consultatif scientifique du Secrétaire général de l'UNESCO
	\item \textbf{Date : } 28 décembre 2016
	\item \textbf{Source : } UNESCO (United Nations Educational, Scientific and Cultural Organization). L'UNESCO a pour objectif selon son acte constitutif de « contribuer au maintien de la paix et de la sécurité en resserrant, par l’éducation, la science et la culture, la collaboration entre nations, afin d’assurer le respect universel de la justice, de la loi, des droits de l’Homme et des libertés fondamentales pour tous, sans distinction de race, de sexe, de langue ou de religion, que la Charte des Nations unies reconnaît à tous les peuples ». Ses scientifiques ont très peu de conflits d'intérêts.
\end{itemize}

Nourrir l’humanité de façon durable est devenu une priorité mondiale majeure pour nos sociétés. À court terme, les préoccupations en matière de sécurité alimentaire dans le monde concernent la faim et la pauvreté parmi les plus démunis, de manière encore plus intense et urgente dans les pays en développement, où quelque 800 millions de personnes sont en proie à la faim et où les enfants risquent de subir des retards de croissance. Le Conseil consultatif a étudié la question de la sécurité alimentaire dans un contexte élargi en prenant notamment en compte l’utilisation et la conservation des ressources naturelles, les modes de production alimentaire et d’exploitation des ressources plus efficaces, les effets du changement climatique, ainsi que la réduction des pertes et du gaspillage alimentaires à l’échelle mondiale. Les changements nécessaires en matière de régime alimentaire, notamment le passage d’une alimentation hautement calorique à une alimentation plus riche en protéines, font partie des sujets abordés dans la note d’orientation.\\

Cette note d’orientation a été élaborée sous la direction de Gebisa Ejeta, membre du Conseil consultatif scientifique. Selon le Conseil consultatif, les capacités humaines et institutionnelles ont grandement besoin d’être renforcées dans de nombreux pays défavorisés afin que ces derniers puissent devenir des acteurs majeurs de la recherche de solutions, dans le cadre d’un nouveau système alimentaire mondial capable de faire face aux besoins croissants de la planète en matière d’alimentation et de nutrition.
Le Conseil consultatif plaide également pour la mise en place de solides partenariats publics et privés, éléments essentiels à l’émergence de « systèmes alimentaires » commerciaux prospères et durables pour favoriser la croissance économique, assurer des emplois rémunérateurs, et répondre aux besoins alimentaires et nutritionnels de la société pour une meilleure santé.
La note d’orientation met en exergue la nécessité de lier la sécurité alimentaire mondiale à des politiques nationales et internationales plus énergiques à l’appui de systèmes de production tenant compte des questions climatiques, avec des entreprises rentables et des systèmes alimentaires également fondés sur une gestion avisée des ressources de la planète Terre.
« En investissant dans la science, nous créons la possibilité de ralentir et d’inverser les phénomènes et tendances néfastes grâce aux décisions que nous prenons dès aujourd’hui », conclut le Conseil scientifique.\\

« L’histoire a montré que les investissements réalisés dans les sciences agricoles au XXe siècle avaient permis d’éviter des catastrophes et avaient largement porté leurs fruits. Dans le cadre des objectifs du Programme de développement durable à l’horizon 2030, il n’est pas impensable que cette seule planète puisse produire suffisamment pour nourrir 9 milliards de personnes de façon durable et respectueuse de l’environnement, grâce à la créativité des sciences et de l’innovation, ainsi qu’à la sagesse locale et à des politiques efficaces », explique Gebisa Ejeta.\\

Créé en 2014 eu égard au rôle essentiel de la science dans la réalisation des objectifs de développement durable, le Conseil consultatif scientifique est une expérience unique en son genre qui permet de fournir des avis scientifiques interdisciplinaires au Secrétaire général de l’ONU. L’UNESCO assure le Secrétariat du Conseil consultatif.  

\end{document}


