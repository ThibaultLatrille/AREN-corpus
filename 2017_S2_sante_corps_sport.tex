\documentclass[8pt]{article}

\usepackage[T1]{fontenc}
\usepackage[utf8]{inputenc}
\usepackage{graphicx}
\usepackage{lmodern}
\usepackage{amsmath}
\usepackage{xfrac}
\usepackage{amsthm}
\usepackage{listings}
\usepackage{enumerate}
\usepackage{amssymb}
\usepackage{cancel}
\usepackage{amsfonts}
\usepackage{float}
\usepackage{fullpage}

\DeclareUnicodeCharacter{200A}{ }
\DeclareUnicodeCharacter{2009}{ } 
\renewcommand*\contentsname{Table des matières}

\PassOptionsToPackage{hyphens}{url}\usepackage{hyperref}

\usepackage{listings}
\author{ControverSciences\textit{ et al} }
\title{Projet AREN - Corpus de ressource \\  Santé, corps et sport.}
\date{15 novembre 2017}

\begin{document}
\maketitle

\tableofcontents
\newpage
\section{Textes à débattre}
\subsection{Une bodybuildeuse australienne meurt d'une overdose de protéines}
\begin{itemize}
	\item \textbf{Lien : }  \url{http://mobile.allodocteurs.fr/alimentation/complements-alimentaires/une-bodybuildeuse-australienne-meurt-d-une-overdose-de-proteines_23066.html} 
	\item \textbf{Auteur : } Arthur Laffargue
	\item \textbf{Date : } 16 août 2017 
	\item \textbf{Source : } Allô docteurs est une émission de télévision française, diffusée en direct du lundi au vendredi à 14 h 30 sur France 5. Les téléspectateurs sont régulièrement invités à se reporter au site internet de l'émission pour des compléments d'information et pour des échanges avec des spécialistes médicaux.
\end{itemize}

Meegan Hefford a été retrouvée inconsciente chez elle le 19 juin. Trois jours plus tard, les médecins la déclaraient en état de mort cérébrale. Ils avaient découvert entretemps, mais trop tard, qu’elle souffrait d’un trouble du cycle de l’urée. Ses reins n’étaient plus en mesure d’éliminer l’ammoniac naturellement métabolisé par son corps après la prise de protéines, ce qui a atteint son cerveau après l’empoisonnement de son sang.\\

La jeune femme de 25 ans souffrait du déficit d’une enzyme, censée détruire l’ammoniac, causé par une malformation génétique. Le problème a été aggravé par la prise de protéines en poudre en grande quantité.\\

Ces compléments alimentaires ne sont utiles qu’à quelques haltérophiles de très haut niveau. Et même eux peuvent satisfaire leurs besoins supplémentaires par un moyen très simple : inclure plus de protéines dans leur alimentation. Grâce à une consommation équilibrée de viande, d’œufs ou de légumineuses. "Il n’y a aucun intérêt à prendre ces protéines, soutient le Dr Arnaud Cocaul, nutritionniste. La plupart d’entre nous sommes des sportifs du dimanche. Les sportifs professionnels eux-mêmes n’en abusent pas, ils contrôlent leurs apports en protéines."\\

Le médecin a déjà soigné un patient possédant un ratio de deux grammes de protéine par kg, quand le taux conseillé se situe entre 0,8 et 1 gramme par kg. Une hérésie, puisque les régimes hyper protéinés forcent les reins à travailler davantage. De quoi poser à terme de graves problèmes de santé.

Les reins sont indispensables : ils éliminent les déchets produits par notre organisme. Une insuffisance, et le poison se répand progressivement dans notre corps, pouvant créer un arrêt cardiaque par accumulation de potassium, un œdème pulmonaire par défaut d’élimination de l’eau ou encore une anémie liée à la baisse de la production d’érythropoïétine (EPO) pour les cas les plus dangereux.\\

L’Agence nationale de sécurité sanitaire de l’alimentation (ANSES), dans un avis publié en novembre 2016, invitait les consommateurs de compléments alimentaires à demander conseil auprès d’un professionnel de santé. L’agence déconseillait même la prise de ces produits aux femmes enceintes, aux enfants et aux adolescents.\\

Il est possible de souffrir d’insuffisance rénale sans le savoir, ou sans que les symptômes (besoin fréquent d’uriner la nuit, perte d’appétit, oedèmes des mains, etc.) n’apparaissent immédiatement. "Les gens qui fréquentent les salles où l’on fait la promotion de ces produits ne sont pas forcément au courant de leurs problèmes rénaux", appuie le Dr Cocaul, qui fustige les endroits où l’on vend des "poudres de perlimpinpin". \\

"Je contre-indique formellement de prendre ces poudres", poursuit le nutritionniste, car elles "épuisent les reins". Il vaut mieux selon lui boire beaucoup après l’effort, et si celui-ci est intense, se réhydrater avec une boisson isotonique, optimale pour l’absorption de l’eau par l’intestin. Pour la nourriture, l’idéal est de consommer de la viande ou des féculents une demi-heure ou une heure après l’effort.


\newpage
\subsection{Pour les sportifs, les seniors, les végétariens…, ils misent sur la protéine bio à diluer}
\begin{itemize}
	\item \textbf{Lien : }  \url{http://www.lamontagne.fr/brive-la-gaillarde/economie/agroalimentaire/2017/10/27/pour-les-sportifs-les-seniors-les-vegetariens-ils-misent-sur-la-proteine-bio-a-diluer_12607058.html} 
	\item \textbf{Auteur : } Caroline Girard
	\item \textbf{Date : } 27 octobre 2017
	\item \textbf{Source : } Le journal \textit{La Montagne}, quotidien régional de la presse écrite française, dont le siège se trouve à Clermont-Ferrand,  possède un site web depuis l'été 2006 (www.lamontagne.fr). Contrairement aux principaux autres groupes de presse, sa particularité est que chaque journaliste peut librement alimenter le site avec la mise en ligne de dépêches et photos.
\end{itemize}



Leur histoire d'amour a trois ans. Celle de leur entreprise presque autant. Après avoir crapahuté en France et en Afrique du nord une fois son école de commerce validée, c'est à l'île de la Réunion, et dans les salles de sport, que James Thomson a fait mûrir l'idée qui le fait vivre désormais, et qu'il partage avec sa moitié, Nadège. « Quand je suis arrivé là-bas, je me suis mis à la musculation assez sérieusement. Et si en parallèle de l'activité physique on ne fait pas attention à la nutrition, les résultats ne suivent pas ». Curieux et surtout studieux, il avale, entre deux encas de sportif, des livres sur l'alimentation et des études, jusqu'à ce qu'il fasse le calcul et réalise qu'« en moyenne, 2 grammes de protéines par kilo du poids de corps sont nécessaires pour ceux qui souhaitent "prendre de la masse". Pour quelqu'un de mon gabarit, cela revenait à manger 150 g de protéines par jour sachant qu'en moyenne, il n'y en a que 20 dans une portion de viande ».\\

\textbf{L'accent sur la provenance européenne et le bio}\\

Pour s'éviter une surconsommation de filet de poulet au goûter ou d'autres aliments riches en protéines, difficiles à caser au quotidien, il s'est alors orienté vers une protéine en poudre. Mais pas n'importe laquelle. « Ce que je cherchais, c'était de la protéine certifiée bio. Et à l'époque, sur le marché, ça n'existait pas. Dans le milieu du sport, la protéine reine, c'est la "whey". Le problème, c'est que la composition est parfois indéchiffrable, et la provenance aussi ».\\

Sur les sachets kraft d'Alter-nutrition, deux petites lignes suffisent pourtant à égrener la composition de cette protéine, extraite du petit-lait puis séchée, pour obtenir la poudre, « 100\% bio et sans émulsifiants, aux arômes issus de l'agriculture biologique ». Le tout, d'origine UE. « Il n'y a pas d'ajouts de sucre ou de superflu. Jusqu'au choix des arômes, nous avons misé sur le bio. Pour avoir la certification sur le produit fini nous n'étions pas obligés de le faire, d'autant que pour la vanille par exemple, l'arôme bio est beaucoup plus cher. Mais notre volonté, c'était de suivre nos principes jusqu'au bout, de faire des produits qui nous plaisent aussi en tant que consommateurs ».\\

Pour mettre à bien ce projet, le couple a quitté le soleil des îles pour dessiner les premiers contours de leur micro-entreprise en région Rhône-Alpes. Une première installation qui leur a permis d'étoffer leur carnet d'adresses et de fournisseurs, avant que leurs locaux ne méritent de l'être eux aussi.\\

\textbf{Une nouvelle gamme sans lactosepour les vegans}\\

« On s'est rapidement rendu compte qu'il allait nous falloir de la place et s'installer dans une pépinière d'entreprise, qui plus est spécialisée dans l'agroalimentaire, c'était l'idéal ». Depuis le mois de juin, ils se sont alors installés sur la zone Novapole à Saint-Viance. Une installation qui, si elle ne leur permet pas encore de mécaniser la transformation et le conditionnement des produits, a néanmoins favorisé le développement de nouvelles gammes. « Le sans lactose est une vraie demande des consommateurs, aussi bien pour les sportifs que pour les vegans. Au début, ce n'était pas possible d'y répondre, pour des problèmes de provenance. Aujourd'hui, nous avons des partenaires européens qui travaillent en bio et nous permettent de proposer des sachets de protéines de pois, de riz, ou de chanvre ».\\

Jusqu'au choix du packaging, qui se veut « beaucoup plus sobre que les grosses boîtes en plastiques virils que l'on retrouvent dans les salles de sport ». Le duo d'Alter-nutrition, lui, surfe sur les nouveaux modes de consommation. Et à ceux qui, dans leur nom, verraient seulement un clin d'œil a une alimentation alternative, d'autres, plus musclés, feront sans doute un lien avec l'altère qui leur a donné l'idée. 


\newpage
\section{Corpus de ressources}
\subsection{Complémenter son alimentation en protéines et/ou en acides aminés ?}
\begin{itemize}
	\item \textbf{Lien : }  \url{https://www.valdemarne.fr/newsletters/sport-sante-et-preparation-physique/complementer-son-alimentation-en-proteines-et/ou-en-acides-amines} 
	\item \textbf{Auteurs : } Rachid Ziane, Romain Chou  \& Guillaume Laffaye
	\item \textbf{Date : } 17 mai 2017
	\item \textbf{Source : } Valdemarne.fr est le portail officiel du Département du Val-de-Marne. Le site contient des lettres d'information, dont la lettre "Sport Santé et Préparation Physique" qui propose des articles autour de la connaissance sportive, de l'entraînement et de la préparation physique.
	L'article est écrit par Romain Chou (diététicien diplômé d’État) et Guillaume Laffaye (maître de conférences et membre du laboratoire Complexité, Innovation, Activités Motrices et Sportives de l'Université Paris-Saclay).
	
\end{itemize}


Les questions au sujet de la consommation de protéines en poudre et d’acides aminés sont récurrentes lors de nos interventions en club. Ces compléments alimentaires seraient pour certains une alternative au dopage.
Sont-ils nécessaires, efficaces, dangereux ?\\


\textbf{Supplémenter en protéines ?}\\

Si certains auteurs parlent de 2 grammes par jour et par kilogramme de poids de corps voire plus, rappelons, avec Trémolières \& Collaborateurs (1961), que les besoins quotidiens sont de 0,8 à 1,2 grammes ! Or, tout ce qui provient du vivant contient des protéines : Viandes, poissons, œufs, légumineuses, produits laitiers, mais aussi céréales, oléagineux, fruits et légumes…\\

En termes de besoins particuliers des sportifs : « En ce qui concerne l'adulte sportif, le besoin en protéines […] la pratique régulière (3 fois 1/2 heure à 1 heure par semaine) d'une activité d'intensité modérée ne modifie pas significativement les besoins indiqués ci-dessus pour l'homme adulte » Chollet-Przednowed, Etiemble, Prigent \& al. (1999).\\


\textbf{Supplémenter en protéines quand on est sportif ?}\\


Lorsque l'on zoome sur une population de sportifs particuliers, que l'on pourra scinder en deux, (les endurants vs les explosifs), il est nécessaire de faire appel à une science particulière, la nutrigénétique pour comprendre l'apport nécessaire en protéines.  Cette science est issue de la médecine préventive introduite en 1980 par le prix Nobel de médecine, Jean Dausset. Elle spécule sur le fait que l'alimentation optimale n'est pas la même pour tous et est fortement dépendante de notre capacité à métaboliser des glucides. Ainsi, la prédisposition aux exercices de force est liée notamment à une difficulté à métaboliser les glucides. Il ressort de cette approche deux grandes idées :

	- Nous sommes inégaux devant le métabolisme des glucides et des protéines.
La quantité nécessaire de glucides et de protéines chez des sportifs dépend du type de fibres musculaires (fibres rapides vs fibres lentes) car ils ont une capacité d'assimilation différente.

	- « Ainsi, pour les sportifs d'endurance de bon à haut niveau, les besoins sont de l'ordre de 1,5 g/kg/j. Les apports habituels les couvrent très largement, l'apport énergétique (et donc protéique) […] Le risque de carences en protéines est donc de facto inexistant » Chollet-Przednowed, Etiemble, Prigent \& al. (Op. Cit.).\\

Pour les sportifs de force, notamment dans une logique de construction du muscle, l'apport peut grimper jusqu'à 2.5 à 3 g/kg/j, soit 30 à 40\% de l’apport total calorique nécessaire, tout en réduisant la quantité de glucides quotidien par rapport à un sportif endurant (Vernesson, 2011). Mais précisons bien que nous sommes ici dans une recherche d'hypertrophie musculaire et non pas dans la gestion de carences.\\

Ainsi : « des besoins de 2 à 3 g/kg/j en période de gain de masse musculaire semblent justifiés, pour certains auteurs. Mais attention, ce niveau d'apport ne se justifie que sur une durée limitée et, là encore, compte tenu de la motivation et des effets de mode dans les milieux de type bodybuilding ou haltérophilie, l'excès de protéines paraît plus à craindre que la carence » Chollet-Przednowed, Etiemble, Prigent et al. (Op. Cit.). Aussi, comme le rappellent ces auteurs, les apports sont largement pourvus par une alimentation variée et équilibrée.\\

Consommer des protéines en complément alimentaire ne présente un intérêt en définitive que lorsque l'alimentation ne peut pas couvrir les besoins engendrés par l’activité physique. Par exemple :
un sportif d'endurance de 70 kg trouvera aisément de quoi couvrir ses besoins basaux dans son bol alimentaire
un sportif de force de 100 kg par contre, pourrait avoir beaucoup de difficultés à couvrir uniquement par son alimentation des besoins avoisinant les 300g de protéines par jour, selon le principe de la nutrigénétique.\\

Attention également car toutes les poudres de protéines ne se valent pas. En effet, les protéines de l’œuf entier (blanc + jaune) sont considérées comme protéines de référence car elles contiennent tous les acides aminés essentiels et qui plus est, répartis dans de bonnes proportions. Aussi, si le marché propose des protéines en poudre à base de lait, de caséine de blanc d’œuf… celles-ci sont pour la plupart, incomplètes.\\


De surcroit, rappelons que tout excédant de protéines consommé est pour partie uriné (fraction azotée) et pour partie stocké sous forme de graisse !\\

Enfin, notez que l’excès de consommation de protéines n’est pas sans risque : Fatigue, problèmes articulaires (goutte), problèmes hépatiques et digestifs.
Donc, en dehors de cas particuliers, tels que celui du sportif de force pesant 100kg, consommer des protéines en poudre est inutile. En effet, il convient de bien distinguer la problématique de la carence et celle de la prise de muscle pour des culturistes aux poids de corps imposants.
Nous conseillons par ailleurs de privilégier les protéines végétales aux protéines animales. Ces dernières sont liées à des minéraux (chlore, soufre, phosphore) qui se transforment dans l'organisme en acides forts, contribuant à acidifier l'organisme.\\


\textbf{Supplémenter en acides aminés ?}\\

Les protéines, qu’elles soient animales ou végétales, ne sont qu’un assemblage plus ou moins long de 22 acides aminés, dont seuls diffèrent l’ordre et la longueur.\\

Sur ces 22 acides aminés, huit sont essentiels, car l'organisme ne peut les synthétiser et l'alimentation doit donc nous les procurer.
Si les viandes sont acidifiantes, elles possèdent l'avantage de les apporter de manière équilibrée. Si l'on mange donc peu de viande ou si on craint de ne pas avoir cet ensemble d'acides aminés essentiels, il peut être alors utile de se complémenter.\\

Mais cette option est discutée. En effet, plusieurs recherches pharmaceutiques ont mis en évidence des effets sur l’organisme propres à chaque acide aminé, lorsqu’ingéré séparément après avoir été synthétisé en laboratoire. Par exemple l’arginine laquelle, selon le dosage, est destiné aux enfants souffrant d’asthénie ou de retard de croissance.\\

Partant ainsi de leurs fonctions dans l’organisme, des commerçants ont spéculé sur les effets de prises isolées pour justifier un argumentaire commercial. Par exemple, la L-carnitine : « Cette molécule intervient au sein de la cellule dans le transport des acides gras du cytosol vers les mitochondries lors du catabolisme des lipides dans le métabolisme énergétique » ce qui explique que « Cette molécule est souvent vendue en tant que complément alimentaire ». Comme si la L-carnitine était le seul déterminant de l’utilisation des graisses, lequel comme par hasard ferait défaut chez les personnes cherchant à perdre de la graisse… et comme si le transport des acides gras était sa seule fonction ! On voit bien que les raccourcis intellectuels n’ont que pour but de servir un discours commercial.\\

Par ailleurs, si quelques effets sont connus sur le court terme, d’autres sont ignorés sur le plus ou moins long terme.\\


\textbf{Conclusion}\\

Alternative au dopage ou au contraire, premier pas y menant, la supplémentation en protéines et/ou en acides aminés interpelle du point de vue des effets que l’on peut en attendre. Ainsi, si l’on peut admettre que certains obtiennent des effets significatifs en termes de prise de volume musculaire, l’effet placebo n’est pas à exclure, ni celui induit par des traces d’anabolisants présents dans certains compléments.\\

Quoi qu’il en soit, on peut s’interroger sur les intentions de ceux qui publient des articles dans des revues de musculation, vantant leurs effets, immédiatement suivi d’une publicité.


\newpage
\subsection{Les protéines, définition, rôle dans l'organisme, sources alimentaires}

\begin{itemize}
	\item \textbf{Lien : }  \url{https://www.anses.fr/fr/content/les-prot%C3%A9ines} 
	\item \textbf{Auteur : } ANSES
	\item \textbf{Date : } 1 janvier 2016
	\item \textbf{Source : } L’Agence nationale de sécurité sanitaire de l'alimentation, de l'environnement et du travail (ANSES) est l'agence nationale française chargée de la sécurité sanitaire. C'est un établissement public d'évaluation des risques dans les domaines de l'alimentation, de l'environnement et du travail. Elle résulte de la fusion en 2010 de l'AFSSA et de l'AFSSET. 
\end{itemize}

Les protéines sont, avec les glucides et les lipides, l’une des trois grandes familles de macronutriments, c’est-à-dire l’un des constituants des aliments qui contribuent à l’apport énergétique.\\

Schématiquement, les protéines peuvent être considérées comme de longues chaînes linéaires ou ramifiés, plus ou moins repliées sur elles-mêmes, organisées dans l’espace ou non.\\

Les acides aminés sont l’unité de base constituant les protéines. Il existe un très grand nombre d’acides aminés différents mais seulement vingt sont utilisés par l’organisme pour la fabrication des protéines (acides aminés dits « protéogènes »). Parmi ces 20 acides aminés, 11 peuvent être fabriqués par le corps humain et les 9 autres sont dits indispensables, car l’organisme est incapable de les synthétiser en quantité suffisante pour satisfaire ses besoins. Ces acides aminés doivent par conséquent être apportés par l’alimentation.\\

La composition en acides aminés des protéines est prise en compte pour évaluer la qualité protéique de notre alimentation. Les acides aminés et donc, les protéines qu’ils composent, sont en outre riches en azote et constituent notre source majoritaire d’apport en cet élément indispensable à l’organisme.\\

\textbf{Rôle des protéines}\\

Dans l'organisme, les protéines jouent des rôles essentiels :
	
	$\bullet$ elles jouent un rôle structural et participent au renouvellement des tissus musculaires, des phanères (cheveux, ongles, poils), de la matrice osseuse, de la peau, etc.
	
	$\bullet$ elles participent à de nombreux processus physiologiques, par exemple sous la forme d'enzymes digestives, d'hémoglobine, d'hormones, de récepteurs ou d'immunoglobulines (anticorps).

Elles constituent, par ailleurs, l'unique source d'azote de l'organisme.\\

\textbf{Sources alimentaires de protéines}\\

La qualité des sources alimentaires de protéines est presque exclusivement définie par leurs capacités à couvrir les besoins en protéines et en acides aminés indispensables. Les protéines animales, majoritaires dans l’alimentation des pays industrialisés, proviennent notamment du lait, de l’œuf, des poissons et de la viande.\\

Les protéines végétales proviennent essentiellement des céréales et des légumineuses. Elles peuvent être naturellement présentes dans les aliments ou être rajoutées pour des raisons nutritionnelles (aliments spécifiques pour nourrissons ou personnes âgées) ou techno-fonctionnelles (propriété gélifiante du blanc d’œuf).\\

\textbf{Protéines d'origine animale}\\

Les protéines animales sont relativement riches en acides aminés indispensables et généralement plus riches que les protéines végétales. En ce qui concerne la digestibilité, elle est en général légèrement plus élevée pour les protéines animales que pour les protéines végétales.\\

Les aliments d’origine animale sont caractérisés par leur forte teneur en protéines de haute qualité nutritionnelle (composition en acides aminés indispensables, digestibilité, etc.).  \\

La viande, le poisson, les œufs, le lait et les produits laitiers sont des aliments riches en protéines.\\

\textbf{Protéines d'origine végétale}\\

Certaines protéines végétales peuvent présenter une teneur limitante en certains acides aminés indispensables, la lysine pour les céréales, et les acides aminés soufrés pour les légumineuses.\\

Pour obtenir une alimentation équilibrée en acides aminés à partir de protéines végétales, il est ainsi nécessaire d'associer différents aliments végétaux : des graines de légumineuses (lentille, fèves, pois, etc.) avec des céréales (riz, blé, maïs, etc.). \\

Les aliments végétaux les plus riches en protéines sont ainsi les graines oléagineuses (cacahuètes, amandes, pistaches, etc.), les légumineuses et leurs dérivés (tofu, pois chiche, haricots…) ou encore les céréales.\\

Au-delà de la problématique de la couverture des apports en acides aminés, l’origine protéique peut avoir une incidence sur la couverture des besoins en d’autres nutriments. Ainsi, une alimentation exclusivement d’origine végétale peut conduire à un risque de déficience en vitamine B12. Et une alimentation riche en protéines animales peut conduire à un apport insuffisant en fibres et excessif en graisses saturées.\\

Dans nos sociétés, les régimes végétariens non stricts (n’excluant pas les produits laitiers et les œufs) permettent d’assurer un apport protéique en quantité et en qualité satisfaisantes pour l’enfant et l’adulte. Chez les végétaliens adultes, une attention particulière doit être apportée à la couverture de l’apport protéino-énergétique et à l’utilisation de sources protéiques qui se complètent. Par exemple, une association entre le riz et le soja permet d’équilibrer l’apport en lysine, faible dans le riz mais élevé dans le soja, et l’apport des acides aminés soufrés, faible dans le soja mais élevé dans le riz.\\

\textbf{Les recommandations de l’Agence, à ce jour}\\

L’Agence a établi l’apport nutritionnel conseillé pour les protéines à 0,83 g/kg/j chez les adultes en bonne santé. 
Il est difficile, compte tenu de l’insuffisance de données disponibles, de définir une limite supérieure de sécurité pour l’apport protéique. Dans l’état actuel des connaissances, des apports entre 0,83 et 2,2 g/kg/j de protéines (soit de 10 à 27\% de l’apport énergétique) peuvent être considérés comme satisfaisants pour un individu adulte de moins de 60 ans.\\

L’apport nutritionnel conseillé est légèrement augmenté chez les personnes âgées, de l’ordre de 1 g/kg/j, ainsi que chez les femmes enceintes et allaitantes, de l’ordre d’au moins 70 g/j ou de 1,2 g/kg/j.\\

D’après l’étude INCA2, les apports quotidiens moyens en protéines sont de 74 g chez les femmes adultes et 100 g chez les hommes adultes et représentent pour les deux sexes environ 17\% des apports énergétiques totaux. Chez les enfants, les apports quotidiens moyens se situent à 63 g chez les 3-10 ans et 74 g chez les 11-17 ans et représentent 15 à 16\% des apports énergétiques totaux.\\

Les produits carnés (viandes, volailles, charcuteries) apportent 31\% des apports en protéines des adultes, suivis par les produits laitiers (17\%), et notamment les fromages (9\%), et les pains et produits de panification (11\%). Les mêmes aliments contribuent à l’apport protéique chez les enfants à quelques points près. Ainsi, les produits carnés sont le premier contributeur aux apports protéiques (28\%), suivis par les produits laitiers (21\%), et notamment le lait (10\%), et les pains et produits de panifications (7\%).

\newpage
\subsection{Compléments alimentaires destinés aux sportifs : des risques pour la santé pour des bénéfices incertains}

\begin{itemize}
	\item \textbf{Lien : }  \url{https://www.anses.fr/fr/content/compl\%C3\%A9ments-alimentaires-destin\%C3\%A9s-aux-sportifs-des-risques-pour-la-sant\%C3\%A9-pour-des-b\%C3\%A9n\%C3\%A9fices} 
	\item \textbf{Auteur : } Rapport d’expertise collective de l'ANSES
	
	\item \textbf{Date : } 20 décembre 2016
	\item \textbf{Source : } L’Agence nationale de sécurité sanitaire de l'alimentation, de l'environnement et du travail (ANSES) est l'agence nationale française chargée de la sécurité sanitaire. C'est un établissement public d'évaluation des risques dans les domaines de l'alimentation, de l'environnement et du travail. Elle résulte de la fusion en 2010 de l'AFSSA et de l'AFSSET. 
\end{itemize}

Le dispositif national de nutrivigilance, piloté par l’Anses, recueille les signalements d’effets indésirables susceptibles d’être liés à la consommation de compléments alimentaires destinés aux sportifs. Ces signalements et la consommation répandue dans plusieurs disciplines sportives de ce type de produits visant le développement musculaire ou la diminution de la masse grasse, conduisent l’Anses à attirer l’attention sur les risques potentiels pour la santé. Des effets potentiellement graves pour certains, majoritairement d’ordre cardiovasculaire (tachycardie, arythmie et accident vasculaire cérébral) et psychique (troubles anxieux et troubles de l’humeur), ont été observés. L’Agence déconseille donc l’usage de ces compléments alimentaires aux personnes présentant des facteurs de risque cardiovasculaire ou souffrant d’une cardiopathie ou d’une altération de la fonction rénale, hépatique ou encore de troubles neuropsychiatriques, aux enfants, adolescents et femmes enceintes ou allaitantes. L’Anses déconseille également la consommation de compléments alimentaires contenant de la caféine avant et pendant une activité sportive, ainsi que la consommation concomitante de plusieurs compléments alimentaires ou leur association avec des médicaments. L’Anses rappelle par ailleurs la nécessité de prendre conseils auprès d’un professionnel de santé avant de consommer des compléments alimentaires.\\



Le dispositif national de nutrivigilance de l’Anses a recueilli quarante-neuf signalements d’effets indésirables susceptibles d’être liés à la consommation de compléments alimentaires visant le développement musculaire ou la diminution de la masse grasse et destinés aux sportifs. Les effets indésirables rapportés étaient majoritairement d’ordre cardiovasculaire (tachycardie, arythmie et accident vasculaire cérébral) et psychiques (troubles anxieux et troubles de l’humeur).\\

Ces signalements d’effets indésirables ont conduit l’Anses à évaluer les risques associés à la consommation de ces compléments et à attirer l’attention des sportifs concernés sur les risques sanitaires induits par ces pratiques.\\

Afin de réduire ces risques, elle recommande aux consommateurs d’être attentifs à l’adéquation de ces compléments alimentaires avec leur statut nutritionnel, leur état de santé et les objectifs visés. De ce fait, un conseil personnalisé par un professionnel de santé, le cas échéant en lien avec l’entraîneur ou le préparateur physique, au regard des périodes et des charges d’entraînement, est indispensable. Afin d’assurer un dialogue interdisciplinaire efficace, il est important que les professionnels de santé bénéficient d’une solide formation initiale et continue en matière de nutrition, et en particulier de nutrition du sportif.\\

Par ailleurs, et plus spécifiquement en cas de recherche de diminution de la masse grasse et/ou d’augmentation de la masse musculaire, les pratiquants doivent être informés des risques liés, d’une part, à la consommation de produits présentant une activité pharmacologique et, d’autre part, des risques sanitaires liés à la pratique de régimes amaigrissants sans accompagnement médical.\\

L’Anses met l’accent sur le fait que des effets de ces compléments alimentaires qui pourraient être revendiqués sur la performance n’excluent en rien le risque sanitaire. De façon générale, l’absence de données d’efficacité scientifiquement démontrée rend les bénéfices escomptés de ces compléments alimentaires très fortement hypothétiques, rendant ainsi l’intérêt des produits les contenant largement discutable au regard des risques encourus. Par ailleurs, l’achat sur internet expose de facto davantage le sportif à la consommation de compléments alimentaires frauduleux ou adultérés, susceptibles de conduire à des contrôles anti-dopage positifs et d’induire des effets sur la santé.\\

\textbf{Les recommandations de l’Agence}\\

Au vu des résultats de son expertise, l’Anses déconseille fortement la consommation de compléments alimentaires visant le développement musculaire ou la diminution de la masse grasse :

	$\bullet$ aux personnes présentant des facteurs de risque cardiovasculaire ou souffrant d’une cardiopathie ou d’une altération de la fonction rénale ou hépatique ou encore de troubles neuropsychiatriques ;
	
	$\bullet$  aux enfants et adolescents ;
	
	$\bullet$ aux femmes enceintes ou allaitantes.\\


\textbf{A l’attention des consommateurs}\\

	$\bullet$ La consommation de compléments alimentaires contenant de la caféine est déconseillée avant et pendant une activité sportive, ainsi que chez les sujets sensibles aux effets de cette substance.
	
	$\bullet$ La consommation concomitante de plusieurs compléments alimentaires ou leur association avec des médicaments est déconseillée.
	
	$\bullet$ Les objectifs de la consommation de compléments alimentaires devraient être discutés avec un professionnel de santé.
	
	$\bullet$ La consommation de compléments alimentaires doit être signalée à son médecin et son pharmacien.
	
	$\bullet$ Les sportifs doivent être attentifs à la composition des produits consommés et privilégier les produits conformes à la norme AFNOR NF V 94-001 (juillet 2012) ainsi que les circuits d’approvisionnement les mieux contrôlés par les pouvoirs publics (conformité à la réglementation française, traçabilité et identification du fabricant).\\


\textbf{A l’attention des cadres sportifs :}\\

	$\bullet$ Le recours aux compléments alimentaires ne doit être envisagé que dans le cadre d’une approche pluridisciplinaire mobilisant tant les cadres sportifs que les professionnels de santé ;
	
	$\bullet$ Une information efficace aux pratiquants, en ciblant plus particulièrement les jeunes sportifs, doit être mise en œuvre.\\

En outre, considérant la banalisation de la consommation de ces compléments alimentaires, l’Agence recommande aux pouvoirs publics de mener une réflexion sur la pertinence de la distribution de ces produits sur les sites de pratique sportive.\\

L’Anses rappelle enfin aux professionnels de santé l’importance de la déclaration auprès de son dispositif de nutrivigilance des effets indésirables susceptibles d’être liés à la consommation de compléments alimentaires destinés aux sportifs dont ils auraient connaissance.
\end{document}


