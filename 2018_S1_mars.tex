\documentclass[8pt]{article}

\usepackage[T1]{fontenc}
\usepackage[utf8]{inputenc}
\usepackage{graphicx}
\usepackage{lmodern}
\usepackage{amsmath}
\usepackage{xfrac}
\usepackage{amsthm}
\usepackage{listings}
\usepackage{enumerate}
\usepackage{amssymb}
\usepackage{cancel}
\usepackage{amsfonts}
\usepackage{float}
\usepackage{fullpage}

\DeclareUnicodeCharacter{200A}{ } 
\renewcommand*\contentsname{Table des matières}

\PassOptionsToPackage{hyphens}{url}\usepackage{hyperref}

\usepackage{listings}
\author{ControverSciences\textit{ et al} }
\title{Projet AREN - Corpus de ressources \\  La conquête de Mars.}
\date{7 Mars 2018}

\begin{document}
\maketitle

\tableofcontents

\newpage
\section{Textes à débattre}
\subsection{}
\begin{itemize}
	\item \textbf{Lien : }  \url{} 
	\item \textbf{Auteur : } 
	\item \textbf{Date : }  
	\item \textbf{Source : }  
\end{itemize}

\newpage

\newpage
\section{Corpus de ressources}

\subsection{Mars One : un aller simple vers Mars ?}
\begin{itemize}
	\item \textbf{Lien : }  \url{https://www.youtube.com/watch?v=i2QtmWFRP7c} 
	\item \textbf{Auteur : } CNES
	\item \textbf{Date : } 20 mai 2015
	\item \textbf{Durée : } 3 minutes 31 secondes
	\item \textbf{Description : } Le projet Mars One prévoit d'envoyer vers Mars une expédition humaine en 2023. Pas moins de 200 000 candidats se sont portés volontaires à travers le monde. Pourquoi un tel engouement ? Réponses avec Jacques Arnould, chargé de mission sur les questions éthiques au CNES.
	\item \textbf{Source : } Le Centre national d'études spatiales (CNES) est en France l'acteur incontournable concernant la recherche spatiale. 
\end{itemize}

\subsection{Comment prépare-t-on l'arrivée de l'homme sur Mars ?}
\begin{itemize}
	\item \textbf{Lien : }  \url{https://www.youtube.com/watch?v=SwEqNH4aLjo} 
	\item \textbf{Auteur : } CNES
	\item \textbf{Date : } 5 juillet 2016
	\item \textbf{Durée : } 2 minutes 41 secondes
	\item \textbf{Description : } Sylvestre Maurice, astrophysicien, planétologue et co-responsable de l'instrument ChemCam sur Curiosity, nous explique comment les missions robotiques actuelles préparent l'arrivée de l'homme sur Mars.
	\item \textbf{Source : } Le Centre national d'études spatiales (CNES) est en France l'acteur incontournable concernant la recherche spatiale. 
\end{itemize}

\subsection{Poser le pied sur Mars, ça se prépare !}
\begin{itemize}
	\item \textbf{Lien : }  \url{https://www.youtube.com/watch?v=AdduX7PV3g0} 
	\item \textbf{Auteur : } CNES
	\item \textbf{Date : } 29 juillet 2016
	\item \textbf{Durée : } 3 minutes 26 secondes
	\item \textbf{Description : } En France, plusieurs organismes étudient la physiologie humaine dans la perspective d'une mission de plusieurs années dans l'espace pour les astronautes qui iront fouler le sol martien. Immersion à Toulouse. 
	\item \textbf{Source : } Le Centre national d'études spatiales (CNES) est en France l'acteur incontournable concernant la recherche spatiale. 
\end{itemize}

\subsection{Bientôt un astronaute sur Mars ?}
\begin{itemize}
	\item \textbf{Lien : }  \url{https://www.youtube.com/watch?v=AdduX7PV3g0} 
	\item \textbf{Auteur : } CNES
	\item \textbf{Date : } 16 novembre 2016
	\item \textbf{Durée : } 2 minutes 38 secondes
	\item \textbf{Description : } 
	\item \textbf{Source : } Le Centre national d'études spatiales (CNES) est en France l'acteur incontournable concernant la recherche spatiale. 
\end{itemize}

\end{document}


