\documentclass[8pt]{article}

\usepackage[T1]{fontenc}
\usepackage[utf8]{inputenc}
\usepackage{graphicx}
\usepackage{lmodern}
\usepackage{amsmath}
\usepackage{xfrac}
\usepackage{amsthm}
\usepackage{listings}
\usepackage{enumerate}
\usepackage{amssymb}
\usepackage{cancel}
\usepackage{amsfonts}
\usepackage{float}
\usepackage{fullpage}
\usepackage{pdfpages}

\DeclareUnicodeCharacter{200A}{ } 
\renewcommand*\contentsname{Table des matières}

\PassOptionsToPackage{hyphens}{url}\usepackage{hyperref}

\usepackage{listings}
\author{ControverSciences\textit{ et al} }
\title{Projet AREN - Corpus de ressources \\  La conquête de Mars}
\date{7 Mars 2018}

\begin{document}
\maketitle

\tableofcontents

\newpage
\section{Textes à débattre}
\subsection{Pourquoi Elon Musk ne doit pas envoyer l'Homme sur Mars}
\begin{itemize}
	\item \textbf{Lien : }  \url{https://www.lexpress.fr/actualite/pourquoi-elon-musk-ne-doit-pas-envoyer-l-homme-sur-mars_1837162.html} 
	\item \textbf{Auteur : } Thomas Jestin, passionné par l'espace et son exploration. Il publie régulièrement des chroniques de prospective dans la presse, et il est l'auteur du livre Pourquoi Elon Musk ne doit pas envoyer l'Homme sur Mars.
	\item \textbf{Date : }  4 octobre 2016
	\item \textbf{Source : } L'Express est un magazine d'actualité hebdomadaire français. Classé au centre-gauche, le journal revendique souvent son refus d'endosser une étiquette, il se veut aujourd'hui moderniste, favorable à l'économie de marché et au projet européen.
\end{itemize}


Avec SpaceX, la société privée de voyage spatial d'Elon Musk, le projet de colonisation de Mars devient de plus en plus concret. Mais est-il réellement souhaitable? Pour notre contributeur, un tel voyage aurait des conséquences irréversibles. 
Elon Musk vient de révéler au grand public les contours de son vaisseau spatial à même d'emmener 100 personnes sur Mars et dont le premier vol habité, si tout fonctionne comme prévu, devrait partir fin 2024 pour se poser sur la surface martienne début 2025.  
Si l'on peut douter de ce calendrier, il serait bien présomptueux de considérer pour autant que la tâche est impossible compte tenu du rythme des progrès techniques accomplis à ce jour par SpaceX. Tôt ou tard, l'Homme aura la possibilité d'aller fouler Mars. Mais est-ce souhaitable?\\

\textbf{Envoyer des hommes sans leurs microbes est impossible}\\

Envoyer des hommes sur Mars accroîtrait démesurément les risques de contamination de la surface par nos milliers de milliards de microbes, ce qui compliquerait drastiquement la recherche de formes de vie martienne. De nombreuses formes d'habitats, c'est-à-dire en biologie des espaces à même d'héberger la vie présente ou passée, sont envisagées. Tout ceci serait fascinant à étudier pour mieux comprendre possiblement comment apparaît la vie et mieux saisir notre place dans l'univers. 
Envoyer des hommes sans leurs microbes serait impossible. Notre corps en héberge près cent mille milliards, répartis en une dizaine de milliers d'espèces. Notre peau elle-même abrite environ mille milliards de microbes! L'air que nous respirons est aussi rempli de nombreux et invisibles micro-organismes. Les occasions de contamination sont nombreuses: fuites d'air du vaisseau vers l'extérieur à travers le sas, fuites d'air à travers la combinaison des marsonautes, accidents...  \\

\textbf{Des conséquences irréversibles}\\

Mais le plus gros des risques est celui de crash au sol du module lors de "l'amarsissage". Poser un appareil à la surface de Mars est un exercice des plus périlleux. Près des deux tiers de la quarantaine de missions envoyées vers Mars à ce jour se sont soldées par des échecs. Dans ce scénario du pire, des centaines de milliers de milliards de microbes se retrouveraient à la surface même de l'astre rouge aux abords des lieux du crash.  
On a longtemps pensé qu'une telle contamination serait brève et réversible, du fait des conditions martiennes hostiles à la vie. Mais la recherche a montré que certains micro-organismes pouvaient résister bien plus que prévu. Alors qu'on croyait qu'ils ne survivraient pas plus d'une minute, un sur un million était toujours viable après un an et demi d'exposition à l'équivalent des rayons solaires reçus par Mars. Du fait des tempêtes de poussière qui balaient régulièrement Mars, nos microbes auraient vite fait de se retrouver disséminés aux quatre coins de la planète. Alors, pourquoi ne pas continuer à explorer efficacement Mars sans y envoyer des hommes? \\

\newpage
\textbf{Garantir la survie de notre espèce n'est pas nécessaire}\\

Autre argument des partisans de la colonisation de Mars: garantir la survie de notre espèce. Or, ce n'est ni nécessaire ni suffisant. Elon Musk redoute qu'une catastrophe vienne éradiquer notre espèce sur Terre, mais en y regardant de plus près, soit ces catastrophes ne sont pas en mesure de nous tuer jusqu'au dernier (guerre nucléaire, épidémie, météorite, éruption volcanique), soit elles sont d'une nature telle (comme l'intelligence artificielle) qu'il n'y aucune raison de penser qu'une colonie sur Mars serait épargnée. 
Elon Musk invoque souvent en premier lieu le risque qu'un astéroïde vienne percuter notre planète et annihile notre espèce du même coup, comme ce fut le cas avec l'extinction des dinosaures il y a 66 millions d'années. Pourtant, aucun astéroïde d'une taille à même d'exterminer aujourd'hui l'humanité sur Terre, n'a impacté ni la Terre, ni la Lune, ni Mars, ni Mercure depuis la fin du grand bombardement tardif il y a des milliards d'années. Notre système solaire s'est très largement apaisé et rien ne justifie de penser que cela devrait changer. Qui plus est, pour une fraction du coût de construction d'une colonie sur Mars, nous pourrions investir dans les moyens d'identifier suffisamment à l'avance et détourner les astéroïdes susceptibles de croiser la Terre.  \\

\textbf{Rien ne presse!}\\

L'exploration humaine de Mars ne semble donc pas justifiée face à ce que l'on risque de perdre au change, à savoir la chance unique de comprendre -grâce à nos robots- ce qu'il s'est passé sur une planète qui a eu des océans pendant un milliard d'années. La vie y est-elle apparue? Que la réponse soit positive ou non, dans tous les cas, elle sera fascinante. 
Que de hardis explorateurs souhaitent prendre des risques avec leur propre vie, admettons, c'est leur choix. Mais compromettre sans retour une opportunité scientifique unique aussi importante est une décision qui ne peut être laissée à l'appréciation de quelques individus, aussi fortunés, audacieux et ingénieux soient-ils! Il appartient à la communauté mondiale de se prononcer après un vif et long débat bien informé.  
Pourquoi nous hâter? Rien, absolument rien ne presse. Aucun danger imminent à l'horizon ne menace l'humanité d'extinction. Prenons le temps d'étudier Mars en long, en large et en travers avec nos machines qui le font déjà plus économiquement que les hommes, et le feront de mieux en mieux. Et ne pas envoyer l'Homme sur Mars ne doit pas nous empêcher de continuer à prospérer dans l'espace où tant reste à faire. Peut-être déjà avec un retour de l'homme sur la Lune? 

\newpage

\subsection{Pourquoi il faut laisser des gens mourir sur Mars}
\begin{itemize}
	\item \textbf{Lien : }  \url{http://leplus.nouvelobs.com/contribution/1560187-pourquoi-il-faut-laisser-des-gens-mourir-sur-mars.html} 
	\item \textbf{Auteur : } Jean-Paul Fritz, écrivain et journaliste freelance principalement sur des sujets scientifiques.
	\item \textbf{Date : }   18 septembre 2016
	\item \textbf{Source : } Le Nouvel Observateur est un magazine d'actualité hebdomadaire français est classé à gauche, avec une ligne « sociale-démocrate ».
\end{itemize}

Mauvaise nouvelle pour les rêveurs : s'installer sur Mars, c'est dangereux. Déjà, pour y aller, il faut survivre intact au voyage. Les radiations cosmiques représentent un risque pour la santé, et il vaut mieux que le vaisseau de transport soit bien protégé. En plus, il y a les météorites. Même petites, elles peuvent faire des trous dans un habitacle, et vous n'auriez pas envie d'être exposés à une déperdition d'atmosphère.
On ajoutera à ça la nécessité de vivre à plusieurs dans un espace très confiné (bien pire que sur la station spatiale) durant des mois, et pour ceux qui voudront revenir, la perspective de remettre ça au voyage retour. De quoi créer des problèmes psychologiques intenses, pour ne pas dire des conflits ouverts. Un combat de boxe en apesanteur, ça doit être difficile à arbitrer.\\

\textbf{Le stratégique lieu d'installation}\\

L'installation, ensuite. Que ce soit pour une mission d'exploration ou une colonisation, il va falloir des bâtiments abrités. Car Mars est loin d'être un paradis. Son atmosphère est ténue, il faut donc un lieu étanche pour y respirer. Le lieu en question doit également être protégé des radiations, car contrairement à la Terre, il n'y a pas de ceintures protectrices dans la haute atmosphère. Pour les mêmes raisons, il faudra limiter les sorties en scaphandre : il y a des limites à l'irradiation cumulée qu'un humain peut supporter. 
On va supposer que le lieu d'installation de la colonie est proche d'un glacier, ou tout au moins d'une source importante et accessible de glace pour fournir l'eau indispensable...pour à peu près tout. L'eau permet de fabriquer l'air respirable, de nourrir les petites plantes, et même de fournir un éventuel carburant pour les promenades exploratoires variées... et même un générateur électrique.
Il faudra aussi des panneaux solaires en quantité, et peut-être un générateur nucléaire, au cas où. Car l'énergie sera aussi indispensable que l'eau, que ce soit pour le chauffage ou pour faire fonctionner les divers appareils assurant la survie de la colonie.\\

\textbf{Produire des ressources locales}\\

La nourriture, bien sûr. Il faudra la cultiver sur place, en prenant bien garde aux métaux lourds présents en abondance dans le sol martien. La bonne nouvelle, c'est que les cultures réalisées sur Terre dans la reproduction d'un terreau de Mars ont donné de bons résultats, et les légumes sont comestibles. Mais les colons seront à la merci d'une année de mauvaises récoltes. La plupart des éléments indispensables devront provenir de Mars. Les imprimantes 3D et le régolithe (sol martien) permettront de fabriquer beaucoup de choses, mais il faudra extraire les minerais, ce qui implique... des mines et des usines, même petites pour commencer. Le plus délicat, ce sera les composants électroniques. Sur Terre, ils sont faciles à fabriquer, mais sur Mars... il faut espérer que les nouvelles technologies d'impression se montreront à la hauteur, mais là encore, il faudra des métaux. En cas de panne d'un composant essentiel, si les pièces de rechange amenées de la Terre sont épuisées (elles seront nécessairement en quantité limitée vu les coûts d'acheminement), cela pourrait menacer la survie de membres de la colonie. Il sera également nécessaire de s'adapter à la gravité, qui n'est que du tiers de celle que nous avons sur Terre, et on n'en connaît pas les conséquences physiques sur le long terme. Enfin, les problèmes psychologiques et organisationnels liés à une vie entière dans un espace restreint et clos seront bien pires que les quelques mois de voyage.\\
\newpage
\textbf{Refuser le risque...}\\

Tous ces éléments sont autant de causes possibles de décès sur Mars, voire d'extinction temporaire d'une colonie (jusqu'à l'arrivée des colons suivants). La bonne nouvelle, c'est qu'ils semblent tous surmontables avec les technologies actuelles, le tout est de les mettre au point avant de partir... et de s'autoriser une marge d'erreur et d'improvisation sur place. Les colons martiens devront donc être des émules de McGyver et de bons agriculteurs, entre autres. Le hic, c'est qu'aujourd'hui les grandes agences spatiales, NASA en tête, ont l'habitude d'envoyer leurs astronautes dans des lieux où ils peuvent espérer être secourus, à portée de la Terre, même si les catastrophes comme celles des navettes Challenger et Columbia ont démontré que le risque est toujours là.\\

\textbf{... ou l'accepter}\\

Il n'en demeure pas moins que les normes de sécurité sont conçues pour des explorateurs, astronautes professionnels et scientifiques en missions temporaires, pas pour des aventuriers se lançant dans un grand bond en avant. L'objectif premier, c'est de faire en sorte qu'un astronaute qui part de la Terre en revienne en vie. Si l'on avait appliqué des principes similaires au 16ème siècle, les navires espagnols seraient toujours à croiser prudemment au large des côtes des Bermudes et les frégates anglaises ravitailleraient avec parcimonie les quelques bases d'exploration de la côte est de l'Amérique du Nord. 
Bien entendu, il ne s'agit pas là d'ériger en modèle un processus qui a conduit à des génocides culturels, des déportations en masses, des conversions forcées au christianisme et à l'esclavage, sans parler du pillage en règle des ressources naturelles. Mais on peut au moins tirer des leçons de l'histoire, d'autant que sur Mars il n'y a pas de vie intelligente qui pourrait en pâtir. La notion de risque acceptable est différente pour une colonisation que pour une exploration.\\

\textbf{Mars aura ses cimetières}

La colonisation est risquée. Les pertes en vies humaines sont non seulement prévisibles, mais elles font partie du processus. Ceux qui se lancent dans de telles aventures le savent, et sont prêts à prendre ce risque. Si on ne l'admet pas, on ne s'établira pas sur Mars avant plusieurs siècles, sans parler d'autres endroits du système solaire ou au-delà. Le futur colon martien prendra un ticket aller simple. Il ne sera pas astronaute, sinon de manière temporaire, le temps du voyage. Mais il devra avoir des compétences en matière de survie sur le terrain, et être prêt à mourir bêtement à cause d'une panne de circuit, une fuite d'air, une irradiation trop importante, ou parce qu'une récolte entière sera perdue. Il ne fera pas "un petit pas pour l'homme, un grand pas pour l'humanité". Les colons ne feront que de petits pas, tous ensemble. Ce n'est que leur oeuvre collective qui, avec le temps et des cimetières martiens bien remplis, aboutira à un établissement permanent sur la planète rouge. Ce sera là un prolongement naturel du chemin qui a mené nos ancêtres hors d'Afrique par vagues successives, pour s'installer sur des terres inhospitalières, les domestiquer, les façonner à leur usage et, au final, y prospérer. Ou y mourir.\\

\newpage

\section{Corpus de ressources}

\subsection{Vidéo - Mars One : un aller simple vers Mars ?}
\begin{itemize}
	\item \textbf{Lien : }  \url{https://www.youtube.com/watch?v=i2QtmWFRP7c} 
	\item \textbf{Auteur : } Témoignage de Jacques Arnould, chargé de mission sur les questions éthiques au CNES.
	\item \textbf{Date : } 20 mai 2015
	\item \textbf{Durée : } 3 minutes 31 secondes
	\item \textbf{Description : } Le projet Mars One prévoit d'envoyer vers Mars une expédition humaine en 2023. Pas moins de 200 000 candidats se sont portés volontaires à travers le monde. Pourquoi un tel engouement? 
	\item \textbf{Source : } Le Centre national d'études spatiales (CNES) propose aux pouvoirs publics la politique spatiale de la France et la met en œuvre dans 5 grands domaines stratégiques: Ariane, les Sciences, l’Observation, les Télécommunications et la Défense. De part son activité civil et militaire, le CNES est placé sous la tutelle conjointe des ministères de la Recherche et des Armées.
	 
\end{itemize}

\subsection{Vidéo - Comment prépare-t-on l'arrivée de l'homme sur Mars ?}
\begin{itemize}
	\item \textbf{Lien : }  \url{https://www.youtube.com/watch?v=SwEqNH4aLjo} 
	\item \textbf{Auteur : } Témoignage de Sylvestre Maurice, astrophysicien, planétologue et co-responsable de l'instrument ChemCam sur Curiosity
	\item \textbf{Date : } 5 juillet 2016
	\item \textbf{Durée : } 2 minutes 41 secondes
	\item \textbf{Description : } Comment les missions robotiques actuelles préparent l'arrivée de l'homme sur Mars.
	\item \textbf{Source : } Centre national d'études spatiales (CNES)
\end{itemize}

\subsection{Vidéo - Poser le pied sur Mars, ça se prépare !}
\begin{itemize}
	\item \textbf{Lien : }  \url{https://www.youtube.com/watch?v=m85lIGR-POk} 
	\item \textbf{Auteur : } CNES
	\item \textbf{Date : } 29 juillet 2016
	\item \textbf{Durée : } 3 minutes 26 secondes
	\item \textbf{Description : } En France, plusieurs organismes étudient la physiologie humaine dans la perspective d'une mission de plusieurs années dans l'espace pour les astronautes qui iront fouler le sol martien.
	\item \textbf{Source : } Centre national d'études spatiales (CNES)
\end{itemize}

\subsection{Vidéo - Bientôt un astronaute sur Mars ?}
\begin{itemize}
	\item \textbf{Lien : }  \url{https://www.youtube.com/watch?v=AdduX7PV3g0} 
	\item \textbf{Auteur : } Témoignage de François Spiero, responsable des vols habités au CNES
	\item \textbf{Date : } 16 novembre 2016
	\item \textbf{Durée : } 2 minutes 38 secondes
	\item \textbf{Source : } Centre national d'études spatiales (CNES)
\end{itemize}

\subsection{En route vers mars}
\begin{itemize}
	\item \textbf{Lien : }  \url{https://www.pourlascience.fr/sd/spatial/en-route-vers-mars-6688.php} 
	\item \textbf{Auteur : } Damon Landau et Nathan Strange travaillent au Jet Propulsion Laboratory pour la nasa.
	\item \textbf{Date : } 23 mars 2012
	\item \textbf{Source : } Pour La Science est la version française du mensuel Scientific American. C'est une revue de vulgarisation scientifique dans toutes les disciplines, dont les articles sont signés par les chercheurs eux-mêmes.
\end{itemize}

\includepdf[pages=-]{PLS84RouteVersMars.pdf}

\end{document}


