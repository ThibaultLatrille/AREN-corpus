\documentclass[8pt]{article}

\usepackage[T1]{fontenc}
\usepackage[utf8]{inputenc}
\usepackage{graphicx}
\usepackage{lmodern}
\usepackage{amsmath}
\usepackage{xfrac}
\usepackage{amsthm}
\usepackage{listings}
\usepackage{enumerate}
\usepackage{amssymb}
\usepackage{cancel}
\usepackage{amsfonts}
\usepackage{float}
\usepackage{fullpage}

\DeclareUnicodeCharacter{200A}{ } 
\renewcommand*\contentsname{Table des matières}

\PassOptionsToPackage{hyphens}{url}\usepackage{hyperref}

\usepackage{listings}
\author{ControverSciences\textit{ et al} }
\title{Projet AREN - Corpus de ressource \\  Les textiles et molécules innovants.}
\date{15 novembre 2017}

\begin{document}
\maketitle

\tableofcontents

\newpage
\section{Textes à débattre}
\subsection{Que contiennent les textiles innovants ?}
\begin{itemize}
	\item \textbf{Lien : }  \url{http://cca.asso.fr/que-contiennent-les-textiles-innovants/} 
	\item \textbf{Auteur : } La Chambre de Consommation d'Alsace
	\item \textbf{Date : }  18 mai 2017
	\item \textbf{Source : }  La Chambre de Consommation d'Alsace se donne pour missions d'informer et soutenir juridiquement les consommateurs, ainsi que d'accompagner les consommateurs vers des modes de consommation plus responsables.
\end{itemize}

Ils envahissent de plus en plus les rayons, et pas seulement dans les enseignes de sport : les vêtements « intelligents » ou « innovants » qui empêchent les odeurs, limitent la transpiration, les frottements, etc. Longtemps réservées au haut de gamme, ces innovations se démocratisent et sont aujourd’hui plus accessibles. Mais les textiles, pour avoir ces propriétés, contiennent bien souvent des nanoparticules, des parfums et autres composés pas toujours très recommandables...\\

Parmi les substances utilisées, on retrouve le zinc pyrithione, le chlorure de dimethyloctadecyl, le polytétrafuoroéthylène (PTFE) ou encore des nanoparticules d’argent, des microcapsules de parfum, etc. Si les fabricants se veulent rassurants, indiquant que tous les tests ont été effectués et que le consommateur n’encourt aucun risque, les associations de consommateurs se montrent bien plus prudentes. C’est le cas également de certains spécialistes, notamment concernant de potentiels risques environnementaux. Selon Éric Gaffet, chercheur spécialiste des nanomatériaux (Institut Jean-Lamour – Nancy), « à partir d’une certaine concentration, les nanoparticules d’argent peuvent tuer les bactéries indispensables au fonctionnement des stations des traitements des eaux. On sait aussi qu’elles sont néfastes pour la reproduction des espèces aquatiques ».\\

Autre problème, les fabricants n’indiquent pas toujours la mention « nano » sur leurs étiquettes contrairement à ce qu’impose la réglementation européenne. En revanche, rien ne les oblige à préciser les molécules utilisées, même si elles sont sous forme de nano (sauf s’il s’agit de biocides).\\

De son côté, l’Agence nationale de sécurité sanitaire de l’alimentation, de l’environnement et du travail (Anses) recommande dans un rapport de 2015 de « limiter l’usage des nanoparticules d’argent (production, transformation, utilisation) aux applications dont l’utilité est clairement démontrée et pour lesquelles la balance des bénéfices pour la santé humaine au regard des risques pour l’environnement est positive ». 

\newpage
\subsection{Des produits dangereux retrouvés dans des articles de sports de plein air}
\begin{itemize}
	\item \textbf{Lien : }  \url{https://www.greenpeace.fr/detox-des-produits-dangereux-retrouves-dans-des-articles-de-sports-de-plein-air/} 
	\item \textbf{Auteur : } Greenpeace
	\item \textbf{Date : }  25 janvier 2016
	\item \textbf{Source : } Greenpeace est une ONGI de protection de l'environnement présente dans plus de quarante pays à travers le monde1. Fondé en 1971, Greenpeace est un groupe de plaidoyer luttant contre ce qu'il estime être les plus grandes menaces pour l'environnement et la biodiversité sur la planète.	
\end{itemize}



Des produits chimiques dangereux ont été détectés dans des vêtements et des équipements de sports de plein air, d’après les résultats d’analyses publiés aujourd’hui par Greenpeace Allemagne, Italie et Suisse. Au total, 90\% des articles testés contenaient des PFC, une famille de substances chimiques persistantes. Les marques de sports de plein air font le bonheur des amoureux de la nature. Mais la respectent-elles vraiment ?\\

Les analyses ont porté sur 40 articles (vêtements, chaussures, tentes, sacs à dos, cordes et sacs de couchage) choisis par environ 30 000 internautes lors d’un vote organisé sur internet par des bureaux de Greenpeace.\\

Ces articles de 11 marques différentes ont été achetés dans 19 pays , et provenaient de différents lieux de production . Des PFC ont été détectés dans 36 articles, et des PFC à chaîne longue particulièrement dangereux dans 18 d’entre eux. Dans leurs communications publiques, la plupart des marques affirment pourtant ne plus avoir recours à ces substances.\\

Des PFOA ont été retrouvés dans des produits de The North Face et de Mammut. Il s’agit d’une catégorie de PFC à chaîne longue, considérée comme une substance « extrêmement préoccupante » par l’Agence européenne des produits chimiques. \\

Les composés perfluorés ou polyfluorés (PFC) sont des produits chimiques artificiels largement utilisés par les industriels du textile pour rendre leurs articles imperméables et antitaches.\\

Une fois présents dans l’environnement, ils se dégradent très lentement et peuvent s’immiscer dans la chaîne alimentaire. Des traces de PFC ont été retrouvées dans le foie de dauphins ou d’ours polaires, mais aussi dans des échantillons de sang humain. Certains PFC peuvent aussi affecter le système reproductif et endocrinien, et faciliter la croissance de tumeurs.\\

Ces produits chimiques sont aussi très volatiles. Au printemps 2015, des équipes de Greenpeace ont mené des expéditions dans huit régions montagneuses et reculées du globe pour y recueillir des échantillons d’eau et de neige. Tous les échantillons prélevés étaient positifs aux PFC, de la Chine à la Russie en passant par les Alpes et la Patagonie. Bien entendu, ces produits polluent aussi l’environnement des régions où les textiles sont fabriqués.

\newpage
\section{Corpus de ressources}
\subsection{Les vêtements : quand les toxiques se cachent}
 
\begin{itemize}
	\item \textbf{Lien : }  \url{http://www.asef-asso.fr/production/les-vetements-quand-les-toxiques-se-cachent-la-synthese-de-lasef/} 
	\item \textbf{Auteur : } Synthèse collaborative de l'ASEF
	\item \textbf{Date : } 17 juin 2017
	\item \textbf{Source : } L’Association Santé Environnement France (ASEF) a été fondée en 2008 par les Drs Pierre Souvet et Patrice Halimi. Elle est basée sur le bénévolat de médecins qui répondent aux problématiques de santé-environnement. L'association rassemble aujourd'hui près de 2 500 professionnels de santé en France. Depuis mars 2010, l'ASEF est reconnue d'intérêt général par l'État.
\end{itemize}

\textbf{Introduction}

Sur les étiquettes des vêtements que nous achetons, il est indiqué la composition en tissu… et c’est tout. Mais qu’en est-il de tous les ajouts destinés à colorer les vêtements, à les rendre plus souples ou plus résistants ? En effet, on peut retrouver de nombreuses substances chimiques lorsqu’on analyse des vêtements; si les concentrations ne sont pas forcément élevées, il est néanmoins important de les prendre en compte étant donné la multiplicité de ces produits, le fait qu’ils puissent avoir des effets à faible dose et surtout l’accumulation des effets. N’oublions pas non plus que ces substances se retrouvent le plus souvent dans l’environnement et les milieux aquatiques lors du lavage des vêtements.\\

\textbf{I. Les principales substances trouvées dans les vêtements}\\

\textbf{I.1 Les éthoxylates de nonylphénol}\\

Les nonylphénols sont des composés organiques synthétiques; ils sont toxiques, bioaccumulables et relativement persistants (demi-vie de plusieurs semaines). Ils sont considérés comme perturbateurs endocriniens et ont un effet sur la fertilité, la reproduction et la croissance. En raison de leur écotoxicité, la concentration maximale en éthoxylates de nonylphénol dans les vêtements en Europe est de 0,1\%.

Selon un rapport de Greenpeace de 2011, sur 78 échantillons de vêtements analysés, 52 contenaient des éthoxylates de nonylphénol. En 2012, sur plus de 140 vêtements, 63\% contenaient encore ces substances.

Certaines marques de vêtements ont annoncé l’exclusion de ces substances dans la fabrication de leurs vêtements; ainsi, Zara ou Levi’s ont présenté des engagements pour éliminer le rejet des produits chimiques d’ici 2020.\\

\textbf{I.2 Le formaldéhyde}\\

Le formaldéhyde est un composé organique volatil (COV) souvent présent dans les vêtements synthétiques; il permet aux tissus d’être infroissables, plus résistants et hydrofuges. Cependant il s’agit d’un gaz nocif pour la santé.

En effet, étant un gaz très volatil, il peut facilement entrer en contact avec les yeux ou le nez et engendre des irritations oculaires et des voies respiratoires. Il est également possible que de faibles expositions au formaldéhyde puissent accroître, à long terme, le risque de développer des pathologies asthmatiques et des sensibilisations allergiques [1]. Des effets loin d’être négligeables car ils peuvent, à terme conduire au développement cancer. En 2004, le Centre International de Recherche sur le Cancer (CIRC) a d’ailleurs classé le formaldéhyde dans le groupe 1 « substance cancérogène avérée pour l’homme » pour les cancers du nasopharynx par inhalation.

Le formaldéhyde présent dans les vêtements est également irritant pour la peau et peut alors déclencher des réactions allergiques, eczéma et dermites de contact (inflammation de la peau) principalement.

En France, les vêtements pour bébés entrant en contact avec la peau ne doivent pas contenir plus de 20 ppm de formaldéhyde. Les textiles en contact direct avec la peau ne doivent pas contenir plus de 100 ppm et ceux qui ne sont pas en contact direct avec la peau peuvent contenir jusqu’à 400 ppm.

L’Allemagne, quant à elle, prévoit une obligation d’étiquetage pour les vêtements entrant en contact avec la peau et libérant plus de 1500 ppm de formaldéhyde, qui doivent porter l’inscription « contient du formaldéhyde. Nous vous recommandons de laver ce vêtement avant de le porter pour éviter toute irritation de la peau »\\

\textbf{I.3 Les composés perfluorés}\\

Dans les vêtements, les composés perfluorés ont pour but d’augmenter les qualités infroissables et imperméables.

Considérés comme perturbateurs endocriniens, ils interfèrent avec le fonctionnement de la glande thyroïde et avec les œstrogènes L’exposition in utero aux perfluorés entraîne des retards de croissance et de développement, des changements comportementaux, un développement anormal des glandes mammaires et une diminution des niveaux de testostérone chez les mâles.\\

\textbf{I.4 Les retardateurs de flamme}\\


Les retardateurs de flamme sont des substances le plus souvent à base de brome, ajoutés à de nombreux objets lors de la fabrication afin de réduire le risque d’incendie. Ils sont présents dans les téléviseurs, les ordinateurs, les matelas, les meubles mais également dans les tissus et donc les vêtements.

Les retardateurs de flamme sont des perturbateurs endocriniens et auraient également un effet sur le développement du système nerveux (autisme, hyperactivité, déficit de l’attention, troubles du comportement…). En effet, plusieurs études ont montré l’action des retardateurs de flamme bromés sur les hormones thyroïdiennes [2]. Or, les hormones thyroïdiennes sont essentielles à de nombreuses fonctions et notamment au développement du système nerveux pendant la vie intra-utérine. En 2009, une étude a analysé les taux d’exposition prénatale aux retardateurs de flamme bromés en mesurant la concentration de ces substances dans le cordon ombilical à la naissance; les enfants ont ensuite été suivis pendant plusieurs années pour évaluer leur développement neurologique [3]. Les enfants présentant à la naissance les concentrations en retardateurs de flamme les plus élevées ont eu les résultats les plus faibles aux tests de développement physique et mental réalisés entre 1 et 6 ans. Ces effets étaient particulièremet remarquables à l’âge de 4 ans (réduction du QI et niveau verbal).\\

\textbf{I.5 Le diméthylfumarate}\\


Le diméthylfumarate (DMF) est une substance produite par l’industrie chimique qui se présente à température ambiante sous forme de cristaux blancs. Ce composé est généralement utilisé pour ses propriétés antifongiques, dans la conservation de semences, de textiles et de mobilier, principalement lors des opérations de stockage et de transport. Les cristaux sont contenus dans des sachés placés à l’intérieur de l’article ou dans l’emballage (ce sont par exemple les petits sachets que l’on retrouve dans les boîtes de chaussures neuves). Le DMF peut donc imprégner les produits eux-même.

Le DMF est connu pour ses propriétés allergisantes. Le contact avec des surfaces contaminées avec cette substance peut entraîner des allergies cutanées de type dermatose de contact : irritation, éruption cutanée, démangeaisons, phlyctènes (cloques)… Il est possible de réaliser un test épicutané pour savoir si une personne est sensibilisée au DMF.

Le DMF n’est plus autorisé dans l’Union Européenne depuis 2009, et a même été interdit en France depuis 2008. Cependant, une dizaine de cas d’allergie sont signalés chaque année, souvent après contact d’un meuble ou d’un objet fabriqué à l’étranger. En 2008, des bottes fabriquées en Chine avaient déjà provoqué des réactions allergiques. Une vingtaine de consommateurs de toute la France se manifestent et évoquent des brûlures, des démangeaisons, des eczémas importants voire des infections nécessitant une hospitalisation. La même année, des fauteuils commercialisés chez Conforama avaient aussi contaminé plusieurs clients, contraints d’être traités avec des crèmes à base de corticoïdes. L’enseigne avait ainsi rappelé plusieurs milliers de fauteuils importés de Chine.Il s’agissait de produits importés illégalement qui avaient échappé aux contrôles des douanes.\\


\textbf{I.6 Les colorants}\\


Les vêtements contiennent souvent de nombreux colorants; mais ceux-ci peuvent provoquer des réactions cutanées allergiques (type dermatite de contact), des dommages au foie et aux reins voire à long terme des cancers. Les colorants azoïques sont particulièrement dangereux; la règlementation européenne les a d’ailleurs interdits en 2002. Mais du fait de leur très bon rapport coût/efficacité, ces colorants continuent d’être régulièrement utilisés dans la fabrication de vêtements dans de nombreux pays.\\

\textbf{I.7 Les phtalates}\\


Les phtalates sont le plus souvent présents dans les dessins, inscriptions et décorations collés sur les vêtements; or, les phtalates sont des perturbateurs endocriniens avérés. Une étude a montré que l’exposition des testicules d’hommes adultes aux phtalates entraînait une inhibition de la production de testostérone, responsable d’une réduction de la taille des testicules [16]. Les phtalates seraient également responsables d’une puberté précoce chez des filles de 6 à 8 ans [17]. Enfin, une étude de 2009 a montré que l’exposition régulière aux phtalates pendant la grossesse (par exemple chez les coiffeuses) entraînait un risque triplé d’une malformation congénitale, l’hypospadias, chez le bébé [18].

En 2012, l’ONG Greenpeace a retrouvé dans certains vêtements pour enfants des taux 370 fois supérieurs aux limites autorisées.\\

\textbf{I.8 Les nanoparticules}\\


Des nanoparticules dans les vêtements ? Cela arrive… l’industrie textile recourt parfois aux nanotechnologies pour améliorer ses produits et leur résistance à l’eau, au feu ou à l’abrasion. Les vêtements de sport sont particulièrement concernés; en effet, il arrive que des nanoparticules d’argent soient intégrées aux fibres pour leurs propriétés bactéricides et pour lutter contre les mauvaises odeurs.

Selon l’avis de l’ANSES, la dispersion de nanoparticules à partir de vêtements est importante [4]. En effet, il est estimé qu’une chaussette anti-odeur libère 144mg de nanoparticules lors d’un lavage. En considérant qu’1 Français sur 10 utilise des chaussettes anti-transpiration et pour un achat de 10 paires par an, le lavage entraînerait le relargage annuel de 18 tonnes de nanoargent dans les milieux aquatiques. Il est donc évident que même s’il est délicat de mesurer le risque pour l’environnement, il y a nécessité d’une attention particulière.

Il n’existe pas d’étude ayant analysé le passage des nanoparticules à travers la barrière cutanée à partir de vêtements. Néanmoins, les nanoparticules sont susceptibles d’altérer les cellules et peuvent entraîner des lésions au niveau de l’ADN, conduisant à des mutations et pouvant avoir des effets cancérigènes. Par précaution, il est préférable pour les sportifs de ne pas utiliser de vêtements anti-transpirants.\\


\textbf{II. Les cibles idéales}\\

\textbf{II.1 Les enfants}\\

Les vêtements destinés aux enfants nécessitent d’être particulièrement vigilants quant à leur composition, en raison de la plus grande susceptibilité des enfants aux substances toxiques. En 2009, l’institut national de la consommation a mené une étude sur la composition de vêtements pour enfants; au total, 40 T-shirts pour enfants ont été testés, et 9 comportaient un taux de phtalates supérieur à la norme européenne. L’étude a également montré un taux élevé de résidus chimiques alcalins aux propriétés irritantes. En effet, des produits de ce style sont souvent utilisés afin de blanchir les tissus.

L’Institut National de la Consommation recommande de systématiquement laver les vêtements pour enfants avant usage afin d’éliminer un maximum de résidus chimiques. Mais d’autres substances comme les phtalates ne peuvent pas s’éliminer au lavage.

N’oubliez pas non plus que les enfants en bas âge vont également mettre leurs vêtements à la bouche; leur exposition n’a donc rien à voir avec celle d’un adulte.\\

\textbf{II.2 Les sportifs}\\


Les vêtements de sport sont souvent dotés de caractéristiques spécifiques, comme des propriétés anti-transpirantes, anti-odeurs, ne retenant pas l’humidité, légères, etc…\\

\textbf{III. Des effets environnementaux}\\


N’oublions pas non plus le coût environnemental de ces vêtements. En effet, ces substances toxiques sont rejetées dans l’environnement lors de la production mais aussi lors du lavage des vêtements, où les composés se retrouvent dans les eaux usées et sont rarement éliminés dans les stations d’épuration. On les retrouve donc dans l’environnement et notamment dans les milieux aquatiques.\\


\textbf{IV. Conseils pour diminuer l’exposition liée aux vêtements}\\


Lisez avec soin les étiquettes sur la composition des vêtements, privilégiez les fibres naturelles avec des labels écologiques « sans solvants »
Privilégiez les fibres biologiques : par exemple la culture du coton traditionnel est très gourmande en pesticides et en eau; le coton biologique aura les mêmes atouts mais contiendra beaucoup de substances néfastes pour la santé et l’environnement
Évitez les vêtements portant des motifs en plastique
Attention aux allégations « anti-odeurs » aux propriétés infroissables ou imperméables
Lavez toujours vos vêtements neufs avant de les porter
Pourquoi ne pas se lancer dans l’achat de vêtements d’occasion ? Comme dans le cas du mobilier, les produits néfastes seront beaucoup moins présents que dans des vêtements neufs.\\

\textbf{Conclusion}\\


Nos vêtements peuvent contenir de nombreuses substances nocives pour notre santé et notre environnement. Veillons donc à les choisir les plus sains possibles, en favorisant des filières biologiques et écoresponsables lorsque c’est possible; notre santé n’en sera que meilleure !

\end{document}


