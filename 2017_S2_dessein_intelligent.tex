\documentclass[8pt]{article}

\usepackage[T1]{fontenc}
\usepackage[utf8]{inputenc}
\usepackage{graphicx}
\usepackage{lmodern}
\usepackage{amsmath}
\usepackage{xfrac}
\usepackage{amsthm}
\usepackage{listings}
\usepackage{enumerate}
\usepackage{amssymb}
\usepackage{cancel}
\usepackage{amsfonts}
\usepackage{float}
\usepackage{fullpage}
 
\DeclareUnicodeCharacter{200A}{ } 
\renewcommand*\contentsname{Table des matières}

\PassOptionsToPackage{hyphens}{url}\usepackage{hyperref}

\usepackage{listings}
\author{ControverSciences\textit{ et al} }
\title{Projet AREN - Corpus de ressource \\ Le dessein intelligent}
\date{15 novembre 2017}

\begin{document}
\maketitle
« L'évolution des espèces, par la mutation et la sélection naturelle, est un fait indéniable scientifiquement. Et d'un point de vue religieux, l'évolution est la plus grande réussite de Dieu. »
\tableofcontents
\newpage
\section{Textes à débattre}
\subsection{Dieu a-t-il utilisé l’évolution pour créer l’homme ?}
\begin{itemize}
	\item \textbf{Lien : }  \url{https://creation.com/did-the-creator-use-evolution-french} 
	\item \textbf{Auteur : } Allan Rosser
	\item \textbf{Date : } Mars 1989
	\item \textbf{Source : } Creation Ministries International est une organisation à but non lucratif qui a pour objectif de promouvoir une interprétation littérale du Livre de la Genèse.
\end{itemize}

 À un moment donné dans votre vie, vous avez probablement rencontré les trois opinions que les gens ont sur l’origine de la vie.

	$\bullet$ L’idée traditionnelle de la création spéciale par le Dieu tout-puissant qui forma l’homme de la poussière du sol. C’est la position créationniste.

	$\bullet$ L’évolution à partir d’éléments non vivants, un développement sans fin en passant par des créatures semblables à des singes jusqu’à ce que, finalement, l’existence humaine devienne différente de celle des primates. C’est la position évolutionniste athée.

	$\bullet$ Un mariage entre 1 et 2. Dieu mit l’évolution en marche, guida le progrès et le développement, et créa ainsi l’homme. Par conséquent, l’homme est quand même responsable devant son Créateur et le scientifique évolutionniste est considéré comme étant crédible. Cette perspective est connue sous le nom d’évolution théiste.

La troisième position, celle de l’évolutionniste théiste, est répandue, et pourtant c’est peut-être la plus difficile à justifier. Le récit biblique de la création doit alors être expliqué différemment, soit peut-être comme une parabole ou en tant qu’histoire mal interprétée dont l’intention serait d’enseigner une leçon théologique.

\newpage

\subsection{Qu’est-ce que l’évolutionnisme théiste ?}
\begin{itemize}
	\item \textbf{Lien : }  \url{https://www.gotquestions.org/Francais/levolution-theiste.html} 
	\item \textbf{Auteur : } gotquestions.org
	\item \textbf{Source : } gotquestions.org est site web géré par une communauté chrétienne qui se donne l'objectif de répondre aux questions de tous concernant la bible et son interprétation. Actuellement le site est traduit en 166 langues pour un total de 35 000 pages traduites.
\end{itemize}

 L’évolution théiste est l’une des trois principales conceptions de l’origine de la vie, les deux autres étant l’évolutionnisme athée (couramment appelée évolution darwinienne ou naturaliste) et le créationnisme.\\
 

L’évolutionnisme athée affirme qu’il n’existe pas de Dieu et que la vie peut apparaître et est apparue naturellement à partir d’éléments inertes préexistants, sous l’influence des lois de la nature (comme la loi de la gravité, etc.), bien que l’origine de ces lois ne soit pas expliquée. Le créationnisme, lui, affirme que Dieu a créé la vie directement, à partir de rien ou de matière préexistante.\\


Il existe deux formes d’évolutionnisme théiste. La première affirme que Dieu existe, mais qu’il n’était pas directement impliqué dans l’origine de la vie. Il a peut-être créé les éléments de base et les lois naturelles, peut-être même en ayant à l’esprit l’apparition éventuelle de la vie, mais à un moment donné, il s’est retiré et a laissé la création prendre le relais. Il l’a laissée fonctionner et la vie a fini par apparaître à partir de la matière inerte. Cette conception ressemble à l’évolutionnisme athée en ce qu’elle défend une origine naturaliste de la vie.\\


La deuxième affirme que Dieu n’a pas fait qu’un ou deux miracles pour faire apparaître la vie telle que nous la connaissons, mais des miracles constants, dirigeant la vie pas à pas de sa simplicité primitive vers la complexité actuelle, selon l’arbre de vie évolutionniste de Darwin (les poissons engendrent des amphibiens, qui engendrent des reptiles, qui engendrent des oiseaux et des mammifères, etc.) Chaque fois que la vie ne pouvait pas évoluer naturellement (comment un membre de reptile peut-il évoluer naturellement en aile d’oiseau ?), Dieu est intervenu. Cette conception ressemble au créationnisme en ce qu’elle suppose que Dieu est intervenu de façon surnaturelle pour faire apparaître la vie telle que nous la connaissons.\\

\newpage
\section{Corpus de ressources}
\subsection{Charles Darwin: de l’origine d’une théorie}

\begin{itemize}
	\item \textbf{Lien : }  \url{https://lejournal.cnrs.fr/articles/charles-darwin-de-lorigine-dune-theorie} 
	\item \textbf{Auteur : } Philippe Testard-Vaillant
	\item \textbf{Date : }  6 août 2015
	\item \textbf{Source : } CNRS Le journal, l'objectif de ce site est de partager largement avec les amateurs de science, les professeurs et leurs élèves, les étudiants et tous les citoyens curieux, des contenus destinés jusque-là à la communauté des agents du CNRS, chercheurs, ingénieurs et techniciens, ceux des labos comme ceux des bureaux.
\end{itemize}

Il y a plus de cent cinquante ans, le célèbre naturaliste révolutionnait l’histoire de la vie en mettant sur pied les théories de l’évolution et de la sélection naturelle. À l’heure où les créationnistes regagnent du terrain, retour sur ses travaux essentiels. \\

La théorie de l’évolution des espèces, échafaudée par le savant à la barbe blanche et sans cesse enrichie, complétée, complexifiée par des générations de chercheurs au prix d’un nombre incalculable de travaux sur le terrain et en laboratoire, paraît indétrônable. Ce que dit Darwin au milieu du XIXe siècle ? Que les organismes vivants sont en perpétuelle évolution, grâce notamment au phénomène de sélection naturelle qui fait qu’au sein d’une même espèce les individus les plus adaptés à leur milieu se reproduisent davantage que les autres. Et que toutes les espèces (l’homme n’est pas exclu de ce schéma) descendent d’un ou de plusieurs ancêtres communs. Un bouleversement dans la vision traditionnelle chrétienne qui prévaut alors, et pour laquelle les créatures en tout genre qui peuplent la planète sont des créations divines, immuables et indépendantes les unes des autres.\\

« La théorie de l’évolution au sens darwinien du terme est actuellement le meilleur cadre conceptuel que nous ayons à notre disposition pour comprendre rationnellement l’instabilité du vivant, pour penser un monde naturel essentiellement dynamique », commente Hervé Le Guyader, du laboratoire Évolution Paris Seine.\\

\textbf{Les grands principes de l’évolution}\\

En ce début de troisième millénaire, l’explication des mécanismes de l’évolution biologique formulée par Darwin et ses successeurs repose sur quatre principes fondamentaux. Premièrement : « Parmi les individus qui se reconnaissent comme partenaires sexuels potentiels, il existe des variations (physiques, génétiques, d’aptitude…). Quelle que soit la cause de cette variation, les espèces vivantes manifestent par conséquent une capacité naturelle à varier », explique Guillaume Lecointre, de l’Institut de Systématique, évolution, biodiversité.\\

Deuxièmement, toute espèce se laisse sélectionner. Les horticulteurs qui créent, par exemple, de nouvelles variétés de roses en croisant entre elles d’anciennes variétés, et les éleveurs, qui ont fait du loup un teckel en 11 000 ans, le savent bien. « Le simple fait que les hommes puissent changer à leur guise la morphologie d’une espèce montre bien que celle-ci est en quelque sorte “plastique”, possède une capacité à être modifiée », dit Guillaume Lecointre.\\

Troisièmement, toutes les espèces se reproduisent aussi longtemps qu’elles trouvent des ressources alimentaires et des conditions optimales d’habitat. Leur taux de reproduction est alors tel qu’elles parviennent toujours aux limites de ces ressources ou trouvent d’autres limites, telles que la prédation qu’elles subissent de la part d’autres espèces. « Il existe ainsi une capacité naturelle de surpeuplement observable lorsque, par exemple, des espèces allogènes envahissent brutalement un milieu fermé comme une île », poursuit Guillaume Lecointre. Meilleur exemple : les lapins introduits au XIXe siècle en Australie s’y sont mis à pulluler, détruisant la végétation et les cultures. Pour autant, la planète n’est pas dominée par une unique espèce hégémonique, « mais bien au contraire peuplée de millions d’espèces en coexistence et cela malgré la capacité naturelle de surpeuplement de chacune d’entre elles. Ainsi, chaque espèce constitue une limite pour les autres soit en occupant leur espace, soit en les exploitant (prédation, parasitisme), soit en partageant les mêmes ressources. Bref, les autres espèces constituent autant de contraintes qui jouent un rôle d’agent sélectif ».\\ 

Quatrièmement, le succès de la croissance et de la reproduction des espèces dépend d’optima physiques (température, humidité, soleil…) et chimiques (pH, molécules odorantes, toxines…). « Ces éléments constituent eux aussi des facteurs contraignants, dit Guillaume Lecointre. S’ils changent, les variants avantagés ne seront plus les mêmes. »\\

En définitive, de multiples facteurs, au sein de l’environnement physique, chimique et biologique dans lequel évolue une espèce, induisent une sélection naturelle à chaque génération, dont le résultat est un « succès reproductif différentiel ». Traduction : au sein d’une même espèce, les individus porteurs d’une variation héritable, momentanément avantageuse par les conditions du milieu, se reproduiront davantage. « Si ces conditions se maintiennent assez longtemps, ajoute Guillaume Lecointre, le variant avantagé finira par avoir une fréquence de 100 \% dans la population. L’espèce aura alors changé. » Conclusion, aucune espèce n’est stable dans le temps.\\

\textbf{Les prédécesseurs}\\

S’il revient à Darwin d’avoir postulé deux grandes idées – la descendance avec modification et le rôle essentiel de la sélection naturelle dans l’adaptation des formes vivantes, donc dans l’évolution –, celles-ci ne lui sont pas venues tout à trac. Le terrain avait été débroussaillé, entre autres, par le zoologiste Jean-Baptiste de Monet, chevalier de Lamarck, et le géologue écossais Charles Lyell. C’est d’ailleurs lesté du premier volume des Principles of Geology, de Lyell, que le jeune Darwin quitte Plymouth fin 1831 pour effectuer un tour du monde à bord du navire Beagle. Un très long voyage d’exploration naturaliste au cours duquel Darwin pose le pied sur les îles Galapagos où s’ébattent des tortues terrestres, des iguanes, des otaries, des pinsons…\\

Ces oiseaux, tout en présentant entre eux de frappantes ressemblances morphologiques, se distinguent par divers détails comme la forme et la taille de leur bec. Darwin comprend que l’isolement de ces volatiles sur des îles les a conduits, à partir d’une souche unique d’origine continentale, à présenter des variations liées probablement à des différences de mode de vie et d’habitudes alimentaires. Plus de vingt ans de labeur vont s’ensuivre avant que ne paraisse De l’origine des espèces. Deux décennies au cours desquelles Darwin « écrit à des correspondants du monde entier, les questionne, leur demande des statistiques, se renseigne sur la systématique des espèces qu’il observe et en tient compte pour ses interprétations. Comme s’il concevait déjà que le principe selon lequel les espèces dérivent d’ancêtres communs devait être utilisé pour étudier l’acquisition des adaptations, comme on le fait aujourd’hui », dit Michel Veuille, de l’Institut de Systématique, évolution, biodiversité.\\

Alors que de nombreux exégètes de Darwin font de 1859 le temps zéro d’un événement scientifique hissant la biologie au rang de science historique, l’épistémologue André Pichot, du Laboratoire de philosophie et d’histoire des sciences-Archives Henri Poincaré, minimise l’importance de Darwin dans l’histoire des sciences. Selon lui, « le darwinisme de 1859 ne consiste guère qu’en la sélection naturelle. Or celle-ci n’était plus vraiment une nouveauté au milieu du XIXe siècle. On trouve par exemple ce concept en 1813 chez William Charles Wells puis, en 1831, chez Patrick Matthew, qui accusera Darwin de plagiat. On sait aussi qu’Alfred Russel Wallace en avait conçu une version comparable à celle de Darwin en même temps que celui-ci. Sans oublier le pasteur, géologue et politologue Joseph Townsend, dont Darwin a quasiment recopié les thèses en ce domaine ». En fait, poursuit André Pichot, l’idée de sélection était déjà plus ou moins dans l’air du temps. Et, si elle a fait le succès de Darwin, c’est que le moment était propice. « La seconde moitié du XIXe siècle a vu le triomphe du libéralisme économique, et Darwin a apporté à celui-ci un argument de poids en lui donnant un fondement naturel. »\\

Une interprétation qui fait bondir les aficionados du grand Charles. « L’idée novatrice de Darwin, plus que la sélection naturelle, c’est la descendance avec modification, le fait que les espèces ont une histoire et sont apparentées, intervient Hervé Le Guyader. La désormais célèbre réunion organisée en juin 1860 à Oxford par l’évêque Samuel Wilberforce porte d’ailleurs sur ce point. Wilberforce, apostrophant le darwinien Thomas Huxley, lui demande si c’est “par son grand-père ou par sa grand-mère qu’(il) descend du singe” et s’attire cette réponse non moins célèbre : mieux vaut un singe qu’un imbécile… »\\

\textbf{La génétique en renfort}\\

Si la théorie de Darwin bouleverse la vision chrétienne traditionnelle du monde, elle souffre d’un lourd handicap : les causes et les lois de l’hérédité, ainsi que la véritable nature de son support matériel, sont encore inconnues. Tout en soutenant que la sélection naturelle est le mécanisme principal de l’évolution, il pense aussi que les caractères acquis au cours de l’existence peuvent se transmettre à la descendance. Pourtant, les contre-exemples sont faciles à trouver : ainsi, un mari devenu cul-de-jatte donne à sa femme des enfants dotés de deux jambes…\\

« La théorie darwinienne de la sélection naturelle connaît une “éclipse” à partir de la mort de Darwin en 1882, intervient Michel Veuille. Après la redécouverte des lois de Mendel sur la transmission héréditaire 5 en 1900, une science nouvelle, la “génétique des populations”, va retrouver toute l’importance de la notion de “sélection naturelle”. Les modèles mathématiques 6 proposés par Fisher, Haldane et Wright reçoivent la reconnaissance de la communauté scientifique en 1932. Ensuite seulement, des expérimentateurs feront de la génétique des populations naturelles une discipline “de terrain” ».\\

Les années 1940 à 1970, quant à elles, vont assister au mariage de la génétique des populations avec la zoologie, la botanique et la paléontologie, qui se regardaient jusqu’ici en chiens de faïence, et à la naissance de la « théorie synthétique de l’évolution ». Ses promoteurs, explique Guillaume Lecointre, « cherchent à décortiquer les mécanismes engendrant la biodiversité en partant des mécanismes décrits par la génétique des populations et en intégrant les savoirs des naturalistes sur les variations naturelles géographiques au sein des espèces et sur la spéciation».\\

\textbf{La postérité}\\

Autre aménagement apporté à la théorie de l’évolution : le modèle dit neutraliste, du généticien japonais Motoo Kimura. « Selon ce chercheur, dit Michel Veuille, la plupart des changements observés entre le génome des diverses espèces ne s’expliquent pas par la sélection naturelle, dont il admet cependant l’existence, mais par le hasard, qui modifie insensiblement la fréquence des variations d’une génération à l’autre. » Aux cours des dernières décennies, de nombreux autres chercheurs ont apporté de l’eau au moulin de la théorie synthétique de l’évolution et l’ont affinée. À commencer par les paléontologues Stephen Jay Gould et Niles Eldredge. Leur nouveau modèle, l’« évolution à équilibres ponctués », montre que la transformation des espèces s’opère par à-coups entrecoupés de longues plages de stagnation, souvent en réponse à des changements dans l’environnement. Pendant la phase « explosive », une petite population de « marginaux » s’isole de sa population souche en occupant un nouvel environnement. Après avoir prospéré, elle étend son territoire et remplace (éventuellement…) la population souche de départ par compétition interspécifique, comme chez les trilobites (des arthropodes marins) de l’ère primaire. « Ainsi interprète-t-on pourquoi, dans une série sédimentaire continue, une espèce stable durant plusieurs millions d’années se trouve brusquement supplantée par une autre espèce qui lui est apparentée », commente Guillaume Lecointre.\\

Associé, cette fois, à Richard Lewontin, Stephen Jay Gould corrige par la suite la vision trop « panglossienne » de la théorie synthétique. Gould et Lewontin font observer que « des variants désavantagés continuent d’apparaître en permanence et amènent les évolutionnistes à relativiser leur impression d’ “une nature bien faite”, précise Guillaume Lecointre. Par ailleurs, ils mettent en évidence que certaines structures qui paraissent handicapantes (tel l’accouchement par le clitoris chez les hyènes tachetées, qui provoque le décès d’une partie des nouveau-nés) sont en fait liées biologiquement à d’autres structures qui fournissent des avantages déterminants (comme l’agressivité des femelles), d’où leur maintien ».\\

Autre étape clé dans la sophistication continue de la théorie synthétique : la méthode mise au point dans les années 1950 par l’entomologiste allemand Willi Hennig pour reconstituer l’histoire évolutive des espèces, c’est-à-dire identifier leurs degrés de parenté et construire l’arbre de la vie, et ses applications informatisées dès les années 1970. Ce remaniement complet de la systématique (la science des classifications des organismes), couplée plus tard avec le séquençage massif des génomes, va permettre de « mettre sur le même “arbre du vivant” tout à la fois des champignons, des bactéries, des animaux… alors que, jusqu’ici, on ne pouvait classer entre eux que des vertébrés ou des végétaux », dit Hervé Le Guyader.\\

\textbf{Les apports de l’embryologie}

Dernier coup de booster en date donné à la théorie de l’évolution : l’essor de l’« évo-dévo », une discipline centrée sur l’identification des gènes à la base du développement embryonnaire, l’étude de leur répartition au sein du monde animal et leur comparaison. De quoi mieux interpréter, en particulier, les homologies d’organes entre grands groupes d’animaux. « Darwin aurait été séduit par la rencontre de l’embryologie, à laquelle il s’est beaucoup intéressé, avec la génétique par le biais de l’évo-dévo qui plonge le développement, et ses gènes associés, dans un cadre évolutif », fait remarquer Hervé Le Guyader.\\

Autant d’axes de recherche qui montrent que les idées pionnières du naturaliste anglais se sont énormément enrichies au cours du XXe siècle. « Les spécialistes de l’évolution ont aujourd’hui à leur disposition une grande palette de modèles et de mécanismes avec lesquels jouer pour rendre compte des phénomènes évolutifs, résume Michel Morange, professeur de biologie à l’UPMC et à l’ENS, directeur du Centre Cavaillès. Leur travail ne consiste pas à tenter de falsifier la théorie darwinienne », mais à mettre à l’épreuve tel ou tel modèle de la galaxie darwinienne.\\

\textbf{Le créationnisme, une dangereuse croisade contre Darwin}


« Je ne suis pas que le chevalier blanc qui pourfend le créationnisme, bien qu’il faille traiter ce sujet », vous répond Pascal Picq, paléoanthropologue Collège de France, un brin agacé d’avoir à commenter une nouvelle fois les méfaits de la croisade que mènent aux États-Unis les milieux fondamentalistes protestants contre la théorie de l’évolution. « Ces Églises, qui professent que l’Univers et la Terre ont été créés par un dieu il y a environ 6 000 ans, ne cessent de gagner du terrain et visent à rien de moins qu’à installer une théocratie, dit-il en retrouvant tout son punch. L’Europe n’est pas à l’abri. Le regain de créationnisme auquel on assiste aujourd’hui ne constitue ni plus ni moins qu’une menace pour la laïcité et la démocratie. » Autre courant de pensée qui a le don d’ulcérer les évolutionnistes : le « dessein intelligent », un « néocréationnisme » qui se présente comme une science et affirme que certains faits de l’évolution (par exemple la formation de dispositifs structuraux et fonctionnels complexes comme l’oeil) seraient à jamais inexplicables par la science, et qu’il faut donc rechercher des causes non naturelles à leur survenue. « Le dessein intelligent invoque l’existence d’une “intelligence supérieure” pour expliquer la fabuleuse diversité du vivant », dit Pascal Picq. Comment repousser les assauts du créationnisme et du dessein intelligent ? En réhabilitant en priorité les concepts fondateurs de la théorie de l’évolution dans les programmes scolaires.\\

\newpage
\subsection{Imaginer une autre évolution de la vie sur Terre}

\begin{itemize}
	\item \textbf{Lien : }  \url{https://lejournal.cnrs.fr/billets/imaginer-une-autre-evolution-de-la-vie-sur-terre} 
	\item \textbf{Auteur : }  Virginie Orgogozo
	\item \textbf{Date : }  31 mars 2016
	\item \textbf{Source : } CNRS Le journal, l'objectif de ce site est de partager largement avec les amateurs de science, les professeurs et leurs élèves, les étudiants et tous les citoyens curieux, des contenus destinés jusque-là à la communauté des agents du CNRS, chercheurs, ingénieurs et techniciens, ceux des labos comme ceux des bureaux.
\end{itemize}

Notre monde vivant est-il juste une alternative parmi tant d’autres ? Au vu des nombreux événements qui se sont répétés au cours de l’histoire de la vie sur Terre, la biologiste Virginie Orgogozo, lauréate du prix de la «Jeune femme scientifique de l’année 2014», s’interroge sur les autres voies qu’aurait pu prendre l’évolution. \\


\textbf{ Sans le hasard, vous ne seriez pas en train de lire cet article}\\

Vingt ans après, je repense à ce que j’ai appris pendant mes études. À l’époque, on nous expliquait – et on le fait encore aujourd’hui – que notre planète Terre avait suivi une évolution désordonnée imprévisible, et que l’existence de telle ou telle espèce était tout simplement fortuite. Il semblait évident que, si les conditions avaient été légèrement modifiées, alors aurait évolué sur la Terre un monde vivant radicalement différent. Effectivement, il paraissait naturel de penser que, si une météorite n’avait pas frappé la surface de la Terre il y a 65 millions d’années, les dinosaures n’auraient pas disparu brutalement et vous ne seriez pas en train de lire cet article aujourd’hui. N’en déplaise à mon émerveillement pour les organismes vivants, il fallait admettre que la trajectoire de la vie sur Terre avait été extrêmement sensible aux conditions initiales.\\

\textbf{Observer et expérimenter pour comprendre à quel point nous serions différents}\\

Bien sûr, si on rembobinait le film de la vie sur Terre et qu’on le relançait en changeant légèrement le début, les êtres vivants seraient forcément dissemblables à ce que nous connaissons. Mais plutôt que de se demander si la vie serait différente, il me semble plus pertinent de chercher à savoir à quel point elle serait différente. Nous n’avons à disposition qu’un seul exemple d’histoire de la vie sur la Terre, donc comment pouvons-nous être sûrs que des conditions initiales différentes auraient conduit à des formes de vie tout à fait extravagantes ? L’idée toute simple que je voudrais avancer ici est que nous ne savons pas à quel point la vie sur Terre aurait été différente avec d’autres conditions initiales. En partant de cette hypothèse, il devient alors envisageable de faire des expériences et des observations pour essayer de lever le voile sur cette affaire.
\\

L’examen minutieux de notre passé révèle qu’à des moments et à des endroits divers, des formes de vie semblables se sont parfois échafaudées de manière indépendante. Ainsi, on rencontre dans les milieux enneigés divers animaux de coloration blanche, dans les milieux aquatiques des corps en forme de poisson et, en Australie des marsupiaux qui ressemblent aux écureuils volants d’Amérique. Les nombres sont surprenants : la photosynthèse en C4 – un métabolisme particulier qui permet aux plantes de mieux affronter la sécheresse – est apparue indépendamment plus de 60 fois, les yeux plus de 45 fois et les rats-taupes aux yeux atrophiés et aux pattes fouisseuses plus de 20 fois. Si le processus évolutif était totalement aléatoire et extrêmement sensible aux conditions initiales, on ne devrait pas observer tant de répétitions.\\

\textbf{Le paradoxe de l’évolution répétée malgré des phénomènes sous-jacents aléatoires}\\

Les données récentes de la biologie indiquent que l’évolution se répète aussi au niveau des gènes et des mutations. Dans des expériences d’évolution expérimentale (où on laisse évoluer des êtres vivants dans un environnement choisi et on répète cette même expérience plusieurs fois), on a pu observer le déploiement des mêmes mutations de façon indépendante. Aussi l’évolution répétée du même caractère chez diverses espèces est-elle souvent causée par des mutations dans le même gène. Par exemple, l’adaptation à une nourriture riche en amidon à la suite du développement de l’agriculture s’est accompagnée de mutations dans les mêmes familles de gènes chez l’homme et chez le chien.\\

Aujourd’hui, nos connaissances ont tellement avancé qu’on peut même deviner les gènes qui ont muté au cours de l’évolution. Ainsi, on peut prédire qu’une plante résistante à l’herbicide imidazolinone a de grandes chances d’avoir une mutation dans le gène ALS. Toutes ces répétitions au cours de l’histoire de la vie suggèrent que l’évolution n’est pas aussi dépendante des conditions initiales que ce qu’on aurait pu croire.\\

Comment un phénomène qui résulte de nombreux processus aléatoires (mutations, rencontres des ovules et des spermatozoïdes, accidents météorologiques, etc.) peut-il être prédictible ? C’est un peu comme un confiseur qui évalue le nombre de boîtes de chocolats qui seront achetées en fonction du mois de l’année alors qu’il ne connaît pas le comportement individuel de chacun des habitants de son quartier. Le temps, en cumulant les effets des processus aléatoires brefs, peut faire émerger des tendances prédictibles. Même si les mutations surviennent de façon imprévisible, celles qui subsistent dans les populations pendant de longues échelles de temps et qui sont responsables de changements évolutifs entre espèces peuvent être pronostiquées. Concernant l’évolution des caractères visibles des êtres vivants, l’enjeu est alors de trouver de nouveaux concepts généraux pour la prédire.\\

\textbf{Réinventer d’autres mondes en partant du nôtre}\\

La recherche en biologie fondamentale s’articule autour de deux interrogations : comment et pourquoi le vivant est-il ainsi ? Traditionnellement, la question du pourquoi a consisté à se demander pourquoi telle structure vivante existe alors qu’elle aurait pu ne pas voir le jour. Depuis quelques années, on voit se dégager un nouveau type de questionnement : pourquoi ce système vivant est-il apparu et pas un autre ? Les biologistes se mettent à imaginer d’autres mondes possibles.\\

Pour les séquences d’ADN, c’est relativement simple : on peut envisager toutes les compositions possibles des quatre lettres A, C, G et T. Pour les caractères visibles, c’est plus compliqué. Il y a au moins trois façons d’imaginer d’autres mondes vivants : on peut faire varier un paramètre (nombre de bras, constante de gravité), combiner des traits de caractère (un reptile avec des ailes de chauve-souris) ou bien transférer une propriété du domaine non vivant au vivant (des organismes qui se déplaceraient sur quatre roues). Quoi que l’on fasse, on a toujours besoin de partir de notre monde pour en inventer d’autres. Trouver les divers chemins qui étaient accessibles à l’évolution n’est donc pas une mince affaire.\\

En résumé, la trajectoire évolutive du vivant n’est pas aussi sensible aux conditions initiales que ce que l’on croyait dans les années 1990. Si les dinosaures n’avaient pas disparu, une intelligence proche de la nôtre aurait peut-être évolué quand même. Faut-il rechercher des créatures à yeux et à cerveau sur ces milliers d’exoplanètes qui pourraient abriter la vie ? Aujourd’hui, la biologie se penche sur la question et elle pourrait nous apporter bientôt des éléments de réponse.\\


\end{document}