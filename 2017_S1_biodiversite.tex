\documentclass[8pt]{article}

\usepackage[T1]{fontenc}
\usepackage[utf8]{inputenc}
\usepackage{graphicx}
\usepackage{lmodern}
\usepackage{amsmath}
\usepackage{xfrac}
\usepackage{amsthm}
\usepackage{listings}
\usepackage{enumerate}
\usepackage{amssymb}
\usepackage{cancel}
\usepackage{amsfonts}
\usepackage{float}
\usepackage{fullpage}

\DeclareUnicodeCharacter{200A}{ } 
\renewcommand*\contentsname{Table des matières}

\PassOptionsToPackage{hyphens}{url}\usepackage{hyperref}

\usepackage{listings}
\author{ControverSciences}
\title{Projet AREN : nourrir l’humanité ou préserver la biodiversité ?}
\date{11 janvier 2017}

\begin{document}
\maketitle

\tableofcontents

\newpage
\section{Textes à débattre}
\subsection{Un chasseur tue l'un des rares rhinocéros noirs pour 350.000 dollars}
\begin{itemize}
	\item \textbf{Lien : }  \url{http://www.huffingtonpost.fr/2015/05/20/un-chasseur-tue-lun-des-rares-rhinoceros-noirs-pour-350-000-dol/} 
	\item \textbf{Auteur : } Rédaction du HuffPost avec AFP
	\item \textbf{Date : } 20 mai 2015
	\item \textbf{Source : } Le Huffington Post est un journal d'information gratuit d'origine américaine qui est publié exclusivement sur Internet. Le Huffington Post est critiqué notamment aux États-Unis pour son style trop sensationnaliste et pour l'utilisation de clickbaits, consistant à s'appuyer sur des éléments sensationnels au détriment de la qualité ou de l'exactitude dans le but de générer des revenus publicitaires.
\end{itemize}

Un chasseur américain du Texas a payé 350.000 dollars le droit de tuer un rhinocéros noir en Namibie, affirmant que l'argent versé aux autorités namibiennes serait utile à la protection de cette espèce menacée d'extinction. \\

Corey Knowlton, 36 ans, a tué l'animal lundi 18 mai, après trois jours de traque en compagnie de guides du gouvernement namibien qui devaient s'assurer que l'animal abattu était bien le bon. Il était aussi accompagné d'une équipe de télévision de la chaîne américaine CNN, qui a filmé la chasse. \\

L'homme avait gagné le droit de tuer son rhinocéros noir -beaucoup plus rare encore que le rhinocéros blanc- lors d'enchères très controversées organisées à Dallas en 2014, qui avaient provoqué la fureur d'associations de protection des animaux. "Le monde entier a entendu parler de ma chasse, et je pense qu'il est très important que les gens sachent que ça s'est bien passé, de la façon la plus scientifique possible", a déclaré Corey Knowlton à CNN, dans un extrait du reportage diffusé mercredi. \\

"Depuis le début, j'ai toujours pensé que c'était quelque chose de bénéfique pour le rhino noir", a-t-il affirmé, juste après avoir délivré les coups de feu mortels. "Faire cette chasse, avec la masse de critiques, d'un côté, et la masse de félicitations, de l'autre... Je ne crois pas qu'on aurait pu faire plus pour faire parler du rhino noir".  \\

"Je pense que les gens ont un problème juste parce que j'aime la chasse (...) Mais je souhaite que le rhino noir soit le plus abondant possible. Je crois à la survie de cette espèce", assure-t-il. Sur les réseaux sociaux, sa démarche a provoqué la colère des défenseurs des animaux. Une page Facebook a même été créé pour contester le projet de Corey Knowlton, en vain.   \\

Depuis 2012, la Namibie a vendu des licences de chasse pour cinq rhinos, assurant que l'argent est utilisé pour financer des projets de protection de la nature et de lutte contre le braconnage. Les rhinos assignés aux chasseurs sont choisis parmi les animaux trop vieux pour se reproduire, affirment les autorités.  \\

Selon l'IUCN (Union internationale pour la conservation de la nature), il restait 850.000 rhinos noirs en Afrique au début du XXe siècle. Leur nombre était tombé à 2.400 en 1995, avant que des efforts de sauvegarde n'augmentent le nombre des survivants à environ 5.000. \\

Les rhinos noirs diffèrent des rhinos blancs par leur comportement et la forme de leur bouche. Les deux espèces survivent essentiellement en Afrique du Sud, mais sont braconnées pour leur corne, vendue à prix d'or sur le marché noir de la médecine traditionnelle asiatique. \\
\newpage
\subsection{Faut-il tuer les loups ? - par Bernard Bruno}
\begin{itemize}
	\item \textbf{Lien : }  \url{http://www.lexpress.fr/actualite/societe/environnement/faut-il-tuer-les-loups_489249.html} 
	\item \textbf{Auteur : } Bernard Bruno
	\item \textbf{Date : } 21 juin 2004
	\item \textbf{Description : } Deux interviews par des protagonistes ayants chacun des conflits d'intérêts très marqués concernant cette question.
	\item \textbf{Source : } Bernard Bruno est Président du Syndicat ovin des Alpes-Maritimes, éleveur à Saint-Vallier-de-Thiey.
\end{itemize}

Le loup, on ne l'a pas voulu, on n'en voudra jamais, mais on n'a pas le choix: il faut faire avec. Pourtant, on ne peut pas non plus se laisser envahir. D'ailleurs, personne n'est d'accord sur les estimations. Certains parlent de 140 bêtes. D'autres minimisent les chiffres.  \\

Depuis le retour des loups, notre mode de vie a radicalement changé. Ce n'est pas un métier qu'on exerce pour l'argent. On le fait par passion, parce qu'on aime cette vie-là. J'ai quitté l'école à 12 ans pour suivre les moutons, comme mon père. Je n'aurais jamais imaginé avoir un jour envie d'arrêter. Le pire, c'est que je n'imagine pas mes enfants prendre la relève. C'est devenu trop dur. Avant, on faisait ça jusqu'à 80 ans. Mais là... Les vieux ne pensent plus qu'à la retraite. Aucun jeune ne s'est installé depuis 1992. Et ceux qui restent n'ont pas le moral. \\

A cause du loup, les contraintes sont devenues tellement énormes qu'on est dégoûtés. C'est sûr, les chiens errants font plus de dégâts, mais, quand un chien attaque un troupeau, on sait qu'il va revenir, et on l'attend avec un fusil. Le loup est plus malin. Il attend la moindre faille, la moindre erreur pour frapper. Nous pratiquons l'élevage extensif sur des alpages où l'herbe n'est pas assez abondante pour contenir les troupeaux dans des enclos, et parquer des moutons, c'est pas une vie pour eux. Les mesures de protection, comme les chiens patous et les parcs de protection, réduisent le nombre de pertes, mais pas celui des attaques. Et, si on protège la nuit, le loup se met à attaquer le jour. C'est sans fin. Alors oui, on aimerait bien que le loup soit limité à des zones spécifiques, quitte à en éliminer quelques-uns. Ça ne va pas être facile: on l'aperçoit très rarement. \\

C'est la vie dans les montagnes qui est en jeu. Si plus personne ne veut faire ce boulot, si les montagnes se vident, elles ne seront plus entretenues. La forêt et les broussailles vont gagner sur les prairies. Il ne faudra pas s'étonner si, dans vingt ou trente ans, les feux de forêt se multiplient. \\

On n'arrive pas à se faire comprendre des gens de la ville, qui ne connaissent la nature qu'à travers leurs livres. Nous, la nature, on vit dedans, et on ne voit plus d'avenir. \\

\newpage
\subsection{Faut-il tuer les loups ? - par Olivier Rousseau}
\begin{itemize}
	\item \textbf{Lien : }  \url{http://www.lexpress.fr/actualite/societe/environnement/faut-il-tuer-les-loups_489249.html} 
	\item \textbf{Auteur : } Olivier Rousseau
	\item \textbf{Date : } 21 juin 2004
	\item \textbf{Description : } Deux interviews par des protagonistes ayants chacun des conflits d'intérêts très marqués concernant cette question.
	\item \textbf{Source : } Olivier Rousseau est Porte-parole de l'Association pour la protection des animaux sauvages (Aspas).
\end{itemize}

Alors que Jacques Chirac fait de grandes déclarations internationales et donne des leçons de protection de la nature pour sauver les éléphants ou les baleines, la France est sur le point de se mettre hors la loi: dans leur projet de «Plan d'action sur le loup 2004-2008», les ministères de l'Ecologie et de l'Agriculture envisagent d'abattre des loups. Non seulement c'est illégal - le loup est une espèce protégée par la convention de Berne, ratifiée par la France - mais cela paraît complètement insensé dans notre époque de crise écologique. Les populations sont encore très faibles, trop fragiles pour supporter les prélèvements de 5 à 7 spécimens envisagés: les chiffres les plus fiables, relevés par le réseau Loup de l'Office national de la chasse et de la faune sauvage (ONCFS), recensent entre 29 et 36 loups. \\

Pour flatter les éleveurs et satisfaire les exigences des syndicats ovins, Hervé Gaymard tient régulièrement des propos très violents à l'encontre des loups. En décembre 2003, il avait dit devant les éleveurs isérois: «A titre personnel, les loups, je les tuerais tous.» Le ministre de l'Agriculture laisse croire qu'ils sont beaucoup plus nombreux que les chiffres annoncés, pour conforter l'opinion publique dans l'idée qu'on peut en abattre quelques-uns sans dommage. L'Aspas et les associations de défense de la nature ne laisseront pas faire: toute autorisation de destruction donnera lieu à une action en justice au niveau européen. \\

La filière ovine a de gros problèmes, mais ce n'est certainement pas de cette façon qu'on les réglera. C'est un mode d'élevage qui survit sous perfusion, à coups de subventions. Le loup, ennemi juré séculaire, n'est qu'un bouc émissaire. Pourquoi ne parle-t-on jamais des ravages causés par les chiens errants? Des épidémies de brucellose? On est en plein Moyen Age! \\

Il existe des solutions, des moyens de protection: les chiens, les filets ou les aides bergers ont fait la preuve de leur efficacité. D'ailleurs, à l'étranger, cette haine héréditaire du loup n'a pas cours. Protéger les loups n'est pas un caprice ou un gadget, c'est une nécessité dans le cadre de la préservation des écosystèmes où les grands prédateurs ont un rôle à jouer. 

\newpage
\section{Corpus de ressources}
\subsection{Biodiversité, l'essentielle différence}
 
\begin{itemize}
	\item \textbf{Lien : }  \url{https://www.youtube.com/watch?v=1F6JGk51_l0} 
	\item \textbf{Auteur : } Julien Goetz
	\item \textbf{Date : } 7 février 2015
	\item \textbf{Description : } L'homme, par ses activités mondialisées, transforme l'ensemble de l'équilibre planétaire. En bâtissant des villes, en ouvrant de nouvelles terres cultivables, en transformant ses cultures en industrie, nous modifions l'habitat de nombreuses espèce vivantes, végétales ou animales. Quand nous ne les éradiquons pas purement et simplement. Or, aucune espèce n'existe de manière isolée. Elles sont toutes un maillon dans une mécanique du vivant plus large. C'est cette biodiversité qui fait la richesse du monde que nous habitons aujourd'hui. La diversité est même ce qui préserve le mieux chaque espèce. Irons-nous consciemment jusqu'à la sixième exctinction massive d'espèces sur la planète
	\item \textbf{Source : } DataGueule, chaque épisode de cette émission hebdomadaire produite par france4 tente de révéler et décrypter les mécanismes de la société et leurs aspects méconnus.
\end{itemize}

\newpage
\subsection{Quand la surpêche, les poissons coulent}
 
\begin{itemize}
	\item \textbf{Lien : }  \url{https://www.youtube.com/watch?v=_-bSD3AHgsQ} 
	\item \textbf{Auteur : } Julien Goetz
	\item \textbf{Date : } 7 février 2015
	\item \textbf{Description : } Les poissons sont de plus en plus nombreux dans nos assiettes et de moins en moins dans les mers et les océans du globe. Petit tour d'horizon d'une expansion qui a transformé en 60 ans la pêche "au gros" pour devenir la pêche "en gros.
	\item \textbf{Source : } DataGueule, chaque épisode de cette émission hebdomadaire produite par france4 tente de révéler et décrypter les mécanismes de la société et leurs aspects méconnus.
\end{itemize}

\newpage
\subsection{Combien vaut la Nature ?}
 
\begin{itemize}
	\item \textbf{Lien : }  \url{https://www.youtube.com/watch?v=-IJnr0nUpVo} 
	\item \textbf{Auteur : } Leo Grasset \& Arnaud Gantier
	\item \textbf{Date : } 7 février 2015
	\item \textbf{Description : } Si la valeur d'une vie est infinie, combien vaut la biosphère toute entière ?
	\item \textbf{Source : } DirtyBiology \& Stupid Economics. Léo Grasset (master de biologie à l'université de Montpellier) anime la chaine DirtyBiology sur Youtube, ses vidéos sont très bien sourcés tout en étant amusante. Pour cette vidéo il est aidé par Arnaud Gantier, animateur de la chaine Stupid Economics sur Youtube et du site\url{stupid-economics.com/}.
\end{itemize}
\newpage
\subsection{Il est possible de nourrir l’humanité de façon durable}
 
\begin{itemize}
	\item \textbf{Lien : }  \url{http://fr.unesco.org/news/il-est-possible-nourrir-humanite-facon-durable-conseil-consultatif-scientifique-du-secretaire} 
	\item \textbf{Auteur : } Conseil consultatif scientifique du Secrétaire général de l'UNESCO
	\item \textbf{Date : } 28 décembre 2016
	\item \textbf{Source : } UNESCO (United Nations Educational, Scientific and Cultural Organization). L'UNESCO a pour objectif selon son acte constitutif de « contribuer au maintien de la paix et de la sécurité en resserrant, par l’éducation, la science et la culture, la collaboration entre nations, afin d’assurer le respect universel de la justice, de la loi, des droits de l’Homme et des libertés fondamentales pour tous, sans distinction de race, de sexe, de langue ou de religion, que la Charte des Nations unies reconnaît à tous les peuples ». Ses scientifiques ont très peu de conflits d'intérêts.
\end{itemize}
	
	Nourrir l’humanité de façon durable est devenu une priorité mondiale majeure pour nos sociétés. À court terme, les préoccupations en matière de sécurité alimentaire dans le monde concernent la faim et la pauvreté parmi les plus démunis, de manière encore plus intense et urgente dans les pays en développement, où quelque 800 millions de personnes sont en proie à la faim et où les enfants risquent de subir des retards de croissance. Le Conseil consultatif a étudié la question de la sécurité alimentaire dans un contexte élargi en prenant notamment en compte l’utilisation et la conservation des ressources naturelles, les modes de production alimentaire et d’exploitation des ressources plus efficaces, les effets du changement climatique, ainsi que la réduction des pertes et du gaspillage alimentaires à l’échelle mondiale. Les changements nécessaires en matière de régime alimentaire, notamment le passage d’une alimentation hautement calorique à une alimentation plus riche en protéines, font partie des sujets abordés dans la note d’orientation.

Cette note d’orientation a été élaborée sous la direction de Gebisa Ejeta, membre du Conseil consultatif scientifique. Selon le Conseil consultatif, les capacités humaines et institutionnelles ont grandement besoin d’être renforcées dans de nombreux pays défavorisés afin que ces derniers puissent devenir des acteurs majeurs de la recherche de solutions, dans le cadre d’un nouveau système alimentaire mondial capable de faire face aux besoins croissants de la planète en matière d’alimentation et de nutrition.

Le Conseil consultatif plaide également pour la mise en place de solides partenariats publics et privés, éléments essentiels à l’émergence de « systèmes alimentaires » commerciaux prospères et durables pour favoriser la croissance économique, assurer des emplois rémunérateurs, et répondre aux besoins alimentaires et nutritionnels de la société pour une meilleure santé.

La note d’orientation met en exergue la nécessité de lier la sécurité alimentaire mondiale à des politiques nationales et internationales plus énergiques à l’appui de systèmes de production tenant compte des questions climatiques, avec des entreprises rentables et des systèmes alimentaires également fondés sur une gestion avisée des ressources de la planète Terre.

« En investissant dans la science, nous créons la possibilité de ralentir et d’inverser les phénomènes et tendances néfastes grâce aux décisions que nous prenons dès aujourd’hui », conclut le Conseil scientifique.

« L’histoire a montré que les investissements réalisés dans les sciences agricoles au XXe siècle avaient permis d’éviter des catastrophes et avaient largement porté leurs fruits. Dans le cadre des objectifs du Programme de développement durable à l’horizon 2030, il n’est pas impensable que cette seule planète puisse produire suffisamment pour nourrir 9 milliards de personnes de façon durable et respectueuse de l’environnement, grâce à la créativité des sciences et de l’innovation, ainsi qu’à la sagesse locale et à des politiques efficaces », explique Gebisa Ejeta.

Créé en 2014 eu égard au rôle essentiel de la science dans la réalisation des objectifs de développement durable, le Conseil consultatif scientifique est une expérience unique en son genre qui permet de fournir des avis scientifiques interdisciplinaires au Secrétaire général de l’ONU. L’UNESCO assure le Secrétariat du Conseil consultatif.        

\end{document}


